\begin{table}[ht]
    \centering
    \caption{\label{tab:pot_anwendungen}Potentielle Anwendungen von Strukturbatterien für verschiedene Einsatzbereiche~\cite{Yang2020,Danzi2021,Asp2021,Wang2020,Johannisson2019,Kalnaus2021,Nie2023,Meng2018}.}
    \begin{tabular}{m{0.15\textwidth} m{0.2\textwidth}<{\centering} m{0.4\textwidth}<{\centering} m{0.1\textwidth}<{\centering}}
        \toprule
        Technologischer Einsatzbereich&Teilbereiche&Anwendungsbeispiele&Einsetz-barkeit\\
        \midrule
        \multirow{3}*{Luftfahrt}   &Tertiärestrukturen&Entertainmentsystem und Innenverkleidungen&kurzfristig\\
                    &Sekundärstruktur&Trennwände und Gepäckfächer; Fahrwerkstüren; Rahmenstrukturen und Stellklappem&mittel-fristig\\
                    &Primärstrukturen&Flugzeugrumpf und Flügelstrukturen&langfristig\\
                    %\hspace{1em}
                    \hline
        \multirow{3}*{Automobil}   &Karosserieteile&Türverkelidungen, Innenraumelemente und Sitzstrukturen&kurzfristig\\
                    &Sekundäre Fahrzeugstrukturen&Armaturenebrett; Dachhimmel; Trennwände&mittel-fristig\\
                    &Tragende Strukturen&Fahrgestell oder Karosserie&langfristig\\
                    \hline
        \multirow{4}*{Robotik}& Leichte Robotikstrukturen & Integration in kleine Roboterarme, Serviceroboter oder Drohnen&kurzfristig\\
                    &Mittellastragende Strukturen&Nutzung in mittellasttragenden Strukturen wie Gelenken und Rahmen von Industrierobotern&mittel-fristig\\
                    &Tragende Hauptstrukturen&Integration in Haupttragstrukturen von größeren Robotern oder autonomen Systemen&langfristig\\
                    &Hochdynamische/bestastete Systeme&Anwendungen in hochbelasteten und sicherheitskritischen Bereichen wie schwerlasttragende Industrieroboter&langfristig\\
        \bottomrule
    \end{tabular}
\end{table}