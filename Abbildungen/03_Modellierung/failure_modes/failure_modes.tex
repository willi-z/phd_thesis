\begin{table}[ht]
    \centering
    \caption{\label{tab:failure_modes}Kritikalität beim Versagen der Einzelschichten einer Batterie im Überblick.}
    \small
    \begin{tabularx}{\textwidth}{lXXX}
    \toprule
    &\makecell[t]{\textbf{Pouchfolien-}\\\textbf{versagen}\\\includegraphics[width=0.24\textwidth]{failure_modes/failure_mode_pouch.png}}
    &\makecell[t]{\textbf{Elektroden-}\\\textbf{versagen}\\\includegraphics[width=0.24\textwidth]{failure_modes/failure_mode_electrode.png}}
    &\makecell[t]{\textbf{Separator-}\\\textbf{versagen}\\\includegraphics[width=0.24\textwidth]{failure_modes/failure_mode_separator.png}}
    \\
    \midrule
    \makecell[t]{\textbf{Funktion-}\\\textbf{versagen}}
        & Versagen gesamten Batterie durch austrocknen
        & Leistungsverlust der Zelle
        & Versagen der Zelle, je nach Verschaltung auch der gesamten Batterie
    \\
    \addlinespace
    \makecell[t]{\textbf{Brandgefahr}}
        & kein Risiko
        & kein Risiko
        & Flammenbildung durch Überhitzung
    \\
    \addlinespace
    \makecell[t]{\textbf{Gesundheits-}\\\textbf{risiko}}
        & hohes Risiko durch austretendes Elektrolyt
        & kein Risiko
        & kein zusätzliches Risiko
    \\
    \bottomrule
    \end{tabularx}\\
\end{table}