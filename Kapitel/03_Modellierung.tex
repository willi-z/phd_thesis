\chapter{\label{sec:modelling_SB}Entwicklung wichtiger Ausgangsgleichungen im Kontext von Strukturbatterien}
Die Modellierung von Strukturbatterien auf der Mikroskalenebene wurde in den letzten Jahren maßgeblich durch die Arbeiten von Carlstedt~\cite{Carlstedt2018,Carlstedt2019,Carlstedt2022a,Carlstedt2023} vorangetrieben. Dabei wurden bereits Modelle sowie die Kopplung mechanischer, elektrochemischer und thermischer Effekte erfolgreich entwickelt~\cite{Carlstedt2022,Carlstedt2022b}. Darüber hinaus existieren weitere Ansätze aus der Forschung zu konventionellen Batterien, die in Kapitel~\ref{sec:existing_micro_models} kurz vorgestellt werden. Die Modelle von Carlstedt konzentrieren sich vorrangig auf das Verhalten auf Mikro- bzw. Partikelebene. Es existiert jedoch eine Vielzahl an Arbeiten, die zeigen, wie diese mikroskaligen Modelle mittels Homogenisierung auf höhere Skalen übertragen werden können. Die aus der Kombination der aus Batterieforschung bekannten Simualtionsansätze mit den Kopplungsansätzen von \textsc{Carlstedt} folgende gesamtheitliche Modellierung übersteigt die bekannten Arbeiten in ihrer komplexität und Detailgrad. Auch erlaubt dieser Ansatz, unter Einbeziehnung existierender Homogenisierungensansätze, die leichtere Entwkikclungen eines kompletten makroskaligen Modellierungsansatzes, siehe Kapitel~\ref{sec:homogenisation}.

% filepath: /home/williz/Promotion/Monografie/Kapitel/03_01_Micromodells.tex

\section{\label{sec:existing_micro_models}Mikroskalenmodelle}

Die auf Mikro- oder Partikelebene ablaufenden Prozesse sind grundsätzlich unabhängig davon, ob eine konventionelle oder eine Strukturbatterie betrachtet wird. Einige dieser Prozesse spielen in konventionellen Batterien jedoch nur eine untergeordnete Rolle und werden daher häufig vernachlässigt oder vereinfacht dargestellt~\cite{Carlstedt2020a}. Der Ionentransport stellt dabei den zentralen Prozess dar~\cite{Carlstedt2019b}. Nach \textsc{Newman} bestehen signifikante Unterschiede im Transportverhalten zwischen flüssigen und festen Phasen~\cite{Newman2021}. Da sowohl konventionelle als auch strukturelle Batterien mit zweiphasigen Elektrolyten einen Ionentransport durch beide Phasen ermöglichen, lässt sich ihr Verhalten in erster Näherung durch die folgenden fünf Differentialgleichungen, auch als \textsc{Doyle}-\textsc{Fuller}-\textsc{Newman}-Modell bezeichnet, beschreiben~\cite{Plett2015}.
\begin{enumerate}
    \item Ladungserhalt in homogenen Festkörpern
    \begin{equation}
        \nabla \cdot \boldsymbol{i}_{\text{s}} = \nabla \cdot \left( - \sigma \cdot \nabla \phi_{\text{s}} \right) = 0
    \end{equation}

    \item Massenerhalt in homogenen Festkörpern
    \begin{equation}
        \frac{\partial c_{\text{s}}}{\partial t}  = \nabla \cdot \left( D_{\text{s}} \nabla c_{\text{s}} \right) = 0
    \end{equation}

    \item Massenerhalt im homogenen Elektrolyt
    \begin{equation}
        \frac{\partial c_e}{\partial t} = \nabla \cdot \left( D_e   \nabla c_e \right) - \frac{\boldsymbol{i}_{\text{e}} \cdot    \nabla t_+^0}{F_{\text{K}}} - \nabla \cdot \left( c_{\text{e}} \boldsymbol{v}_0\right)
    \end{equation}

    \item Ladungserhalt im homogenen Elektrolyt
    \begin{equation}
        \nabla \cdot \boldsymbol{i}_{\text{e}} = \nabla \cdot \left(    - \kappa \nabla \phi_{\text{e}}  -\frac{2\kappa R_{\text{K}} T}{F_{\text{K}}} \left(  1+ \frac{\partial \ln f_\pm}{\partial \ln c_{\text{e}}}\right)   \left( t_+^0-1\right) \nabla \ln c_{\text{e}} \right) = 0
    \end{equation}

    \item Ionentransport zwischen fester und flüssiger Phase
    \begin{align}
        j &= \frac{i_0}{F}\left( \exp \left(\frac{\left(1-\alpha\right)  F}{RT}\eta \right) - \exp \left(-\frac{\alpha F_{\text{K}}}{R_{\text{K}} T}  \eta\right) \right)\\
        i_0 &= n F_{\text{K}} k_{0,\text{K}} \left(\prod_i c_{o,i}\right)^{1-\alpha} \left( \prod_i c_{r,i}\right)^\alpha\\
        \eta &= (\phi_{\text{s}}-\phi_{\text{e}}) - U_{\text{ocp}}
    \end{align}
\end{enumerate}
Dabei beschreibt $\boldsymbol{i}$ die Stromdichte, $\sigma$ die elektrische Leitfähigkeit des Materials, $\phi$ das elektrische Potenzial, $c$ die Konzentration der Ladungsträger, $D$ den effektiven Diffusionskoeffizienten, $\boldsymbol{t}^0_+$ die Hittorfsche Überführungszahl der Kationen bezogen auf das Elektrolytsystem, $F_{\text{K}}$ die Faraday-Konstante, $\boldsymbol{v}_0$ die Geschwindigkeit des Elektrolyten, $\kappa$ die ionische Leitfähigkeit, $R_{\text{K}}$ die ideale Gaskonstante, $T$ die Temperatur, $f_{\pm}$ den mittleren molaren Aktivitätskoeffizienten, $j$ die molare Flussdichte der Ionen, $i_0$ die Austauschstromdichte\footnote{Vereinfacht sich für Lithium und Natrium zu: $i_0 = F_{\text{K}} k_{0,\text{K}}  c_e^{1-\alpha} (c_{s,\text{max}} - c_{s,e})^{1-\alpha} c_{s,e}^\alpha$}, $\eta$ das Reaktionsüberpotenzial, $k_{0,\text{K}}$ die effektive Reaktionsratenkonstante, $U_{\text{ocp}}$ das Open-Circuit-Potenzial (Leerlaufspannung) und $\alpha$ den asymmetrischen Ladungstransferkoeffizienten. Letzterer ist durch das Verhältnis aus Änderung der Aktivierungsenergie der Reduktionsmittel ($\Delta E_{\text{a,red}}$) und Änderung der Gibbs-Energie der Oxidationsmittel ($\Delta G_0$) definiert:
\begin{equation}
        \alpha = \left|\frac{\Delta E_{\text{a,red}}}{\Delta G_0}\right|
\end{equation}
und ist dadurch auf den Wertebereich $0 < \alpha < 1$ beschränkt.

Neben dem Ladungstransport beeinflussen auch die Temperaturentwicklung sowie die Entstehung mechanischer Spannungen das Systemverhalten. Die Temperaturverteilung in der festen und flüssigen Phase wird dabei durch die Dichte $\rho$, die spezifische Wärmekapazität $c_\text{P}$, die Wärmeleitfähigkeit $\lambda$ sowie den elektrischen Strom bestimmt~\cite{Gao2021,Katrasnik2021}.
\begin{align}
    \rho_{\text{s}} c_{\text{P,s}} \frac{\partial T_{\text{s}}}{\partial t} &= \nabla \cdot (\lambda_{\text{s}} \nabla T_{\text{s}}) - \boldsymbol{i}_{\text{s}} \cdot \nabla \phi_{\text{s}}\\
    \rho_{\text{e}} c_{\text{P,e}} \frac{\partial T_{\text{e}}}{\partial t} &= \nabla \cdot (\lambda_{\text{e}} \nabla T_{\text{e}}) - \boldsymbol{i}_{\text{e}} \cdot \nabla \phi_{\text{e}}
\end{align}

Mechanischen Spannungen kommt insbesondere im Kontext von Strukturbatterien eine zentrale Rolle zu~\cite{Carlstedt2020b}. Auch bei konventionellen Batterien werden sie als ein entscheidender Faktor für bestimmte Alterungsmechanismen berücksichtigt~\cite{Mueller2019}. Dabei können mechanische Spannungen ausschließlich in der Festkörperphase auftreten~\cite{Kaliaperumal2021,Berg2022}. Für statische und rein mechanische Problemstellungen folgt ihre Beschreibung durch die lokale Impulsbilanz:
\begin{equation}\label{eq:stress_gov}
    -\nabla \cdot \boldsymbol{\sigma} + f = \boldsymbol{0}.
\end{equation}
Für kleine Deformationen und homogene Werkstoffe kann das Deformationsverhalten durch das \textsc{Hook}sche Gesetz beschrieben werden:
\begin{equation}\label{eq:stress_material}
    \boldsymbol{\sigma} = \boldsymbol{C} \boldsymbol{\varepsilon}_{mech}
\end{equation}
Der Elastizitätstensor $\boldsymbol{C}$ wird im Kontext von Strukturbatterien in Abhängigkeit vom Material als isotrop\footnote{z.\,B. Metallelektrode, Aktivmaterial, Polymerphase},
\begin{align}
\boldsymbol{C}^{-1}_{\text{iso}} &= 
\begin{bmatrix}
    \frac{1}{E} & -\frac{\nu}{E} & -\frac{\nu}{E} & 0 & 0 & 0 \\
    -\frac{\nu}{E}& \frac{1}{E} & -\frac{\nu}{E} & 0 & 0 & 0 \\
    -\frac{\nu}{E} & -\frac{\nu}{E} & \frac{1}{E} & 0 & 0 & 0 \\
    0 & 0 & 0 & \frac{2(1+\nu)}{E} & 0 & 0 \\
    0 & 0 & 0 & 0 & \frac{2(1+\nu)}{E} & 0 \\
    0 & 0 & 0 & 0 & 0 & \frac{2(1+\nu)}{E} \\
\end{bmatrix}
\end{align}
transversal-isotrop\footnote{z.\,B. einzelne Kohlenstofffaser},
\begin{align}
\boldsymbol{C}^{-1}_{\text{trans}} &= 
\begin{bmatrix}
    \frac{1}{E_{1}} & -\frac{\nu_{12}}{E_{1}} & -\frac{\nu_{13}}{E_{1}} & 0 & 0 & 0 \\
    -\frac{\nu_{12}}{E_{1}}& \frac{1}{E_{2}} & -\frac{\nu_{23}}{E_{2}} & 0 & 0 & 0 \\
    -\frac{\nu_{13}}{E_{1}} & -\frac{\nu_{23}}{E_{2}} & \frac{1}{E_{2}} & 0 & 0 & 0 \\
    0 & 0 & 0 & \frac{2(1+\nu_{23})}{E_{2}} & 0 & 0 \\
    0 & 0 & 0 & 0 & \frac{1}{G_{31}} & 0 \\
    0 & 0 & 0 & 0 & 0 & \frac{1}{G_{12}} \\
\end{bmatrix}
\end{align}
oder orthotrop\footnote{z.\,B. Kohlenstofffasergewebe, Glasfaserseparator}:
\begin{align}
\boldsymbol{C}^{-1}_{\text{ortho}} &= 
\begin{bmatrix}
    \frac{1}{E_{1}} & -\frac{\nu_{12}}{E_{1}} & -\frac{\nu_{13}}{E_{1}} & 0 & 0 & 0 \\
    -\frac{\nu_{12}}{E_{1}}& \frac{1}{E_{2}} & -\frac{\nu_{23}}{E_{2}} & 0 & 0 & 0 \\
    -\frac{\nu_{13}}{E_{1}} & -\frac{\nu_{23}}{E_{2}} & \frac{1}{E_{3}} & 0 & 0 & 0 \\
    0 & 0 & 0 & \frac{1}{G_{23}} & 0 & 0 \\
    0 & 0 & 0 & 0 & \frac{1}{G_{31}} & 0 \\
    0 & 0 & 0 & 0 & 0 & \frac{1}{G_{12}} \\
\end{bmatrix}
\end{align}
beschrieben.

Besonders bei den Materialien, die als Interkalationsort dienen, haben Untersuchungen von \textsc{Duan}~\cite{Duan2021} gezeigt, dass die Elastizitätsmodule $E_i$, mit $i \in [1,2,3]$, näherungsweise linear von der Ionenkonzentration abhängig sind:
\begin{equation}
    E_i(c_{s}) = E_{i,0} + \frac{c_{s}}{c_{s,1}} (E_{i,1} - E_{i,0}).
\end{equation}

Die Gesamtdehnung $\boldsymbol{\varepsilon}$ ergibt sich dabei aus der Summe der elektrochemischen, thermischen und mechanischen Dehnungsanteile:
\begin{equation}\label{eq:strain_total}
    \boldsymbol{\varepsilon} = \boldsymbol{\varepsilon}_{echem} +\boldsymbol{\varepsilon}_{th} + \boldsymbol{\varepsilon}_{mech}
\end{equation}
und wird direkt aus dem Verschiebungsfeld $u$ bestimmt:
\begin{equation}\label{eq:strain_total_displacement}
    \boldsymbol{\varepsilon} = \frac{1}{2}\left[\left(\nabla u\right)^T + \left(\nabla u\right)\right].
\end{equation}
Die thermischen und elektrochemischen Dehnungsanteile hängen von den jeweiligen Ausdehnungskoeffizienten $\boldsymbol{\alpha}$ linear von der Veränderung der Temperatur beziehungsweise Konzentration ab:
\begin{align}
    \boldsymbol{\varepsilon}_{echem} &= \boldsymbol{\alpha}_{echem} \left(c_{\pm}-c_{\pm,0}\right),\\
    \boldsymbol{\varepsilon}_{th}  &= \boldsymbol{\alpha}_{th}\left( T - T_0\right).
\end{align}
\begin{figure}[!ht]
    %\raggedleft
        %\def\svgwidth{\columnwidth}
        \center
        \includegraphics[width=1.0\textwidth, angle=0]{cahn-hilliard.pdf}
        \caption{\label{fig:cahn-hilliard}a) Zufallsverteilte Ausgangswerte, b) Funktion $f(c)$ zur Separation der Konzentrationen, c)--d) Ergebnisse mit $M=0{,}2$, $\lambda = 0{,}5$}
\end{figure}
Die aus den Gleichungen abgeleitete mikroskalige Modellierung kann eingesetzt werden, um Halbzellen mit Geometrien im vergleichbaren Größenspektrum zu analysieren~\cite{Plett2015}. Zur realitätsnahen Simulation einer Strukturbatteriezelle aus zwei Fasern ist es jedoch erforderlich, auch die Struktur des Zweiphasen-Elektrolyten adäquat abzubilden~\cite{Tu2020}. Die zugrunde liegende Geometrie ergibt sich aus dem Prozess der Phasenseparierung, welcher durch die \textsc{Cahn-Hilliard}-Gleichung beschrieben werden kann~\cite{Carolan2015,Grant1993}.
\begin{align}
    \frac{\partial c}{\partial t} - \nabla \cdot M \left( \nabla \left( \frac{df}{dc} - \lambda \nabla^2 c\right) \right) &= 0 \text{ in }\Omega\\
    M\left( \nabla \left( \frac{df}{dc} - \lambda \nabla^2 c \right)\right) \cdot \boldsymbol{n} &= 0 \text{ auf }\partial\Omega
\end{align}
Als Ausgangspunkt wird dabei häufig eine zufallsbasierte Konzentrationsverteilung $c(\boldsymbol{x})$ genommen, siehe Bild~\ref{fig:cahn-hilliard}a, die nach Konvention den folgenden Zusammenhang zum Phasenanteil $\varepsilon$ aufweist:
\begin{equation}
    \varepsilon \hat{=} \frac{1}{\left| \Omega \right|}\int_\Omega c(\boldsymbol{x}) \partial \boldsymbol{x}.
\end{equation}
Dabei wird die Phasenseparation der Konzentration $c$\footnote{Nach Konvention gehören Konzentrationswerte kleiner 0{,}5 zur ersten Phase, während Werte größer 0{,}5 der zweiten Phase zugeordnet werden.}, siehe Bild~\ref{fig:cahn-hilliard}c--d, allein durch zwei Parameter $f$\footnote{Eine in $c$ nicht-konvexe Polynomfunktion 4. Grades.}, siehe Bild~\ref{fig:cahn-hilliard}b, und $M$ beschrieben. Da die \textsc{Cahn-Hilliard}-Gleichung jedoch eine Differentialgleichung vierter Ordnung ist, führt dies in der schwachen Formulierung zu Ortsableitungen zweiter Ordnung, was mit Standard-Lagrange-Elementen nicht direkt lösbar ist. Eine häufig verwendete Herangehensweise zur Lösung dieses Problems besteht darin, die Gleichung mittels Operatorzerlegung umzuformulieren:
\begin{align}
    \frac{\partial c}{\partial t} - \nabla \cdot M \nabla \mu &= 0 \text{ in }\Omega\\
    \mu - \frac{\partial f}{\partial c} + \lambda \nabla^2 c &= 0 \text{ in }\Omega
\end{align}

Das resultierende Mikroskalenmodell besteht aus einer einfasrigen Kohlenstofffanode und einer mit LFP beschichteten Kohlenstofffkathode, die durch einen Strukturelektrolyt verbunden sind, siehe Bild~\ref{fig:micro_model}a. Dabei befinden sich die Poren des Strukturelektrolyten im Nanometerbereich, während die typischen Partikelgrößen der LFP-Komponenten und Kohlenstofffasern $1~\mu\text{m}$ beziehungsweise $10~\mu\text{m}$ betragen~\cite{Chaudhary2024a,Huson2014}. Dies erfordert ein äußerst feines Rechennetz\footnote{Hier $180 \times 180 \times 640 = 20\,736\,000$ Elemente.}, um die relevanten Mikrostrukturen adäquat abzubilden, siehe Bild~\ref{fig:micro_model}b.
\begin{figure}[!ht]
    %\raggedleft
        %\def\svgwidth{\columnwidth}
        \center
        \includegraphics[width=1.0\textwidth, angle=0]{micro_model.pdf}
        \caption{\label{fig:micro_model}a) Simulation einer Zweifaser-Batterie aus einer Kohlenstofffaser als Anode und einer mit LFP beschichteten Kohlenstofffaser als Kathode, b) Blockvernetzung und Zuweisung der Domänen für die gekoppelte FE-Simulation, c) Der angelegte Strom als treibende Randbedingung über die Zeit, d) Die elektrische Spannung und Stromdichte über die Zeit, e) Die gemittelte Temperatur über die Zeit. Die Lithiumkonzentration (f), die mechanische Spannung (g) und die Temperaturverteilung (h) bei $t = 2000~\text{s}$.}
\end{figure}
In Kombination mit den nichtlinearen Differentialgleichungen und den vielfältigen physikalischen Kopplungen resultiert daraus ein komplexes Simulationsmodell\footnote{Um die Parallelisierbarkeit von Blocknetzen möglichst gut auszunutzen, werden alle benötigten Parameter allen Knoten zugewiesen. Bereiche, die nicht an den jeweiligen Prozessen teilnehmen, bekommen dafür um mehrere Größenordnungen größere beziehungsweise kleinere Parameter. Außerdem verhindert dieser Ansatz Singularitäten in der Matrix, die sonst bei isolierten Bereichen entstehen können.}. Die Simulation eines vollständigen Entlade- und Beladevorgangs wird dadurch sehr rechenintensiv\footnote{Berechnungsserver der HTWK unter Ausnutzung von zwei integrierten AMD EPYC 7F753 CPUs mit einer Taktfrequenz von $2{,}95~\text{GHz}$ und jeweils 32 Rechenkernen.} (siehe Bild~\ref{fig:micro_model}).
\section{\label{sec:homogenisation}Homogenisierung von Mikroskalenmodellen}

Die Modellierung der einzelnen physikalischen Prozesse ist auf der Mikroskala häufig einfacher umzusetzen~\cite{Plett2015}. Mithilfe mikroskaliger Modelle lassen sich die Einflüsse der Geometrie, Verteilung und Clusterbildung präzise ermitteln~\cite{Newman2021}. Aufgrund der hohen Komplexität, die mit den verschiedenen Skalenbereichen einhergeht, ist der damit verbundene Berechnungsaufwand jedoch zu groß, um eine Vielzahl von Zellen effizient zu simulieren~\cite{Liu2019}. Daher sind makroskalige Modelle erforderlich, welche den Rechenaufwand durch Homogenisierung und geeignete Modellvereinfachungen deutlich reduzieren~\cite{Plett2015}. Darüber hinaus bestehen Abweichungen durch Skalierungseffekte sowie durch die richtige Abbildung der untersuchten Mikrostruktur und durch nicht hinreichend bestimmte Materialkennwerte.

Ein häufig verwendeter Ansatz stellt die Mittelung der physikalischen Eigenschaften über ein repräsentatives Volumenelement~(RVE) dar~\cite{Burow2016,Arunachalam2019,Li2020}. Die dazugehörigen mathematischen Grundlagen basieren auf drei Volumenmittelungstheoremen~\cite{Gray1977}.
\begin{enumerate}
    \item Volumenmittelung für ein skalares Feld $\psi$ 
    \begin{equation}
        \varepsilon_{\alpha} \overline{\nabla \psi_{\alpha}} = \nabla \left(\varepsilon_{\alpha} \bar{\psi}_{\alpha} \right) + \frac{1}{V} \iint_{A_{\alpha \beta(\boldsymbol{x},t)}}\psi_{\alpha} \hat{\boldsymbol{n}}_{\alpha} \,\mathrm{d}A,
    \end{equation}
    \item Volumenmittelung für ein Vektorfeld $\boldsymbol{\psi}$
    \begin{equation}
        \varepsilon_{\alpha} \overline{\nabla \cdot \boldsymbol{\psi}_{\alpha}} = \nabla \cdot \left(\varepsilon_{\alpha} \bar{\boldsymbol{\psi}}_{\alpha} \right) + \frac{1}{V} \iint_{A_{\alpha \beta(\boldsymbol{x},t)}}\boldsymbol{\psi}_{\alpha} \cdot \hat{\boldsymbol{n}}_{\alpha} \,\mathrm{d}A,
    \end{equation}
    \item Volumenmittelung für die zeitliche Änderung eines skalaren Feldes $\psi$ 
    \begin{equation}
        \varepsilon_{\alpha} \overline{\left[\frac{\partial \psi_{\alpha}}{\partial t}\right]} = \frac{\partial \left(\varepsilon_{\alpha} \bar{\psi}_{\alpha} \right)}{\partial t} - \frac{1}{V} \iint_{A_{\alpha \beta(\boldsymbol{x},t)}}\psi_{\alpha} \boldsymbol{v}_{\alpha \beta} \cdot \hat{\boldsymbol{n}}_{\alpha} \,\mathrm{d}A.
    \end{equation}
\end{enumerate}
Dabei beschreibt $\bar{\psi}_{\alpha}$ bzw. $\bar{\boldsymbol{\psi}}_{\alpha}$ die intrinsische Mittelung über Phase $\alpha$. Diese Mittelung wird nur über das von Phase $\alpha$ eingenommene Volumen\footnote{Hier als Zwei-Phasen-System mit der zweiten Phase $\beta$ betrachtet.} ermittelt. Die intrinsische Mittelung bietet gegenüber einer klassischen Mittelung $\langle \psi_{\alpha} \rangle$, die sich auf das Volumen des gesamten Gebiets bezieht, größere Flexibilität und Wiederverwendbarkeit\footnote{Intrinsische Werte können wegen der Unabhängigkeit vom Phasenanteil für beliebige Phasenanteile wiederverwendet werden.}. Mittels des Volumenanteils $\varepsilon_{\alpha}$
\begin{equation}
    \varepsilon_{\alpha} = \frac{V_{\alpha}(\boldsymbol{x},t)}{V} 
\end{equation}
können die beiden Mittelungsarten ineinander umgewandelt werden:
\begin{equation}
    \langle \psi_{\alpha} \rangle = \varepsilon_{\alpha} \bar{\psi}_{\alpha}.
\end{equation}

Mithilfe der drei Volumenmittelungstheoreme lassen sich die folgenden vier Gleichungen herleiten~\cite{Doyle1995}.
\begin{enumerate}
    \item Volumengemittelte Näherung des Ladungserhalts in der festen Phase der porösen Elektrode
    \begin{equation}
        \nabla \cdot \left(\sigma_{\text{eff}} \nabla \hat{\phi}_{s} \right) = a_s F_{\text{K}} \hat{j},
    \end{equation}
    \item Volumengemittelte Näherung des Ladungserhalts in der Elektrolytphase der porösen Elektrode
    \begin{equation}
        \nabla \cdot \left(\kappa_{\text{eff}} \nabla \hat{\phi}_e + \kappa_{D, \text{eff}} \nabla \ln \hat{c}_e\right) + a_s F_{\text{K}} \hat{j} = 0,
    \end{equation}
    \item Volumengemittelte Näherung des Massenerhalts in der Elektrolytphase der porösen Elektrode
    \begin{equation}
        \frac{\partial \left(\varepsilon_e \hat{c}_e \right)}{\partial t} = \nabla \cdot \left(D_{e,\text{eff}}\nabla\hat{c}_e\right) + a_s (1+t^0_+) \hat{j},
    \end{equation}
    \item Volumengemittelte Näherung der mikroskopischen Butler-Volmer-Beziehung für den Ionenphasenwechsel
    \begin{equation}
        \hat{j} = j(c_{s,e},\hat{c}_e,\hat{\phi}_s,\hat{\phi}_e).
    \end{equation}
\end{enumerate}

Analog lassen sich für die mechanische Spannung und die Temperatur die folgenden Zusammenhänge aufstellen.
\begin{enumerate}
    \item Homogenisierung der mechanischen Spannung
    \begin{equation}
    \boldsymbol{\sigma} = \boldsymbol{C}_{\text{eff}} \boldsymbol{\varepsilon}_{\text{mech}},
    \end{equation}
    \item Volumengemittelte Darstellung der Temperaturentwicklung
    \begin{equation}
        \frac{\partial (\rho c_{\text{P}} T)}{\partial t} = \nabla \cdot (\lambda \nabla T) + q.
    \end{equation}
\end{enumerate}

Die eingeführte Wärmequelle $q$ kann dabei aus den folgenden fünf Beiträgen zusammengesetzt werden~\cite{Plett2015}.
\begin{enumerate}
    \item Irreversible Wärmeentstehung durch chemische Reaktionen
    \begin{equation}
        q_i = a_{\text{s}} F_{\text{K}} \hat{j}_j \eta_{j},
    \end{equation}
    \item Reversible Wärmebildung durch Entropieänderung
    \begin{equation}
    q_{r} = a_{\text{s}} F_{\text{K}} \hat{j}_j T \frac{\partial U_{\text{ocp},j}}{\partial T},
    \end{equation}
    \item Joule-Wärme im Feststoff
    \begin{equation}
    q_{s} = \sigma_{\text{eff}}(\nabla\hat{\phi}_{\text{s}} \cdot \nabla\hat{\phi}_{\text{s}}),
    \end{equation}
    \item Joule-Wärme im Elektrolyt
    \begin{equation}
        q_{e} = \kappa_{\text{eff}}(\nabla\hat{\phi}_{\text{e}} \cdot \nabla\hat{\phi}_{\text{e}}) + \kappa_{D,\text{eff}} (\nabla \ln \hat{c}_e \cdot \nabla \hat{\phi}_{\text{e}}),
    \end{equation}
    \item Wärmeentstehung durch Kontaktwiderstände\footnote{$q_c$ gilt nur für die Elektrodenfläche und ist daher auf die Einheitsfläche bezogen; die anderen Terme sind auf das Einheitsvolumen bezogen.}
    \begin{equation}
        q_{c} = i_{\text{app}}^2 R_{\text{Kontakt}}.
    \end{equation}
\end{enumerate}

\begin{figure}[!ht]
    \center
    \includegraphics[width=0.8\textwidth, angle=0]{carlstedt.pdf}
    \caption{\label{fig:carlstedt}a) Beispielhafte Darstellung der untersuchten Kohlenstofffaser-Strukturbatterie und der LFP-Zelle nach~\cite{Carlstedt2022b}, b) zweidimensionales Modell zur Durchführung der FEM-Simulation, c) Zeitverlauf des angelegten Stroms als treibende Randbedingung, d) elektrische Spannung und Stromdichte im zeitlichen Verlauf sowie die Lithiumkonzentration zu den Zeitpunkten $t_1 = 2000\,\text{s}$ und $t_2 = 6000\,\text{s}$, e) gemittelte Temperatur über die Zeit sowie Temperaturverteilungen bei $t_1$ und $t_2$, f) mechanische Spannungskomponenten $\sigma_{11}$ und $\sigma_{22}$ zu den Zeitpunkten $t_1$ und $t_2$.}
\end{figure}

Angelehnt an Arbeiten von \textsc{Carlstedt}~\cite{Carlstedt2022b}\footnote{Die Materialwerte, Geometrie und Randbedingungen wurden der Arbeit entnommen, um einen Vergleich zu ermöglichen.} können diese Gleichungen bereits verwendet werden, um das Verhalten ganzer Zellen zu beschreiben\footnote{Hier: eine Kohlenstofffaser-LFP-Zelle} (Bild~\ref{fig:carlstedt}). Die Zelle durchläuft dabei einen Entlade- und Ladezyklus innerhalb von 2,2\,h. Die Simulationszeit betrug 34,6\,h auf einem Berechnungsserver der HTWK\footnote{Unter voller Ausnutzung von zwei eingebauten CPUs der Marke AMD EPYC 75F3 mit einer Taktrate von 2,95\,GHz und jeweils 32 Kernen.}. Der hohe Rechenaufwand bereits für einen Ladezyklus macht diesen Ansatz jedoch ungeeignet, um eine Vielzahl an Varianten und größere, mehrzellige Batteriesysteme auszulegen.

Durch Ermittlung effektiver physikalischer Eigenschaften werden die Inhomogenitäten auf der Mikroskala durch ein Kontinuum auf der Makroskala beschrieben~\cite{Plett2024}. Die Genauigkeit dieses Ansatzes hängt jedoch stark von den zu betrachtenden Längenskalen ab~\cite{Plett2015}. Lokal erhöhte Porendichten oder ähnliche inhomogene Effekte lassen sich nur aufwendig berücksichtigen~\cite{Mei2019}. Bei der Analyse deutlich größerer Skalen als die Inhomogenitäten zeigen diese Modelle hingegen eine höhere Effizienz und ausreichende Genauigkeit~\cite{Plett2015}. 

Um die Berechnungszeit weiter zu reduzieren, kann aufgrund der Butler-Volmer-Randbedingung keine Volumenmittelung für die Massenerhaltung in der festen Phase\footnote{Die Materialien, die als Interkalationsort dienen.} verwendet werden~\cite{Plett2015}. Durch Geometrievereinfachungen lassen sich jedoch Freiheitsgrade reduzieren und zusätzlicher Rechenaufwand vermeiden. Im Kontext von Strukturbatterien ist der interkalationsaktiv teilnehmende Bereich partikel- oder faserförmig und kann durch Kugeln bzw. Zylinder approximiert werden~\cite{Newman2021}. Daraus ergeben sich die nachfolgenden Gleichungen:
\begin{enumerate}
    \item Spezialfall Massenerhalt in kugelförmigen Festkörpern
    \begin{equation}
        \label{eq:diffusion_sphere}
    \frac{\partial c_{\text{s}}}{\partial t} = \frac{1}{r^2} \frac{\partial}{ \partial r} \left[ D_{\text{s}} r^2 \frac{\partial c_{\text{s}}}{\partial r}\right],
    \end{equation}
    \item Spezialfall Massenerhalt in zylindrischen Festkörpern
    \begin{equation}
        \label{eq:diffusion_cylinder}
    \frac{\partial c_{\text{s}}^{\pm}}{\partial t} = \frac{1}{r} \frac{\partial}{ \partial r} \left[ D_{\text{s}} r \frac{\partial c_{\text{s}}}{\partial r}\right] + \frac{\partial}{ \partial z}\left[D_{\text{s}}  \frac{\partial c_{\text{s}}}{\partial z}\right].
    \end{equation}
\end{enumerate}
Dabei ist für viele Szenarien die Verteilung der Konzentration in $z$-Richtung näherungsweise konstant~\cite{Wang2020c}. In diesem Fall kann der Massenerhalt in zylindrischen Festkörpern weiter vereinfacht werden:
\begin{equation}
    \frac{\partial c_{\text{s}}^{\pm}}{\partial t} = \frac{1}{r} \frac{\partial}{ \partial r} \left[ D_{\text{s}} r \frac{\partial c_{\text{s}}}{\partial r}\right].
\end{equation}
In beiden Fällen lässt sich das Interkalationsverhalten durch die Randbedingungen
\begin{align}
    \left.\frac{\partial c_{\text{s}}^{\pm}}{\partial r}\right\vert_{r=0} &= 0, \\
    \left.\frac{\partial c_{\text{s}}^{\pm}}{\partial r}\right\vert_{r=R_{\text{p,s}}^{\pm}} &= -\frac{1}{ D_{\text{s}}^\pm} j_{n}^{\pm}(x,t),
\end{align}
darstellen, wobei im Falle einer stromgesteuerten Be- und Entladung
\begin{equation}
j_{n}^{\pm}(t) = \mp \frac{I(t)}{F a^{\pm} L^{\pm}}
\end{equation}
ist~\cite{Plett2015}.

\begin{figure}[!ht]
    \center
    \includegraphics[width=0.99\textwidth, angle=0]{p2d_model.pdf}
    \caption{\label{fig:p2d_model}a) Vereinfachung und Überführung einer NMC-Zelle zu einem 2D-Modell für die FEM-Berechnung, b) elektrische Spannung über mehrere durch den Strom geprägte Lade- und Entladezyklen, c) Temperaturverlauf während der Zyklen, d) maximale und minimale mechanische Spannung über den betrachteten Zeitraum.}
\end{figure}

Die daraus folgenden zweidimensionalen Modelle\footnote{Eine Dimension in Dicken-/Höhenrichtung und eine weitere in Radialrichtung der Partikel oder Fasern.} (Bild~\ref{fig:p2d_model}) gelten als die effizientesten physikalisch basierten Batteriemodelle. Mit diesen lassen sich mehrere Zyklen über 65\,h in unter 43\,min simulieren\footnote{Unter voller Ausnutzung von zwei eingebauten CPUs der Marke AMD EPYC 75F3 mit einer Taktrate von 2,95\,GHz und jeweils 32 Kernen.}. Die Genauigkeit dieser Modelle ist dabei hoch und zeigt meist Abweichungen von unter 0,5~\%~\cite{Pistorio2023}. Wie in anderen Modellen werden die schwer zu bestimmenden kinetischen Parameter häufig durch Anpassung an die ersten Zyklenverläufe identifiziert, wobei als Startwerte Literaturwerte verwendet werden~\cite{Sauerteig2018,Shui2023}. Eine einheitliche Bestimmung und ein konsistenter Austausch dieser Parameter zwischen verschiedenen Modellen ist aufgrund der unterschiedlichen Modellannahmen jedoch schwierig~\cite{Madani2018}. Für eine breite Werkstoff- bzw. Komponentenauswahl im Sinne einer Vorauslegung von Strukturbatterien sind diese Modelle aufgrund der hohen Anzahl zu bestimmender Parameter oft ungeeignet~\cite{Li2022}.

\chapter{\label{sec:sim_sbe}Bestimmung des effektiven Diffusions- und Steifigkeitsverhaltens von zweiphasigen Elektrolytsystemen}

Zweiphasige Elektrolyte besitzen eine komplexe Porenstruktur, die im Durchmesser meist zwischen 1 bis 400 $\si{\nano \metre}$ schwanken (Bild~\ref{fig:sphere_cylinder_model_RVE_generation}a). Bereits verwendete Ansätze, wie etwa Lösen der \textsc{Cahn-Hilliard}-Gleichung, mit anschließendem lösen der linear elastischen Verformung in der festen Domaine (Gleichungen~\ref{eq:stress_gov}, \ref{eq:stress_material}, \ref{eq:strain_total_displacement})
sowie die Simualtion des Diffusionsverhaltens nach \textsc{Fick}\footnote{Gilt unter den Annahmen, dass die Eigenschaften der Flüssigkeit innerhalb des Netzwerks einheitlich sind und das elektrische Feld das gesamte Medium durchdringt.}
\begin{equation}\label{eq:fick}
    J = -D \frac{\partial c}{\partial x}
\end{equation}
in der flüssigen Domain kommt hierbei an mehrer Grenzen. Für einen gute Repräsentation muss das ausgewählte Repräsentationselement ausreichend groß sein, um den Einfluss von lokalen Unterschieden auszugleichen. Durch die große Porenvarianz ist der Generierungs und Vernetzungsaufwand jedoch aufwendig und sorgt für lange Berechnungszeiten. 

\section{Modellierung des effektiven Leitfähigkeit als Porennetzwerk}
Ein urspünnglich aus der Geologie stammender Ansatz der eigenständig für zweiphasige Elektrolyte adaptiert wurde nutzt ein sogenanntes Pore-Netzwerk-Modell~\cite{Xiong2016,Gostick2016}. Dabei wird angenommen, dass kein Ionentransport im festen Polymernetzwerk stattfindet~\cite{Tu2020}.

\begin{figure}[!ht]
	%\raggedleft
		%\def\svgwidth{\columnwidth}
        \center
		\includegraphics[width=0.99\textwidth, angle=0]{sphere_cylinder_model.pdf}
		\caption{\label{fig:sphere_cylinder_model}Das Porennetzwerkmodel mit Kugelförmigen Poren und zylindrischen Verbindungselementen für beschleunigte Berechnungen der effktiven Transporteigenschaften von zweiphasigen Strukturelektrolyten.
        }
\end{figure}

Durch Multiplikation von Gleichung~\ref{eq:fick} mit dem Querschnitt $A$ ergibt sich die Massenflussrate $\dot{m}$.
\begin{equation}
\dot{m} = -D \cdot A(x) \frac{\partial c}{\partial x}
\end{equation}
Aus der Massenbilanz für jede Pore $i$  mit allen benachbarten Poren $ j \in \text{Nb}_i $ folgt
\begin{equation}
\sum_{j \in \text{Nb}_i} \dot{m}_{ij} = \sum_{j \in \text{Nb}_i} D \cdot S_{ij} (c_i - c_j).
\end{equation}
Dabei ist $S_{ij}$ der Formfaktor, welcher wie folgt definiert ist.
\begin{equation}
\frac{1}{S_{ij}} = \frac{1}{S^p_i} + \frac{1}{S^v_{ij}} + \frac{1}{S^p_j}
\end{equation}
Für kugelförmige Poren $p$ und die zylindrische Verbindung $v$ ergibt sich, der Formfaktor
\begin{align}
S^p_i &= \frac{\pi (3d_i^2 - 4l_{ij,i}^2)}{12l_{ij,i}} \\
S^p_j &= \frac{\pi (3d_j^2 - 4l_{ij,j}^2)}{12l_{ij,j}} \\
S^v_{ij} &= \frac{\pi d_{ij}^2}{4l_{ij}}
\end{align}

Die Umrechnung in ionische Leitfähigkeit erfolgt über die Nernst-Einstein-Gleichung.
\begin{equation}
D = \frac{kT}{nq^2} \sigma
\end{equation}
Analog ergibt sich für die modifizierte molare Flussrate $\dot{m}'$
\begin{equation}
\dot{m}' = \frac{nq^2}{kT} \dot{m}
\end{equation}
und
\begin{equation}
\sum_{j \in \text{Nb}_i} \dot{m}'_{ij} = \sum_{j \in \text{Nb}_i} \sigma \cdot S_{ij} (c_i - c_j) = 0
\end{equation}

Damit lassen sich die drei wichtigsten Größen für Porennetzwerke bestimmen:
\begin{enumerate}
    \item der effektive Diffusionskoeffizient
    \begin{equation}
    D_\text{eff} = \frac{\dot{m} \cdot L}{A \Delta c}
    \end{equation}
    \item die effektive Leitfähigkeit
    \begin{equation}
    \sigma_\text{eff} = \frac{\dot{m}' \cdot L}{A \Delta c}
    \end{equation}
    \item und die Tortuosität.
    \begin{equation}
    \tau = \epsilon \frac{D}{D_\text{eff}} = \epsilon \frac{\sigma}{\sigma_\text{eff}}
    \end{equation}
\end{enumerate}
Dabei ist die Porosität $\epsilon$ definiert durch das Verhältnis von Volumen flüssiger Phase $V_f$ zum Gesamtvolumen $V_{\text{gesamt}}$.
\begin{equation}
\epsilon = \frac{V_f}{V_{\text{gesamt}}} = \frac{V_p + V_v}{V_{\text{gesamt}}}
\end{equation}

\begin{figure}[!ht]
	%\raggedleft
		%\def\svgwidth{\columnwidth}
        \center
		\includegraphics[width=0.99\textwidth, angle=0]{RVE_generation.pdf}
		\caption{\label{fig:sphere_cylinder_model_RVE_generation}Die Generierung möglichst akurater Repräsentationen des Strukturelektrolytes beinhaltet (a) die experimenteller Bestimmung der Porenverteilung durch Gasabsorptionsmessungen verschiedener Oligomere (O1-O4), (b) die Konvertierung in eine kommulative Wahrscheinlichkeitsverteilung, (c) das zufällige Auswählen einer vorher festgelegten Anzahl an Poren, (d) die Vernetzung und Skalierung der Porenverbindungen und (e) die Simulation der Ionentransportes und vergleich mit den Experimentellen Ergebnissen.
        }
\end{figure}

Das Modell wurde mit experimentellen Porengrößenverteilungen validiert, die über Gasabsorptionsmessungen gewonnen wurden (Bild~\ref{fig:sphere_cylinder_model_RVE_generation}a). Dazu wurde die Verteilung auf das Intervall [0,1] normalisiert, um eine Wahrscheinlichkeitsverteilung für die zufällige Generierung zu erhalten (Bild~\ref{fig:sphere_cylinder_model_RVE_generation}b). Allerdings muss das Volumenverhältnis $r$ zwischen Verbindungs- und Porenvolumen vorgegeben werden, welches durch
\begin{equation}
r = \frac{V_v}{V_p}
\end{equation}
definiert ist.
Daraus ergibt sich das Porenvolumen alternativ als
\begin{equation}
V_f = V_p (1 + r).
\end{equation}
Durch die Wahl eines würfelförmigen repräsentativen Volumenelementes (RVE) können folgende Zusammenhänge für die Seitenlänge $L$ und die Querschnittsfläche $A$ aufgestellt werden.
\begin{align}
L &= \sqrt[3]{V_{\text{gesamt}}} \\
A &= L^2
\end{align}
Die Poren werden dann der Größe nach absteigend, in das RVE hineingelegt (Bild~\ref{fig:sphere_cylinder_model_RVE_generation}c). Befindet sich eine Pore am Rand muss, zum erhalt der Symmetrie, diese enstprechend dupliziert werden. Dabei wird ein KD-Tree-Algorithmus benutzt, um Porenkollisionen zu verhindern. Anschließend wurden die nächsten $n$ Nachbarn jeder Pore verbunden. Dabei wird als Ausgangsdurchmesser der Durchmesser der kleinsten Pore genommen und abschließend alle Verbindungsradien entsprechend skalliert\footnote{Das nicht-lineare Verhältnis macht ein iteratives Verfahren notwendig. In dieser Arbeit wurde dazu, dass Intervallhalbierungsverfahren benutzt.}, um dem vorher definierten Volumenverhältnis zu entsprechen (Bild~\ref{fig:sphere_cylinder_model_RVE_generation}d).
Zur Bestimmung der effektiven Größen wurde eine konstante molarere Flussrate an zwei gegenüberliegenden Seiten vorgegeben (Bild~\ref{fig:sphere_cylinder_model_RVE_generation}e). Aus den ermittelten Größen wurde der relative Fehler $\text{Err}_{\text{rel}}$ nach folgendem Prinzip bestimmt.
\begin{align}
    \text{Err}_{\text{rel}} &= \frac{\sigma_{\text{eff,PNM}} - \sigma_{\text{eff,exp}}}{\sigma_{\text{eff,PNM}}}\\
    &= \frac{D_{\text{eff,PNM}} - D_{\text{eff,exp}}}{D_{\text{eff,PNM}}}
\end{align}

\begin{figure}[!ht]
	%\raggedleft
		%\def\svgwidth{\columnwidth}
        \center
		\includegraphics[width=0.99\textwidth, angle=0]{convergence.pdf}
		\caption{\label{fig:sphere_cylinder_model_convergence}Der relative Fehler verschiedener Porennetzwerke mit (a) einer festen Anazahl an Verbindungen über verschieden Volumenverhältnise zwischen Verbindungs- und Porenvolumen, (b) Verschiedene Anzahl an verbunden Porennachbarn und (c)-(d) Konvergenzverhalten bei vermehrter Porenanzahl bei optimalen Volumenverhältnisen.
        }
\end{figure}

\input{Abbildungen/03_Modellierung/poren_network_results.tex}

Die Validierung der Methode wurde durch mehrere Studien mit verschiedenen Konfigurationen hinsichtlich Stichprobengröße und Modellparametern wurde für drei unterschiedliche bicontinuierliche oligomere Elektrolyte (O1\_50, O2\_40 und O4\_40) erbracht (Bild~\ref{fig:sphere_cylinder_model_RVE_generation}a)~\cite{Emilsson2023}. Für jede Konfiguration wurden auf Basis der gasadsorptiv gemessenen Porengrößenverteilung verschiedene Netzwerkmodelle erstellt. Die Ionentransporteigenschaft dieser Netzwerke wurde unter Verwendung der jeweiligen ionischen Leitfähigkeit der reinen Flüssigphase\footnote{O1: \SI{0,24}{\milli\siemens\per\centi\meter}, O2: \SI{0,026}{\milli\siemens\per\centi\meter} und O4: \SI{0,0026}{\milli\siemens\per\centi\meter}} simuliert.
Zur Analyse der Genauigkeit wurden die berechneten Leitfähigkeiten mit den experimentell gemessenen Werten verglichen (Tabelle~\ref{tab:pore_network_result}).

Für jede Konfiguration wurden 50 unterschiedliche Netzwerke mit zufälliger Porenverteilung generiert. In der ersten Untersuchung wurde der durchschnittliche relative Fehler für jede Substanz über verschiedene Volumenverhältnisse hinweg verfolgt (Bild~\ref{fig:sphere_cylinder_model_convergence}a). Das Volumenverhältnis mit dem geringsten durchschnittlichen Fehler wurde anschließend für eine Konvergenzstudie ausgewählt.

Diese Ergebnisse zeigt, dass der Fehlerbereich und der durchschnittliche Fehler mit zunehmender Netzwerkgröße kleiner werden. Bereits ab etwa 3000 Poren nähert sich der Fehler einem Wert unterhalb von \SI{10}{\percent} an und lässt sich mit steigender Porenzahl weiter reduzieren, siehe Bild~\ref{fig:sphere_cylinder_model_convergence}c-e.

Abschließend wurde die Konnektivität zwischen benachbarten Poren variiert. Dabei zeigte sich ein signifikanter Einfluss auf die Gesamtleitfähigkeit des Netzwerks: Eine höhere Konnektivität führte zu einer deutlichen Steigerung der effektiven Leitfähigkeit, sihe Bild~\ref{fig:sphere_cylinder_model_convergence}b.

Die Ergebnisse deuten darauf hin, dass die vorgeschlagene Methode die ionische Leitfähigkeit mit guter Genauigkeit modellieren kann, insbesondere bei zunehmender Porenzahl, was zu einer verbesserten Konvergenz führt. Allerdings ist nur die Porenkonnektivität direkt experimentell messbar. Die Beziehung zwischen Konnektivität und das Verhältnis von Hals- zu Porenvolumen — beides entscheidende Parameter zur Vorhersage der ionischen Leitfähigkeit — sind derzeit nicht direkt experimentell erfassbar. 
Allerdings könnten beide Parameter mithilfe von kleinmaßstäblichen RVE-Modellen abgeschätzt werden, die den Phasentrennungsprozess simulieren. 

Des Weiteren konnte auch ein Zusammenhang zwischen der Zusammensetzung des Elektrolyten und dem am besten passenden Verhältnis von Hals- zu Porendurchmesser gefunden werden. Ein möglicher Grund hierfür könnte sein, dass das zugrunde liegende Prinzip, den Halsdurchmesser anhand des kleineren Durchmessers der verbundenen Poren zu skalieren, nicht dem tatsächlichen physikalischen Verhalten entspricht.

Abschließend lassen sich mit diesem Ansatz nur schwer die effektiven Eigenschaften der festen Phase bestimmen, wie etwa die Steifigkeit. Hierfür benötigt es andere Methoden, wie etwa Walk-On-Stars~\cite{Sawhney2023a}.


\section{Parallele Berechnung des effektiven Diffusionskoeffizient und Steifigkeit für fraktale Geometrien mit der Walk-on-Stars Methode}

Walk-on-Stars (WoSt) ist eine netzfrei Methode zum Lösen linearer Differentialgleichungen, die mittels der Benutzung von Grafikkarten besonders gut parallelisiet werden kann und auch in Domainen funktioniert, die fraktale Strukturen mit vielen feinen Details aufweisen~\cite{Sawhney2023a}. Die mathematische Grundlage basiert dabei auf den Arbeiten von \textsc{Feyman} und \textsc{Kac}, die erstmal den Zusammenhang zwischen parabolischen partiellen Differentialgleichungen und storastischten Prozessen darstellten~\cite{Pascucci2024}. 
WoSt stellt dabei die Weiterentwicklung von Walk on Spheres (WoS) dar, welcher Brownische Teilchenbewegungen durch Zufallsbewegungen innerhalb von Kugeln annähert (Bild~\ref{fig:wost_method}a)~\cite{Sawhney2020}. WoSt basiert auf drei Kernmechnismen: die Bestimmung des größtmöglichen Sterngebietes\footnote{Ein (Teil-)Gebiet bei der alle Punkte von einem Punkt aus sichtbar sind.} für jeden Abfragepunkt, die Reflektion von Pfaden die durch Flächen mit Neumann Randbedingung durchlaufen würden und die Beednung und Aggregation der beendeten Pfade nach treffen auf eine Dirchlet-Randbedingung oder einer maximalen Anzahl an Schritten~\cite{Sawhney2023a}.

\begin{figure}[!ht]
	%\raggedleft
		%\def\svgwidth{\columnwidth}
        \center
		\includegraphics[width=0.99\textwidth, angle=0]{wost_method.pdf}
		\caption{\label{fig:wost_method}a) Brownschen Bewegung mit absorbierenden Dirichlet Randbedingungen und reflektierenden Neumann Randbedingung. b)-f) Ablauf der Walk-on-Star Methode zur Annäherung der Lösung einer linearen Differentialgleichung in einem Punkt. e) Berücksichtigung des Einflusses unterschiedlicher Materialkoeefizienten durch Abtastung an der Stelle $y_{k+1}$.
        }
\end{figure}

Vorrausgesetzt, die Geometrie des Strukturelektrolyten ist bekannt, kann diese Methode adaptiert werden um schnell eine sehr gute Annäherung an die Diffusionskoefizienten in der flüssigen Phase und eine durch die feste Phase erzeugte Steifigkeit zu bestimmen.

Ausgang der Annäherung der Verschiebung $\boldsymbol{u}$ in der festen Phase an der Stelle $\boldsymbol{x}$ ist dabei der folgende Zusammenhang.
\begin{equation}
    \boldsymbol{u}(\boldsymbol{x}) = \boldsymbol{E} \left( \int_{0}^{\tau} \boldsymbol{G}(\boldsymbol{x},\boldsymbol{X}_t) \boldsymbol{f}(\boldsymbol{X}_t)dt + \boldsymbol{G}(\boldsymbol{x},\boldsymbol{X}_{\tau}) \boldsymbol{g}(\boldsymbol{X}_{\tau}) \right)
\end{equation}
Dabi ist $\boldsymbol{G}(\boldsymbol{x},\boldsymbol{y})$ der elastische greensche Tensor\footnote{Häfig auch als Kelvinlösung oder Kelvinlet bezeichnet.}, $\tau$ die Endzeit beim Erreichen der Dirchlet Randbedingung und $\boldsymbol{g}$ die vorgegeben Verschiebung an dieser. Für isotope Medien ist greensche Tensor definiert\footnote{Die hier benutzte Version hat eine Singularität bei $r=0$. Diese kann zu bei kleinen Radien zu numerischen instabilitäten führen, weshalb in der Praxis oft regulierte Versionen verwendet werden~\cite{DeGoes2017,Chen2022b,Ringel2024}.} als~\cite{Lazar2014,Chen2022b}
\begin{equation}
    \boldsymbol{G}(\boldsymbol{x}, \boldsymbol{y}) = \frac{1}{4\pi \mu r} \begin{bmatrix}
        1-\frac {1}{2b}+\frac {1}{2b}\frac {x^2}{r^2} & {\frac {1}{2b}}{\frac {xy}{r^{2}}} & {\frac {1}{2b}}{\frac {xz}{r^{2}}}\\
        {\frac {1}{2b}}{\frac {yx}{r^{2}}} & 1-{\frac {1}{2b}}+{\frac {1}{2b}}{\frac {y^{2}}{r^{2}}} & {\frac {1}{2b}}{\frac {yz}{r^{2}}}\\
        {\frac {1}{2b}}{\frac {zx}{r^{2}}} & {\frac {1}{2b}}{\frac {zy}{r^{2}}} & 1-{\frac {1}{2b}}+{\frac {1}{2b}}{\frac {z^{2}}{r^{2}}}
    \end{bmatrix}
\end{equation}
wobei $\boldsymbol{r} = \boldsymbol{x} - \boldsymbol{y}$, $r = \lVert \boldsymbol{r} \rVert$, $a = 1-2 \nu$, $b = 2(1-\nu)$ und $\mu$ das Schubmodul ist, was für isotrope Materialien als 
\begin{equation}
    \mu = \frac{E}{2(1+\nu)}
\end{equation} definiert ist.
Unter der Annahme, dass außer an den Rändern keine Käfte im Körper auftreten und die Eigenschaften in der festen Phase ortsunabhängig sind, kann die Verschiebung an einem Punkt durch 
\begin{equation}
    \boldsymbol{u}(\boldsymbol{x}) = \frac{1}{N} \sum_{i=1}^{N} \sum_{K_i}^{k=1} \boldsymbol{G}(\boldsymbol{x}, \boldsymbol{X}_{i,k}) w_{i,k}
\end{equation}
angenähert werden~\cite{Kulkarni2003,Taylor2013,Chen2024b}. Wobei $K_i$ die Anzahl an Schritten der $i$-ten Stichprobe ist und $w_{i,k}$ die Wichtung darstellt. Die Wichtung bei der Reflektion durch Neumannflächen ist 
\begin{equation}
    w_{k,N} = \boldsymbol{t}(\boldsymbol{X}_{k+1}) = \boldsymbol{\sigma} \boldsymbol{n},
\end{equation}
während für Dirichlet Randbedingungen
\begin{equation}
    w_{k,D} = \boldsymbol{g}(\boldsymbol{X}_{k+1}) = \boldsymbol{u}
\end{equation} 
gilt und die Iteration beendet wird~\cite{Shia2000,Lazar2014,Sawhney2023a}.

Um die Steifikeit des RVEs mittels WoSt zu bestimmen wird in der würfelförmigen Domain $\Omega = [0,L] \times [0,L] \times [0,L]$ eine Dirichletbedingung am unteren Rand ($\boldsymbol{u}(x,y,0) = \boldsymbol{0}$) und eine Neumannbedingung am oberen Rand ($\boldsymbol{n} \boldsymbol{\sigma} = -p \boldsymbol{e}_z$) benutzt. Die Verschiebungen am oberen Rand können direkt und damit sehr effizient über WoSt approximiert werden. Die Gesamtverschiebung oben wird dann aus dem gemittelten Wert angenommen. Anschließend wird die Steifigkeit durch 
\begin{equation}
E = \frac{\Delta \sigma}{\Delta \varepsilon} = \frac{pL}{u_z(z=L)}  
\end{equation}
angenähert.

\begin{figure}[!ht]
	%\raggedleft
		%\def\svgwidth{\columnwidth}
        \center
		\includegraphics[width=0.99\textwidth, angle=0]{wost_results.pdf}
		\caption{\label{fig:wost_result}a) Simulation des Diffusionsverhaltens durch Walk-on-Stars. b) Simulation des Verformungsverhaltens. c-d) Konvergenzverhaltens des berrechneten Diffusions- und Steifigkeitsfehlers mit höherer Schrittmenge pro Pixel.
        }
\end{figure}

Analog kann WoSt benutzt werden, um in der flüssigen Phase die effektive Diffusion zu bestimmen.



\section{Automatisierte Generierung von repräsentativen Volumenelementen für zweiphasige Elektrolytsysteme aus Raster Elektronen Aufnahmen}

Sowohl die Porennetzwerkmethode, als auch die Annäherung durch Walk on Stars benötigt eine geometrische Rräsetnation der zweiphasige Strukturelektroyten. In den Erläuterungen zur Porennetzwerkmethode wurde bereits eine Möglichkeit aus Gasabsorptionsmessungen ein Ersatzmodel aus Kugeln und zylindirsichen Verbindungen zu genieren näher beschrieben. Während diese Methode auch für den WoSt Ansatz benutzt werden kann\footnote{Dazu die Domain innerhalb der Kugeln und Hälse für die fluid Phase wählen oder die außerhalb liegende, aber immer noch im RVE-Würfel liegende Domain für die feste Phase benutzen.} sind Gasabsorptionsmessungen in der Literatur nicht immer gegeben. Allerdings können aus Computertomographen Aufnahmen von ausreichender Auflösung benutzt werden, um mithilfe machinellen Lernens (ML) die Randbedingung der \textsc{Cahn-Hilliard}-Gleichungen, sowie die Stoppzeit abzuschätzen.
\begin{figure}[!ht]
	%\raggedleft
		%\def\svgwidth{\columnwidth}
        \center
		\includegraphics[width=0.99\textwidth, angle=0]{nn_metod_rve_se.pdf}
		\caption{\label{fig:nn_method_rve_se}Ausnutzung eines Convolutional Neural Network zur Abschätzung geeigneter Ausgangsparameter der \textsc{Cahn-Hilliard}-Gleichung.
        }
\end{figure}

Die Trainingsdaten bestehen aus Grauwert-REM-Scans, die in Schwarz-Weiß-Darstellung vorliegen, sowie den zugehörigen realen Abmessungen der Bildausschnitte in Nanometern. Zusätzlich werden experimentell bestimmte Diffusionskoeffizienten $D_i$ und elastische Steifigkeiten $E_i$ sowohl für die Einzelkomponenten als auch für den Verbundmaterial gemessen und den entsprechenden REM-Bildabschnitten zugeordnet.

Ein Convolutional Neural Network extrahiert aus den REM-Bildern textur- und strukturrelevante Merkmale. Diese Merkmale, kombiniert mit den realen Abmessungen, dienen als Eingabe für ein nachgeschaltetes Fully Connected Network, das die Parameter für die \textsc{Cahn-Hilliard}-Gleichung vorhersagt, namentlich die Randbindung $M$ und die Mobilität $L$, siehe Bild~\ref{fig:nn_method_rve_se}. %\citep{Zhang2021CHparameters, Müller2017MLforCH}.

\begin{figure}[!ht]
	%\raggedleft
		%\def\svgwidth{\columnwidth}
        \center
		\includegraphics[width=0.6\textwidth, angle=0]{training_progress.pdf}
		\caption{\label{fig:nn_method_result}Konvergenz der Verlustfunktion über den Trainingsprogress des neuronalen Netzwerkes.
        }
\end{figure}
Die \textsc{Cahn-Hilliard}-Gleichung wird mit einem Finite-Elemente Methode gelöst, wobei die Parameter $L(\mathbf{x})$ und $\varepsilon$ werden durch das ML-Modell bereitgestellt %\citep{Elliott1989CHFEM, Schneider2022FEMCH}.

Aus der zeitabhängigen Lösung $c(\mathbf{x},t)$ werden die Phasengrenzen als Isoflächen $c(\mathbf{x},t^*) = c_\mathrm{threshold}$ extrahiert. Dies erfolgt mit Hilfe von Marching Cubes~\cite{Lorensen1987}.

\begin{table}[ht!]
    \centering
    \caption{\label{tab:nn_method_rve_se_results}Von einem Neuronalen Netzwerk abgeschätzte \textsc{Cahn-Hilliard}-Parameter und resultierende RVE-Elemente basierend auf REM-Aufnahmen von Strukturelektrolyten.}
    \begin{tabularx}{\textwidth}{
        >{\centering\arraybackslash}m{0.2\textwidth}  % SEM column
        >{\centering\arraybackslash}m{0.08\textwidth} % M
        >{\centering\arraybackslash}m{0.08\textwidth} % λ
        >{\centering\arraybackslash}m{0.11\textwidth} % T_end
        >{\centering\arraybackslash}m{0.08\textwidth} % c0
        >{\centering\arraybackslash}m{0.08\textwidth} % L                                          % L
        >{\centering\arraybackslash}m{0.2\textwidth}  % RVE column
    }
    \toprule
    \textbf{REM-Aufnahme}
    & \textbf{M}
    & $\boldsymbol{\mathrm{\lambda}}$
    & $\boldsymbol{\mathrm{T_{end}}}$
    & $\boldsymbol{\mathrm{c_0}}$
    & $\boldsymbol{\mathcal{L}}$
    & \textbf{RVE}
    \\
    \midrule
    \makecell{\includegraphics[width=0.2\textwidth]{generated_rve_se/SEM_60DGEBA.png}\\60DEBA\footnotemark}
        & 0,94 & 0,013 & $\mathrm{5,5 \times 10^{-5}}$ & 0,92 & 0,97 
        & \includegraphics[width=0.22\textwidth]{generated_rve_se/RVE_spheres.png} \\
    \makecell{\includegraphics[width=0.2\textwidth]{generated_rve_se/SEM_50MTM57_2.3.png}\\50MTM57/2.3\footnotemark}
        & 3,22 & 0,061 & $\mathrm{36,0 \times 10^{-5}}$ & 0,94 & 0,98 
        & \includegraphics[width=0.22\textwidth]{generated_rve_se/RVE_Cahn_Hilbert.png} \\
    \makecell{\includegraphics[width=0.2\textwidth]{generated_rve_se/SEM_polyMIPE.png}\\polyMIPE\footnotemark}
        & 1,02 & 0,012 & $\mathrm{7,5 \times 10^{-5}}$ & 0,06 & 0,88 
        & \includegraphics[width=0.22\textwidth]{generated_rve_se/RVE_Template.png} \\
    \bottomrule
    \end{tabularx}\\
    %\noindent{\footnotesize{\textsuperscript{*} Gemessen gegenüber \ce{Li}/\ce{Li+}.}}
\end{table}

% WICHTIG: Die Reihenfolge muss exakt der in der Tabelle entsprechen!
\addtocounter{footnote}{-2} % Counter zurücksetzen, um die erste Markierung zu treffen
\footnotetext{Bezeichnet ein poröses Elektrolytsystem bestehend aus 60 Gew.-\% flüssigem Elektrolyten und einer festen Matrix aus \textit{Diglycidylether von Bisphenol A} (DGEBA), einem gängigen Epoxidharz, das hier mittels PIPS (\textit{Polymerisation-Induced Phase Separation}) hergestellt wurde.}
\stepcounter{footnote}
\footnotetext{Ein strukturelles Elektrolytsystem mit 50 Gew.-\% Elektrolytanteil. Die Matrix besteht aus dem trifunktionellen Epoxidharz \textit{MY0510}, dem Härter \textit{MNA} und dem Beschleuniger \textit{Tertiary amine} (zusammengefasst als MTM), infiltriert mit einem 2,3 M Lithium-Salz-Elektrolyten.}
\stepcounter{footnote}
\footnotetext{Steht für \textit{polymerized Medium Internal Phase Emulsion}. Im Gegensatz zu polyHIPE (High Internal Phase, $>74\%$ interne Phase) beschreibt polyMIPE ein poröses Polymer, das aus einer Emulsion mit einem mittleren Volumenanteil der inneren Phase (typischerweise zwischen 30 \% und 74 \%) synthetisiert wurde.}
Die extrahierte Geometrie definiert zwei Domänen: die Matrix- und die Flüssigphase. Auf jeder Domäne wird mittels linear-elastischer Finite-Elemente-Analyse die effektive Steifigkeit $E_\mathrm{eff}$ berechnet:
\begin{equation}
\nabla \cdot \bigl(\mathbf{C} : \nabla \mathbf{u}\bigr) = \mathbf{0}, 
\end{equation}
wobei $\mathbf{C}$ das Materialsteifigkeits-Tensor ist %\citep{Ciarlet2002FEM, Smith2018CompositeElasticity}. 
Parallel dazu wird die Diffusionsgleichung von \textsc{Fick}
zur Bestimmung des effektiven Diffusionskoeffizienten $D_\mathrm{eff}$ gelöst %\citep{Crank1979Diffusion, Kim2016MLDiffusion}.

Das gesamte Modell wird end-to-end trainiert, indem der Vergleich der berechneten $E_\mathrm{eff}$ und $D_\mathrm{eff}$ mit den experimentellen Werten in den Verlustfunktionsterm
\begin{equation}
    \mathcal{L} =  \frac{1}{1+\sqrt{\lVert \frac{E_\mathrm{eff}^\mathrm{pred} - E_\mathrm{eff}^\mathrm{exp}}{E_\mathrm{rein}^\mathrm{exp}}\rVert^2
+ \lVert \frac{D_\mathrm{eff}^\mathrm{pred} - D_\mathrm{eff}^\mathrm{exp}}{D_\mathrm{rein}^\mathrm{exp}}\rVert^2}}
\end{equation}
aufgenommen wird, siehe Bild~\ref{fig:nn_method_result}.  %\citep{Goodfellow2016DeepLearning, Bishop2006PatternRecognition}.

Die in Tabelle~\ref{tab:nn_method_rve_se_results} gezeigten Resultate verdeutlichen, dass das neuronale Netzwerk die Parameter der \textsc{Cahn-Hilliard}-Gleichung konsistent aus den jeweiligen REM-Aufnahmen ableiten kann. Besonders relevant ist dabei der Wert der Verlustfunktion $\mathcal{L}$, der durch die Funktionskonstruktion zwischen 0 und 1 liegt. Ein hoher Wert von $L$ entspricht einer sehr guten Übereinstimmung zwischen den experimentell bestimmten und den vom Modell vorhergesagten effektiven Materialeigenschaften. Alle drei untersuchten Strukturelektrolyte erreichen hohe $\mathcal{L}$-Werte zwischen 0,88 und 0,98, was die Robustheit des end-to-end Ansatzes bestätigt.

Die Parameter $M$ und $T_\mathrm{end}$ zeigen in allen Fällen eine deutliche Korrelation. Ein hoher Mobilitätskoeffizient $M$ führt zu einer schnelleren Koarsening-Dynamik, die sich häufig in einer größeren Phasenverbindung äußert. Da $T_\mathrm{end}$ direkt die Simulationsdauer bis zur Ausbildung der charakteristischen Mikrostruktur beschreibt, sind beide Größen weitgehend überbestimmt. Dies zeigt sich besonders beim System 50MTM57/2.3, das zugleich den höchsten Mobilitätswert und die längste Ausscheidungszeit aufweist. Die resultierende Morphologie besteht aus vielen großen Poren.

Der Wert von $c_0$ beeinflusst das Anfangsverhältnis von fester und flüssiger Phase. Ein niedriger Wert - wie bei polyMIPE - weist auf einen höheren Anteil der festen Phase hin, während hohe Werte, wie bei 60DGEBA und 50MTM57/2.3, eine stärker flüssigkeitsdominierte Ausgangsverteilung anzeigen. Dies wirkt sich maßgeblich auf die entstehenden Phasenvolumina im RVE aus und entspricht den beobachteten Strukturelementen.

Der Parameter $\lambda$ bestimmt die Feinheit der Grenzflächenauflösung und damit den Detaillierungsgrad der Mikrostruktur. Größere Werte wie bei 50MTM57/2.3 führen zu klarer ausgeprägten Grenzflächen, während kleinere Werte, wie bei polyMIPE, feinere und diffuser konturierte Strukturen erzeugen. Die Unterschiede in $\lambda$ spiegeln sich direkt in den generierten RVEs wider und stimmen qualitativ mit den REM-Beobachtungen überein.

Insgesamt zeigt die Tabelle, dass das Netzwerk in der Lage ist, sowohl grobe als auch feine strukturelle Merkmale der Eingangsbilder zuverlässig in physikalisch sinnvolle Modellparameter zu überführen. Die erzeugten RVE-Geometrien unterscheiden sich klar zwischen den Materialsystemen und repräsentieren die realen Mikrostrukturen überzeugend. Dies bestätigt, dass die ML-basierte Parametrisierung eine leistungsfähige Alternative zu experimentell schwer zugänglichen Methoden darstellt und eine konsistente Kopplung zwischen Bildgebung, physikalischem Modell und effektiven Materialeigenschaften ermöglicht.

\chapter{Entwicklung einer effizienten multiskaligen Auslegungsmethodik von Strukturbatterien}
Die Modelle, die sich aus den Arbeiten von \textsc{Carlstedt}, \textsc{Doyle}, \textsc{Newman}, \textsc{Fuller} und \textsc{Plett} ergeben, sind mit einem hohen Detailgrad versehen. Dieser Detailgrad erlaubt eine hohe physikalische Präzision, gleichzeitig führt er jedoch zu einer großen Anzahl an Parametern, die aufwendig bestimmt werden müssen. Daraus ergibt sich das Problem, dass es in bestimmten Konstellationen schneller und günstiger ist, direkt Experimente mit allen Materialkombinationen durchzuführen, statt zuerst alle benötigten Material- und Interaktionsparameter zu bestimmen. Daher wird für eine möglichst optimale Entwicklungsstrategie in Kapitel~\ref{sec:efficent_development} ein entsprechendes Auswahl- und Bewertungsrahmenwerk entwickelt. Mithilfe dieses Rahmens wird dann in Kapitel~\ref{sec:reduction_and_parallelization} eine gezielte Reduktion weniger signifikanter Einflussfaktoren vorgenommen. Um die Einsatzfähigkeiten verschiedener Strukturbatterien automatisiert bewerten zu können, wird in Kapitel~\ref{sec:automated_failure} eine Versagens- und Risikoabschätzung vorgestellt. Abschließend werden in Kapitel~\ref{sec:validation} die simulierten Ergebnisse dieses Ansatzes mit experimentellen Ergebnissen verglichen.

\section{\label{sec:efficent_development}Konzeption einer effizienten Entwicklung von Strukturbatterien}
Für den effektiven Einsatz der Modelle zur Vorhersage mechanischer und elektrochemischer Eigenschaften wurden zahlreiche Anforderungen an die Modellierung zusammengestellt:
\begin{itemize}
    \item Modellierung basierend auf physikalischen Prozessen (kein reines Fitting),
    \item geringe Anzahl an Materialparametern und keine Einführung neuer, schwer bestim-mbarer Größen,
    \item ausreichend präzise für Vergleichbarkeit zwischen Ergebnissen,
    \item schnelle Berechnungen (keine Wochenlaufzeiten).
\end{itemize} 

Diese Anforderungen beruhen auf der Annahme, dass experimentell gewonnene Ergebnisse das reale Verhalten des untersuchten Objekts abbilden. Daraus folgt, dass Simulationsergebnisse diesem idealerweise möglichst nahekommen, jedoch stets ungenauer sind, solange experimentelle Messfehler vernachlässigt werden können~\cite{Morris2024}. Der mit hochwertigen Experimenten verbundene Aufwand ($k_{\mathrm{exp}}$) hinsichtlich Material- und Zeitkosten ist in vielen Fällen deutlich höher als der Aufwand für eine Computersimulation ($k_{\mathrm{sim}}$):
\begin{equation}
    k_{\mathrm{exp}} \gg k_{\mathrm{sim}}.
\end{equation}
Für einen rein experimentellen Ansatz, der jede mögliche Materialkombination ($n_{\mathrm{Kombis}}$) einer bestimmten Anzahl an experimentellen Bestimmungen ($n_{\mathrm{exp,Bestimmungen}}$) unterzieht, ergibt sich der gesamte Aufwand ($k_{\mathrm{exp, gesamt}}$) etwa als
\begin{align}
    k_{\mathrm{exp, gesamt}} &\approx k_{\mathrm{exp}} \cdot n_{\mathrm{Kombis}} \cdot n_{\mathrm{exp,Bestimmungen}}.
\end{align}
Der schnelle Anstieg des experimentellen Aufwands bei vielen Kombinationen kann zu hohen zeitlichen und finanziellen Kosten führen. Insbesondere im Kontext von Strukturbatterien liegt der Einzelaufwand $k_{\mathrm{exp}}$ für Präparation, Zusammenbau und Validierung einzelner Zellen häufig im Bereich mehrerer Monate.

Um viele potenzielle Materialkombinationen zu testen, sind simulationsbasierte Modelle daher unverzichtbar. Dabei muss jedoch sichergestellt werden, dass der Aufwand zur Bestimmung der Materialkennwerte und die Gesamtrechenzeit der Modellierung ($k_{\mathrm{sim, gesamt}}$) nicht den reinen experimentellen Aufwand übersteigen. Eine grobe Abschätzung des Gesamtaufwands für eine Simulationsstrategie lautet:
\begin{align}
    k_{\mathrm{sim, gesamt}}
    &= k_{\mathrm{sim}} \cdot n_{\mathrm{Kombis}} \cdot n_{\mathrm{Rechnungen}} \nonumber \\
    &\quad + \sum_{m=1}^{n_{\mathrm{Material}}} \Big( n_{\mathrm{exp, Bestimmung,m}} \cdot k_{\mathrm{exp}} + n_{\mathrm{lit, Bestimmung,m}} \cdot k_{\mathrm{lit}} \Big).
\end{align}
Für eine große Zahl an Materialkandidaten erhöht die Menge der experimentell zu bestimmenden Größen den Simulationsaufwand schnell. Unter Abwägung dieser Aufwände stellt ein mehrstufiges Verfahren oft einen guten Kompromiss zwischen Aussagegenauigkeit und Bestimmungsaufwand dar: Zunächst grobe, schnelle Simulationen zur Vorauswahl, dann detaillierte Simulationen für aussichtsreiche Kandidaten und abschließend gezielte Experimente zur Validierung und Feinkalibrierung.

\section{\label{sec:reduction_and_parallelization}Entkopplung und Parallelisierungsstrategie zur Reduktion des Simulationsaufwandes von Strukturbatterien}
\begin{figure}[!ht]
    \center
    \includegraphics[width=0.99\textwidth, angle=0]{simulation_model.pdf}
    \caption{\label{fig:homogenisation}Mehrskalige Homogenisierung der Strukturbatterie durch Abstraktion der Geometrie.}
\end{figure}
Strukturbatterien neigen aufgrund des etwa hundertfach geringeren effektiven Diffusionskoeffizienten von Kohlenstofffasern im Vergleich zu Graphit dazu, bei schnellen Lade- und Entladezyklen ineffizient zu werden~\cite{Uchida1996,Kim2021,Johansen2024}. Dies führt zu vergleichsweise langsamen Auf- und Entladeprozessen. In diesem Bereich verlieren insbesondere temperatureinflussabhängige Effekte an Einfluss auf das Gesamtverhalten und können in erster Näherung vernachlässigt werden~\cite{Carlstedt2019a,Carlstedt2018}. Auch die meisten Interaktionskoeffizienten zwischen den unterschiedlichen physikalischen Domänen haben nur geringen Einfluss und fallen oft mehrere Größenordnungen kleiner aus\footnote{Der Einfluss der Interkalation auf das Elastizitätsmodul liegt unter 1\,\%\cite{Carlstedt2019}.}~\cite{Carlstedt2022b}. Zwar existieren sowohl chemische als auch mechanische Wechselwirkungen; diese können jedoch aufgrund sicherheitstechnischer Restriktionen praktisch kaum experimentell untersucht und damit nur begrenzt validiert werden~\cite{Asp2024}. Aus diesem Grund erscheint eine gezielte Entkopplung der gekoppelten Feldgrößen gerechtfertigt und zweckmäßig.

Die für die betrachteten Belastungsfälle relevante Pouch-Folienbauform kann weiter vereinfacht werden. Insbesondere die in der realen Zelle vorhandenen Siegelstellen in der Mitte des Batteriestacks werden im abstrahierten Modell weggelassen, siehe Bild~\ref{fig:homogenisation}. Da die Diffusion durch den Elektrolyten im Vergleich zu den mechanisch induzierten Längenänderungen sehr schnell erfolgt, kann die Kopplung zwischen Diffusion und geometrischer Deformation ebenfalls vernachlässigt werden. 

\begin{figure}[!ht]
    \center
    \includegraphics[width=0.7\textwidth, angle=0]{bending.pdf}
    \caption{\label{fig:bending_electroylte_tests}Validierung des 3-Punkt-Biegeversuchs: a) visueller Vergleich von Simulation und Experiment, b) Kraftverlauf in Abhängigkeit der Durchbiegung für Pouchzellen mit und ohne Elektrolyt.}
\end{figure}

Da der Fokus dieser Arbeit primär auf den Batterieeffekten und nicht auf möglichen sensorischen Eigenschaften liegt, und da diese, wie in \cite{Carlstedt2023} gezeigt, selbst bei Anregung nur sehr kleine Ströme ausbilden, kann die Rückkopplung mechanischer Spannungen auf die elektrochemische Funktionalität entkoppelt werden. Infolge dieser Reduktion der Kopplungen kann die Gesamtverformung der Strukturbatterie als Summe zweier getrennter Beiträge beschrieben werden: der rein mechanisch verursachten Ausdehnung und der elektrochemisch bedingten Volumenänderung.

\begin{figure}[!ht]
    \center
    \includegraphics[width=0.65\textwidth, angle=0]{simulation_electro_chem.pdf}
    \caption{\label{fig:simulation_electro_chem}Reduktion des Rechenaufwands der elektrochemischen Simulation durch Modellreduktion und Homogenisierung.}
\end{figure}

Für die mechanische Simulation wird ausgenutzt, dass der gesamte Zellstapel bei einem 3-Punkt-Biegeversuch lediglich entlang der Biegelinie unterschiedliche Verschiebungen erfährt, während sich die Elemente in der dazu senkrechten Richtung aufgrund der vorhandenen Symmetrie ähnlich verhalten. Daher genügt es, einen dünnen, repräsentativen Streifen zu betrachten, der anteilig belastet wird. Zusätzlich können die einzelnen Schichten des Stapels durch die Kombination mehrerer repräsentativer Volumenelemente als Blockelemente beschrieben werden. Nichtlineare Eigenschaften dieser Volumenelemente können an ausgewählten Stellen ausgewertet und anschließend mittels Interpolation ohne zusätzlichen Rechenaufwand rekonstruiert werden. Eine vergleichende Studie des Deformationsverhaltens eines Batteriestacks mit und ohne Elektrolyt zeigte dabei gute Übereinstimmung mit den simulierten Ergebnissen, siehe Bild~\ref{fig:bending_electroylte_tests}.

Für die elektrochemische Simulation wird eine analoge Strategie verfolgt: Es wird nur ein Streifen der Batterie betrachtet, wobei jeder Zellstapel als Kombination mehrerer, auf jeweils ein Partikel reduzierter Ersatzschaltungen repräsentiert wird, siehe Bild~\ref{fig:simulation_electro_chem}. Dieser Ansatz basiert auf einem von \textsc{Moura et al.} modifizierten Single-Particle-Modell~\cite{Moura2017}. Als Basis dienen die Diffusionsgleichungen (\ref{eq:diffusion_sphere}) und (\ref{eq:diffusion_cylinder}) für die Partikel in den Elektroden sowie die Transportgleichung im Elektrolyten für die Bereiche Anode ($-$), Separator (sep) und Kathode ($+$):
\begin{equation}
    \frac{\partial c_{e,j}}{\partial t} = \frac{\partial}{\partial x}\left[\frac{D_{e}^{\text{eff}}(c_{e,j})}{\varepsilon_{e,j}} \frac{\partial c_{e,j}}{\partial x} (x,t)\right]-\operatorname{sign}(j)\frac{1-t_c^0}{\varepsilon_{e,j} F L_j} I(t),
\end{equation}
mit $j \in \{-,\text{sep},+\}$ und
\begin{equation}
    \operatorname{sign}(j) = \begin{cases}
        -1 & j = - ,\\
         0 & j = \text{sep},\\
         1 & j = +.
    \end{cases}
\end{equation}
Die Differentialgleichungen sind über folgende Randbedingungen verbunden:
\begin{align}
\frac{\partial c_{e,-}}{\partial x} (0_-,t) &= \frac{\partial c_{e,+}}{\partial x} (0_+,t) = 0,\\
D_{e,\text{eff},-}\frac{\partial c_{e,-}}{\partial x} (L_-,t) &= D_{e,\text{eff,sep}} \frac{\partial c_{e,\text{sep}}}{\partial x} (0_{\text{sep}},t),\\
D_{e,\text{eff,sep}}\frac{\partial c_{e,\text{sep}}}{\partial x} (L_{\text{sep}},t) &= D_{e,\text{eff},+} \frac{\partial c_{e,+}}{\partial x} (L_+,t),\\
c_e(L_-,t) &= c_e(0_{\text{sep}}, t),\\
c_e(L_{\text{sep}},t) &= c_e(L_+, t).
\end{align}
Die effektiven Eigenschaften werden, wie bereits im mechanischen Simulationsteil, durch Homogenisierung der zugrunde liegenden Mikrostruktur bestimmt. Auf diese Weise lässt sich der numerische Aufwand erheblich reduzieren, während die maßgeblichen physikalischen Effekte weiterhin mit ausreichender Genauigkeit erfasst werden.

\section{\label{sec:automated_failure}Versagensanalyse für Strukturbatterien und Risikoeinschätzung}
Mithilfe des entwickelten Modells kann der lokale Spannungszustand
\begin{equation}
\boldsymbol{\sigma} =
\begin{bmatrix}
\sigma_{xx} & \tau_{xy} & \tau_{xz} \\
\tau_{xy}   & \sigma_{yy} & \tau_{yz} \\
\tau_{xz}   & \tau_{yz}   & \sigma_{zz}
\end{bmatrix}
\end{equation}
für alle relevanten Materialschichten der Strukturbatterie bestimmt werden. Aufbauend darauf wird eine automatisierte Versagensanalyse durchgeführt, um kritische Bereiche zu identifizieren und das mit einem Schichtversagen verbundene Gesamtrisiko der Batterie zu bewerten.

\subsection{Versagenskriterien und plastisches Fließverhalten der Einzelschichten}
Für isotrope Materialien, etwa metallische Stromableiter oder Teile der Polymerhülle, wird das von-Mises-Kriterium~\cite{Hill1998} verwendet. Die Vergleichsspannung lautet
\begin{equation}
\sigma_\mathrm{v} = \sqrt{\frac{1}{2}
\left [
(\sigma_{xx}-\sigma_{yy})^2 +
(\sigma_{yy}-\sigma_{zz})^2 +
(\sigma_{zz}-\sigma_{xx})^2
\right ]
+ 3\left ( \tau_{xy}^2 + \tau_{yz}^2 + \tau_{xz}^2 \right )}.
\end{equation}
Versagen bzw. plastisches Fließen tritt ein, wenn
\begin{equation}
\sigma_\mathrm{v} \ge \sigma_\mathrm{y}
\end{equation}
gilt. Für ideal plastisches Verhalten ist die Fließspannung konstant und entspricht der Anfangsfließgrenze $\sigma_{\mathrm{y}0}$.

Zur realistischeren Beschreibung metallischer Werkstoffe wird ein isotrop verfestigendes Materialmodell verwendet. Die Fließspannung wächst mit zunehmender plastischer Deformation:
\begin{equation}
\sigma_\mathrm{y} = \sigma_{\mathrm{y}0} + H \, \bar{\varepsilon}^p ,
\end{equation}
wobei $H$ der isotrope Verfestigungsmodul und $\bar{\varepsilon}^p$ die äquivalente plastische Dehnung ist. Diese ergibt sich inkrementell aus
\begin{equation}
\Delta \bar{\varepsilon}^p = \sqrt{\frac{2}{3} \, \Delta \boldsymbol{\varepsilon}^p : \Delta \boldsymbol{\varepsilon}^p},
\end{equation}
und wird über den Belastungsverlauf aufsummiert:
\begin{equation}
\bar{\varepsilon}^{p}_{n+1} = \bar{\varepsilon}^{p}_{n} + \Delta \bar{\varepsilon}^{p}.
\end{equation}
Die Fließbedingung lautet damit
\begin{equation}
f(\boldsymbol{\sigma}, \bar{\varepsilon}^p) =
\sigma_\mathrm{v} - \left( \sigma_{\mathrm{y}0} + H \, \bar{\varepsilon}^p \right) \le 0.
\end{equation}
Bei Überschreitung ($f>0$) tritt plastische Deformation auf; die plastische Dehnungsrate wird nach dem assoziativen Fließgesetz berechnet:
\begin{equation}
\dot{\boldsymbol{\varepsilon}}^p
= \dot{\lambda}
\frac{\partial f}{\partial \boldsymbol{\sigma}}
= \dot{\lambda} \frac{3}{2} \frac{\boldsymbol{s}}{\sigma_\mathrm{v}},
\end{equation}
mit dem Deviatorspannungstensor
\begin{equation}
\boldsymbol{s} = \boldsymbol{\sigma} - \frac{1}{3}\operatorname{tr}(\boldsymbol{\sigma}) \boldsymbol{I}
\end{equation}
und dem plastischen Multiplikator $\dot{\lambda}$.

Für anisotrope, faserverstärkte Werkstoffe, insbesondere die strukturellen Verstärkungslagen, wird das Tsai-Wu-Kriterium verwendet. Es lautet:
\begin{equation}
F_1 \sigma_1 + F_2 \sigma_2 + F_{11} \sigma_1^2 + F_{22} \sigma_2^2
+ 2F_{12} \sigma_1 \sigma_2 + F_{66} \tau_{12}^2 \geq 1,
\end{equation}
wobei $\sigma_1$ und $\sigma_2$ die Normalspannungen in Faserrichtung und quer zur Faserrichtung sowie $\tau_{12}$ die Schubspannung in der Materialebene darstellen. Die Koeffizienten $F_i$ und $F_{ij}$ werden aus den Zug- und Druckfestigkeiten in Faserrichtung ($R^\text{z}_{||}$, $R^\text{d}_{||}$) sowie quer zur Faser ($R^\text{z}_{\bot}$, $R^\text{d}_{\bot}$) und der Schubfestigkeit ($R_{||\bot}$) bestimmt:
\begin{align}
F_1 &= \frac{1}{R^\text{z}_{||}} - \frac{1}{R^\text{d}_{||}}, &
F_{11} &= \frac{1}{R^\text{z}_{||} R^\text{d}_{||}}, \\
F_2 &= \frac{1}{R^\text{z}_{\bot}} - \frac{1}{R^\text{d}_{\bot}}, &
F_{22} &= \frac{1}{R^\text{z}_{\bot} R^\text{d}_{\bot}}, \\
F_{66} &= \frac{1}{R_{||\bot}^2}, &
F_{12} &\approx -\frac{1}{2}\sqrt{F_{11} F_{22}}.
\end{align}

\subsection{Verknüpfte Sicherheitsrisiken}
\begin{table}[ht]
    \centering
    \caption{\label{tab:failure_modes}Überischt des mit Versagens der Einzelschichten verknüpften Sichheitsrisikos.}
    \begin{tabularx}{\textwidth}{lXXX}
    \toprule
    &\makecell{Pouchfolienversagen\\\includegraphics[width=0.2\textwidth]{failure_modes/failure_mode_pouch.png}}
    &\makecell{Elektrodenversagen\\\includegraphics[width=0.2\textwidth]{failure_modes/failure_mode_electrode.png}}
    &\makecell{Separatorversagen\\\includegraphics[width=0.2\textwidth]{failure_modes/failure_mode_separator.png}}
    \\
    \midrule
    Funktion
        & Funktionsversagen der gesamten Batterie durch austrocknen
        & Leistungsverlust der Zelle
        & Funktionsversagen der Zelle, je nach Verschaltung auch der gesamten Batterie
    \\
    Brandgefahr
        & kein Risiko
        & kein Risiko
        & Flammenbildung durch Überhitzung
    \\
    Gesundheit
        & hohes Risiko durch austretendes Elektrolyt
        & kein Risiko
        & kein zusätzliches Risiko
    \\
    \bottomrule
    \end{tabularx}\\
    %\noindent{\footnotesize{\textsuperscript{*} Gemessen gegenüber \ce{Li}/\ce{Li+}.}}
\end{table}%

Wesentlich kritischer als das lokale Versagen einer einzelnen Schicht ist das damit verbundene Sicherheitsrisiko für die gesamte Strukturbatterie und ihre Umgebung. Ein rein mechanisches Versagen kann, abhängig von der betroffenen Lage, direkte Auswirkungen auf Funktionalität, Sicherheit und Umweltverträglichkeit haben. Tabelle~\ref{tab:failure_modes} gibt einen Überblick über die wichtigsten Versagensarten und deren Konsequenzen.

Besonders kritisch ist das Versagen des Separators, da dies zu einem internen Kurzschluss mit lokaler Überhitzung und im Extremfall zu thermischem Durchgehen führen kann. Ein Versagen der Pouchfolie kann zum Austreten des Elektrolyten und damit zu Funktionsverlust und Gesundheitsrisiko führen. Ein Versagen der Elektroden führt primär zu Leistungsverlust und verringerter Kapazität.

Im Rahmen der automatisierten Versagensanalyse wird nicht nur ein binärer Schadenszustand (intakt/versagt) bewertet, sondern zusätzlich eine qualitative Risikoklassifikation vorgenommen. Diese erlaubt eine Zuordnung von Simulationsergebnissen zu sicherheitsrelevanten Zuständen und dient als Grundlage für das strukturelle Batterie-Design unter Sicherheitsaspekten.

\section{\label{sec:validation}Validierung der Eigenschaftsvorhersagen}
Die Validierung der theoretischen Modelle erfolgt durch Vergleich mit experimentellen Daten eines Schichtverbunds aus neun Lagen, der in einer Stapelanordnung vier funktionale Einzelzellen (Anode–Separator–Kathode) realisiert. Als Referenz dient ein System mit Graphitanode auf Kupferfolie, NMC622-Kathode auf Aluminium und Celgard-2400-Separator mit flüssigem LP30-Elektrolyten. Die zu validierende Strukturbatterie basiert auf einem PX-35 Kohlenstofffasergelege mit Kupfer-Primer und Hardcarbon-Beschichtung; die mechanische Integrität wurde durch Modifikation des Elektrolyten mit KYNAR FLEX 28 erhöht.

Die experimentelle Charakterisierung erfolgte mittels elektrochemischer Zyklierungsversuche und 3-Punkt-Biegeversuchen. Die elektrische Prüfung wurde galvanostatisch durchgeführt, die Stromraten wurden schrittweise von $C/100$ bis $2C$ variiert, um Ratenfähigkeit und Energiedichte über bis zu 35 Zyklen zu erfassen. Parallel wurde die mechanische Tragfähigkeit im 3-Punkt-Biegeversuch bestimmt; die Proben wurden mit konstanter Traversengeschwindigkeit belastet, um die resultierende Kraftaufnahme in Abhängigkeit von der Durchbiegung zu dokumentieren.

\begin{figure}[!ht]
    \center
    \includegraphics[width=0.8\textwidth, angle=0]{electrical_sim_final.pdf}
    \caption{\label{fig:electrical_sim_final}Vergleich der experimentellen und simulierten gravimetrischen Energiedichte über 40 Zyklen bei variierenden Entladeraten für Referenz- und Strukturbatterie.}
\end{figure}

In Bild~\ref{fig:electrical_sim_final} erreicht die Referenzzelle eine Energiedichte von ca. 75,0 [Wh/kg], während die Strukturbatterie aufgrund zusätzlicher passiver Masse der strukturellen Komponenten etwa 43 [Wh/kg] erreicht. Die simulierten Vorhersagen zeigen gute Übereinstimmung mit den experimentellen Daten über das betrachtete Entladespektrum ($C/10$ bis $2C$).

\begin{figure}[!ht]
    \center
    \includegraphics[width=0.99\textwidth, angle=0]{mech_sim_final.pdf}
    \caption{\label{fig:mech_sim_final}Kraft-Durchbiegungs-Diagramm: experimentelle Validierung und korrigierte Simulation im 3-Punkt-Biegeversuch.}
\end{figure}

Die mechanische Validierung (Bild~\ref{fig:mech_sim_final}) zeigt: Die Referenzzelle erreicht eine maximale Kraftaufnahme von ca. 4,5 [N], die Strukturbatterie etwa 9,6 [N]. Geringe Abweichungen im Post-Peak-Bereich der Strukturbatterie lassen sich auf lokal begrenzte Versagensmechanismen im Hardcarbon-Slurry zurückführen, die über die globale Modellierung hinausgehen.

Zur Optimierung der Übereinstimmung wurde ein gezieltes Parameterfitting durchgeführt: Modellparameter wurden anhand der Messdaten der ersten Zyklen (elektrisch) sowie der linearen Phase der Dehnung (mechanisch) kalibriert. Dadurch können fertigungsbedingte Toleranzen (z. B. Schichtdickenvariationen oder Infiltrationsqualität des Elektrolyten) kompensiert werden.

\begin{figure}[!ht]
    \center
    \includegraphics[width=0.69\textwidth, angle=0]{plasticity.pdf}
    \caption{\label{fig:plasticity}Post-mortem-Analyse nach dem 3-Punkt-Biegeversuch im Vergleich zu simulierten Ergebnissen der plastischen Verformung: a) erste Schicht, b) siebte (mittlere) Schicht, c) fünfzehnte (letzte) Schicht.}
\end{figure}

Die postmortale Analyse zeigt gute Übereinstimmung zwischen experimentell beobachteten Deformationsmechanismen und den modellbasierten Vorhersagen. Plastische Verformungszonen in den einzelnen Schichten lassen sich in Lokalisation und Ausprägung mit den Prognosen korrelieren, was die Tauglichkeit des verwendeten plastischen Materialmodells bestätigt.

\chapter{Herleitung Analytischer Ansätze für perspektivische Vorauslegungen}
\section{\label{sec:improve_elchem}Analytische Vorhersage der Energiedichte}
% \begin{itemize}
%     \item Diffusionskoeffizient wird durch equivalente Schaltung ermittelt, die % konstanten Wert vorraussetzen
%     \item Diffusionskoeeffizeint ist eigentlich stark von der Lithierung abhängig
%     \item aufwendig zu ermitteln
%     \item außerdem abweichungen durch Bildung Elektrolyteinterface
%     \item daher für vorhersagen ist die benutzung eher ungeeignet
%     \item für Batterien ist Energidichte wichtiger als Leisungsdichte
%     \item Lösung quasistatische Be- und Entladung, also warten bis vorher
%     \item dies reduziert die vereinfacht die oberen Gleichungen enorm
% \end{itemize}

Die präzise elektrochemische Modellierung von Lithium-Ionen-Zellen erfordert in der Regel die Berücksichtigung eines diffusionsabhängigen Stofftransportes im aktiven Material. Traditionell wird der Diffusionskoeffizient über eine äquivalente elektrische Schaltung angenähert, die einen konstanten, d.h. lithierungsunabhängigen Wert voraussetzt. Diese Annahme führt jedoch zu erheblichen Vereinfachungen, da der tatsächliche Diffusionskoeffizient stark von der lokalen Lithierung abhängt und zudem im Betrieb Temperaturänderungen sowie Alterungsmechanismen, insbesondere die Ausbildung der SEI, berücksichtigt werden müssten. Die exakte Ermittlung eines dynamischen, lithierungsabhängigen Diffusionskoeffizienten ist erfahrungsgemäß rechenaufwendig und experimentell schwer zugänglich. Für schnelle, echtzeitfähige Vorhersagemodelle ist dieser Ansatz daher nur bedingt geeignet.

Für viele Anwendungen im Bereich der elektrochemischen Energiespeicher ist jedoch nicht primär die Leistungsdichte, sondern vielmehr die Energiedichte der Batterie maßgebend. Dieser Fokus ermöglicht es, den Modellierungsaufwand erheblich zu reduzieren. Eine praktikable Lösung besteht darin, Be- und Entladeprozesse quasistatisch zu betrachten. Dabei wird nach jedem kleinen Lade- oder Entladeschritt eine Relaxationsphase angenommen, in der sich das Konzentrationsfeld im aktiven Material vollständig ausgleicht. Die Diffusionsprozesse müssen somit nicht explizit zeitaufgelöst simuliert werden, was die oberen elektrochemischen Gleichungen drastisch vereinfacht und den Berechnungsaufwand deutlich reduziert.Auf dieser Grundlage lassen sich die relevanten Kapazitäts-, Massen- und Energiekennwerte des Batteriestacks durch folgende vereinfachte Beziehungen beschreiben.

Ein zentraler Schritt zur Bewertung der maximal verfügbaren Energiedichte eines elektrochemischen Systems ist die Bestimmung der Oberflächenkapazität der Zelle. Die Oberflächenkapazität beschreibt die umgesetzte Stoffmenge pro aktiver Elektrodenfläche und ist damit direkt an die stöchiometrisch erreichbare Lithiumaufnahme der Elektroden gekoppelt. Da eine Lithium-Ionen-Zelle stets aus einer Anode und einer Kathode besteht, die über den Elektrolyten Lithiumionen austauschen, bestimmt die jeweils limitierende Elektrode die insgesamt nutzbare Kapazität. 

Genau dieses limitierende Verhalten wird durch die folgende Beziehung beschrieben
\begin{equation}
    C_{\text{A, Zelle}} = \min \left( C_{\text{A, -}} , C_{\text{A, +}}\right).
\end{equation}

Hierbei stehen $C_{\text{A, -}}$ und $C_{\text{A, +}}$ für die spezifischen Oberflächenkapazitäten der negativen  bzw. positiven  Elektrode. Die $\min$-Funktion reflektiert, dass eine Elektrode stets vor der anderen ihre maximal mögliche Lithierung bzw. Delithierung erreicht. Sobald die Kapazität der limitierenden Elektrode ausgeschöpft ist, kann der elektrochemische Prozess nicht weitergeführt werden, unabhängig davon, ob die Gegenelektrode theoretisch noch Ladung aufnehmen oder abgeben könnte. 

Die Wahl der minimalen Kapazität stellt somit sicher, dass das Modell physikalisch konsistent bleibt und die tatsächliche, durch die Zellchemie begrenzte Leistungsfähigkeit korrekt beschreibt. Dieser Ansatz ist insbesondere für vereinfachte Modelle essenziell, da er ohne detaillierte Konzentrationsprofile oder komplexe Diffusionsberechnungen auskommt und dennoch die reale Einschränkung durch die elektrochemisch schwächere Elektrode akkurat abbildet.

Da in der Praxis ausschließlich identische Zellen innerhalb eines Stacks verschaltet werden, addieren sich deren Einzelkapazitäten linear, sodass die Gesamtkapazität des Stacks einfach als Produkt aus der Anzahl der Zellen und der Oberflächenkapazität einer Einzelzelle beschrieben werden kann
\begin{equation}
    C_{\text{A, Stack}} = n_{\text{Zellen}} \cdot C_{\text{A, Zelle}}.
\end{equation}

Die Masse des im Stack enthaltenen Elektrolyten lässt sich direkt aus der Gesamtoberflächenkapazität ableiten, da pro umgesetzter Kapazität ein charakteristisches Elektrolytvolumen $V_{\text{C,E}}$ benötigt wird. Multipliziert man dieses volumenbezogene Kapazitätsmaß mit der Gesamtoberflächenkapazität des Stacks sowie der Elektrolytdichte $\rho_{\text{E}}$, ergibt sich die Gesamtmasse des erforderlichen Elektrolyten
\begin{equation}
    m_{\text{A, Stack, E}} = C_{\text{A, Stack}} \cdot V_{\text{C,E}} \cdot \rho_{\text{E}}.
\end{equation}

Durch Summierung der Masse des Elektrolytens und Einzelschichten folgt kann die Gesamtmasse des Batteriestacks $m_{\text{A, Stack}}$ bestimmt werden 
\begin{equation}
    m_{\text{A, Stack}} = m_{\text{A, Stack, E}} + \sum_{i}^{n_{\text{Schichten}}} m_{\text{A,i}}.
\end{equation}

Während die zuvor hergeleiteten Größen auf die elektrochemisch aktive Fläche bezogen sind, erfordert die Bewertung der Energiedichte eine Normierung auf die Gesamtmasse des Stacks. Erst durch diese Umrechnung lässt sich beurteilen, wie viel Kapazität pro Masseeinheit tatsächlich bereitgestellt werden kann. Die massenbezogene Kapazität ergibt sich daher als Quotient aus der gesamten Oberflächenkapazität des Stacks und seiner Gesamtmasse
\begin{equation}
    C_{\text{m, Stack}} = \frac{C_{\text{A, Stack}} }{ m_{\text{A, Stack}}}.
\end{equation}

Daraus ergibt sich die gravimetrische Energiedichte des Stacks unmittelbar aus der massenbezogenen Kapazität und der nutzbaren Zellspannung. Durch Multiplikation dieser beiden Größen erhält man den spezifischen Energieinhalt des Systems.
\begin{equation}
    \Gamma_{\text{Stack}} = C_{\text{m, Stack}} \cdot \left(U_{+} - U_{-}\right)
\end{equation}

% Simultaneously Coupled Mechanical-Electrochemical- Thermal Simulation of Lithium-Ion Cells
Darüber hinaus können Kurzschluss- und Versagensmechanismen weiterhin mit reduzierten Modellen abgebildet werden,
\begin{align}
    R_{\text{Kurz}} &= A_{\text{Kurz}} \sum_{i} \frac{1}{K_i}\\
    A_{\text{Kurz}} &= \sum_{i}^{n_{\text{Versagen}}} A_{i}
\end{align}
wobeit $K_i$ die effektive Leitfähigkeit bzw. Permeabilität des jeweiligen Kurzschluss- oder Versagenspfades beschreibt~\cite{Zhang2016}.

Die Zellspannung ergibt sich unter Berücksichtigung vereinfachter Konzentrationsüberspannungen sowie des Kurzschlusswiderstandes~\cite{Daigle2013} zu
\begin{equation}
    V_{\text{Zelle}} = U_{+} - U_{-} + \frac{RT}{F} \ln \left( \frac{1-x}{x}\right) - i_{app} R_{\text{Kurz}}.
\end{equation}
Das thermische Verhalten kann in der quasistatischen Betrachtung ebenfalls reduziert werden, da reversible Wärmequellen dominieren und dissipative Beiträge klein sind. Damit ergibt sich
\begin{equation}
    \rho v c_p \frac{\partial T}{\partial t} = i_{app}\left(V_{\text{Zelle}} - U_{+} + U_{-} + i_{app} R_{Kurz} \right) -q.
\end{equation}
Da beim quasistatischen Be- und Entladen keine nennenswerten Konzentrationsgradienten und damit auch keine irreversiblen Relaxationsprozesse auftreten, können wärmeerzeugende Nebenreaktionen vernachlässigt werden. Somit entfallen dissipative Wärmequellen vollständig, und es gilt in guter Näherung von
\begin{equation}
    q = 0.
\end{equation}
Durch diese Annahmen lassen sich die elektrochemischen und thermischen Teilmodelle signifikant vereinfachen, wodurch der Gesamtberechnungsaufwand stark reduziert wird, ohne die Aussagekraft für energiedichteorientierte Anwendungen wesentlich einzuschränken.


\section{\label{sec:improve_mech}Analytische Bestimmung des Verformungsverhaltens von Strukturbatterien unter Berücksichtigung verschiederner Elektrolytarten}
Unter der Annahme, dass alle Einzelschichten bei der Bestimmung der Zugsteifigkeit auf beiden Seiten in der Klemmung mit aufgenommen werden und keiner Vordehnung der Einzelschichten sind die Dehnungen in Zugrichtungen für alle Schichten gleich.
\begin{equation}
    \varepsilon_{x,ges} = \varepsilon_{x,i}\\
\end{equation}

Der Struktur von konventionellen Batterien oder Strukturbatterie mit Gel oder flüssigem Elektrolytsystemen kann vereinfacht als Schichtung, lastentragende Materialien betrachtet werden, in deren Zwischnraum eine nicht-lastentragenden Substanz in Form eines Flüssigen oder Gelartigen Zustandes infiltriert wurde.
Die einzelnen Schichten sind nicht direkt mit einander verbundnen und halten einzig durch den Druck der durch die äußere Pouchfolie aneinander. Unter der Annahme, dass die Sichten sich lückenlos anschmiegen ist davon aus zugehen, dass die Krümmung $\kappa$ mit
\begin{equation}
    \kappa = \frac{1}{r} = \frac{M_y}{E I_y}
\end{equation}
in jeder Schicht gleichgroßt ist.
\begin{equation}
    \kappa = \kappa_1 = \kappa_2 = \dots = \kappa_i = \dots = \kappa_n
\end{equation}
Des Weiteren folgt aus dem Momentengleichgewicht, dass das außen angreifende Biegemoment $M_{b}$ gleich der Summe der Schnittmomente in den Einzelschichten sein muss.
\begin{equation}
    M_{b} = \sum_{i}^{n}M_{y,i}
\end{equation}
Unter Annahme von rechticken Querschnitten mit Breite $b_i$ und Höhe $h_i$ und der Annhame, dass alle Elektroden näherungsweise gleich Breit sind, also $b_i = b$ gilt, folgt für die Belastung einer Einzelschicht durch das Moment $M_i$:
\begin{align}
    M_{b} &= M_i \sum_{k}^{n}\frac{E_k I_{yy,k}}{E_i I_{yy,i}}\\
    M_{b} &= M_i \frac{\sum_{k}^{n} E_k h_k^3}{E_i h_i^3}\\
    M_i &= M_{b} \frac{ E_i h_i^3} { \sum_{k}^{n}E_k h_k^3}
\end{align}
Durch einsetzen Einzelschichtbelastung in die Formel zur Bestimmung der Biegespannung erhält man einen Zusammenhang zwischen Einzelschichtspannung und Biegemomentenbelastung:
\begin{align}
    \sigma_{b,i} &= \frac{M_y,i}{I_{yy}/h_i} \\
    \sigma_{b,i} &= 12 \frac{ M_y,i}{b h_i^2}\\
    \sigma_{b,i} &= 12 \frac{M_{b} E_i h_i^3}{b h_i^2 \sum_{k}^{n}E_k h_k^3}\\
    \sigma_{b,i} &= 12 \frac{M_{b} E_i h_i}{b \sum_{k}^{n}E_k h_k^3}
\end{align}

Für die Bestimmung der Durchbiegung $u$ beim 3-Punkt-Biegeversuch kann 
unter der
\begin{equation}
\frac{\frac{\partial^2 u(x)}{\partial x^2}}{\left(1 + \left(\frac{\partial u(x)}{\partial x} \right)^2 \right)^{3/2}} = -\frac{M_y}{E I_{yy}}
\end{equation}
Diese Gleichung kann für kleine Verformungen, so dass $(\frac{\partial u(x)}{\partial x})^2 \ll 1$ durch die folgende Näherung ersetzt werden.
\begin{equation}
    \frac{\partial^2 u(x)}{\partial x^2} \approx -\frac{M_y(x)}{E I_{yy}}
\end{equation}

Unter der Annhame kleiner Verformung und konstantem Querschnitt und Steifigkeit lässt sich die Durchbiegung infolge der Kraft F durch folgende Gleichung annähern.
\begin{align}
    u_\text{flüssig} (x) &= \frac{F L^3}{48 \sum_{k}^{n} E_k I_{yy,k}} \left[ 3 \frac{x}{L} - 4\left(\frac{x}{L}\right)^3 \right] \text{für} \; 0 \leq x \leq L/2 \\
    u_\text{max,flüssig} (x = L/2) &= \frac{FL^3}{48 \sum_{k}^{n} E_k I_{yy,k}} \label{eq:bending_sbe_0}
\end{align}



An dieser Stelle ist zu bemerken, dass für Spezialfall wo alle $n$ Schichten gleich dick sind und aus dem gleichen Material bestehen, die Spannung sich wie folgend ergibt.
\begin{equation}
    \sigma = \sigma_i = \frac{12 M_{b}}{n b h^2},
\end{equation}
Für diesen Spezialfall ergibt sich die maximale Druchbiegung $u_\text{max,flüssig} $ als
\begin{equation}
    u_\text{max,flüssig}  = \frac{L^3 Q}{4 n b h^3 E} = \frac{L^2 \sigma}{6 h E}.
\end{equation}


Da die bisherigen Annäherung vorallem bei konventionellen Batterien mit nichttragenden Elektrolytschichten eine bessere Gültigkeit besitzen ist auch über den Fall zu sprechen wo die Schichten durch das Elektrolyt fest verbunden sind, wie es vorrangig bei Strukturelektrolyten der Fall ist. Die resultierende Laminatstruktur kann beliebig viele Lagen mit unterschiedlichen Orientierungen aufweisen und wird durch die klassische Laminattheorie (CLT) beschrieben~\cite{Carlstedt2018}. Dabei wird angenommen, dass das Material linear-elastisch ist und kleine Durchbiegungen auftreten.


In der CLT wird die Dehnung über der Dicke $z$ des Laminats beschrieben durch
\begin{equation}
\boldsymbol{\varepsilon}(z) = \boldsymbol{\varepsilon}^0 + z\,\boldsymbol{\kappa},
\end{equation}
wobei $\boldsymbol{\varepsilon}^0$ die Dehnungen in der Mittelfläche und $\boldsymbol{\kappa}$ die Krümmungen sind. Die Beziehung zwischen den Schnittgrößen (Kräfte und Momente pro Breite) und den Dehnungen/Krümmungen ergibt sich aus der ABD-Steifigkeitsmatrix:
\begin{equation}
\begin{pmatrix}
\mathbf{N} \\
\mathbf{M}
\end{pmatrix}
=
\begin{pmatrix}
\mathbf{A} & \mathbf{B} \\
\mathbf{B} & \mathbf{D}
\end{pmatrix}
\begin{pmatrix}
\boldsymbol{\varepsilon}^0 \\
\boldsymbol{\kappa}
\end{pmatrix},
\end{equation}
wobei $\mathbf{A}$ die Membransteifigkeit, $\mathbf{D}$ die Biegesteifigkeit und  $\mathbf{B}$ die Kopplung zwischen Membran- und Biegebeanspruchung des Laminats beschreibt.

Für reine Biegung, in der keine Normalkräfte wirken, gilt $\mathbf{N} = \mathbf{0}$. Daraus folgt
\begin{equation}
\mathbf{A}\boldsymbol{\varepsilon}^0 + \mathbf{B}\boldsymbol{\kappa} = \mathbf{0}
\quad\Rightarrow\quad
\boldsymbol{\varepsilon}^0 = -\mathbf{A}^{-1} \mathbf{B} \boldsymbol{\kappa}.
\end{equation}
Durch Einsetzen in die Momentengleichung ergibt
\begin{equation}
\mathbf{M} = \mathbf{B} \boldsymbol{\varepsilon}^0 + \mathbf{D} \boldsymbol{\kappa} = 
\left( \mathbf{D} - \mathbf{B} \mathbf{A}^{-1} \mathbf{B} \right) \boldsymbol{\kappa}.
\end{equation}

Die Größe
\begin{equation}
\mathbf{D}^* = \mathbf{D} - \mathbf{B} \mathbf{A}^{-1} \mathbf{B}.
\end{equation}
wird als effektive Biegesteifigkeit eines unsymmetrischen Laminats bezeichnet~\cite{Jones2018}. 
Damit ergibt sich der Zusammenhang
\begin{equation}
\boldsymbol{\kappa} = (\mathbf{D}^*)^{-1} \mathbf{M}.
\end{equation}

Für reine Biegung um die $x$-Achse gilt $M_x \ne 0$, reduziert sich das Gleichungssystem zu
\begin{equation}
\kappa_x = \frac{M_x}{D^*_{11}}.
\end{equation}

Unter einer mittigen Krafteinwirkung $F$ auf einen Balken ergibt sich ein Momentenverlauf von
\begin{equation}
M_x(x) = \frac{F}{2}x \quad\text{für } 0 \le x \le \frac{L}{2}.
\end{equation}

Somit ergibt sich die Krümmung zu:
\begin{equation}
\kappa_x(x) = \frac{F x}{2 D^*_{11}}.
\end{equation}

Nach zweimaliger Integration gemäß der Euler-Bernoulli-Theorie entsteht die maximale Durchbiegung in der Balkenmitte zu
\begin{equation}\label{eq:bending_sbe_100}
u_\text{max,fest} = \frac{F L^3}{48 D^*_{11}}, \quad
D^*_{11} = D_{11} - [\mathbf{B} \mathbf{A}^{-1} \mathbf{B}]_{11}.
\end{equation}

Für einen symmetrischen Laminataufbau ($\mathbf{B} = \mathbf{0}$) vereinfacht sich $D^*_{11} = D_{11}$, und die Durchbiegung wird:
\begin{equation}
u_\text{max,fest} = \frac{F L^3}{48 D_{11}}.
\end{equation}

Die Dehnung in $x$-Richtung in einer Schicht bei Höhe $z$ ergibt sich zu
\begin{equation}
\varepsilon_x(z) = \varepsilon_x^0 + z\,\kappa_x.
\end{equation}

\begin{figure}[!ht]
	%\raggedleft
		%\def\svgwidth{\columnwidth}
        \center
		\includegraphics[width=0.99\textwidth, angle=0]{bending_pre_tests.pdf}
		\caption{\label{fig:bending_pre_tests}Zur Entwicklung einer vereinfachten Bestimmung des Verformungsverhaltens wurden \textbf{a)} Probekörper aus 11 Schichten einer $100~\mu m$ dicken Aluminiumfolie hergestellt, welche \textbf{b)} flächig mit Verschiedenen Anteilen an PVDF und Luft zusammengefügt wurden. Die \textbf{c-d)} 3-Punkt-Biegeversuche zeigten \textbf{e)} wurden mit den äqivalenten Kraft-Verschiebungskurven der entwickelten Methode anschließend verglichen.}
\end{figure}

Zur bestimmung der Schichtspannungen wird diese globale Dehnung in die lokale Koordinatenrichtung jeder Schicht transformiert\footnote{je nach Faserwinkel $\theta_i$} und mittels Anwendung des materialgesetzes bestimmt
\begin{equation}
\boldsymbol{\sigma}_i = \mathbf{Q}^{(i)}\, \boldsymbol{\varepsilon}_i.
\end{equation}

Dabei ist $\mathbf{Q}^{(i)}$ die Steifigkeitsmatrix der $i$-ten Schicht im lokalen Koordinatensystem. Die Spannungsverteilung ist innerhalb jeder Schicht linear, aber mit unterschiedlichen Steigungen, jedoch abhängig von Materialparametern und Faserorientierung.

Die hergeleiteten Gleichungen beschreiben das 3-Punkt-Biegungsverhalten eines aus mehreren Schichten aufgebauten Stacks, wobei sowohl der Fall rein aufeinanderliegender Lagen (\ref{eq:bending_sbe_0}) als auch der einer lasttragend über das Elektrolyt verbundenen Struktur (\ref{eq:bending_sbe_100}) berücksichtigt wird. Da Strukturelektrolyte in der Regel aus einer Mischung aus fester und flüssiger Phase bestehen, wurde zur Abschätzung des intermediären mechanischen Verhaltens eine Reihe von Probekörpern (80\,mm $\times$ 10\,mm) gefertigt, bestehend aus elf 0,1\,mm dicken Aluminiumfolien (vgl. Bild~\ref{fig:bending_pre_tests}a).

Durch Vergleich des Verformungsverhaltens von glatten und angerauten Aluminiumfolien, die ohne PVDF auf einander gelegt wurden, konnte gezeigt werden, dass die vernachlässigte Oberflächenrauheit keinen relevanten Einfluss auf das Biegeverhalten ausübt. Wie in Bild~\ref{fig:bending_pre_tests}e dargestellt, ist der Unterschied vernachlässigbar. 

Wegen der besseren Adhäsion wurden anschließend die glatten Aluminiumfolien durch verschiedene poröse PVDF-Folien miteinander verbunden (Bild~\ref{fig:bending_pre_tests}b). Aus den Versuchen ließ sich ableiten, dass sich das resultierende maximale Durchbiegungsverhalten als lineare Kombination der beiden Grenzfälle beschreiben lässt. Diese Überlagerung kann über den Phasenvolumenanteil $\varphi$ der flüssigen Phase formuliert werden zu
\begin{equation}
    u_\text{max} = \varphi \cdot u_{\text{max,flüssig}} + (1-\varphi) \cdot u_{\text{max,fest}}.
\end{equation}