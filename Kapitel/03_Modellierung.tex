\chapter{\label{sec:modelling_SB}Modellierung von Strukturbatterien}
Die Modellierung von Strukturbatterien auf der Mikroskalenebene wurde in den letzten Jahren maßgeblich durch die Arbeiten von Carlstedt~\cite{Carlstedt2018,Carlstedt2019,Carlstedt2022a,Carlstedt2023} vorangetrieben. Dabei wurden bereits Modelle sowie die Kopplung mechanischer, elektrochemischer und thermischer Effekte erfolgreich entwickelt~\cite{Carlstedt2022,Carlstedt2022b}. Darüber hinaus existieren weitere Ansätze aus der Forschung zu konventionellen Batterien, die in Kapitel~\ref{sec:existing_micro_models} kurz vorgestellt werden. Die Modelle von Carlstedt konzentrieren sich vorrangig auf das Verhalten auf Mikro- bzw. Partikelebene. Es existiert jedoch eine Vielzahl an Arbeiten, die zeigen, wie diese mikroskaligen Modelle mittels Homogenisierung auf höhere Skalen übertragen werden können. Die aus der Kombination der aus Batterieforschung bekannten Simualtionsansätze mit den Kopplungsansätzen von \textsc{Carlstedt} folgende gesamtheitliche Modellierung übersteigt die bekannten Arbeiten in ihrer komplexität und Detailgrad. Auch erlaubt dieser Ansatz, unter Einbeziehnung existierender Homogenisierungensansätze, die leichtere Entwkikclungen eines kompletten makroskaligen Modellierungsansatzes, siehe Kapitel~\ref{sec:homogenisation}.

\section{\label{sec:existing_micro_models}Bestehende Mikroskalen-Modellierungen}

Die auf Mikro- oder Partikelebene ablaufenden Prozesse sind grundsätzlich unabhängig davon, ob eine konventionelle oder eine Strukturbatterie betrachtet wird. Einige dieser Prozesse spielen in konventionellen Batterien jedoch nur eine untergeordnete Rolle und werden daher häufig vernachlässigt oder vereinfacht dargestellt~\cite{Carlstedt2020a}. Der Ionentransport stellt dabei den zentralen Prozess dar~\cite{Carlstedt2019b}. Nach \textsc{Newman} bestehen signifikante Unterschiede im Transportverhalten zwischen flüssigen und festen Phasen~\cite{Newman2021}. Da sowohl konventionelle als auch strukturelle Batterien mit zweiphasigen Elektrolyten einen Ionentransport durch beide Phasen ermöglichen, lässt sich ihr Verhalten in erster Näherung durch die folgenden fünf Differentialgleichungen\footnote{auch unter dem Namen \textsc{Doyle}-\textsc{Fuller}-\textsc{Newman}-Modell bekannt} beschreiben~\cite{Plett2015}.
\begin{enumerate}
    \item Ladungserhalt in homogenen Festkörpern
    \begin{equation}
        \nabla \cdot \boldsymbol{i}_{\text{s}} = \nabla \cdot \left( - \sigma \cdot \nabla \phi_{\text{s}} \right) = 0
    \end{equation}

    \item Massenserhalt in homogenen Festkörpern
    \begin{equation}
        \frac{\partial c_{\text{s}}}{\partial t}  = \nabla \cdot \left( D_{\text{s}} \nabla c_{\text{s}} \right) = 0
    \end{equation}

    \item Massenerhalt in dem homogenen Elektrolyt
    \begin{equation}
        \frac{\partial c_e}{\partial t} = \nabla \cdot \left( D_e   \nabla c_e \right) - \frac{\boldsymbol{i}_{\text{e}} \cdot    \nabla t_+^0}{F_{\text{K}}} - \nabla \cdot \left( c_{\text{e}} \boldsymbol{v}_0\right)
    \end{equation}

    \item Ladungserhalt  in dem homogenen Elektrolyt
    \begin{equation}
        \nabla \cdot \boldsymbol{i}_{\text{e}} = \nabla \cdot \left(    - \kappa \nabla \phi_{\text{e}}  -\frac{2\kappa R_{\text{K}} T}{F_{\text{K}}} \left(  1+ \frac{\partial \ln f_\pm}{\partial \ln c_{\text{e}}}\right)   \left( t_+^0-1\right) \nabla \ln c_{\text{e}} \right) = 0
    \end{equation}

    \item Ionentrasport zwischen fester und flüssiger Phase
    \begin{align}
        j &= \frac{i_0}{F}\left( \exp \left(\frac{\left(1-\alpha\right)  F}{RT}\eta \right) - \exp \left(-\frac{\alpha F_{\text{K}}}{R_{\text{K}} T}  \eta\right) \right)\\
        i_0 &= n F_{\text{K}} k_0 \left(\prod_i c_{o,i}\right)^{1-\alpha} \left( \prod_i c_{r,i}\right)^\alpha\\
        \eta &= (\phi_{\text{s}}-\phi_{\text{e}}) - U_{\text{ocp}}
    \end{align}
\end{enumerate}
Dabei beschreibt $\boldsymbol{i}$ die Stromdichte, $\sigma$ die elektrische Leitfähigkeit des Materials, $\phi$ das elektrische Potenzial, $c$ die Konzentration der Ladungsträger, $D$ den effektiven Diffusionskoeffizienten, $\boldsymbol{t}^0_+$ die hittorfsche Überführungszahl der Kationen bezogen auf das Elektrolytsystem, $F_{\text{K}}$ die Faraday-Konstante, $\boldsymbol{v}_0$ die Geschwindigkeit des Elektrolyten, $\kappa$ die ionische Leitfähigkeit, $R_{\text{K}}$ die ideale Gaskonstante, $T$ die Temperatur, $f_{\pm}$ den mittleren molaren Aktivitätskoeffizienten, $j$ die molare Flussdichte der Ionen, $i_0$ die Austauschstromdichte\footnote{Vereinfacht sich für Lithium und Natrium zu: $i_0 = F_{\text{K}} k_0 c_e^{1-\alpha} (c_{s,\text{max}} - c_{s,e})^{1-\alpha} c_{s,e}^\alpha$}, $\eta$ das Reaktionsüberpotenzial, $k_{0,K}$ die effektive Reaktionsratenkonstante, $U_{\text{ocp}}$ das Open-Circuit-Potenzial (Leerlaufspannung) und $\alpha$ den asymmetrischen Ladungstransferkoeffizienten im Bereich $0 < \alpha < 1$, welches durch
\begin{equation}
        \alpha = \left|\frac{\Delta E_{\text{a,red}}}{\Delta G_0}\right|
\end{equation}
dem Verhältnis aus Änderung der Aktivierungsenergie der Reduktionsmittel ($\Delta E_{\text{a,red}}$) und Änderung der Gibbs-Energy der Oxidationsmittel ($\Delta G_0$) definiert ist.

Neben dem Ladungstransport beeinflussen auch die Temperaturentwicklung sowie die Entstehung mechanischer Spannungen das Systemverhalten.  
Die Temperaturverteilung in der festen und flüssigen Phase wird dabei durch die Dichte $\rho$, die spezifische Wärmekapazität $c_\text{P}$, die Wärmeleitfähigkeit $\lambda$ sowie den elektrischen Strom bestimmt~\cite{Gao2021,Katrasnik2021}.
\begin{align}
    \rho_{\text{s}} c_{\text{P,s}} \frac{T_{\text{s}}}{\partial t} &= \nabla \cdot (\lambda_{\text{s}} \nabla T_{\text{s}}) - \boldsymbol{i}_{\text{s}} \cdot \nabla \phi_{\text{s}}\\
    \rho_{\text{e}} c_{\text{P,e}} \frac{T_{\text{e}}}{\partial t} &= \nabla \cdot (\lambda_{\text{e}} \nabla T_{\text{e}}) - \boldsymbol{i}_{\text{e}} \cdot \nabla \phi_{\text{e}}
\end{align}

Mechanischer Spannung kommt insbesondere im Kontext von Strukturbatterien eine zentrale Rolle zu~\cite{Carlstedt2020b}. Auch bei konventionellen Batterien wird sie als ein entscheidender Faktor für bestimmte Alterungsmechanismen berücksichtigt~\cite{Mueller2019}. Dabei kann mechanische Spannung ausschließlich in der Festkörperphase auftreten~\cite{Kaliaperumal2021,Berg2022}.
\begin{equation}\label{eq:stress_gov}
    -\nabla \cdot \boldsymbol{\sigma} + f = \boldsymbol{0}
\end{equation}
Durch \textsc{Hook} lässt sich außerdem die mechanische Spannung mit der Dehnung als lineare Abhängigkeit darstellen.
\begin{equation}\label{eq:stress_material}
    \boldsymbol{\sigma} = \boldsymbol{C} \boldsymbol{\varepsilon}_{mech}
\end{equation}
Der Elastizitätstensor $\boldsymbol{C}$ wird im Kontext von Strukturbatterien in Abhängigkeit vom Material als isotrop\footnote{z.B. Metallelektrode, Aktivmaterial, Polymerphase}, transversal-isotrop\footnote{z.B. einzelne Kohlenstofffaser} oder orthotrop\footnote{z.B. Kohlenstofffasergewebe, Glasfaserseparator} beschrieben.
\begin{align}
\boldsymbol{C}^{-1}_{\text{iso}} &= 
\begin{bmatrix}
    \frac{1}{E} & -\frac{\nu}{E} & -\frac{\nu}{E} & 0 & 0 & 0 \\
    -\frac{\nu}{E}& \frac{1}{E} & -\frac{\nu}{E} & 0 & 0 & 0 \\
    -\frac{\nu}{E} & -\frac{\nu}{E} & \frac{1}{E} & 0 & 0 & 0 \\
    0 & 0 & 0 & \frac{2(1+\nu)}{E} & 0 & 0 \\
    0 & 0 & 0 & 0 & \frac{2(1+\nu)}{E} & 0 \\
    0 & 0 & 0 & 0 & 0 & \frac{2(1+\nu)}{E} \\
\end{bmatrix}\\
\boldsymbol{C}^{-1}_{\text{trans}} &= 
\begin{bmatrix}
    \frac{1}{E_{1}} & -\frac{\nu_{12}}{E_{1}} & -\frac{\nu_{13}}{E_{1}} & 0 & 0 & 0 \\
    -\frac{\nu_{12}}{E_{1}}& \frac{1}{E_{2}} & -\frac{\nu_{23}}{E_{2}} & 0 & 0 & 0 \\
    -\frac{\nu_{13}}{E_{1}} & -\frac{\nu_{23}}{E_{2}} & \frac{1}{E_{2}} & 0 & 0 & 0 \\
    0 & 0 & 0 & \frac{2(1+\nu_{23})}{E_{2}} & 0 & 0 \\
    0 & 0 & 0 & 0 & \frac{1}{G_{31}} & 0 \\
    0 & 0 & 0 & 0 & 0 & \frac{1}{G_{12}} \\
\end{bmatrix}\\
\boldsymbol{C}^{-1}_{\text{ortho}} &= 
\begin{bmatrix}
    \frac{1}{E_{1}} & -\frac{\nu_{12}}{E_{1}} & -\frac{\nu_{13}}{E_{1}} & 0 & 0 & 0 \\
    -\frac{\nu_{12}}{E_{1}}& \frac{1}{E_{2}} & -\frac{\nu_{23}}{E_{2}} & 0 & 0 & 0 \\
    -\frac{\nu_{13}}{E_{1}} & -\frac{\nu_{23}}{E_{2}} & \frac{1}{E_{3}} & 0 & 0 & 0 \\
    0 & 0 & 0 & \frac{1}{G_{23}} & 0 & 0 \\
    0 & 0 & 0 & 0 & \frac{1}{G_{31}} & 0 \\
    0 & 0 & 0 & 0 & 0 & \frac{1}{G_{12}} \\
\end{bmatrix}
\end{align}

Besonders bei den Materialien, die als Interkalationsort dienen, haben Untersuchungen von \textsc{Duan}~\cite{Duan2021} gezeigt, dass die Elastizitätsmodule näherungsweise linear von der Ionenkonzentration abhängig sind.
\begin{equation}
    E(c_{s}) = E_0 + \frac{c_{s}}{c_{s,1}} (E_1 - E_0)
\end{equation}

Die Gesamtdehnung $\boldsymbol{\varepsilon}$ ergibt sich dabei aus der Summe der elektrochemischen, thermischen und mechanischen Einflüsse
\begin{equation}\label{eq:strain_total}
    \boldsymbol{\varepsilon} = \boldsymbol{\varepsilon}_{echem} +\boldsymbol{\varepsilon}_{th} + \boldsymbol{\varepsilon}_{mech}
\end{equation}
und wird direkt aus dem Verschiebungsfeld $u$ bestimmt werden.
\begin{equation}\label{eq:strain_total_displacement}
    \boldsymbol{\varepsilon} = \frac{1}{2}\left[\left(\nabla u\right)^T + \left(\nabla u\right)\right]
\end{equation}
Die themische und elektrochemischen Dehnungsanteile hängen dabei durch den jeweiligen Ausdehnungskoeffizeinten $\boldsymbol{\alpha}$ linear von der Veränderung der Temperatur bzw. Konzentration ab.
\begin{align}
    \boldsymbol{\varepsilon}_{echem} &= \boldsymbol{\alpha}_{echem} \left(c_{\pm}-c_{\pm,0}\right)\\
    \boldsymbol{\varepsilon}_{th}  &= \boldsymbol{\alpha}_{th}\left( T - T_0\right)
\end{align}

\begin{figure}[!ht]
	%\raggedleft
		%\def\svgwidth{\columnwidth}
        \center
		\includegraphics[width=0.8\textwidth, angle=0]{micro_model.pdf}
		\caption{\label{fig:micro_model}a) Eine Zwei-Faser-Batterie aus einer Kohlenstofffaser als Anode und einer mit LFP beschichten Kohlenstofffaser als Kathode. b) Blockvernetzung und Zuweisung der Domainen für die gekoppelte FE-Simulation c) Der angelegte Strom als treibende Randbedingung über die Zeit. d) Die elektrische Spannung und Stromdichte über die Zeit. e) Die gemittelte Temperatur über die Zeit. Die Lithiumkonzentration (f), die mechanische Spannung (g) und die die Temperaturverteilung (h) bei t = 2000s.}
\end{figure}

Die aus den Gleichungen abgeleitete mikroskalige Modellierung kann eingesetzt werden, um Halbzellen mit Geometrien im vergleichbaren Größenspektrum zu analysieren~\cite{Plett2015}. Zur realitätsnahen Simulation einer Strukturbatteriezelle aus zwei Fasern ist es jedoch erforderlich, auch die Struktur des Zwei-Phasen-Elektrolyten adäquat abzubilden\cite{Tu2020}. Die zugrunde liegende Geometrie ergibt sich aus dem Prozess der Phasenseparierung, welcher durch die \textsc{Cahn-Hilliard}-Gleichung beschrieben werden kann\cite{Carolan2015,Grant1993}.
\begin{align}
    \frac{\partial c}{\partial t} - \nabla \cdot M \left( \nabla \left( \frac{df}{dc} - \lambda \nabla^2 c\right) \right) &= 0 \text{ in }\Omega\\
    M\left( \nabla \left( \frac{df}{dc} - \lambda \nabla^2 c \right)\right) \cdot n &= 0 \text{ auf }\partial\Omega\\
    M \lambda \nabla c \cdot n &= 0 \text{ auf }\partial\Omega
\end{align}
Dabei wird die Phasenseparation der Konzentration $c$\footnote{Konzentrationswerte nahe 0 gehören zur ersten Phase, während Werte nahe 1 der zweiten Phase zugeordnet werden.} allein durch zwei Parameter $f$\footnote{Häufig eine in $c$ nicht-konvexe Polynomfunktion 4. Grades.} und $M$\footnote{Skalarer Wert} beschrieben. Da die \textsc{Cahn-Hilliard}-Gleichung jedoch eine Differentialgleichung vierter Ordnung ist, führt dies in der schwachen Formulierung zu Ortsableitungen zweiter Ordnung, was mit Standard-Lagrange-Elementen nicht direkt lösbar ist. Eine häufig verwendete Herangehensweise zur Lösung dieses Problems besteht darin, die Gleichung mittels Operatorzerlegung umzuformulieren.
\begin{align}
    \frac{\partial c}{\partial t} - \nabla \cdot M \nabla \mu &= 0 \text{ in }\Omega\\
    \mu - \frac{\partial f}{\partial c} + \lambda \nabla^2 c &= 0 \text{ in }\Omega
\end{align}
Die Poren des resultierenden Strukturelektrolyten befinden sich im Nanometerbereich. Die typischen Partikelgrößen der LFP-Komponenten und Kohlenstofffasern betragen hingegen etwa 1~$\mu m$ bzw. 10~$\mu m$. Dies erfordert ein äußerst feines Rechennetz\footnote{Im gezeigten Beispiel besteht das Netz aus $180 \times 180 \times 640 = 20\,736\,000$ Elementen.}, um die relevanten Mikrostrukturen adäquat abzubilden.
In Kombination mit den nichtlinearen Differentialgleichungen und den vielfältigen physikalischen Kopplungen resultiert daraus ein erheblicher Rechenaufwand\footnote{Um die Parallelisierbarkeit von Blocknetzten möglichst gut auszunutzen werden alle benötigten Parameter allen Knotten zu gewiesen. Bereiche die nicht an den jeweiligen Prozessen teilnehmen bekommen dafür um mehrer Größenordnungen größerer bzw. kleiner Parameter. Außerdem verhindert dieser Ansatz Sigularitäten in der Matrize, die sonst bei isolierten Bereichen entstehen können.}. Die Simulation eines vollständigen Entlade- und Beladevorgangs kann mehr als zwei Wochen in Anspruch nehmen\footnote{Berechnungsserver der HTWK unter Ausnutzung von zwei integrierten AMD EPYC 75F3 CPUs mit einer Taktfrequenz von 2{,}95~GHz und jeweils 32 Rechenkernen.} (Bild~\ref{fig:micro_model}).


\section{\label{sec:homogenisation}Überführung der Mikroskaligen Modellierungsansätze in makroskalige Modelle durch Homogenisierung}

Die Modellierung der einzelnen physikalischen Prozesse ist auf der Mikroskala häufig einfacher umzusetzen~\cite{Plett2015}. Mithilfe mikroskaliger Modelle lassen sich Einflüsse der Geometrie, Verteilung und Clusterbildung präzise ermitteln~\cite{Newman2021}. Aufgrund der hohen Komplexität, die mit den verschiedenen Skalenbereichen einhergeht, ist der damit verbundene Berechnungsaufwand jedoch zu groß, um eine Vielzahl von Zellen effizient zu simulieren~\cite{Liu2019}. Daher sind makroskalige Modelle erforderlich, welche den Rechenaufwand durch Homogenisierung und geeignete Modellvereinfachungen deutlich reduzieren~\cite{Plett2015}.

Ein häufig verwendeter Ansatz stellt dabei die Mittelung der physikalischen Eigenschaften über ein repräsentatives Volumenelement dar~\cite{Burow2016,Arunachalam2019,Li2020}. Die dazu mathematischen Grundlagen basieren auf drei Volumenmittlungstheoremen~\cite{Gray1977}.
\begin{enumerate}
    \item Volumenmittlung für ein skalares Feld $\psi$ 
    \begin{equation}
        \varepsilon_{\alpha} \overline{\nabla \psi_{\alpha}} = \nabla \left(\varepsilon_{\alpha} \bar{\psi}_{\alpha} \right) + \frac{1}{V} \iint_{A_{\alpha \beta(\boldsymbol{x},t)}}\psi_{\alpha} \hat{\boldsymbol{n}}_{\alpha} \text{d}A
    \end{equation}
    \item Volumenmittlung für ein Vektorfeld $\boldsymbol{\psi}$
    \begin{equation}
        \varepsilon_{\alpha} \overline{\nabla \cdot \boldsymbol{\psi}_{\alpha}} = \nabla \cdot \left(\varepsilon_{\alpha} \bar{\boldsymbol{\psi}}_{\alpha} \right) + \frac{1}{V} \iint_{A_{\alpha \beta(\boldsymbol{x},t)}}\boldsymbol{\psi}_{\alpha} \cdot \hat{\boldsymbol{n}}_{\alpha} \text{d}A
    \end{equation}
    \item Volumenmittlung für die zeitliche Änderung eines skalaren Feldes $\psi$ 
    \begin{equation}
        \varepsilon_{\alpha} \overline{\left[\frac{\partial \psi_{\alpha}}{\partial t}\right]} = \frac{\partial \left(\varepsilon_{\alpha} \bar{\psi}_{\alpha} \right)}{\partial t} - \frac{1}{V} \iint_{A_{\alpha \beta(\boldsymbol{x},t)}}\psi_{\alpha} \boldsymbol{v}_{\alpha \beta} \cdot \hat{\boldsymbol{n}}_{\alpha} \text{d}A
    \end{equation}
\end{enumerate}
Dabei beschreibt $\bar{\psi}_{\alpha}$ bzw. $\bar{\boldsymbol{\psi}}_{\alpha}$ die intrinsischen Mittelung über Phase $\alpha$. Diese Art der Mittelung wird nur über das von Phase $\alpha$ eingenommene Volumen\footnote{Hier als zwei Phasensystem mit der zweiten Phase $\beta$ betrachtet.} ermittelt. Die intrinische Mittelung erlaubt gegenüber einer klassischen Mittelung $\langle \psi_{\alpha} \rangle$, welcher auf das Volumen des gesamten Gebietes bezogen ist, eine größere Flexibilität und wieder Verwendbarkeit. Mittels des Volumenanteils $\varepsilon_{\alpha}$
\begin{equation}
    \varepsilon_{\alpha} = \frac{V_{\alpha}(\boldsymbol{x},t)}{V} 
\end{equation}
können die beiden Mittelungsarten in einander umgewandelt werden.
\begin{equation}
    \langle \psi_{\alpha} \rangle = \varepsilon_{\alpha} \bar{\psi}_{\alpha}
\end{equation}

Mit Hilfe der drei Volumenmittelungstheoreme lassen sich die folgenden vier Gleichungen herleiten~\cite{Doyle1995}.
\begin{enumerate}
    \item Volumengemittelte Annäherung des Ladungserhaltes in der festen Phase der porösen Elektrode
    \begin{equation}
        \nabla \cdot \left(\sigma_{\text{eff}} \nabla \hat{\phi}_{s} \right) = a_s F_{\text{K}} \hat{j}
    \end{equation}
    \item Volumengemittelte Annäherung des Ladungserhaltes in der Elektrolytphase der porösen Elektrode
    \begin{equation}
        \nabla \cdot \left(\kappa_{\text{eff}} \nabla \hat{\phi}_e + \kappa_{D, \text{eff}} \nabla ln \hat{c}_e\right) + a_s F_{\text{K}} \hat{j} = 0
    \end{equation}
    \item Volumengemittelte Annäherung des Massenerhaltes in der Elektrolytphase der porösen Elektrode
    \begin{equation}
        \frac{\partial \left(\varepsilon_e \hat{c}_e \right)}{\partial t} = \nabla \cdot \left(D_{e,\text{eff}}\nabla\hat{c}_e\right) + a_s (1+t^0_+) \hat{j}
    \end{equation}
    \item Volumengemittelte Annäherung der mikroskopischen Butler-Volmer Beziehung für den Ionenphasenwechsel
    \begin{equation}
        \hat{j} = j(c_{s,e},\hat{c}_e,\hat{\phi}_s,\hat{\phi}_e)
    \end{equation}
\end{enumerate}

Analog lassen sich für die mechansiche Spannung und die Temparatur die folgenden Zusammenhänge aufstellen.
\begin{enumerate}
    \item Homogenisierung der mechansichen Spannung
    \begin{equation}
    \boldsymbol{\sigma} = \boldsymbol{C}_{\text{eff}} \boldsymbol{\varepsilon}_{\text{mech}} 
    \end{equation}
    \item Volumengemittelte Annäherung der Temperatur
    \begin{equation}
        \frac{\partial (\rho c_{\text{P}} T)}{\partial t} = \nabla \cdot (\lambda \nabla T) + q
    \end{equation}
\end{enumerate}


Der neu eingeführte Wärmegenerierungsterm $q$ kann dabei aus den folgenden fünf Quellen zusammengesetz werden~\cite{Plett2015}.
\begin{enumerate}
    \item Irreversible Wärmeentstehung durch chemische Reaktionen\footnote{Für jede chemische Reaktion $j$.}
    \begin{equation}
        q_i = a_{\text{s}} F_{\text{K}} \hat{j}_j \eta_{j}
    \end{equation}
    \item Reversible Wärmebildung durch Veränderung der Entropie\footnote{Für jede chemische Reaktion $j$.}
    \begin{equation}
    q_{r} = a_{\text{s}} F_{\text{K}} \hat{j}_j \eta_{j} T \frac{\partial U_{\text{ocp},j}}{\partial T}
    \end{equation}
    \item Joule-Wärmeentstehung durch Gradient des elektrischen Potenzials im Feststoff
    \begin{equation}
    q_{s} = \sigma_{\text{eff}}(\nabla\hat{\phi}_{\text{s}} \cdot \nabla\hat{\phi}_{\text{s}})
    \end{equation}
    \item Joule-Wärmeentstehung durch Gradient des elektrischen Potenzials im Elektrolyt
    \begin{equation}
        q_{e} = \kappa_{\text{eff}}(\nabla\hat{\phi}_{\text{e}} \cdot \nabla\hat{\phi}_{\text{e}}) + \kappa_{D,\text{eff}} (\nabla ln \hat{c}_e \cdot \nabla \hat{\phi}_{\text{e}})
    \end{equation}
    \item Warmeentstehung durch Kontaktwiderstände\footnote{$q_c$ gilt nur für die Elektodenfläche und ist daher bezogen auf die Einheitsfläche und nicht wie die anderen Therme auf das Einheitsvolumen}
    \begin{equation}
        q_{c} = i_{\text{app}}^2 R_{\text{Kontakt}}
    \end{equation}
\end{enumerate}

\begin{figure}[!ht]
	%\raggedleft
		%\def\svgwidth{\columnwidth}
        \center
		\includegraphics[width=0.8\textwidth, angle=0]{carlstedt.pdf}
		\caption{\label{fig:carlstedt}a) Schematische Darstellung der untersuchten Kohlenstofffaser-Strukturbatterie und der LFP-Zelle, basierend auf den Arbeiten von \textsc{Carlstedt}~\cite{Carlstedt2022b}. b) Zwei-dimensionales Modell zur Durchführung der FEM-Simulation. c) Zeitverlauf des angelegten Stroms als treibende Randbedingung. d) Elektrische Spannung und Stromdichte im zeitlichen Verlauf sowie die Lithiumkonzentration zu den Zeitpunkten $t_1 = 2000\,\text{s}$ und $t_2 = 6000\,\text{s}$. e) Gemittelte Temperatur über die Zeit sowie Temperaturverteilungen bei $t_1$ und $t_2$. f) Mechanische Spannungskomponenten $\sigma_{11}$ und $\sigma_{22}$ zu den Zeitpunkten $t_1$ und $t_2$.
        }
\end{figure}

Angelehnt an Arbeiten von \textsc{Carlstedt}~\cite{Carlstedt2022b}\footnote{Die Materialwerte und Geometrie sowie die Randbedingungen wurden aus der Arbeit entnommen, um einen Vergleich zu ermöglichen.} können diese Gleichungen bereits verwendet werden, um das Verhalten ganzer Zellen zu beschreiben\footnote{Hier eine Kohlenstofffaser-LFP-Zelle} (Bild~\ref{fig:carlstedt}). Die Zelle durchläuft dabei einen Entlade- und Ladezyklus innerhalb von 2,2 h. Die Simulationszeit betrug dabei 34,6 h auf einem Berechnungsserver der HTWK\footnote{Unter voller Ausnutzung von zwei eingebauten CPUs der Marke AMD EPYC 75F3 mit einer Taktrate von 2,95 GHz und jeweils 32 Kernen}. Der hohe Berechnungsaufwand bereits für einen Ladezyklus macht diesen Ansatz jedoch ungeeignet, um eine Vielzahl an Varianten und größere, mehrzellige Batteriesysteme auszulegen.

Durch Ermittlung effektiver physikalischer Eigenschaften werden dabei die Inhomogenitäten auf der Mikroskala durch ein Kontinuum auf der Makroskala beschrieben~\cite{Plett2024}. Die Genauigkeit dieses Ansatzes hängt jedoch stark von den zu betrachtenden Dimensionen ab~\cite{Plett2015}. Durch die Beschreibung als Kontinuum können Einflüsse wie etwa eine lokal höhere Porendichte nur aufwendig berücksichtigt werden~\cite{Mei2019}. Bei der Analyse deutlich größerer Skalen als der Inhomogenitäten zeigen diese Modelle hingegen eine hohe Genauigkeit~\cite{Plett2015}. 

Um die Berechnungszeit weiter zu reduzieren, kann wegen der Butler-Volmer-Randbedingung keine Volumenmittelung für die Massenerhaltung in der festen Phase\footnote{Die Materialien, die als Interkalationsort dienen} verwendet werden~\cite{Plett2015}. Jedoch können durch Geometrievereinfachungen Freiheitsgrade reduziert und zusätzlicher Berechnungsaufwand vermieden werden. Im Kontext von Strukturbatterien ist der an der Interkalation aktiv teilnehmende Teil partikel- oder faserförmig, welcher durch Kugeln oder Zylinder approximiert werden kann~\cite{Newman2021}.
\begin{enumerate}
    \item Spezialfall Massenserhalt in kugelförmigen Festkörpern
    \begin{equation}
    \frac{\partial c_{\text{s}}}{\partial t} = \frac{1}{r^2} \frac{\partial}{ \partial r} \left[ D_{\text{s}} r^2 \frac{\partial c_{\text{s}}}{\partial r}\right]
    \end{equation}
    \item Spezialfall Massenserhalt in zylindrischen Festkörpern
    \begin{equation}
    \frac{\partial c_{\text{s}}^{\pm}}{\partial t} = \frac{1}{r} \frac{\partial}{ \partial r} \left[ D_{\text{s}} r \frac{\partial c_{\text{s}}}{\partial r}\right] + \frac{\partial}{ \partial z}\left[D_{\text{s}}  \frac{\partial c_{\text{s}}}{\partial z}\right]
    \end{equation}
\end{enumerate}
Dabei ist für viele Szenarien die Verteilung der Konzentration in z-Richtung näherungsweise gleich~\cite{Wang2020c}. In diesem Fall kann der Massenerhalt in zylindirschen Festkörpern weiter zu 
\begin{equation}
    \frac{\partial c_{\text{s}}^{\pm}}{\partial t} = \frac{1}{r} \frac{\partial}{ \partial r} \left[ D_{\text{s}} r \frac{\partial c_{\text{s}}}{\partial r}\right]
\end{equation}
vereinfacht werden.
In beiden Fällen lässt sich das Interkalationsverhalten durch die beiden Randbedingungen
\begin{align}
    \left.\frac{\partial c_{\text{s}}^{\pm}}{\partial r}\right\vert_{r=0} &= 0 \\
    \left.\frac{\partial c_{\text{s}}^{\pm}}{\partial r}\right\vert_{r=R_{\text{p,s}}^{\pm}} &= -\frac{1}{ D_{\text{s}}^\pm} j_{n}^{\pm}(x,t)
\end{align}
darstellen, wobei im Falle einer Stromgesteuerten Be- und Entladung
\begin{equation}
j_{n}^{\pm}(t) = \mp \frac{I(t)}{F a^{\pm} L^{\pm}}
\end{equation}
ist~\cite{Plett2015}.

\begin{figure}[!ht]
	%\raggedleft
		%\def\svgwidth{\columnwidth}
        \center
		\includegraphics[width=0.99\textwidth, angle=0]{p2d_model.pdf}
		\caption{\label{fig:p2d_model}a) Vereinfachung und Überführung von einer NMC-Zelle zu einem 2D-Modell für die FEM-Berechnung. b) Die el. Spannung über mehrere durch den Strom geprägte sich steigernde Lade- und Entladezyklen. c) Der Temperaturverlauf während der Zyklen. d) Die maximale und minimale mechanische Spannung über den zu betrachtenden Zeitraum.
        }
\end{figure}

Die daraus folgenden zweidimensionalen Modelle\footnote{Eine Dimension in Dicken-/Höhenrichtung und eine weitere in Radialrichtung der Partikel oder Fasern.} (Bild~\ref{fig:p2d_model}) gelten als die effizientesten physikalisch basierten Batteriemodelle. Mit diesen lassen sich mehrere Zyklen über 65,h in unter 43\,min simulieren\footnote{Unter voller Ausnutzung von zwei eingebauten CPUs der Marke AMD EPYC 75F3 mit einer Taktrate von 2,95\,GHz und jeweils 32 Kernen}. Die Genauigkeit dieser Modelle ist dabei sehr hoch und zeigt meist Abweichungen unter 0,5~\%~\cite{Pistorio2023}. Jedoch werden, wie auch in anderen Modellen, die nur schwer zu bestimmenden kinetischen Parameter durch Annäherung während der ersten Zyklenverläufe bestimmt, wobei als Startwerte für die Optimierung Literaturwerte verwendet werden können~\cite{Sauerteig2018,Shui2023}. Eine Möglichkeit, diese Parameter einheitlich zu bestimmen und zwischen verschiedenen Modellen auszutauschen, unterliegt aufgrund der teilweise unterschiedlichen Beziehungen keiner Allgemeingültigkeit~\cite{Madani2018}. Für eine breite Vorauslegung verschiedener Strukturbatterien sind diese Modelle aufgrund ihrer hohen Parameteranzahl, welche oft nicht direkt bestimmt werden können, ungeeignet~\cite{Li2022}.


\chapter{Entwicklung einer hybriden Auslegung von Strukturbatterien}
Die Modelle, die sich aus den Arbeiten von \textsc{Carlstedt}, \textsc{Doyle}, \textsc{Newman}, \textsc{Fuller} und \textsc{Plett} ergeben sind mit einem hohen hohen Detailgrad versehen. Dieser Detailgrad erlaubt eine hohe physikalische Präzesion, jedoch sorgt dies gleichzeitig für eine hohe Anzahl an Parametern, die aufwenig bestimmt werden müssen. Damit entsteht das Problem, dass in bestimmten Konstellationen es schneller und günstiger ist direkt Experimente mit allen Materialkombinationen zu machen, als erst alle benötigten Material- und Interaktionsparamter zu bestimmen. Um dies einzuschätzen und eine möglichst optimale Entwicklungsstrategie wird in Kapitel~\ref{sec:efficent_development} ein entsprechendes Auswahl und Bewertungsrahmenwerk entwickelt. Mit Hilfe dieses werden in den Kapiteln~\ref{sec:improve_elchem} und \ref{sec:improve_mech} eine Reihe an häufig gültigen Vereinfachungen der elektrochemische und mechansichen Modellierung unternommen. Um die Einsatzfähigkeiten der verschiedenen Strukturbatterie automatisiert bewerten zu können wird in Kapitel~\ref{sec:automated_failure} ein Versagens- und Risikoabschätzung vorgestellt. Abschließend wird in Kapitel~\ref{sec:digitalisation} auf die Umsetzung der digitalen und automatisierten Vorauswahl geeigneter Strukturbatteriekonfigurationen eingegangen.

\section{\label{sec:efficent_development}Konzeptionierung eines effizienten Entwicklung von Strukturbatterien}
Für den effektiven Einsatz der Modellen für die Vorhersage von mechansichen und elektroschmeischen Eigenschaften wurde eine Vielzahl an Anforderungen an die Modellierung gesammelt:
\begin{itemize}
    \item Moddelierung baiserend auf physikalischen Prozessen, % kein Fitting
    \item geringe Materialparameteranzahl und keine Einführung Neuer, % aufwendige bestimmung
    \item präzise genug für Vergleichbarkeit zwischen mehreren Ergebnissen, % 
    \item schnelle Berechnungen. % nicht wochenlang rechnen
\end{itemize} 

Diese Anforderungen folgen aus der Annahme, dass experimentell erworbene Ergebnisse das reale Verhalten des Objektes unter Beobachtung darstellen. Aus dieser Annhame folgt, dass sich simulative Ergbenisse maximal den experimentellen Ergebnissen annähern und damit folglich stehts ungenauer sind, solange experimentelle Messfehler vernachlässigt werden können~\cite{Morris2024}. Jedoch ist mit diesen hochwerigen Experimenten verbunde Aufwand ($k_{\mathrm{exp}}$) hinsichtlich Material-  und Zeitkosten oftmals um einiges höher als der Aufwand die Simulation mithilfe eines Computers zu berechnen ($k_{\mathrm{sim}}$).
\begin{equation}
    k_{\mathrm{exp}} \ll k_{\mathrm{sim}} 
\end{equation}
Für einen rein experimentellen Ansatz der jede mögliche Materialkombination ($n_{\mathrm{Kombis}}$) einer bestimmten Anzahl an experimentellen Bestimmungen ($n_{\mathrm{exp,Bestimmungen}}$) unterzieht ergibt sich der gesamte Aufwand ($k_{\mathrm{exp, gesamt}}$) wie folgend.
\begin{align}
    k_{\mathrm{exp, gesamt}} &= \prod_{i}^{n_{\mathrm{Kombis}}}\prod_{j}^{n_{\mathrm{Bestimmungen,i}}} k_{\mathrm{exp,j}}\\
    &\approx k_{\mathrm{exp}} \cdot n_{\mathrm{Kombis}} \cdot n_{\mathrm{exp,Bestimmungen}}
\end{align}
Der sich daraus ergebene schnelle Anstieg des experimentellen Aufwandes, bei einer hohen Anzahl an potenziellen Kombinationen oder Versuchen zur Bestimmung der resultierenden Eigenschaften, kann zu hohen zeitlichen und finanzieleln Kosten führen. Insbesondere im Kontext von Strukturbatterie zeigt sich aus der Erfahrung, dass der Einzelaufwand $k_{\mathrm{exp}}$ zur Präperation, Zusammenbau und Validierung einzelnener Zellen im Rahmen von Monaten liegt.
Um die zahlreichen potenziellen Materialkombination zu testen sind Modelle also unablässig. Dennoch muss darauf geachtet werden, dass durch den Bestimmungsaufwand der Materialkennwerte $(n_{\mathrm{exp, Bestimmung}}$) und den Berechnungsaufwand ($n_{\mathrm{Rechnungen}}$) der Auffwand der Modellierungstrategie ($k_{\mathrm{sim, gesamt}}$) nicht den Aufwand eines rein experimentellen Ansatzes übersteigt.
\begin{align}
    k_{\mathrm{sim, gesamt}}
    &= k_{\mathrm{sim}} \cdot n_{\mathrm{Kombis}} \cdot n_{\mathrm{Rechnungen}} \nonumber \\
    &+ \sum_{m}^{n_{\mathrm{Material}}} \left( \prod_{i}^{n_{\mathrm{exp, Bestimmung, m}}}k_{\mathrm{exp,i}} + \prod_{j}^{n_{\mathrm{lit, Bestimmung, m}}} k_{\mathrm{lit,j}} \right)\\
    &\approx k_{\mathrm{sim}} \cdot n_{\mathrm{Kombis}} \cdot n_{\mathrm{Rechnungen}} \nonumber \\
    &+ \sum_{}^{n_{\mathrm{Material}}} \left( n_{\mathrm{exp, Bestimmung}} \cdot k_{\mathrm{exp}} + n_{\mathrm{lit, Bestimmung}} \cdot k_{\mathrm{lit}} \right)
\end{align}
Für eine große Menge an möglichen Materialkandidaten zeigt sich an mit diesem Modell, dass die Menge der experimentell zu bestimmenden Größen den Simulationsaufwand schnell erhöht. Da von auszugehen ist, dass die Genauigkeit der Simulation sich den Experimente maximal annähern, ist es unter Berücksichtigungen der verschiedenen Aufwände stellt ein mehrstufiges Verfahren oft einen guten Kompromiss zwischen Aussagegenauigkeit und Bestimmungsaufwand dar. Durch im Umfang reduzierte Berechnungen können am Anfang viele wenig versprechende Kombinationen ausgeschlossen werden. Mittels detailierter Simulationen können diese auf möglichst wenige Aussichtsreiche Kandidaten eingeschränkt werden. Anschließend können durch umfangreiche Experimenten die Simulationen teilweise validiert und letzte Detailabwägungen getroffen werden. Da für die Simulation und die Experimente bereits hinreichende Ansätze exitieren ist es notwendig das Reduktionspotenzial für die Anwendung bei Strukturbatterien zu identifizieren und den experimentellen Bestimmungsaufwand von schwer zubestimmenden physikalischen Größen\footnote{z.B. Der Diffusionskoeffizient und die Steifigkeit bei verschiedenen zweiphasigen Elektrolytsystemen.} mit hilfe schnellerer Methoden abzuschätzen.


\section{\label{sec:params_elchem}Identifizierung elektrochemischer Materialparameter}

Das hergeleitet Modell zur multi-physikalischen Beschreibung der Strukturbatterie auf der Mikroskala benötigt in der kompletten ausführung 18 zu bestimmende Parameter für jede faserbasierte Elektrode, 11 Parameter pro Elektrolytesystem und faserbasierten Separator, und zwei Interaktionskoeffizienten für jede Kombination an Elektrode und Elektrolyte. Hinzukommen 10 Parameter die für transversal Isotropematerialien, die als Pouchbag und damit nicht an der Reaktion teilnehmen.

\section{\label{sec:sim_sbe}Untersuchung des Geometrieeinflusses von zweiphasigen Elektrolytsystemen auf das Diffusions- und Steifigkeitsverhalten}

Zweiphasige Elektrolyte besitzen eine komplexe Porenstruktur, die im Durchmesser meist zwischen 1 bis 400 $\si{\nano \metre}$ schwanken (Bild~\ref{fig:sphere_cylinder_model_RVE_generation}a). Bereits verwendete Ansätze, wie etwa Lösen der \textsc{Cahn-Hilliard}-Gleichung, mit anschließendem lösen der linear elastischen Verformung in der festen Domaine (Gleichungen~\ref{eq:stress_gov}, \ref{eq:stress_material}, \ref{eq:strain_total_displacement})
sowie die Simualtion des Diffusionsverhaltens nach \textsc{Fick}\footnote{Gilt unter den Annahmen, dass die Eigenschaften der Flüssigkeit innerhalb des Netzwerks einheitlich sind und das elektrische Feld das gesamte Medium durchdringt.}
\begin{equation}\label{eq:fick}
    J = -D \frac{\partial c}{\partial x}
\end{equation}
in der flüssigen Domain kommt hierbei an mehrer Grenzen. Für einen gute Repräsentation muss das ausgewählte Repräsentationselement ausreichend groß sein, um den Einfluss von lokalen Unterschieden auszugleichen. Durch die große Porenvarianz ist der Generierungs und Vernetzungsaufwand jedoch aufwendig und sorgt für lange Berechnungszeiten. 

\subsection{Modellierung des effektiven Leitfähigkeit als Porennetzwerk}
Ein urspünnglich aus der Geologie stammender Ansatz der eigenständig für zweiphasige Elektrolyte adaptiert wurde nutzt ein sogenanntes Pore-Netzwerk-Modell~\cite{Xiong2016,Gostick2016}. Dabei wird angenommen, dass kein Ionentransport im festen Polymernetzwerk stattfindet~\cite{Tu2020}.

\begin{figure}[!ht]
	%\raggedleft
		%\def\svgwidth{\columnwidth}
        \center
		\includegraphics[width=0.99\textwidth, angle=0]{sphere_cylinder_model.pdf}
		\caption{\label{fig:sphere_cylinder_model}Das Porennetzwerkmodel mit Kugelförmigen Poren und zylindrischen Verbindungselementen für beschleunigte Berechnungen der effktiven Transporteigenschaften von zweiphasigen Strukturelektrolyten.
        }
\end{figure}

Durch Multiplikation von Gleichung~\ref{eq:fick} mit dem Querschnitt $A$ ergibt sich die Massenflussrate $\dot{m}$.
\begin{equation}
\dot{m} = -D \cdot A(x) \frac{\partial c}{\partial x}
\end{equation}
Aus der Massenbilanz für jede Pore $i$  mit allen benachbarten Poren $ j \in \text{Nb}_i $ folgt
\begin{equation}
\sum_{j \in \text{Nb}_i} \dot{m}_{ij} = \sum_{j \in \text{Nb}_i} D \cdot S_{ij} (c_i - c_j).
\end{equation}
Dabei ist $S_{ij}$ der Formfaktor, welcher wie folgt definiert ist.
\begin{equation}
\frac{1}{S_{ij}} = \frac{1}{S^p_i} + \frac{1}{S^v_{ij}} + \frac{1}{S^p_j}
\end{equation}
Für kugelförmige Poren $p$ und die zylindrische Verbindung $v$ ergibt sich, der Formfaktor
\begin{align}
S^p_i &= \frac{\pi (3d_i^2 - 4l_{ij,i}^2)}{12l_{ij,i}} \\
S^p_j &= \frac{\pi (3d_j^2 - 4l_{ij,j}^2)}{12l_{ij,j}} \\
S^v_{ij} &= \frac{\pi d_{ij}^2}{4l_{ij}}
\end{align}

Die Umrechnung in ionische Leitfähigkeit erfolgt über die Nernst-Einstein-Gleichung.
\begin{equation}
D = \frac{kT}{nq^2} \sigma
\end{equation}
Analog ergibt sich für die modifizierte molare Flussrate $\dot{m}'$
\begin{equation}
\dot{m}' = \frac{nq^2}{kT} \dot{m}
\end{equation}
und
\begin{equation}
\sum_{j \in \text{Nb}_i} \dot{m}'_{ij} = \sum_{j \in \text{Nb}_i} \sigma \cdot S_{ij} (c_i - c_j) = 0
\end{equation}

Damit lassen sich die drei wichtigsten Größen für Porennetzwerke bestimmen:
\begin{enumerate}
    \item der effektive Diffusionskoeffizient
    \begin{equation}
    D_\text{eff} = \frac{\dot{m} \cdot L}{A \Delta c}
    \end{equation}
    \item die effektive Leitfähigkeit
    \begin{equation}
    \sigma_\text{eff} = \frac{\dot{m}' \cdot L}{A \Delta c}
    \end{equation}
    \item und die Tortuosität.
    \begin{equation}
    \tau = \epsilon \frac{D}{D_\text{eff}} = \epsilon \frac{\sigma}{\sigma_\text{eff}}
    \end{equation}
\end{enumerate}
Dabei ist die Porosität $\epsilon$ definiert durch das Verhältnis von Volumen flüssiger Phase $V_f$ zum Gesamtvolumen $V_{\text{gesamt}}$.
\begin{equation}
\epsilon = \frac{V_f}{V_{\text{gesamt}}} = \frac{V_p + V_v}{V_{\text{gesamt}}}
\end{equation}

\begin{figure}[!ht]
	%\raggedleft
		%\def\svgwidth{\columnwidth}
        \center
		\includegraphics[width=0.99\textwidth, angle=0]{RVE_generation.pdf}
		\caption{\label{fig:sphere_cylinder_model_RVE_generation}Die Generierung möglichst akurater Repräsentationen des Strukturelektrolytes beinhaltet (a) die experimenteller Bestimmung der Porenverteilung durch Gasabsorptionsmessungen verschiedener Oligomere (O1-O4), (b) die Konvertierung in eine kommulative Wahrscheinlichkeitsverteilung, (c) das zufällige Auswählen einer vorher festgelegten Anzahl an Poren, (d) die Vernetzung und Skalierung der Porenverbindungen und (e) die Simulation der Ionentransportes und vergleich mit den Experimentellen Ergebnissen.
        }
\end{figure}

Das Modell wurde mit experimentellen Porengrößenverteilungen validiert, die über Gasabsorptionsmessungen gewonnen wurden (Bild~\ref{fig:sphere_cylinder_model_RVE_generation}a). Dazu wurde die Verteilung auf das Intervall [0,1] normalisiert, um eine Wahrscheinlichkeitsverteilung für die zufällige Generierung zu erhalten (Bild~\ref{fig:sphere_cylinder_model_RVE_generation}b). Allerdings muss das Volumenverhältnis $r$ zwischen Verbindungs- und Porenvolumen vorgegeben werden, welches durch
\begin{equation}
r = \frac{V_v}{V_p}
\end{equation}
definiert ist.
Daraus ergibt sich das Porenvolumen alternativ als
\begin{equation}
V_f = V_p (1 + r).
\end{equation}
Durch die Wahl eines würfelförmigen repräsentativen Volumenelementes (RVE) können folgende Zusammenhänge für die Seitenlänge $L$ und die Querschnittsfläche $A$ aufgestellt werden.
\begin{align}
L &= \sqrt[3]{V_{\text{gesamt}}} \\
A &= L^2
\end{align}
Die Poren werden dann der Größe nach absteigend, in das RVE hineingelegt (Bild~\ref{fig:sphere_cylinder_model_RVE_generation}c). Befindet sich eine Pore am Rand muss, zum erhalt der Symmetrie, diese enstprechend dupliziert werden. Dabei wird ein KD-Tree-Algorithmus benutzt, um Porenkollisionen zu verhindern. Anschließend wurden die nächsten $n$ Nachbarn jeder Pore verbunden. Dabei wird als Ausgangsdurchmesser der Durchmesser der kleinsten Pore genommen und abschließend alle Verbindungsradien entsprechend skalliert\footnote{Das nicht-lineare Verhältnis macht ein iteratives Verfahren notwendig. In dieser Arbeit wurde dazu, dass Intervallhalbierungsverfahren benutzt.}, um dem vorher definierten Volumenverhältnis zu entsprechen (Bild~\ref{fig:sphere_cylinder_model_RVE_generation}d).
Zur Bestimmung der effektiven Größen wurde eine konstante molarere Flussrate an zwei gegenüberliegenden Seiten vorgegeben (Bild~\ref{fig:sphere_cylinder_model_RVE_generation}e). Aus den ermittelten Größen wurde der relative Fehler $\text{Err}_{\text{rel}}$ nach folgendem Prinzip bestimmt.
\begin{align}
    \text{Err}_{\text{rel}} &= \frac{\sigma_{\text{eff,PNM}} - \sigma_{\text{eff,exp}}}{\sigma_{\text{eff,PNM}}}\\
    &= \frac{D_{\text{eff,PNM}} - D_{\text{eff,exp}}}{D_{\text{eff,PNM}}}
\end{align}

\begin{figure}[!ht]
	%\raggedleft
		%\def\svgwidth{\columnwidth}
        \center
		\includegraphics[width=0.99\textwidth, angle=0]{convergence.pdf}
		\caption{\label{fig:sphere_cylinder_model_convergence}Der relative Fehler verschiedener Porennetzwerke mit (a) einer festen Anazahl an Verbindungen über verschieden Volumenverhältnise zwischen Verbindungs- und Porenvolumen, (b) Verschiedene Anzahl an verbunden Porennachbarn und (c)-(d) Konvergenzverhalten bei vermehrter Porenanzahl bei optimalen Volumenverhältnisen.
        }
\end{figure}

\input{Abbildungen/03_Modellierung/poren_network_results.tex}

Die Validierung der Methode wurde durch mehrere Studien mit verschiedenen Konfigurationen hinsichtlich Stichprobengröße und Modellparametern wurde für drei unterschiedliche bicontinuierliche oligomere Elektrolyte (O1\_50, O2\_40 und O4\_40) erbracht (Bild~\ref{fig:sphere_cylinder_model_RVE_generation}a)~\cite{Emilsson2023}. Für jede Konfiguration wurden auf Basis der gasadsorptiv gemessenen Porengrößenverteilung verschiedene Netzwerkmodelle erstellt. Die Ionentransporteigenschaft dieser Netzwerke wurde unter Verwendung der jeweiligen ionischen Leitfähigkeit der reinen Flüssigphase\footnote{O1: \SI{0,24}{\milli\siemens\per\centi\meter}, O2: \SI{0,026}{\milli\siemens\per\centi\meter} und O4: \SI{0,0026}{\milli\siemens\per\centi\meter}} simuliert.
Zur Analyse der Genauigkeit wurden die berechneten Leitfähigkeiten mit den experimentell gemessenen Werten verglichen (Tabelle~\ref{tab:pore_network_result}).

Für jede Konfiguration wurden 50 unterschiedliche Netzwerke mit zufälliger Porenverteilung generiert. In der ersten Untersuchung wurde der durchschnittliche relative Fehler für jede Substanz über verschiedene Volumenverhältnisse hinweg verfolgt (Bild~\ref{fig:sphere_cylinder_model_convergence}a). Das Volumenverhältnis mit dem geringsten durchschnittlichen Fehler wurde anschließend für eine Konvergenzstudie ausgewählt.

Diese Ergebnisse zeigt, dass der Fehlerbereich und der durchschnittliche Fehler mit zunehmender Netzwerkgröße kleiner werden. Bereits ab etwa 3000 Poren nähert sich der Fehler einem Wert unterhalb von \SI{10}{\percent} an und lässt sich mit steigender Porenzahl weiter reduzieren (Bild~\ref{fig:sphere_cylinder_model_convergence}c-e).

Abschließend wurde die Konnektivität zwischen benachbarten Poren variiert. Dabei zeigte sich ein signifikanter Einfluss auf die Gesamtleitfähigkeit des Netzwerks: Eine höhere Konnektivität führte zu einer deutlichen Steigerung der effektiven Leitfähigkeit (Bild~\ref{fig:sphere_cylinder_model_convergence}b).

Die Ergebnisse deuten darauf hin, dass die vorgeschlagene Methode die ionische Leitfähigkeit mit guter Genauigkeit modellieren kann, insbesondere bei zunehmender Porenzahl, was zu einer verbesserten Konvergenz führt. Allerdings ist nur die Porenkonnektivität direkt experimentell messbar. Die Beziehung zwischen Konnektivität und das Verhältnis von Hals- zu Porenvolumen — beides entscheidende Parameter zur Vorhersage der ionischen Leitfähigkeit — sind derzeit nicht direkt experimentell erfassbar. 
Allerdings könnten beide Parameter mithilfe von kleinmaßstäblichen RVE-Modellen abgeschätzt werden, die den Phasentrennungsprozess simulieren. 

Des Weiteren konnte auch ein Zusammenhang zwischen der Zusammensetzung des Elektrolyten und dem am besten passenden Verhältnis von Hals- zu Porendurchmesser gefunden werden. Ein möglicher Grund hierfür könnte sein, dass das zugrunde liegende Prinzip, den Halsdurchmesser anhand des kleineren Durchmessers der verbundenen Poren zu skalieren, nicht dem tatsächlichen physikalischen Verhalten entspricht.

Abschließend lassen sich mit diesem Ansatz nur schwer die effektiven Eigenschaften der festen Phase bestimmen, wie etwa die Steifigkeit. Hierfür benötigt es andere Methoden, wie etwa Walk-On-Stars~\cite{Sawhney2023a}.


\subsection{Parallele Berechnung des effektiven Diffusionskoeffizient und Steifigkeit für fraktale Geometrien mit der Walk-on-Stars Methode}

Walk-on-Stars (WoSt) ist eine netzfrei Methode zum Lösen linearer Differentialgleichungen, die mittels der Benutzung von Grafikkarten besonders gut parallelisiet werden kann und auch in Domainen funktioniert, die fraktale Strukturen mit vielen feinen Details aufweisen~\cite{Sawhney2023a}. Die mathematische Grundlage basiert dabei auf den Arbeiten von \textsc{Feyman} und \textsc{Kac}, die erstmal den Zusammenhang zwischen parabolischen partiellen Differentialgleichungen und storastischten Prozessen darstellten~\cite{Pascucci2024}. 
WoSt stellt dabei die Weiterentwicklung von Walk on Spheres (WoS) dar, welcher Brownische Teilchenbewegungen durch Zufallsbewegungen innerhalb von Kugeln annähert (Bild~\ref{fig:wost_method}a)~\cite{Sawhney2020}. WoSt basiert auf drei Kernmechnismen: die Bestimmung des größtmöglichen Sterngebietes\footnote{Ein (Teil-)Gebiet bei der alle Punkte von einem Punkt aus sichtbar sind.} für jeden Abfragepunkt, die Reflektion von Pfaden die durch Flächen mit Neumann Randbedingung durchlaufen würden und die Beednung und Aggregation der beendeten Pfade nach treffen auf eine Dirchlet-Randbedingung oder einer maximalen Anzahl an Schritten~\cite{Sawhney2023a}.

\begin{figure}[!ht]
	%\raggedleft
		%\def\svgwidth{\columnwidth}
        \center
		\includegraphics[width=0.99\textwidth, angle=0]{wost_method.pdf}
		\caption{\label{fig:wost_method}a) Brownschen Bewegung mit absorbierenden Dirichlet Randbedingungen und reflektierenden Neumann Randbedingung. b)-f) Ablauf der Walk-on-Star Methode zur Annäherung der Lösung einer linearen Differentialgleichung in einem Punkt. e) Berücksichtigung des Einflusses unterschiedlicher Materialkoeefizienten durch Abtastung an der Stelle $y_{k+1}$.
        }
\end{figure}

Vorrausgesetzt, die Geometrie des Strukturelektrolyten ist bekannt, kann diese Methode adaptiert werden um schnell eine sehr gute Annäherung an die Diffusionskoefizienten in der flüssigen Phase und eine durch die feste Phase erzeugte Steifigkeit zu bestimmen.

Ausgang der Annäherung der Verschiebung $\boldsymbol{u}$ in der festen Phase an der Stelle $\boldsymbol{x}$ ist dabei der folgende Zusammenhang.
\begin{equation}
    \boldsymbol{u}(\boldsymbol{x}) = \boldsymbol{E} \left( \int_{0}^{\tau} \boldsymbol{G}(\boldsymbol{x},\boldsymbol{X}_t) \boldsymbol{f}(\boldsymbol{X}_t)dt + \boldsymbol{G}(\boldsymbol{x},\boldsymbol{X}_{\tau}) \boldsymbol{g}(\boldsymbol{X}_{\tau}) \right)
\end{equation}
Dabi ist $\boldsymbol{G}(\boldsymbol{x},\boldsymbol{y})$ der elastische greensche Tensor\footnote{Häfig auch als Kelvinlösung oder Kelvinlet bezeichnet.}, $\tau$ die Endzeit beim Erreichen der Dirchlet Randbedingung und $\boldsymbol{g}$ die vorgegeben Verschiebung an dieser. Für isotope Medien ist greensche Tensor definiert\footnote{Die hier benutzte Version hat eine Singularität bei $r=0$. Diese kann zu bei kleinen Radien zu numerischen instabilitäten führen, weshalb in der Praxis oft regulierte Versionen verwendet werden~\cite{DeGoes2017,Chen2022b,Ringel2024}.} als~\cite{Lazar2014,Chen2022b}
\begin{equation}
    \boldsymbol{G}(\boldsymbol{x}, \boldsymbol{y}) = \frac{1}{4\pi \mu r} \begin{bmatrix}
        1-\frac {1}{2b}+\frac {1}{2b}\frac {x^2}{r^2} & {\frac {1}{2b}}{\frac {xy}{r^{2}}} & {\frac {1}{2b}}{\frac {xz}{r^{2}}}\\
        {\frac {1}{2b}}{\frac {yx}{r^{2}}} & 1-{\frac {1}{2b}}+{\frac {1}{2b}}{\frac {y^{2}}{r^{2}}} & {\frac {1}{2b}}{\frac {yz}{r^{2}}}\\
        {\frac {1}{2b}}{\frac {zx}{r^{2}}} & {\frac {1}{2b}}{\frac {zy}{r^{2}}} & 1-{\frac {1}{2b}}+{\frac {1}{2b}}{\frac {z^{2}}{r^{2}}}
    \end{bmatrix}
\end{equation}
wobei $\boldsymbol{r} = \boldsymbol{x} - \boldsymbol{y}$, $r = \lVert \boldsymbol{r} \rVert$, $a = 1-2 \nu$, $b = 2(1-\nu)$ und $\mu$ das Schubmodul ist, was für isotrope Materialien als 
\begin{equation}
    \mu = \frac{E}{2(1+\nu)}
\end{equation} definiert ist.
Unter der Annahme, dass außer an den Rändern keine Käfte im Körper auftreten und die Eigenschaften in der festen Phase ortsunabhängig sind, kann die Verschiebung an einem Punkt durch 
\begin{equation}
    \boldsymbol{u}(\boldsymbol{x}) = \frac{1}{N} \sum_{i=1}^{N} \sum_{K_i}^{k=1} \boldsymbol{G}(\boldsymbol{x}, \boldsymbol{X}_{i,k}) w_{i,k}
\end{equation}
angenähert werden~\cite{Kulkarni2003,Taylor2013,Chen2024b}. Wobei $K_i$ die Anzahl an Schritten der $i$-ten Stichprobe ist und $w_{i,k}$ die Wichtung darstellt. Die Wichtung bei der Reflektion durch Neumannflächen ist 
\begin{equation}
    w_{k,N} = \boldsymbol{t}(\boldsymbol{X}_{k+1}) = \boldsymbol{\sigma} \boldsymbol{n},
\end{equation}
während für Dirichlet Randbedingungen
\begin{equation}
    w_{k,D} = \boldsymbol{g}(\boldsymbol{X}_{k+1}) = \boldsymbol{u}
\end{equation} 
gilt und die Iteration beendet wird~\cite{Shia2000,Lazar2014,Sawhney2023a}.

Um die Steifikeit des RVEs mittels WoSt zu bestimmen wird in der würfelförmigen Domain $\Omega = [0,L] \times [0,L] \times [0,L]$ eine Dirichletbedingung am unteren Rand ($\boldsymbol{u}(x,y,0) = \boldsymbol{0}$) und eine Neumannbedingung am oberen Rand ($\boldsymbol{n} \boldsymbol{\sigma} = -p \boldsymbol{e}_z$) benutzt. Die Verschiebungen am oberen Rand können direkt und damit sehr effizient über WoSt approximiert werden. Die Gesamtverschiebung oben wird dann aus dem gemittelten Wert angenommen. Anschließend wird die Steifigkeit durch 
\begin{equation}
E = \frac{\Delta \sigma}{\Delta \varepsilon} = \frac{pL}{u_z(z=L)}  
\end{equation}
angenähert.

\begin{figure}[!ht]
	%\raggedleft
		%\def\svgwidth{\columnwidth}
        \center
		\includegraphics[width=0.99\textwidth, angle=0]{wost_results.pdf}
		\caption{\label{fig:wost_result}a) Simulation des Diffusionsverhaltens durch Walk-on-Stars. b) Simulation des Verformungsverhaltens. c-d) Konvergenzverhaltens des berrechneten Diffusions- und Steifigkeitsfehlers mit höherer Schrittmenge pro Pixel.
        }
\end{figure}

Analog kann WoSt benutzt werden, um in der flüssigen Phase die effektive Diffusion zu bestimmen.



\subsection{Automatisierte Generierung von repräsentativen Volumenelementen für zweiphasige Elektrolytsysteme aus Raster Elektronen Aufnahmen}

Sowohl die Porennetzwerkmethode, als auch die Annäherung durch Walk on Stars benötigt eine geometrische Rräsetnation der zweiphasige Strukturelektroyten. In den Erläuterungen zur Porennetzwerkmethode wurde bereits eine Möglichkeit aus Gasabsorptionsmessungen ein Ersatzmodel aus Kugeln und zylindirsichen Verbindungen zu genieren näher beschrieben. Während diese Methode auch für den WoSt Ansatz benutzt werden kann\footnote{Dazu die Domain innerhalb der Kugeln und Hälse für die fluid Phase wählen oder die außerhalb liegende, aber immer noch im RVE-Würfel liegende Domain für die feste Phase benutzen.} sind Gasabsorptionsmessungen in der Literatur nicht immer gegeben. Allerdings können aus Computertomographen Aufnahmen von ausreichender Auflösung benutzt werden, um mithilfe machinellen Lernens (ML) die Randbedingung der \textsc{Cahn-Hilliard}-Gleichungen, sowie die Stoppzeit abzuschätzen.

Die Trainingsdaten bestehen aus Grauwert-REM-Scans, die in Schwarz-Weiß-Darstellung vorliegen, sowie den zugehörigen realen Abmessungen der Bildausschnitte in Nanometern. Zusätzlich werden experimentell bestimmte Diffusionskoeffizienten $D_i$ und elastische Steifigkeiten $E_i$ sowohl für die Einzelkomponenten als auch für den Verbundmaterial gemessen und den entsprechenden REM-Bildabschnitten zugeordnet.

Ein Convolutional Neural Network extrahiert aus den REM-Bildern textur- und strukturrelevante Merkmale. Diese Merkmale, kombiniert mit den realen Abmessungen, dienen als Eingabe für ein nachgeschaltetes Fully Connected Network, das die Parameter für die \textsc{Cahn-Hilliard}-Gleichung vorhersagt, namentlich die Randbindung $M$ und die Mobilität $L$ %\citep{Zhang2021CHparameters, Müller2017MLforCH}.

Die \textsc{Cahn-Hilliard}-Gleichung wird mit einem Finite-Elemente Methode gelöst, wobei die Parameter $L(\mathbf{x})$ und $\varepsilon$ werden durch das ML-Modell bereitgestellt %\citep{Elliott1989CHFEM, Schneider2022FEMCH}.

Aus der zeitabhängigen Lösung $c(\mathbf{x},t)$ werden die Phasengrenzen als Isoflächen $c(\mathbf{x},t^*) = c_\mathrm{threshold}$ extrahiert. Dies erfolgt durch Level-Set- oder Schwellenwertverfahren %\citep{Osher1988LevelSet, Lee2015InterfaceExtraction}.

Die extrahierte Geometrie definiert zwei Domänen: die Matrix- und die Flüssigphase. Auf jeder Domäne wird mittels linear-elastischer Finite-Elemente-Analyse die effektive Steifigkeit $E_\mathrm{eff}$ berechnet:
\begin{equation}
\nabla \cdot \bigl(\mathbf{C} : \nabla \mathbf{u}\bigr) = \mathbf{0}, 
\end{equation}
wobei $\mathbf{C}$ das Materialsteifigkeits-Tensor ist %\citep{Ciarlet2002FEM, Smith2018CompositeElasticity}. 
Parallel dazu wird die Diffusionsgleichung von \textsc{Fick}
zur Bestimmung des effektiven Diffusionskoeffizienten $D_\mathrm{eff}$ gelöst %\citep{Crank1979Diffusion, Kim2016MLDiffusion}.

Das gesamte Modell wird end-to-end trainiert, indem der Vergleich der berechneten $E_\mathrm{eff}$ und $D_\mathrm{eff}$ mit den experimentellen Werten in den Verlustfunktionsterm
\begin{equation}
    \mathcal{L} =  \frac{1}{1+\sqrt{\lVert \frac{E_\mathrm{eff}^\mathrm{pred} - E_\mathrm{eff}^\mathrm{exp}}{E_\mathrm{rein}^\mathrm{exp}}\rVert^2
+ \lVert \frac{D_\mathrm{eff}^\mathrm{pred} - D_\mathrm{eff}^\mathrm{exp}}{D_\mathrm{rein}^\mathrm{exp}}\rVert^2}}
\end{equation}
aufgenommen wird.  %\citep{Goodfellow2016DeepLearning, Bishop2006PatternRecognition}.


\section{\label{sec:improve_elchem} Reduktion des Berechnungsaufwandes}
\begin{itemize}
    \item Diffusionskoeffizient wird durch equivalente Schaltung ermittelt, die konstanten Wert vorraussetzen
    \item Diffusionskoeeffizeint ist eigentlich stark von der Lithierung abhängig
    \item aufwendig zu ermitteln
    \item außerdem abweichungen durch Bildung Elektrolyteinterface
    \item daher für vorhersagen ist die benutzung eher ungeeignet
    \item für Batterien ist Energidichte wichtiger als Leisungsdichte
    \item Lösung quasistatische Be- und Entladung, also warten bis vorher
    \item dies reduziert die vereinfacht die oberen Gleichungen enorm
\end{itemize}

\begin{equation}
    C_{\text{A, Zelle}} = \min \left( C_{\text{A, -}} , C_{\text{A, +}}\right)
\end{equation}

\begin{equation}
    C_{\text{A, Stack}} = n_{\text{Zellen}} \cdot C_{\text{A, Zelle}}
\end{equation}

\begin{equation}
    m_{\text{A, Stack, E}} = C_{\text{A, Stack}} \cdot V_{\text{C,E}} \cdot \rho_{\text{E}}
\end{equation}

\begin{equation}
    m_{\text{A, Stack}} = m_{\text{A, Stack, E}} + \sum_{i}^{n_{\text{Schichten}}} m_{\text{A,i}} 
\end{equation}

\begin{equation}
    C_{\text{m, Stack}} = \frac{C_{\text{A, Stack}} }{ m_{\text{A, Stack}}}
\end{equation}

\begin{equation}
    \Gamma_{\text{Stack}} = C_{\text{m, Stack}} \cdot \left(U_{+} - U_{-}\right)
\end{equation}

% aus: Simultaneously Coupled
% Mechanical-Electrochemical-
% Thermal Simulation of Lithium-
% Ion Cells
\begin{align}
    R_{\text{Kurz}} &= A_{\text{Kurz}} \sum_{i} \frac{1}{K_i}\\
    A &= \sum_{i}^{n_{\text{Versagen}}} A_{i}
\end{align}

\begin{equation}
    V_{\text{Zelle}} = (U_{+} - U_{-} + \sum_{j=+,-} \frac{2 RT}{F} ln\left(\frac{\sqrt{m_j^2 +4} + m_j}{2}\right) - i_{app} R_{\text{Kurz}}
    m_j = \frac{i_{app}}{F k_j S_j c_{s,j}^{max} \sqrt{c_e (1-x_{Li,j}) x_{Li,j}}} 
\end{equation}

\begin{equation}
    \rho v c_p \frac{\partial T}{\partial t} = i_{app}\left(V_{\text{Zelle}} - U_{+} + U_{-} + i_{app} R_{Kurz} \right) -q
\end{equation}

\begin{equation}
    q = 0
\end{equation}

\section{\label{sec:improve_mech}Identifizierung mechanischer Materialparameter}
Unter der Annahme, dass alle Einzelschichten bei der Bestimmung der Zugsteifigkeit auf beiden Seiten in der Klemmung mit aufgenommen werden und keiner Vordehnung der Einzelschichten sind die Dehnungen in Zugrichtungen für alle Schichten gleich.
\begin{equation}
    \varepsilon_{x,ges} = \varepsilon_{x,i}\\
\end{equation}



\subsection*{Reduktion des Berechnungsaufwandes für 3-Punkt-Biegebelastungen unter Berücksichtigung verschiederner Elektrolytarten}

Der Struktur von konventionellen Batterien oder Strukturbatterie mit Gel oder flüssigem Elektrolytsystemen kann vereinfacht als Schichtung, lastentragende Materialien betrachtet werden, in deren Zwischnraum eine nicht-lastentragenden Substanz in Form eines Flüssigen oder Gelartigen Zustandes infiltriert wurde.
Die einzelnen Schichten sind nicht direkt mit einander verbundnen und halten einzig durch den Druck der durch die äußere Pouchfolie aneinander. Unter der Annahme, dass die Sichten sich lückenlos anschmiegen ist davon aus zugehen, dass die Krümmung $\kappa$ mit
\begin{equation}
    \kappa = \frac{1}{r} = \frac{M_y}{E I_y}
\end{equation}
in jeder Schicht gleichgroßt ist.
\begin{equation}
    \kappa = \kappa_1 = \kappa_2 = \dots = \kappa_i = \dots = \kappa_n
\end{equation}
Des Weiteren folgt aus dem Momentengleichgewicht, dass das außen angreifende Biegemoment $M_{b}$ gleich der Summe der Schnittmomente in den Einzelschichten sein muss.
\begin{equation}
    M_{b} = \sum_{i}^{n}M_{y,i}
\end{equation}
Unter Annahme von rechticken Querschnitten mit Breite $b_i$ und Höhe $h_i$ und der Annhame, dass alle Elektroden näherungsweise gleich Breit sind, also $b_i = b$ gilt, folgt für die Belastung einer Einzelschicht durch das Moment $M_i$:
\begin{align}
    M_{b} &= M_i \sum_{k}^{n}\frac{E_k I_{yy,k}}{E_i I_{yy,i}}\\
    M_{b} &= M_i \frac{\sum_{k}^{n} E_k h_k^3}{E_i h_i^3}\\
    M_i &= M_{b} \frac{ E_i h_i^3} { \sum_{k}^{n}E_k h_k^3}
\end{align}
Durch einsetzen Einzelschichtbelastung in die Formel zur Bestimmung der Biegespannung erhält man einen Zusammenhang zwischen Einzelschichtspannung und Biegemomentenbelastung:
\begin{align}
    \sigma_{b,i} &= \frac{M_y,i}{I_{yy}/h_i} \\
    \sigma_{b,i} &= 12 \frac{ M_y,i}{b h_i^2}\\
    \sigma_{b,i} &= 12 \frac{M_{b} E_i h_i^3}{b h_i^2 \sum_{k}^{n}E_k h_k^3}\\
    \sigma_{b,i} &= 12 \frac{M_{b} E_i h_i}{b \sum_{k}^{n}E_k h_k^3}
\end{align}

Für die Bestimmung der Durchbiegung $u$ beim 3-Punkt-Biegeversuch kann 
unter der
\begin{equation}
\frac{\frac{\partial^2 u(x)}{\partial x^2}}{\left(1 + \left(\frac{\partial u(x)}{\partial x} \right)^2 \right)^{3/2}} = -\frac{M_y}{E I_{yy}}
\end{equation}
Diese Gleichung kann für kleine Verformungen, so dass $(\frac{\partial u(x)}{\partial x})^2 \ll 1$ durch die folgende Näherung ersetzt werden.
\begin{equation}
    \frac{\partial^2 u(x)}{\partial x^2} \approx -\frac{M_y(x)}{E I_{yy}}
\end{equation}

Unter der Annhame kleiner Verformung und konstantem Querschnitt und Steifigkeit lässt sich die Durchbiegung infolge der Kraft F durch folgende Gleichung annähern.
\begin{align}
    u(x) &= \frac{F L^3}{48 \sum_{k}^{n} E_k I_{yy,k}} \left[ 3 \frac{x}{L} - 4\left(\frac{x}{L}\right)^3 \right] \text{für} \; 0 \leq x \leq L/2 \\
    u_{max} (x = L/2) &= \frac{FL^3}{48 \sum_{k}^{n} E_k I_{yy,k}} 
\end{align}



An dieser Stelle ist zu bemerken, dass für Spezialfall wo alle $n$ Schichten gleich dick sind und aus dem gleichen Material bestehen, die Spannung sich wie folgend ergibt.
\begin{equation}
    \sigma = \sigma_i = \frac{12 M_{b}}{n b h^2},
\end{equation}
Diese Formel ist bereits im Kontext geschichteter Blattfedern bekannt und ist, was 

Druchbiegung $u_max$
\begin{equation}
    u_{max} = \frac{L^3 Q}{4 n b h^3 E} = \frac{L^2 \sigma}{6 h E}
\end{equation}
ergibt sich die 

\section*{Dreipunkt-Biegebelastung eines Laminats mit der klassischen Laminattheorie}

Wir betrachten einen einfach gelagerten Laminatbalken der Länge $L$, auf den eine Einzelkraft $F$ exakt in der Mitte wirkt (Dreipunkt-Biegung). Die Laminatstruktur kann beliebig viele Lagen mit unterschiedlichen Orientierungen aufweisen und wird durch die klassische Laminattheorie (CLT) beschrieben. Es wird angenommen, dass das Material linear-elastisch ist und kleine Durchbiegungen auftreten.

\subsection*{Kinematik und Spannung-Dehnungs-Zusammenhang}

In der CLT wird die Dehnung über der Dicke $z$ des Laminats beschrieben durch:

\[
\boldsymbol{\varepsilon}(z) = \boldsymbol{\varepsilon}^0 + z\,\boldsymbol{\kappa},
\]

wobei $\boldsymbol{\varepsilon}^0$ die Dehnungen in der Mittelfläche und $\boldsymbol{\kappa}$ die Krümmungen sind. Die Beziehung zwischen den Schnittgrößen (Kräfte und Momente pro Breite) und den Dehnungen/Krümmungen ergibt sich aus der ABD-Matrix:

\[
\begin{pmatrix}
\mathbf{N} \\
\mathbf{M}
\end{pmatrix}
=
\begin{pmatrix}
\mathbf{A} & \mathbf{B} \\
\mathbf{B} & \mathbf{D}
\end{pmatrix}
\begin{pmatrix}
\boldsymbol{\varepsilon}^0 \\
\boldsymbol{\kappa}
\end{pmatrix},
\]

wobei:
\begin{itemize}
  \item $\mathbf{A}$: Membranstetigkeit (in-plane),
  \item $\mathbf{B}$: Kopplung zwischen Membran- und Biegebeanspruchung,
  \item $\mathbf{D}$: Biegesteifigkeit.
\end{itemize}

\subsection*{Biegebelastung ohne Normalkräfte}

Für reine Biegung (keine Normalkräfte) gilt $\mathbf{N} = \mathbf{0}$. Daraus folgt:

\[
\mathbf{A}\boldsymbol{\varepsilon}^0 + \mathbf{B}\boldsymbol{\kappa} = \mathbf{0}
\quad\Rightarrow\quad
\boldsymbol{\varepsilon}^0 = -\mathbf{A}^{-1} \mathbf{B} \boldsymbol{\kappa}.
\]

Einsetzen in die Momentengleichung ergibt:

\[
\mathbf{M} = \mathbf{B} \boldsymbol{\varepsilon}^0 + \mathbf{D} \boldsymbol{\kappa} = 
\left( \mathbf{D} - \mathbf{B} \mathbf{A}^{-1} \mathbf{B} \right) \boldsymbol{\kappa}.
\]

Wir definieren daraus die effektive Biegesteifigkeit:

\[
\mathbf{D}^* = \mathbf{D} - \mathbf{B} \mathbf{A}^{-1} \mathbf{B}.
\]

Somit gilt:

\[
\boldsymbol{\kappa} = (\mathbf{D}^*)^{-1} \mathbf{M}.
\]

Für reine Biegung um die $x$-Achse gilt $M_x \ne 0$, alle anderen Momente sind null, also:

\[
\kappa_x = \frac{M_x}{D^*_{11}}.
\]

\subsection*{Durchbiegung bei Dreipunkt-Biegung}

Die Momentenverteilung über den Balken ist:

\[
M_x(x) = \frac{F}{2}x \quad\text{für } 0 \le x \le \frac{L}{2}.
\]

Somit ergibt sich die Krümmung zu:

\[
\kappa_x(x) = \frac{F x}{2 D^*_{11}}.
\]

Nach zweimaliger Integration (Euler–Bernoulli-Theorie) ergibt sich die maximale Durchbiegung in der Balkenmitte zu:

\[
\delta_\mathrm{max} = \frac{F L^3}{48 D^*_{11}}.
\]

Für einen symmetrischen Laminataufbau ($\mathbf{B} = \mathbf{0}$) vereinfacht sich $D^*_{11} = D_{11}$, und die Durchbiegung wird:

\[
\delta_\mathrm{max} = \frac{F L^3}{48 D_{11}}.
\]

\subsection*{Spannungen und Dehnungen im Laminat}

Die Dehnung in $x$-Richtung in einer Schicht bei Höhe $z$ ergibt sich zu:

\[
\varepsilon_x(z) = \varepsilon_x^0 + z\,\kappa_x.
\]

Diese globale Dehnung wird in die lokale Koordinatenrichtung jeder Schicht transformiert (je nach Faserwinkel $\theta_i$), und die Spannungen ergeben sich aus dem Stoffgesetz:

\[
\boldsymbol{\sigma}_i = \mathbf{Q}^{(i)}\, \boldsymbol{\varepsilon}_i.
\]

Dabei ist $\mathbf{Q}^{(i)}$ die Steifigkeitsmatrix der Schicht im lokalen Koordinatensystem. Die Spannungsverteilung ist innerhalb jeder Schicht linear, aber mit unterschiedlichen Steigungen je nach E-Modul und Orientierung.

\subsection*{Zusammenfassung}

Die maximale Durchbiegung des Laminatbalkens unter einer mittigen Kraft $F$ beträgt:

\[
\boxed{
\delta_\mathrm{max} = u_{max}(x=L/2) = \frac{F L^3}{48\,D^*_{11}}, \quad
D^*_{11} = D_{11} - [\mathbf{B} \mathbf{A}^{-1} \mathbf{B}]_{11}
}
\]

Für symmetrische Laminataufbauten (mit $\mathbf{B} = \mathbf{0}$) vereinfacht sich dies zu:

\[
\delta_\mathrm{max} = \frac{F L^3}{48\,D_{11}}.
\]

Die Spannungsverteilung im Laminat ergibt sich aus den berechneten Krümmungen und den Materialeigenschaften jeder Schicht.



Im Verhältniss zum stofflichen Verbund ist davon auszugehen, dass diese Steifigkeitssteigerung deutlich geringer ist, wenn aber auch nicht komplett vernachlässigbar.
\begin{equation}
    u_\text{max} = \varphi \cdot u_{\text{max,flüssig}} + (1-\varphi) \cdot u_{\text{max,fest}}
\end{equation}

\begin{figure}[!ht]
	%\raggedleft
		%\def\svgwidth{\columnwidth}
        \center
		\includegraphics[width=0.5\textwidth, angle=0]{simulation_model.pdf}
		\caption{\label{fig:homogenisation}Mehrskalige Homogenisierung der Strukturbatterie durch Abstraktion der Geometrie, }
\end{figure}

\begin{figure}[!ht]
	%\raggedleft
		%\def\svgwidth{\columnwidth}
        \center
		\includegraphics[width=0.5\textwidth, angle=0]{bending.pdf}
		\caption{\label{fig:bending}Validierung des 3-Punkt-Biegefalls durch: a) Visueller Vergleich von Simulation und Experiment und b) Kraftverlauf in Abhängigkeit der Durchbiegung für Pouchzellen mit und ohne Elektrolyt.}
\end{figure}


\section{\label{sec:automated_failure}Automatisierte Versagensanalyse für Strukturbatterien und Risikoeinschätzung}

Mit Hilfe der entwickelten Modelle, kann der Spannungszustand $\boldsymbol{\sigma}$ in den einzelnen Materialien innerhalb der Strukturbatterie bestimmt werden. Durch geeignet Versagenskriterien kann ein mechanisches Versagen der Schichten ermittelt werden.  
Typsiche Versagenskriterein für isotrope 


\begin{table}[ht]
    \centering
    \caption{\label{tab:failure_modes}Überischt des mit Versagens der Einzelschichten verknüpften Sichheitsrisikos.}
    \begin{tabularx}{\textwidth}{lXXX}
    \toprule
    &\makecell{Pouchfolienversagen\\\includegraphics[width=0.2\textwidth]{failure_modes/failure_mode_pouch.png}}
    &\makecell{Elektrodenversagen\\\includegraphics[width=0.2\textwidth]{failure_modes/failure_mode_electrode.png}}
    &\makecell{Separatorversagen\\\includegraphics[width=0.2\textwidth]{failure_modes/failure_mode_separator.png}}
    \\
    \midrule
    Funktion
        & Funktionsversagen der gesamten Batterie durch austrocknen
        & Leistungsverlust der Zelle
        & Funktionsversagen der Zelle, je nach Verschaltung auch der gesamten Batterie
    \\
    Brandgefahr
        & kein Risiko
        & kein Risiko
        & Flammenbildung durch Überhitzung
    \\
    Gesundheit
        & hohes Risiko durch austretendes Elektrolyt
        & kein Risiko
        & kein zusätzliches Risiko
    \\
    \bottomrule
    \end{tabularx}\\
    %\noindent{\footnotesize{\textsuperscript{*} Gemessen gegenüber \ce{Li}/\ce{Li+}.}}
\end{table}%

\section{\label{sec:digitalisation}Erstellung einer Materialdatenbank für Strukturbatterien}