\chapter{\label{sec:lightweight_applications}Anwendungspotenzial von Strukturbatterien im Leichtbau}

Der Leichtbau verfolgt das Ziel, durch eine konsequente Reduktion der Systemmasse bei gleichbleibender oder verbesserter Funktionalität Effizienzgewinne zu erzielen. In vielen technischen Systemen wird dieses Ziel jedoch zunehmend durch funktionale Zusatzanforderungen konterkariert: Elektrifizierung, Sensorik, Aktorik und autonome Funktionen führen zu einem stetigen Zuwachs an Komponenten, insbesondere an Energiespeichern. Genau an dieser Schnittstelle entfalten Strukturbatterien ihr besonderes Potenzial für den Leichtbau, da sie es erlauben, Energieversorgung nicht länger als additiven Massenblock, sondern als integralen Bestandteil der Struktur zu begreifen \cite{Yang2020,Asp2021a}.

Aus Leichtbausicht liegt der zentrale Vorteil weniger in der absoluten Energiedichte der Strukturbatterie, sondern in der systemischen Massen- und Volumeneffizienz. Durch die Mehrfachnutzung von Bauteilen können klassische Funktionsgrenzen aufgelöst werden: Tragende Strukturen übernehmen gleichzeitig Aufgaben der Energiespeicherung, wodurch separate Batteriemodule, Gehäuse, Halterungen und Teile der Verkabelung entfallen. Diese Funktionsintegration führt nicht nur zu einer Reduktion der Gesamtmasse, sondern eröffnet neue Freiheitsgrade im Design, etwa hinsichtlich Bauraumaufteilung, Modularität und Lastpfadgestaltung \cite{Danzi2021,Johannisson2019}.

Besonders relevant für den Leichtbau ist zudem der Effekt der indirekten Gewichtseinsparung. In mobilen Systemen, wie etwa in Flugzeugen, Fahrzeugen oder Robotern, führt jede eingesparte Masse zu einer Kettenreaktion: geringere Strukturmassen ermöglichen leichtere Aufhängungen, reduzierte Antriebsleistungen und kleinere konventionelle Energiespeicher. Strukturbatterien adressieren diesen Effekt gezielt, indem sie dort eingesetzt werden, wo Strukturmasse ohnehin notwendig ist, und diese funktional aufwerten, anstatt zusätzliche Masse einzubringen \cite{Johannisson2019}.

Vor diesem Hintergrund wird deutlich, dass der Nutzen von Strukturbatterien im Leichtbau stark anwendungsabhängig ist. Ihr Mehrwert ist besonders dort hoch, wo Gewicht und Bauraum kritische Systemparameter darstellen und wo die mechanischen Anforderungen eine schrittweise Integration erlauben. Die Entwicklung folgt dabei keinem disruptiven, sondern einem evolutionären Pfad, der sich entlang steigender struktureller Verantwortung vollzieht. Zu diesem Zweck wurde ein Phasenmodell für die Technologieeinführung erstellt, das die Anwendungen realistisch nach kurz-, mittel- und langfristiger Umsetzbarkeit einordnet \cite{Kuehnelt2021}. Im weiteren Verlauf des Kapitels werden daher konkrete Anwendungsfelder betrachtet, in denen sich der leichtbauspezifische Mehrwert von Strukturbatterien besonders klar zeigt.


\section{Phasenmodell der Technologieeinführung}

Die Einführung von Strukturbatterien in Leichtbauanwendungen erfordert ein klar strukturiertes Phasenmodell, da es sich um eine hochgradig interdisziplinäre Technologie mit stark anwendungsabhängiger Reife handelt. Im Gegensatz zu etablierten Batteriekonzepten ist die Leistungsfähigkeit von Strukturbatterien nicht isoliert über klassische Kennwerte wie Energiedichte zu bewerten, sondern stets im Zusammenspiel mit mechanischen Anforderungen, Sicherheitsaspekten, Fertigungsprozessen und regulatorischen Rahmenbedingungen \cite{Asp2021a,Danzi2021,Jaeger2016}. Das Modell dient daher nicht nur der zeitlichen Einordnung technischer Entwicklungen, sondern insbesondere der systematischen Bewertung realistischer Einsatzpfade.

\begin{table}[!ht]
    \centering
    \caption{\label{tab:pot_anwendungen}Potentielle Anwendungen von Strukturbatterien für verschiedene Einsatzbereiche~\cite{Yang2020,Danzi2021,Asp2021,Wang2020,Johannisson2019,Kalnaus2021,Nie2023,Meng2018}.}
    \begin{tabular}{m{0.15\textwidth} m{0.2\textwidth}<{\centering} m{0.4\textwidth}<{\centering} m{0.1\textwidth}<{\centering}}
        \toprule
        \textbf{Technologi-scher Einsatzbereich}&\textbf{Teilbereiche}&\textbf{Anwendungsbeispiele}&\textbf{Einsetz-barkeit [Jahre]}\\
        \midrule
        \multirow{3}*{Luftfahrt}
                    &Tertiärestrukturen&Entertainmentsystem und Innenverkleidungen&$< 5$\\
                    \addlinespace
                    &Sekundärstruktur&Trennwände und Gepäckfächer; Fahrwerkstüren; Rahmenstrukturen und Stellklappem&5 - 15\\
                    \addlinespace
                    &Primärstrukturen&Flugzeugrumpf und Flügelstrukturen&$> 15$\\
                    %\hspace{1em}
                    \hline
        \multirow{3}*{Automobil}
                    &Karosserieteile&Türverkelidungen, Innenraumelemente und Sitzstrukturen&$< 5$\\
                    \addlinespace
                    &Sekundäre Fahrzeugstrukturen&Armaturenebrett; Dachhimmel; Trennwände&5 - 15\\
                    \addlinespace
                    &Tragende Strukturen&Fahrgestell oder Karosserie&$> 15$\\
                    \hline
        \multirow{4}*{Robotik}
                    &Leichte Robotikstrukturen & Integration in kleine Roboterarme, Serviceroboter oder Drohnen&$< 5$\\
                    \addlinespace
                    &Mittellastragende Strukturen&Nutzung in mittellasttragenden Strukturen wie Gelenken und Rahmen von Industrierobotern&5 - 15\\
                    \addlinespace
                    &Tragende Hauptstrukturen&Integration in Haupttragstrukturen von größeren Robotern oder autonomen Systemen&$> 15$\\
                    \addlinespace
                    &Hochdynamische/ bestastete Systeme&Anwendungen in hochbelasteten und sicherheitskritischen Bereichen wie schwerlasttragende Industrieroboter&$> 20$\\
        \bottomrule
    \end{tabular}
\end{table}

Die in diesem Kaptiel zugrunde gelegte Klassifizierung basiert auf einer umfassenden Literaturrecherche und einer vergleichenden Evaluation bestehender Forschungsarbeiten zu Strukturbatterien. Dabei wurden nicht einzelne Demonstratoren isoliert betrachtet, sondern Anwendungsfelder über verschiedene Industrien hinweg analysiert und hinsichtlich ihrer strukturellen Anforderungen, ihres Leichtbaupotenzials sowie ihres technologischen Reifegrads eingeordnet \cite{Yang2020,Johannisson2019}. Das zentrale Ergebnis dieser Analyse ist die in Tabelle~\ref{tab:pot_anwendungen} dargestellte Zuordnung potenzieller Anwendungen zu kurz-, mittel- und langfristigen Einsatzzeiträumen.

Ein zentrales Strukturierungsmerkmal dieses Phasenmodells ist die Unterscheidung zwischen tertiären, sekundären und primären Strukturen. Diese Differenzierung erlaubt eine leichtbauspezifische Betrachtung, da mit zunehmender struktureller Verantwortung nicht nur die mechanischen Anforderungen steigen, sondern auch die sicherheitsrelevanten und regulatorischen Hürden deutlich zunehmen. Entsprechend zeigt sich in der Tabelle ein konsistentes Muster über alle betrachteten Branchen hinweg: Anwendungen in gering belasteten Strukturen sind kurzfristig realisierbar, während hochbelastete Hauptstrukturen eindeutig als langfristige Zielanwendungen identifiziert werden \cite{Kuehnelt2021,Wang2020}.

Kurzfristige Einsatzmöglichkeiten konzentrieren sich demnach auf Bauteile, bei denen der leichtbauliche Mehrwert durch Funktionsintegration hoch ist, die mechanischen Lasten jedoch vergleichsweise gering bleiben. Diese Anwendungen sind besonders geeignet, um erste industrielle Erfahrungen zu sammeln, Fertigungs- und Integrationsprozesse zu validieren und Akzeptanz bei Anwendern und Zulassungsstellen aufzubauen. Das Modell versteht diese Phase explizit als Lern- und Demonstrationsphase, nicht als Endzustand der Technologie \cite{Meng2018}.

Mittelfristige Anwendungen verschieben den Fokus auf sekundäre Strukturen mit moderater Lastübertragung. Hier wird deutlich, dass der Leichtbauvorteil von Strukturbatterien zunehmend systemisch wirkt: Gewichtseinsparungen ergeben sich nicht nur durch den Wegfall konventioneller Batteriemodule, sondern auch durch eine optimierte Lastpfadführung und eine reduzierte Systemkomplexität. Die Literatur zeigt jedoch, dass für diese Anwendungen noch substanzielle Fortschritte in der Materialzuverlässigkeit, der elektro-mechanischen Kopplung sowie in der Auslegung und Simulation erforderlich sind \cite{Johannisson2019,Nie2023}.

Langfristige Einsatzszenarien betreffen schließlich primäre tragende Strukturen, bei denen Strukturbatterien integraler Bestandteil der Gesamtstruktur werden. Die zeitliche Einordnung macht deutlich, dass diese Anwendungen nicht allein durch inkrementelle Verbesserungen erreichbar sind, sondern grundlegende Weiterentwicklungen in Materialsystemen, Sicherheitskonzepten, Inspektions- und Reparaturstrategien sowie in Normung und Zertifizierung voraussetzen \cite{Asp2021a,Kuehnelt2021}. Entsprechend werden diese Anwendungen in der Tabelle konsistent als langfristig kategorisiert.

Abschließend ist zu erkennen, dass sich durch das Phasenmodell ein kohärentes Gesamtbild über mehrere Forschungsstudien hinweg ergibt. Es wird deutlich, dass der Einsatz von Strukturbatterien im Leichtbau keinem singulären Durchbruch folgt, sondern einer klar strukturierten, evolutionären Entwicklung. Diese systematische Einordnung bildet damit die Grundlage für die nachfolgende detaillierte Betrachtung ausgewählter Anwendungsfelder und deren spezifisches Leichtbaupotenzial.

\section{Anwedungsbeispiele}
\subsection{Flugzeuginnenraum}

Der Flugzeugkabineninnenraum ist durch besonders strenge Anforderungen an Gewicht, Bauraum, Sicherheit und Funktionalität geprägt~\cite{EASA_CS25, RTCA_DO160, ABD0031, Mensen2013}. Jedes zusätzliche Kilogramm wirkt sich direkt auf den Treibstoffverbrauch und damit auf Betriebskosten und Emissionen aus\footnote{Eine Analyse von Dynamic Metals zeigt, dass pro 1~\% Gewichtsreduzierung eines Flugzeugs der Treibstoffverbrauch um etwa 0,75~\% sinkt. Bei einem Langstreckenjet kann dies zu einer jährlichen Einsparung von 300.000 Litern Treibstoff führen, was die \ce{CO2}-Emissionen um etwa 750 Tonnen reduziert und die Treibstoffkosten pro Flugzeug um 475.000~\$ senkt~\cite{DynamicMetals2023}.}. Gleichzeitig steigt der Bedarf an elektrischer Energie kontinuierlich, insbesondere durch Bordmediasysteme und die zunehmende Erwartung der Passagiere an individuelle Stromversorgung~\cite{Aguida2024}. Konventionelle Lösungen basieren häufig auf zentralen Stromverteilungen, umfangreicher Verkabelung und nichttragenden Kunststoffbauteilen, was sowohl das Gewicht als auch die Systemkomplexität erhöht. Vor diesem Hintergrund bieten Strukturbatterien ein hohes Potenzial, um strukturelle Funktionen mit dezentraler Energiespeicherung zu kombinieren und damit neue Freiheitsgrade in der Kabinenauslegung zu eröffnen.

Zur Untersuchung dieses Potenzials wurde im Rahmen dieser Arbeit eine Fallstudie zur Integration von Strukturbatterien in den Flugzeugkabineninnenraum durchgeführt. Als Referenzszenario diente ein Airbus A321neo mit einer standardisierten Bestuhlung von 200 Sitzplätzen~\cite{AirbusA321Spec}. Der Fokus der Untersuchung lag auf dem Ausklapptisch des Passagiersitzes, der gemäß den Anforderungen der EASA CS-25 für strukturelle Innenraumkomponenten ausgelegt sein muss~\cite{EASA_CS25_SubpartD}. In der konventionellen Ausführung besteht dieser Tisch aus einem nicht-funktionalen Kunststoff\footnote{hier PLA} und verfügt über keinerlei energetische Kapazität. Durch den Ersatz dieser Strukturkomponente durch eine Strukturbatterie wurde der Ausklapptisch so umgestaltet, dass er gleichzeitig als tragendes Bauteil und als dezentrale Energiespeichereinheit zur Versorgung des Bordmediasystems fungiert.

\begin{figure}[!ht]
    \center
	\includegraphics[width=0.8\textwidth, angle=0]{cabine_usecase.pdf}
	\caption{\label{fig:cabin}Der Einsatz von Strukturbatterien kann in Flugzeugkabinen dazu eingesetz werden um eine kabelose Stromversorgung am Passagiersitz zu ermöglichen, was nicht nur Kabel einsparrt, sonder auch bei der Gewichtsverteilung hilft.}
\end{figure}

Als Strukturbatterie wurde die Variante aus ElViS gewählt (siehe Tabelle~\ref{tab:combinations}). 
Die Ergebnisse der Studie zeigen deutliche Vorteile gegenüber der konventionellen Lösung. Pro Sitzplatz steht eine zusätzliche Speicherkapazität von 3700 mAh zur Verfügung, wodurch eine dezentrale und bedarfsgerechte Energieversorgung ermöglicht wird. Gleichzeitig kann die Kabinenverkabelung signifikant reduziert werden: Insgesamt lassen sich etwa 390 m an Kabeln einsparen, was nicht nur das Gewicht senkt, sondern auch die Systemkomplexität und den Wartungsaufwand reduziert. Auf struktureller Ebene weist der Strukturbatterie-Ausklapptisch eine um 520~\% höhere Biegesteifigkeit sowie eine um 320~\% höhere Festigkeit im Vergleich zur konventionellen PLA-Ausführung auf. Trotz dieser deutlich verbesserten mechanischen Eigenschaften ergibt sich eine Gewichtseinsparung von 57~g pro Sitz.

Die Ergebnisse dieser Fallstudie verdeutlichen das erhebliche Potenzial von Strukturbatterien für den Flugzeugkabineninnenraum. Die Kombination aus Gewichtsreduktion, erhöhter struktureller Leistungsfähigkeit und zusätzlicher Energiespeicherung adressiert zentrale Herausforderungen der Kabinenauslegung. Insbesondere die Möglichkeit, elektrische Energie dezentral und direkt in strukturellen Bauteilen bereitzustellen, eröffnet neue Konzepte für modulare, wartungsarme und energieeffiziente Kabinensysteme. Langfristig können Strukturbatterien damit einen wichtigen Beitrag zur Reduktion des Systemgewichts, zur Vereinfachung der Bordelektrik und zur Steigerung des Passagierkomforts leisten.

\section{Robotik}
Mobile Roboter unterliegen ausgeprägten systemischen Zielkonflikten zwischen Energieversorgung, Masse, struktureller Integrität und dynamischer Leistungsfähigkeit. Insbesondere bei laufenden Robotern bestimmt die verfügbare Energiemenge direkt die Einsatzdauer und Reichweite, während eine Erhöhung der Masse unmittelbar negative Auswirkungen auf Geschwindigkeit, Agilität und Energieeffizienz hat. Konventionelle Ansätze zur Reichweitensteigerung basieren meist auf der Integration zusätzlicher Batterien, was zwar die Kapazität erhöht, jedoch das Gesamtgewicht steigert und damit die mechanische und energetische Effizienz des Systems reduziert. Vor diesem Hintergrund stellen Strukturbatterien einen vielversprechenden Ansatz dar, da sie das Potenzial besitzen, tragende Strukturen und Energiespeicher zu vereinen und somit den klassischen Zielkonflikt zwischen Struktur und Energie zumindest teilweise aufzulösen.

Zur Bewertung dieses Potenzials wurde im Rahmen dieser Arbeit eine Fallstudie an einem vierbeinigen mobilen Roboter durchgeführt. Untersucht wurden ausgewählte Chassis-Komponenten eines Roboterhundes, der auf dem Open-Source-Design von KDY0523 basiert~\cite{SpotMiniOpenSource2025}. Mit einem Gesamtgewicht von ca. 16~kg und einer maximalen Fortbewegungsgeschwindigkeit von 3~m/s stellt das System hohe Anforderungen an die strukturelle Integrität und die Energiedichte. Die aktuelle Struktur des Roboters ist primär additiv aus konventionellen Thermoplasten gefertigt, was zwar eine schnelle Prototypenerstellung ermöglicht, jedoch aufgrund der Materiallimitationen bei einer Betriebsdauer von etwa 1,2~Stunden und den auftretenden dynamischen Lasten an mechanische Grenzen stößt. Verglichen wurden drei Konfigurationen: eine Referenzkonfiguration mit konventioneller Batterie, eine Variante mit zusätzlich integrierten konventionellen Pouchzellen sowie eine Variante, bei der Teile des Chassis durch die ElViS-Strukturbatterie ersetzt wurden (siehe Tabelle~\ref{tab:combinations}). Ziel war es, den Einfluss der beiden Integrationsstrategien auf Betriebsdauer, Masse, Dynamik und Reichweite systematisch zu analysieren.

\begin{figure}[!ht]
    \center
	\includegraphics[width=0.99\textwidth, angle=0]{dog_robot_sb_study.pdf}
	\caption{\label{fig:robot_dog}Gegenüberstellung konventioneller und struktureller Energiespeicherkonzepte am Beispiel eines vierbeinigen Roboters: Evaluation von Gewichtseffizenz und missionsspezifischen Parametern unter Einsatz der \textit{ElViS}-Strukturbatterie.}
\end{figure}

Die Ergebnisse zeigen, dass beide Ansätze eine signifikante Erhöhung der verfügbaren elektrischen Kapazität ermöglichen. Die Betriebsdauer konnte von 1,2~h in der Referenzkonfiguration auf 1,8~h durch die Integration zusätzlicher Pouchzellen und auf 1,6~h durch den Einsatz von Strukturbatterien gesteigert werden. Aufgrund des geringen Ausgangsgewichts des Roboters und der vergleichsweise niedrigen strukturellen Lasten führen jedoch beide Konzepte zu einer Erhöhung der Gesamtmasse. Das Gesamtgewicht steigt von 16~kg auf 16,9~kg bei integrierten Pouchzellen, während es bei Verwendung von Strukturbatterien lediglich 16,2~kg beträgt. Damit zeigt sich bereits in diesem frühen Anwendungsfall ein klarer Vorteil der Strukturbatterien, da die zusätzliche Energiefunktion mit einer deutlich geringeren Gewichtszunahme realisiert wird.

Die Auswirkungen auf die Bewegungsgeschwindigkeit und Reichweite wurden mittels dynamischer Simulationen in ROS\footnote{ROS ist ein quelloffenes, meta-betriebssystemähnliches Framework für die Roboterentwicklung.} und Gazebo\footnote{Gazebo ist ein leistungsfähiger 3D-Simulator für die Robotik, der die physikalisch korrekte Interaktion zwischen Robotern und deren Umgebung berechnet.} untersucht~\cite{ROS2, Gazebo}. Während die maximale Laufgeschwindigkeit in der Referenzkonfiguration bei 3,0~m/s liegt, reduziert sie sich auf 2,6~m/s bei integrierten Pouchzellen und auf 2,8~m/s beim Einsatz von Strukturbatterien. Trotz des erhöhten Gewichts bleibt die dynamische Leistungsfähigkeit bei Strukturbatterien somit näher am Ausgangszustand. Gleichzeitig zeigt die Reichweitenanalyse eine deutliche Verbesserung: Die Reichweite erhöht sich von 13 km auf 16,8~km bei integrierten Pouchzellen und auf 16,7~km bei Strukturbatterien. Bemerkenswert ist hierbei, dass mit Strukturbatterien nahezu die gleiche Reichweite wie mit zusätzlichen konventionellen Batterien erzielt wird, obwohl die Masse signifikant geringer ausfällt.

Diese Ergebnisse unterstreichen das enorme Potenzial von Strukturbatterien für die mobile Robotik. Bereits jetzt überzeugen sie durch exzellente funktionale Integration und Systemeffizienz. Besonders bei steigenden Anforderungen an Nutzlast oder Missionsdauer wird dieser technologische Vorsprung voll ausgespielt. Strukturbatterien ermöglichen so signifikante Reichweitensteigerungen und längere Einsatzzeiten, ohne die Mobilität durch das Zusatzgewicht konventioneller Batterien zu limitieren.