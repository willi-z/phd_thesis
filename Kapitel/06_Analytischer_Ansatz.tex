\chapter{Herleitung Analytischer Ansätze für perspektivische Vorauslegungen}
Während die vorangegangenen Kapitel detaillierte numerische Modelle zur Abbildung lokaler Effekte und komplexer Kopplungsphänomene behandelten, steht in diesem Kapitel die praktische Anwendbarkeit für die Systemauslegung im Vordergrund. Für die Dimensionierung multifunktionaler Leichtbaustrukturen unter komplexen Lastprofilen (gemäß TZ 4) ist es essenziell, bereits in frühen Entwurfsphasen die Wechselwirkungen zwischen Zellchemie und mechanischer Integrität quantifizieren zu können, ohne auf rechenintensive Simulationen angewiesen zu sein.

Der Fokus liegt hierbei auf der Identifikation von methodischen Modellreduktionen, die physikalisch konsistente Vorhersagen über die Energiedichte und das Verformungsverhalten erlauben. Durch die Überführung der elektrochemischen Dynamik in eine quasistatische Betrachtungsweise sowie die Anwendung modifizierter Laminattheorien für das Biegeverhalten wird eine Brücke zwischen Materialkennwerten und der globalen Systemcharakteristik geschlagen. Dies ermöglicht eine effiziente Variantenstudie und bildet die Grundlage für die zielgerichtete Auslegung von Strukturbatterien in realen Leichtbauanwendungen.

\section{\label{sec:improve_elchem}Analytische Vorhersage der Energiedichte}
Die präzise elektrochemische Modellierung von Lithium-Ionen-Zellen erfordert in der Regel die Berücksichtigung eines diffusionsabhängigen Stofftransportes im aktiven Material. Traditionell wird der Diffusionskoeffizient über eine äquivalente elektrische Schaltung angenähert, die einen konstanten, d.h. lithierungsunabhängigen Wert voraussetzt. Diese Annahme führt jedoch zu erheblichen Vereinfachungen, da der tatsächliche Diffusionskoeffizient stark von der lokalen Lithierung abhängt und zudem im Betrieb Temperaturänderungen sowie Alterungsmechanismen, insbesondere die Ausbildung der SEI, berücksichtigt werden müssten. Die exakte Ermittlung eines dynamischen, lithierungsabhängigen Diffusionskoeffizienten ist erfahrungsgemäß rechenaufwendig und experimentell schwer zugänglich. Für schnelle, echtzeitfähige Vorhersagemodelle ist dieser Ansatz daher nur bedingt geeignet.

Für viele Anwendungen im Bereich der elektrochemischen Energiespeicher ist jedoch nicht primär die Leistungsdichte, sondern vielmehr die Energiedichte der Batterie maßgebend. Dieser Fokus ermöglicht es, den Modellierungsaufwand erheblich zu reduzieren. Eine praktikable Lösung besteht darin, Be- und Entladeprozesse quasistatisch zu betrachten. Dabei wird nach jedem kleinen Lade- oder Entladeschritt eine Relaxationsphase angenommen, in der sich das Konzentrationsfeld im aktiven Material vollständig ausgleicht. Die Diffusionsprozesse müssen somit nicht explizit zeitaufgelöst simuliert werden, was die vorherigen elektrochemischen Gleichungen drastisch vereinfacht und den Berechnungsaufwand deutlich reduziert.

Ein zentraler Schritt zur Bewertung der maximal verfügbaren Energiedichte eines elektrochemischen Systems ist die Bestimmung der Oberflächenkapazität der Zelle. Die Oberflächenkapazität beschreibt die umgesetzte Stoffmenge pro aktiver Elektrodenfläche und ist damit direkt an die stöchiometrisch erreichbare Lithiumaufnahme der Elektroden gekoppelt. Da eine Lithium-Ionen-Zelle stets aus einer Anode und einer Kathode besteht, die über den Elektrolyten Lithiumionen austauschen, bestimmt die jeweils limitierende Elektrode die insgesamt nutzbare Kapazität. 

Genau dieses limitierende Verhalten wird durch die folgende Beziehung beschrieben
\begin{equation}
    C_{\text{A, Zelle}} = \min \left( C_{\text{A, -}} , C_{\text{A, +}}\right).
\end{equation}

Hierbei stehen $C_{\text{A, -}}$ und $C_{\text{A, +}}$ für die spezifischen Oberflächenkapazitäten der negativen  bzw. positiven  Elektrode. Die $\min$-Funktion reflektiert, dass eine Elektrode stets vor der anderen ihre maximal mögliche Lithierung bzw. Delithierung erreicht. Sobald die Kapazität der limitierenden Elektrode ausgeschöpft ist, kann der elektrochemische Prozess nicht weitergeführt werden, unabhängig davon, ob die Gegenelektrode theoretisch noch Ladung aufnehmen oder abgeben könnte. 

Die Wahl der minimalen Kapazität stellt somit sicher, dass das Modell physikalisch konsistent bleibt und die tatsächliche, durch die Zellchemie begrenzte Leistungsfähigkeit korrekt beschreibt. Dieser Ansatz ist insbesondere für vereinfachte Modelle essenziell, da er ohne detaillierte Konzentrationsprofile oder komplexe Diffusionsberechnungen auskommt und dennoch die reale Einschränkung durch die elektrochemisch schwächere Elektrode akkurat abbildet.

Da in der Praxis ausschließlich identische Zellen innerhalb eines Stacks verschaltet werden, addieren sich deren Einzelkapazitäten linear, sodass die Gesamtkapazität des Stacks einfach als Produkt aus der Anzahl der Zellen und der Oberflächenkapazität einer Einzelzelle beschrieben werden kann
\begin{equation}
    C_{\text{A, Stack}} = n_{\text{Zellen}} \cdot C_{\text{A, Zelle}}.
\end{equation}

Die Masse des im Stack enthaltenen Elektrolyten lässt sich direkt aus der Gesamtoberflächenkapazität ableiten, da pro umgesetzter Kapazität ein charakteristisches Elektrolytvolumen $V_{\text{C,E}}$ benötigt wird. Multipliziert man dieses volumenbezogene Kapazitätsmaß mit der Gesamtoberflächenkapazität des Stacks sowie der Elektrolytdichte $\rho_{\text{E}}$, ergibt sich die Gesamtmasse des erforderlichen Elektrolyten
\begin{equation}
    m_{\text{A, Stack, E}} = C_{\text{A, Stack}} \cdot V_{\text{C,E}} \cdot \rho_{\text{E}}.
\end{equation}

Durch Summierung der Masse des Elektrolyten und der Einzelschichten kann die Gesamtmasse des Batteriestacks $m_{\text{A, Stack}}$ wie folgt bestimmt werden.
\begin{equation}
    m_{\text{A, Stack}} = m_{\text{A, Stack, E}} + \sum_{i=1}^{n_{\text{Schichten}}} m_{\text{A,i}}
\end{equation}

Während die zuvor hergeleiteten Größen auf die elektrochemisch aktive Fläche bezogen sind, erfordert die Bewertung der Energiedichte eine Normierung auf die Gesamtmasse des Stacks. Erst durch diese Umrechnung lässt sich beurteilen, wie viel Kapazität pro Masseeinheit tatsächlich bereitgestellt werden kann. Die massenbezogene Kapazität ergibt sich daher als Quotient aus der gesamten Oberflächenkapazität des Stacks und seiner Gesamtmasse
\begin{equation}
    C_{\text{m, Stack}} = \frac{C_{\text{A, Stack}} }{ m_{\text{A, Stack}}}.
\end{equation}

Daraus ergibt sich die gravimetrische Energiedichte des Stacks unmittelbar aus der massenbezogenen Kapazität und der nutzbaren Zellspannung. Durch Multiplikation dieser beiden Größen erhält man den spezifischen Energieinhalt des Systems.
\begin{equation}
    \Gamma_{\text{Stack}} = C_{\text{m, Stack}} \cdot \left(U_{+} - U_{-}\right)
\end{equation}

% Simultaneously Coupled Mechanical-Electrochemical- Thermal Simulation of Lithium-Ion Cells
Darüber hinaus können Kurzschluss- und Versagensmechanismen weiterhin mit reduzierten Modellen abgebildet werden,
\begin{align}
    R_{\text{Kurz}} &= A_{\text{Kurz}} \sum_{i} \frac{1}{K_i}\\
    A_{\text{Kurz}} &= \sum_{i}^{n_{\text{Versagen}}} A_{i}
\end{align}
wobei $K_i$ die effektive Leitfähigkeit bzw. Permeabilität des jeweiligen Kurzschluss- oder Versagenspfades beschreibt~\cite{Zhang2016}.

Die Zellspannung ergibt sich unter Berücksichtigung vereinfachter Konzentrationsüberspannungen sowie des Kurzschlusswiderstandes~\cite{Daigle2013} zu
\begin{equation}
    V_{\text{Zelle}} = U_{+} - U_{-} + \frac{RT}{F} \ln \left( \frac{1-x}{x}\right) - i_{app} R_{\text{Kurz}}.
\end{equation}
Das thermische Verhalten kann in der quasistatischen Betrachtung ebenfalls reduziert werden, da reversible Wärmequellen dominieren und dissipative Beiträge klein sind. Damit ergibt sich
\begin{equation}
    \rho v c_p \frac{\partial T}{\partial t} = i_{app}\left(V_{\text{Zelle}} - U_{+} + U_{-} + i_{app} R_{Kurz} \right) -q.
\end{equation}
Da beim quasistatischen Be- und Entladen keine nennenswerten Konzentrationsgradienten und damit auch keine irreversiblen Relaxationsprozesse auftreten, können wärmeerzeugende Nebenreaktionen vernachlässigt werden. Somit entfallen dissipative Wärmequellen vollständig, und es gilt in guter Näherung
\begin{equation}
    q = 0.
\end{equation}
Durch diese Annahmen lassen sich die elektrochemischen und thermischen Teilmodelle signifikant vereinfachen, wodurch der Gesamtberechnungsaufwand stark reduziert wird, ohne die Aussagekraft für energiedichteorientierte Anwendungen wesentlich einzuschränken.


\section{\label{sec:improve_mech}Analytische Bestimmung des Verformungsverhaltens}
Unter der Annahme, dass alle Einzelschichten beidseitig kraft- oder formschlüssig mit der Einspannung verbunden sind und keine Vordehnung dieser vorliegt, kann für die Bestimmung der Zugsteifigkeit von einem einheitlichen Dehnungszustand in Zugrichtung ausgegangen werden.
\begin{equation}
    \varepsilon_{x,ges} = \varepsilon_{x,i}\\
\end{equation}

Kennzeichnend für Batterien mit reinen Flüssig- oder Gelelektrolytsystemen ist, dass diese, im Gegensatz zu Festkörperelektrolyten, über keine mechanisch belastbare Feststoffphase verfügen. Die Zellstruktur kann daher als Schichtung lasttragender Strukturelemente, während der Elektrolyt als mechanisch passive Infiltration in den Zwischenschichten vorliegt.
Dadurch sind die einzelnen Schichten nicht direkt mit einander verbunden und halten einzig durch den Druck, der durch die äußere Pouchfolie ampliziert wird, aneinander. Unter der Annahme der Verschiebungsstetigkeit an den Schichtgrenzflächen folgt, dass die Krümmung $\kappa$ mit
\begin{equation}
    \kappa = \frac{1}{r} = \frac{M_y}{E I_y}
\end{equation}
in jeder Schicht einheitlich ist.
\begin{equation}
    \kappa = \kappa_1 = \kappa_2 = \dots = \kappa_i = \dots = \kappa_n
\end{equation}
Des Weiteren folgt aus dem Momentengleichgewicht, dass das außen angreifende Biegemoment $M_{b}$ gleich der Summe der Schnittmomente in den Einzelschichten sein muss.
\begin{equation}
    M_{b} = \sum_{i}^{n}M_{y,i}
\end{equation}
Unter Annahme von rechteckigen Querschnitten mit Breite $b_i$ und Höhe $h_i$ und der Annhame, dass alle Elektroden näherungsweise gleich Breit sind, also $b_i = b$ gilt, folgt für die Belastung einer Einzelschicht durch das Moment $M_i$:
\begin{align}
    M_{b} &= M_i \sum_{k}^{n}\frac{E_k I_{yy,k}}{E_i I_{yy,i}}\\
    M_{b} &= M_i \frac{\sum_{k}^{n} E_k h_k^3}{E_i h_i^3}\\
    M_i &= M_{b} \frac{ E_i h_i^3} { \sum_{k}^{n}E_k h_k^3}
\end{align}
Durch einsetzen Einzelschichtmomente in die Formel zur Bestimmung der Biegespannung erhält man einen Zusammenhang zwischen Einzelschichtspannung und Biegemomentenbelastung:
\begin{align}
    \sigma_{b,i} &= \frac{M_y,i}{I_{yy}/h_i} \\
    \sigma_{b,i} &= 12 \frac{ M_y,i}{b h_i^2}\\
    \sigma_{b,i} &= 12 \frac{M_{b} E_i h_i^3}{b h_i^2 \sum_{k}^{n}E_k h_k^3}\\
    \sigma_{b,i} &= 12 \frac{M_{b} E_i h_i}{b \sum_{k}^{n}E_k h_k^3}
\end{align}

Für die Bestimmung der Durchbiegung $u$ beim 3-Punkt-Biegeversuch kann 
unter der Annahme der Bernoulli-Hypothese der exakte Ausdruck für die Krümmung linearisiert werden:
\begin{equation}
\frac{\frac{\partial^2 u(x)}{\partial x^2}}{\left(1 + \left(\frac{\partial u(x)}{\partial x} \right)^2 \right)^{3/2}} = -\frac{M_y}{E I_{yy}}
\end{equation}
Diese Gleichung kann für kleine Verformungen ($(\frac{\partial u(x)}{\partial x})^2 \ll 1$) durch die folgende Näherung ersetzt werden.
\begin{equation}
    \frac{\partial^2 u(x)}{\partial x^2} \approx -\frac{M_y(x)}{E I_{yy}}
\end{equation}

Unter der Annhame kleiner Verformung und konstantem Querschnitt und Steifigkeit lässt sich die Durchbiegung infolge der Kraft F durch folgende Gleichung annähern.
\begin{align}
    u_\text{flüssig} (x) &= \frac{F L^3}{48 \sum_{k}^{n} E_k I_{yy,k}} \left[ 3 \frac{x}{L} - 4\left(\frac{x}{L}\right)^3 \right] \text{für} \; 0 \leq x \leq L/2 \\
    u_\text{max,flüssig} (x = L/2) &= \frac{FL^3}{48 \sum_{k}^{n} E_k I_{yy,k}} \label{eq:bending_sbe_0}
\end{align}



An dieser Stelle ist zu bemerken, dass für den Spezialfall, in dem alle $n$ Schichten gleich dick sind und aus dem gleichen Material bestehen, die Spannung sich wie folgend ergibt.
\begin{equation}
    \sigma = \sigma_i = \frac{12 M_{b}}{n b h^2},
\end{equation}
Für diesen Spezialfall ergibt sich die maximale Durchbiegung $u_\text{max,flüssig} $ als
\begin{equation}
    u_\text{max,flüssig}  = \frac{L^3 Q}{4 n b h^3 E} = \frac{L^2 \sigma}{6 h E}.
\end{equation}


Da die bisherigen Annäherungen vor allem bei konventionellen Batterien mit nichttragenden Elektrolytschichten eine bessere Gültigkeit besitzen ist auch über den Fall zu sprechen wo die Schichten durch das Elektrolyt fest verbunden sind, wie es vorrangig bei Strukturelektrolyten der Fall ist. Die resultierende Laminatstruktur kann beliebig viele Lagen mit unterschiedlichen Orientierungen aufweisen und wird durch die klassische Laminattheorie (CLT) beschrieben~\cite{Carlstedt2018}. Dabei wird angenommen, dass das Material linear-elastisch ist und kleine Durchbiegungen auftreten.


In der CLT wird die Dehnung über der Dicke $z$ des Laminats beschrieben durch
\begin{equation}
\boldsymbol{\varepsilon}(z) = \boldsymbol{\varepsilon}^0 + z\,\boldsymbol{\kappa},
\end{equation}
wobei $\boldsymbol{\varepsilon}^0$ die Dehnungen in der Mittelfläche und $\boldsymbol{\kappa}$ die Krümmungen sind. Die Beziehung zwischen den Schnittgrößen (Kräfte und Momente pro Breite) und den Dehnungen/Krümmungen ergibt sich aus der ABD-Steifigkeitsmatrix:
\begin{equation}
\begin{pmatrix}
\mathbf{N} \\
\mathbf{M}
\end{pmatrix}
=
\begin{pmatrix}
\mathbf{A} & \mathbf{B} \\
\mathbf{B} & \mathbf{D}
\end{pmatrix}
\begin{pmatrix}
\boldsymbol{\varepsilon}^0 \\
\boldsymbol{\kappa}
\end{pmatrix},
\end{equation}
wobei $\mathbf{A}$ die Membransteifigkeit, $\mathbf{D}$ die Biegesteifigkeit und  $\mathbf{B}$ die Kopplung zwischen Membran- und Biegebeanspruchung des Laminats beschreibt.

Für reine Biegung, in der keine Normalkräfte wirken, gilt $\mathbf{N} = \mathbf{0}$. Daraus folgt
\begin{equation}
\mathbf{A}\boldsymbol{\varepsilon}^0 + \mathbf{B}\boldsymbol{\kappa} = \mathbf{0}
\quad\Rightarrow\quad
\boldsymbol{\varepsilon}^0 = -\mathbf{A}^{-1} \mathbf{B} \boldsymbol{\kappa}.
\end{equation}
Durch Einsetzen in die Momentengleichung ergibt
\begin{equation}
\mathbf{M} = \mathbf{B} \boldsymbol{\varepsilon}^0 + \mathbf{D} \boldsymbol{\kappa} = 
\left( \mathbf{D} - \mathbf{B} \mathbf{A}^{-1} \mathbf{B} \right) \boldsymbol{\kappa}.
\end{equation}

Die Größe
\begin{equation}
\mathbf{D}^* = \mathbf{D} - \mathbf{B} \mathbf{A}^{-1} \mathbf{B}.
\end{equation}
wird als effektive Biegesteifigkeit eines unsymmetrischen Laminats bezeichnet~\cite{Jones2018}. 
Damit ergibt sich der Zusammenhang
\begin{equation}
\boldsymbol{\kappa} = (\mathbf{D}^*)^{-1} \mathbf{M}.
\end{equation}

Für reine Biegung um die $x$-Achse gilt $M_x \ne 0$, reduziert sich das Gleichungssystem zu
\begin{equation}
\kappa_x = \frac{M_x}{D^*_{11}}.
\end{equation}

Unter einer mittigen Krafteinwirkung $F$ auf einen Balken ergibt sich ein Momentenverlauf von
\begin{equation}
M_x(x) = \frac{F}{2}x \quad\text{für } 0 \le x \le \frac{L}{2}.
\end{equation}

Somit ergibt sich die Krümmung zu:
\begin{equation}
\kappa_x(x) = \frac{F x}{2 D^*_{11}}.
\end{equation}

Nach zweimaliger Integration gemäß der Euler-Bernoulli-Theorie entsteht die maximale Durchbiegung in der Balkenmitte zu
\begin{equation}\label{eq:bending_sbe_100}
u_\text{max,fest} = \frac{F L^3}{48 D^*_{11}}, \quad
D^*_{11} = D_{11} - [\mathbf{B} \mathbf{A}^{-1} \mathbf{B}]_{11}.
\end{equation}

Für einen symmetrischen Laminataufbau ($\mathbf{B} = \mathbf{0}$) vereinfacht sich $D^*_{11} = D_{11}$, und die Durchbiegung wird:
\begin{equation}
u_\text{max,fest} = \frac{F L^3}{48 D_{11}}.
\end{equation}

Die Dehnung in $x$-Richtung in einer Schicht bei Höhe $z$ ergibt sich zu
\begin{equation}
\varepsilon_x(z) = \varepsilon_x^0 + z\,\kappa_x.
\end{equation}

\begin{figure}[!ht]
	%\raggedleft
		%\def\svgwidth{\columnwidth}
        \center
		\includegraphics[width=0.99\textwidth, angle=0]{bending_pre_tests.pdf}
		\caption{\label{fig:bending_pre_tests}Zur Entwicklung einer vereinfachten Bestimmung des Verformungsverhaltens wurden \textbf{a)} Probekörper aus 11 Schichten einer $100~\mu m$ dicken Aluminiumfolie hergestellt, welche \textbf{b)} flächig mit Verschiedenen Anteilen an PVDF und Luft zusammengefügt wurden. Die \textbf{c-d)} 3-Punkt-Biegeversuche \textbf{e)} wurden mit den äquivalenten Kraft-Verschiebungskurven der entwickelten Methode anschließend verglichen.}
\end{figure}

Zur Bestimmung der Schichtspannungen wird diese globale Dehnung in die lokale Koordinatenrichtung jeder Schicht transformiert\footnote{je nach Faserwinkel $\theta_i$} und mittels Anwendung des Materialgesetzes bestimmt
\begin{equation}
\boldsymbol{\sigma}_i = \mathbf{Q}^{(i)}\, \boldsymbol{\varepsilon}_i.
\end{equation}

Dabei ist $\mathbf{Q}^{(i)}$ die Steifigkeitsmatrix der $i$-ten Schicht im lokalen Koordinatensystem. Die Spannungsverteilung ist innerhalb jeder Schicht linear, aber mit unterschiedlichen Steigungen, jedoch abhängig von Materialparametern und Faserorientierung.

Um die Gültigkeit der hergeleiteten Grenzfälle zu verifizieren und das reale Verhalten zwischen rein flüssigen (\ref{eq:bending_sbe_0}) und festen (\ref{eq:bending_sbe_100}) Elektrolytsystemen zu quantifizieren, wurden experimentelle Untersuchungen durchgeführt. Diese Versuche entstanden in einer Kooperation zwischen dem Fraunhofer IWS und der TU Dresden im Rahmen des Forschungsprojektes ElViS. Ziel war es, die mechanische Kopplungsgüte der Laminatschichten in Abhängigkeit der Elektrolytbeschaffenheit zu isolieren.

Als Probekörper dienten Stapel aus elf Aluminiumfolien mit einer Einzeldicke von 0,1\,mm einer Geometrie von 80\,mm $\times$ 10\,mm (Bild~\ref{fig:bending_pre_tests}a). Die Wahl von Aluminium als Substrat begründet sich durch seine Rolle als klassisches Elektrodenmaterial; zudem findet es in der hier betrachteten Strukturbatterie direkt als Trägermaterial der Kathode Verwendung.

Zunächst wurde der Einfluss der rein mechanischen Oberflächenreibung untersucht. Ein Vergleich zwischen glatten und künstlich angerauten Aluminiumfolien, die ohne Bindemittel aufeinandergelegt wurden, zeigte keine signifikanten Unterschiede im Kraft-Weg-Verlauf (Bild~\ref{fig:bending_pre_tests}e). Dies belegt, dass die Oberflächenrauheit der Stromsammler für das globale Biegeverhalten vernachlässigbar ist und eine nennenswerte Lastübertragung erst durch eine stoffschlüssige Verbindung realisiert wird.

Der Übergang von der idealen Gleitschicht (flüssiger Elektrolyt) zum schubstarren Verbund (Festkörperelektrolyt) wird maßgeblich durch die Morphologie der porösen PVDF-Matrix bestimmt. In den durchgeführten 3-Punkt-Biegeversuchen (vgl. Bild~\ref{fig:bending_pre_tests}c-d) zeigte sich, dass die Probekörper ein mechanisches Verhalten aufweisen, das zwischen den theoretischen Grenzwerten der freien Verschieblichkeit (\ref{eq:bending_sbe_0}) und des idealen Verbunds (\ref{eq:bending_sbe_100}) skaliert.

Um diesen intermediären Zustand für die Vorauslegung mathematisch zugänglich zu machen, wird die effektive Durchbiegung $u_{\text{max}}$ als gewichtete Überlagerung der Grenzfälle modelliert. Als Gewichtungsfaktor dient der Phasenvolumenanteil $\psi$ der flüssigen, mechanisch passiven Phase:

\begin{equation}
    u_{\text{max}} = \psi \cdot u_{\text{max,flüssig}} + (1-\psi) \cdot u_{\text{max,fest}}.
\end{equation}

In dieser Formulierung fungiert $\psi$ als physikalisch interpretierbarer Kopplungsparameter. Ein Wert von $\psi = 1$ repräsentiert das Verhalten einer konventionellen Batterie mit rein flüssigem Elektrolyten, während $\psi \to 0$ den idealen schubfesten Verbund einer monolithischen Strukturbatterie beschreibt. Dieser analytische Mischungsansatz ermöglicht es, die mechanische Wirksamkeit des infiltrierten Strukturelektrolyten ohne numerische Kontaktmodellierung direkt in die Dimensionierung der Hochleistungsstruktur (TZ 4) einzubeziehen.