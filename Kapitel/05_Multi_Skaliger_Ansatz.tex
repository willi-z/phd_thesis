\chapter{Mehrskalige Berechnungsmethodik für Strukturbatterien}

Nachdem die theoretischen Grundlagen sowie die Homogenisierung der Struktelektrolyte mittels repräsentativer Volumenelemente bereits etabliert wurden, erfolgt nun die Übertragung dieser Konzepte auf das Gesamtsystem der Strukturbatterie. In der ingenieurwissenschaftlichen Praxis ergibt sich dabei oft die Herausforderung, dass die physikalische Genauigkeit hochdetaillierter Modelle durch einen sehr hohen Bedarf an Material- und Interaktionsparametern erkauft wird. Wenn der Aufwand zur Parameteridentifikation den Zeitrahmen der eigentlichen Bauteilauslegung überschreitet, verliert die Simulation ihren praktischen Nutzen als effizientes Entwicklungswerkzeug.

Für eine anwendungsorientierte Simulation ist es daher entscheidend, den Fokus auf die maßgeblichen physikalischen Effekte zu legen und weniger signifikante Kopplungen gezielt zu vernachlässigen. Eine Modellierungsstrategie ist nur dann effektiv, wenn sie mit einer überschaubaren Anzahl an experimentell zugänglichen Parametern auskommt und gleichzeitig eine hohe Recheneffizienz bietet. Dies erlaubt es, die numerische Analyse nicht nur als punktuelle Validierung, sondern als proaktives Werkzeug zur schnellen Bewertung zahlreicher Materialkombinationen einzusetzen.\footnote{Die Reduktion der Modellkomplexität verhindert, dass die Parameteridentifikation zum eigentlichen Flaschenhals im Entwicklungsprozess von multifunktionalen Leichtbaustrukturen wird.}

Die nachfolgende Methodik nutzt die zuvor definierten RVE-Strukturen, um das komplexe Zusammenspiel von Mechanik und Elektrochemie auf Systemebene abzubilden. Dabei steht zunächst die gezielte Entkopplung von Feldgrößen im Vordergrund, die unter realen Lastbedingungen nur geringe Wechselwirkungen zeigen. Zur ganzheitlichen Bewertung der Einsatzfähigkeit wird darüber hinaus ein multifunktionales Versagenskriterium vorgestellt. Dieses ermöglicht eine direkte Verknüpfung des mechanischen Strukturversagens mit dem daraus resultierenden Funktionsverlust, um eine effiziente und praxisnahe Sicherheitsbewertung der Strukturbatterie zu gewährleisten.

\section{\label{sec:reduction_and_parallelization}Entkopplung und Parallelisierungsstrategie zur Reduktion des Simulationsaufwandes}

Die Umsetzung einer anwendungsorientierten Simulation erfordert eine zweistufige Reduktionsstrategie, um sowohl die Anzahl der benötigten Materialparameter als auch die notwendigen Rechenressourcen zu minimieren. Zunächst werden die physikalischen Wirkzusammenhänge auf die maßgeblichen Kopplungen beschränkt, wodurch das Modell parameterarm bleibt und die experimentelle Identifikationslast gesenkt wird. Darauf aufbauend ermöglicht eine gezielte geometrische Abstraktion unter Ausnutzung von Symmetrien, den Berechnungsraum signifikant zu verkleinern, um eine hohe Recheneffizienz für iterative Optimierungsprozesse zu erzielen. Dieser kombinierte Ansatz stellt sicher, dass die Modellierung trotz der inhärenten Multiphysik ein handhabbares Werkzeug für die Bauteilauslegung bleibt.
\begin{figure}[!ht]
    \center
    \includegraphics[width=0.99\textwidth, angle=0]{simulation_model.pdf}
    \caption{\label{fig:homogenisation}Mehrskalige Homogenisierung der Strukturbatterie durch Abstraktion der Geometrie.}
\end{figure}
\subsection{Physikalische Domänenentkopplung und Parameterreduktion}
Strukturbatterien neigen aufgrund des im Vergleich zu Graphit signifikant geringeren effektiven Diffusionskoeffizienten $D$ [$\si{\metre\squared\per\second}$] der Kohlenstofffasern dazu, bei schnellen Lade- und Entladezyklen zunehmend ineffizient zu werden~\cite{Uchida1996,Kim2021}. Dies bedingt vergleichsweise langsame elektrochemische Prozesse, wodurch temperatureinflussabhängige Effekte an Bedeutung verlieren und in erster Näherung vernachlässigt werden können~\cite{Carlstedt2019a}. Ebenfalls einen geringen Einfluss auf das Gesamtsystem zeigen die meisten Interaktionskoeffizienten zwischen den physikalischen Domänen, welche oft mehrere Größenordnungen unter den Hauptkopplungen liegen.\footnote{Beispielsweise liegt der Einfluss der Lithium-Interkalation auf den Elastizitätsmodul $E$ unter 1\,\%~\cite{Carlstedt2019}.}

Infolge dieser Reduktion lässt sich die Gesamtverformung der Strukturbatterie als Summe zweier getrennter Beiträge beschreiben: der rein mechanisch verursachten Ausdehnung und der elektrochemisch bedingten Volumenänderung. Während diese physikalischen Entkopplungen allgemeingültig für verschiedene Belastungszustände sind, werden für die hier betrachteten Belastungsfälle der Zug- und 3-Punkt-Biegebelastung zusätzliche Annahmen zur Steigerung der Recheneffizienz getroffen.

Die Folien-Pouchbauform wird dahingehend abstrahiert, dass die Siegelstellen des Batteriestacks vernachlässigt werden, siehe Bild~\ref{fig:homogenisation}. Da die ionische Diffusion im Elektrolyten im Vergleich zu den mechanisch induzierten Längenänderungen sehr schnell erfolgt, wird die Kopplung zwischen Diffusion und geometrischer Deformation entkoppelt. Mit dem Schwerpunkt dieser Arbeit auf der strukturellen Batteriefunktionalität und sensorische Effekte laut \textsc{Carlstedt} lediglich vernachlässigbare kleine Ströme ausbilden~\cite{Carlstedt2023}, wird die Rückwirkung mechanischer Spannungen $\boldsymbol{\sigma}$ [$\si{\pascal}$] auf die Elektrochemie entkoppelt. Diese Vereinfachung trägt maßgeblich zu einem parameterarmen Modell bei, da auf die Identifikation komplexer piezoresistiver oder elektromechanischer Interaktionskoeffizienten verzichtet werden kann.

\subsection{Reduziertes elektrochemisches Ersatzmodell}
\begin{figure}[!ht]
    \center
    \includegraphics[width=0.65\textwidth, angle=0]{simulation_electro_chem.pdf}
    \caption{\label{fig:simulation_electro_chem}Reduzierung des Berechnugnsaufwandes der elektro-chemischen Simulation durch Verwenung eines Single-Particle-Models.}
\end{figure}

Analog hierzu wird für die elektrochemische Simulation ein Single-Partikel-Modell verwendet~\cite{Moura2017}. Jeder Zellstapel wird als Kombination reduzierter Ersatzschaltungen repräsentiert, siehe Bild~\ref{fig:simulation_electro_chem}. Als Basis dienen die Transportgleichungen innerhalb der Domänen $\Omega$ für Anode ($-$), Kathode ($+$) und Separator ($\text{sep}$):
\begin{equation}
    \frac{\partial c_{e,j}}{\partial t} = \frac{\partial}{\partial x}\left[\frac{D_{\text{e}}^{\text{eff}}(c_{e,j})}{\psi_{e,j}} \frac{\partial c_{e,j}}{\partial x} (x,t)\right]-\text{sign}(j)\frac{1-t_+^0}{\psi_{e,j} F_{\text{K}} L_j} I(t).
\end{equation}
mit $j \in \{-, \text{sep}, +\}$ und der Richtungsfunktion
\begin{equation}
    \text{sign}(j) = \left\{
        \begin{array}{ll}
            -1 & j = - \\
            0 & j = \text{sep} \\
            1 & j = +
        \end{array}
    \right. .
\end{equation}
Hierbei bezeichnen $\psi_{e,j}$ [-] den Volumenanteil der Elektrolytphase, $t_+^0$ [-] die Hittorfsche Überführungszahl, $F_{\text{K}}$ [$\si{\coulomb\per\mole}$] die Faraday-Konstante und $L_j$ [$\si{\metre}$] die Schichtdicke. Die Kopplung erfolgt über die Randbedingungen an den Grenzflächen:
\begin{align}
\frac{\partial c_{\text{e},-}}{\partial x} (0_-,t) &= \frac{\partial c_{\text{e},+}}{\partial x} (0_+,t) = 0\\
D_{\text{e,eff,-}}\frac{\partial c_{e,-}}{\partial x} (L_-,t) &= D_{\text{e,eff,sep}} \frac{\partial c_{\text{e,sep}}}{\partial x} (0_{\text{sep}},t)\\
D_{\text{e,eff,sep}}\frac{\partial c_{\text{e,sep}}}{\partial x} (L_{\text{sep}},t) &= D_{\text{e,eff,+}} \frac{\partial c_{e,+}}{\partial x} (L_{\text{+}},t)\\
c_e(L_{\text{-}},t) &= c_e(0_{\text{sep}}, t)\\
c_e(L_{\text{sep}},t) &= c_e(L_{\text{+}, t})
\end{align}
Die effektiven Eigenschaften werden konsistent zum mechanischen Teil durch die Homogenisierung der Mikrostruktur bestimmt, wodurch der numerische Aufwand bei Erhalt der maßgeblichen physikalischen Genauigkeit minimiert wird.

\subsection{Lastfallspezifische geometrische Abstraktion}
\begin{figure}[!ht]
    \center
    \includegraphics[width=0.7\textwidth, angle=0]{bending.pdf}
    \caption{\label{fig:bending_electroylte_tests}Validierung des 3-Punkt-Biegefalls durch: a) visuellen Vergleich von Simulation und Experiment und b) Kraftverlauf in Abhängigkeit der Durchbiegung für Pouchzellen mit und ohne Elektrolyt.\protect\footnote{Die experimentellen Untersuchungen wurden von den Projektpartnern im Rahmen des ElViS-Projekts am Institut für Leichtbau und Kunststofftechnik (ILK) der TU Dresden sowie am Fraunhofer IWS Dresden durchgeführt.}}
\end{figure}
Für die Untersuchung der mechanischen Integrität unter Zug- und Biegebelastung wird der Berechnungsraum unter Ausnutzung von Symmetrien radikal reduziert. In der 3-Punkt-Biegung ermöglicht die Identität der Spannungszustände quer zur Biegelinie die Abstraktion auf einen repräsentativen zweidimensionalen Streifen. Diese geometrische Vereinfachung reduziert die Anzahl der finiten Elemente drastisch, ohne die Vorhersagegüte für die globale Kraft-Durchbiegungs-Charakteristik zu mindern. Die Einzelschichten des Stapels werden durch RVE als Blockelemente beschrieben. Nichtlineare Eigenschaften dieser RVE können an diskreten Stützstellen ausgewertet und mittels Interpolation ohne nennenswerten Rechenaufwand über den gesamten Bereich rekonstruiert werden. Vergleichende Studien zeigten hierbei eine gute Übereinstimmung mit experimentellen Daten\footnote{Die Versuchsreihe wurde im Rahmen des ElViS-Projektes durchgeführt. Dabei erfolgte die Zellherstellung durch das Fraunhofer IWS, die Biegeversuche wurden am Institut für Leichtbau und Kunststofftechnik (ILK) der TU Dresden durchgeführt.}, siehe Bild~\ref{fig:bending_electroylte_tests}.

\section{\label{sec:automated_failure}Analyse des Versagens der Multifunktionalität und Kritikalitätsbewertung}

Auf Basis der entwickelten Methodik kann der lokale mechanische Spannungszustand $\boldsymbol{\sigma}$ [$\si{\pascal}$] für jede funktionale Schicht der Strukturbatterie bestimmt werden:
\begin{equation}
\boldsymbol{\sigma} =
\begin{bmatrix}
\sigma_{xx} & \tau_{xy} & \tau_{xz} \\
\tau_{xy}   & \sigma_{yy} & \tau_{yz} \\
\tau_{xz}   & \tau_{yz}   & \sigma_{zz}
\end{bmatrix}.
\end{equation}
Anstatt einer rein strukturellen Betrachtung wird eine Analyse des Versagens der Multiunktionalität durchgeführt. Ziel ist es, kritische Zustände zu identifizieren, die über das mechanische Bauteilversagen hinaus einen unmittelbaren Funktionsverlust der elektrochemischen Domäne oder sicherheitstechnische Havarien auslösen.

\subsection{Versagenskriterien und plastisches Fließverhalten der Einzelschichten}

Für isotrope Komponenten, wie die metallischen Stromableiter oder die Polymerhülle, wird das von-Mises-Kriterium~\cite{Hill1998} angewendet. Die Vergleichsspannung $\sigma_\mathrm{v}$ [$\si{\pascal}$] berechnet sich zu:
\begin{equation}
\sigma_\mathrm{v} = \sqrt{\frac{1}{2}
\left [
(\sigma_{xx}-\sigma_{yy})^2 +
(\sigma_{yy}-\sigma_{zz})^2 +
(\sigma_{zz}-\sigma_{xx})^2
\right ]
+ 3\left ( \tau_{xy}^2 + \tau_{yz}^2 + \tau_{xz}^2 \right )}.
\end{equation}
Zur realistischen Abbildung der metallischen Ableiter wird ein isotropes Verfestigungsmodell genutzt, bei dem die Fließspannung $\sigma_\mathrm{y}$ [$\si{\pascal}$] mit zunehmender äquivalenter plastischer Dehnung $\bar{\varepsilon}^p$ [-] ansteigt:
\begin{equation}
\sigma_\mathrm{y} = \sigma_{\mathrm{y}0} + H \, \bar{\varepsilon}^p .
\end{equation}
Hierbei ergibt sich die äquivalente plastische Dehnung inkrementell aus dem plastischen Dehnungstensor $\boldsymbol{\varepsilon}^p$:
\begin{equation}
\Delta \bar{\varepsilon}^p = \sqrt{\frac{2}{3} \, \Delta \boldsymbol{\varepsilon}^p : \Delta \boldsymbol{\varepsilon}^p}, \quad \bar{\varepsilon}^{p}_{n+1} = \bar{\varepsilon}^{p}_{n} + \Delta \bar{\varepsilon}^{p}.
\end{equation}
Die Fließbedingung $f \le 0$ definiert den elastischen Bereich. Bei Überschreitung ($f > 0$) wird die plastische Dehnungsrate über das assoziative Fließgesetz bestimmt:
\begin{equation}
f = \sigma_\mathrm{v} - \sigma_\mathrm{y} \le 0, \quad \dot{\boldsymbol{\varepsilon}}^p = \dot{\lambda} \frac{\partial f}{\partial \boldsymbol{\sigma}} = \dot{\lambda} \frac{3}{2} \frac{\boldsymbol{s}}{\sigma_\mathrm{v}},
\end{equation}
wobei $\boldsymbol{s} = \boldsymbol{\sigma} - \frac{1}{3}\text{tr}(\boldsymbol{\sigma}) \boldsymbol{I}$ den Deviatorspannungstensor und $\dot{\lambda}$ den plastischen Multiplikator darstellt.

Für die anisotropen Verstärkungslagen wird das \textsc{Tsai}-\textsc{Wu}-Kriterium~\cite{Tsai1971} herangezogen:
\begin{equation}
F_1 \sigma_1 + F_2 \sigma_2 + F_{11} \sigma_1^2 + F_{22} \sigma_2^2 + 2F_{12} \sigma_1 \sigma_2 + F_{66} \tau_{12}^2 \geq 1,
\end{equation}
wobei $\sigma_1$ und $\sigma_2$ die Normalspannungen in Faserrichtung und quer zur Faserrichtung sowie $\tau_{12}$ die Schubspannung in der Materialebene darstellen. Die Koeffizienten $F_i$ und $F_{ij}$ werden aus den Zug- und Druckfestigkeiten in Faserrichtung ($R^\text{z}_{||}$, $R^\text{d}_{||}$) sowie quer zur Faser ($R^\text{z}_{\bot}$, $R^\text{d}_{\bot}$) und der Schubfestigkeit ($R_{||\bot}$) bestimmt:
\begin{align}
F_1 &= \frac{1}{R^\text{z}_{||}} - \frac{1}{R^\text{d}_{||}}, &
F_{11} &= \frac{1}{R^\text{z}_{||} R^\text{d}_{||}}, \\
F_2 &= \frac{1}{R^\text{z}_{\bot}} - \frac{1}{R^\text{d}_{\bot}}, &
F_{22} &= \frac{1}{R^\text{z}_{\bot} R^\text{d}_{\bot}}, \\
F_{66} &= \frac{1}{R_{||\bot}^2}, &
F_{12} &\approx -\frac{1}{2}\sqrt{F_{11} F_{22}}.
\end{align}

\subsection{Multifunktionale Kritikalitätsbewertung}

Das Kernstück der hier vorgestellten Analyse bildet die Bewertung des Versagens der Multifunktionalität. Im Gegensatz zu konventionellen Strukturen, bei denen lokales Materialversagen primär die Resttragfähigkeit mindert, bedingt ein Defekt in Strukturbatterien stets eine Beeinträchtigung der elektrochemischen Domäne. Um diese Interdependenzen systematisch zu erfassen, wurde im Rahmen dieser Arbeit ein Bewertungsschema entwickelt, das mechanische Schädigungszustände direkt mit funktionalen Konsequenzen korreliert. Die daraus resultierende Kritikalitätsmatrix (siehe Tab.~\ref{tab:failure_modes}) ordnet den Einzelschichten spezifische Gefährdungspotenziale zu.

\begin{table}[ht]
    \centering
    \caption{\label{tab:failure_modes}Überischt des mit Versagens der Einzelschichten verknüpften Sichheitsrisikos.}
    \begin{tabularx}{\textwidth}{lXXX}
    \toprule
    &\makecell{Pouchfolienversagen\\\includegraphics[width=0.2\textwidth]{failure_modes/failure_mode_pouch.png}}
    &\makecell{Elektrodenversagen\\\includegraphics[width=0.2\textwidth]{failure_modes/failure_mode_electrode.png}}
    &\makecell{Separatorversagen\\\includegraphics[width=0.2\textwidth]{failure_modes/failure_mode_separator.png}}
    \\
    \midrule
    Funktion
        & Funktionsversagen der gesamten Batterie durch austrocknen
        & Leistungsverlust der Zelle
        & Funktionsversagen der Zelle, je nach Verschaltung auch der gesamten Batterie
    \\
    Brandgefahr
        & kein Risiko
        & kein Risiko
        & Flammenbildung durch Überhitzung
    \\
    Gesundheit
        & hohes Risiko durch austretendes Elektrolyt
        & kein Risiko
        & kein zusätzliches Risiko
    \\
    \bottomrule
    \end{tabularx}\\
    %\noindent{\footnotesize{\textsuperscript{*} Gemessen gegenüber \ce{Li}/\ce{Li+}.}}
\end{table}%

Ein Bruch der Pouchfolie führt zur Leckage des Elektrolyten, was durch das Austrocknen der Zelle einen unmittelbaren Funktionsstopp nach sich zieht. Im Gegensatz dazu wird eine Schädigung der Elektrodenlagen primär als Degradation der Kapazität eingestuft. Als kritischster Zustand innerhalb des entwickelten Schemas ist das strukturelle Versagen des Separators definiert: Eine mechanische Perforation ermöglicht den direkten Kontakt der Elektroden, was einen internen Kurzschluss induziert und im Extremfall ein thermisches Durchgehen\footnote{Thermal Runaway} auslösen kann. Diese methodische Einordnung ermöglicht ein sicherheitsorientiertes Design, bei dem kritische Funktionsschichten gezielt in mechanisch neutrale Zonen der Gesamtstruktur verschoben werden können.

\section{\label{sec:validation}Validierung des skalenübergreifenden Modelles}

Die Validierung des hier vorgestellten mehrskaligen Berechnungsansatzes erfolgt durch den direkten Vergleich mit experimentellen Daten eines Schichtverbunds, der in einer Stapelanordnung vier funktionale Einzelzellen realisiert. Die Herstellung und elektrochemische Charakterisierungder Testzellen erfolgte am Fraunhofer-Institut für Werkstoff- und Strahltechnik (IWS) in Dresden, während die mechanische Untersuchung am Institut für Leichtbau und Kunststofftechnik (ILK) der TU Dresden durchgeführt wurde. Um die Vorhersagegüte der Methodik zu prüfen, wird ein konventionelles Referenzsystem der neu entwickelten Strukturbatterie gegenübergestellt. Die spezifischen Materialzusammensetzungen beider Systeme, extrahiert aus den experimentellen Datensätzen, sind in Tabelle~\ref{tab:material_comp} zusammengefasst.
\begin{table}[ht]
    \centering
    \caption{\label{tab:material_comp}Materialkomposition der Referenz- und Strukturbatterie.}
    \begin{tabularx}{\textwidth}{lXX}
    \toprule
    \textbf{Komponente} & \textbf{Referenzsystem} & \textbf{Strukturbatterie} \\
    \midrule
    Anode & Graphit auf Kupferfolie (\SI{14}{\micro\metre}) & Hardcarbon auf PX-35 CF-Gelege inkl. Cu-Primer \\
    \addlinespace
    Kathode & NMC622 auf Aluminium (\SI{15}{\micro\metre}) & NMC622 auf Aluminium (\SI{15}{\micro\metre}) \\
    \addlinespace
    Separator & Celgard 2400 (PP, \SI{25}{\micro\metre}) & Celgard 2400 (PP, \SI{25}{\micro\metre}) \\
    \addlinespace
    Elektrolyt & LP30 (flüssig) & LP30 + \SI{10}{\%} KYNAR FLEX 28 (SBE) \\
    \addlinespace
    Gehäuse & Standard Pouchfolie (\SI{111}{\micro\metre}) & Standard Pouchfolie (\SI{111}{\micro\metre}) \\
    \bottomrule
    \end{tabularx}
\end{table}

Das Referenzsystem repräsentiert den Stand der Technik für Hochenergie-Lithium-Ionen-Zellen. Bei der Strukturbatterie kommt hingegen ein multifunktionaler Ansatz zum Einsatz: Das PX-35 Kohlenstofffasergelege dient gleichzeitig als Stromableiter und mechanisches Verstärkungselement. Die Hardcarbon-Beschichtung der Anode ist speziell auf die Interkalation in Kohlenstofffasern optimiert. Zur Steigerung der mechanischen Integrität (Schubübertragung) wurde der flüssige Elektrolyt durch die Zugabe von KYNAR FLEX 28\footnote{PVDF-HFP-Copolymer} zu einem strukturellen Gelelektrolyten modifiziert.

Die experimentelle Charakterisierung erfolgte durch eine Kombination aus elektrochemischen Zyklierungsversuchen und 3-Punkt-Biegeversuchen. Die elektrochemische Charakteristik wurde mittels galvanostatischer Entladung bestimmt, wobei die Stromraten stufenweise von $C/100$ bis $2C$ variiert wurden, um die Ratenfähigkeit über 35 Zyklen zu erfassen. Anschließend dazu wurde die mechanische Tragfähigkeit im 3-Punkt-Biegeversuch bei konstanter Traversengeschwindigkeit ermittelt. 

\begin{figure}[!ht]
    \center
    \includegraphics[width=0.8\textwidth, angle=0]{electrical_sim_final.pdf}
    \caption{\label{fig:electrical_sim_final}Vergleich der experimentellen und simulativen Energiedichte über 40 Zyklen bei variierenden Entladeraten.}
\end{figure}

In Bild~\ref{fig:electrical_sim_final} wird die elektrochemische Leistungsfähigkeit anhand der gravimetrischen Energiedichte dargestellt. Während die Referenzzelle eine Energiedichte von ca. \SI{75,0}{Wh/kg} erreicht, weist die Strukturbatterie aufgrund der passiven Masse der Verstärkungskomponenten etwa \SI{43}{Wh/kg} auf. Die simulative Vorhersage zeigt eine exzellente Übereinstimmung mit den experimentellen Daten über das gesamte Entladespektrum. Besonders hervorzuheben ist die präzise Abbildung der Ratenfähigkeit, die durch die modellbasierte Berücksichtigung der diffusionsbedingten Limitierungen in den Kohlenstofffasern ermöglicht wird. Diese Ergebnisse bestätigen die Validität des mehrskaligen Ansatzes zur Beschreibung der elektrochemischen Dynamik in Strukturbatterien.

\begin{figure}[!ht]
    \center
    \includegraphics[width=0.9\textwidth, angle=0]{mech_sim_final.pdf}
    \caption{\label{fig:mech_sim_final}Kraft-Durchbiegungs-Diagramm der experimentellen Validierung und der korrigierten Simulation.}
\end{figure}

Die mechanische Validierung im 3-Punkt-Biegeversuch (Bild~\ref{fig:mech_sim_final}) verdeutlicht die signifikante Steigerung der Kraftaufnahme durch den Kohlefaserverbund auf \SI{9,6}{N}. Auffallend ist hierbei, dass eine rein theoretische Vorhersage allein auf Basis von Material-Einzelwerten bei der Strukturbatterie zu deutlichen Abweichungen führt. Im Gegensatz zum Referenzsystem, dessen Herstellungsprozesse am Fraunhofer IWS bereits langjährig erprobt und dessen Materialparameter präzise charakterisiert sind, unterliegt die neuartige Strukturbatterie größeren fertigungsbedingten Toleranzen. 

Ein deutliches Indiz für diese Hypothese ist der unerwartete Steifigkeitseinbruch bei einer der untersuchten Proben (siehe Bild~\ref{fig:mech_sim_final}), welcher auf stochastische Defekte in der Strukturbatterie hindeutet. Als weitere Ursachen für die Abweichungen zwischen Idealmodell und Experiment wurden Variationen in der Schichtdicke, lokale Infiltrationsunterschiede des versteiften Elektrolyten sowie eine unzureichende Grenzflächenhaftung des Hardcarbon-Slurrys auf dem CF-Gelege identifiziert\footnote{Diese Faktoren erwiesen sich bereits in Vorversuchen der Projektpartner am Fraunhofer IWS und dem ILK der TU Dresden als prozesskritisch.}. 

Um diese Effekte zu kompensieren und eine hochpräzise Abbildung des realen Systems zu ermöglichen, wurde die Simulation durch ein gezieltes Parameterfitting kalibriert. Diese „korrigierte Simulation“ nutzt die Messdaten der ersten Zyklen (elektrochemisch) sowie den linear-elastischen Bereich der ersten Prozent Dehnung (mechanisch) als Stützstellen. Dieser Abgleich ist notwendig, um die Diskrepanz zwischen idealisierter Modellwelt und realem Prototyp zu schließen und bildet die Grundlage für die anschließende Versagensanalyse.

\begin{figure}[!ht]
    \center
    \includegraphics[width=0.69\textwidth, angle=0]{plasticity.pdf}
    \caption{\label{fig:plasticity}Postmortale Analyse der plastischen Verformungszonen im Vergleich zur Simulation.}
\end{figure}

Die postmortale Analyse (Bild~\ref{fig:plasticity}) bestätigt die prognostizierten Deformationsmechanismen. Die Identifikation plastischer Zonen in den Einzelschichten korrespondiert mit den modellbasierten Vorhersagen und untermauert die Gültigkeit des verwendeten mehrskaligen Materialmodells.
