\chapter{\label{sec:sim_sbe}Bestimmung des effektiven Diffusions- und Steifigkeitsverhaltens von zweiphasigen Elektrolytsystemen}

Zweiphasige Elektrolyte besitzen eine komplexe Porenstruktur, die im Durchmesser meist zwischen 1 bis 400 $\si{\nano \metre}$ schwanken (Bild~\ref{fig:sphere_cylinder_model_RVE_generation}a). Bereits verwendete Ansätze, wie etwa Lösen der \textsc{Cahn-Hilliard}-Gleichung, mit anschließendem lösen der linear elastischen Verformung in der festen Domaine (Gleichungen~\ref{eq:stress_gov}, \ref{eq:stress_material}, \ref{eq:strain_total_displacement})
sowie die Simualtion des Diffusionsverhaltens nach \textsc{Fick}\footnote{Gilt unter den Annahmen, dass die Eigenschaften der Flüssigkeit innerhalb des Netzwerks einheitlich sind und das elektrische Feld das gesamte Medium durchdringt.}
\begin{equation}\label{eq:fick}
    J = -D \frac{\partial c}{\partial x}
\end{equation}
in der flüssigen Domain kommt hierbei an mehrer Grenzen. Für einen gute Repräsentation muss das ausgewählte Repräsentationselement ausreichend groß sein, um den Einfluss von lokalen Unterschieden auszugleichen. Durch die große Porenvarianz ist der Generierungs und Vernetzungsaufwand jedoch aufwendig und sorgt für lange Berechnungszeiten. 

\section{Modellierung des effektiven Leitfähigkeit als Porennetzwerk}
Ein urspünnglich aus der Geologie stammender Ansatz der eigenständig für zweiphasige Elektrolyte adaptiert wurde nutzt ein sogenanntes Pore-Netzwerk-Modell~\cite{Xiong2016,Gostick2016}. Dabei wird angenommen, dass kein Ionentransport im festen Polymernetzwerk stattfindet~\cite{Tu2020}.

\begin{figure}[!ht]
	%\raggedleft
		%\def\svgwidth{\columnwidth}
        \center
		\includegraphics[width=0.99\textwidth, angle=0]{sphere_cylinder_model.pdf}
		\caption{\label{fig:sphere_cylinder_model}Das Porennetzwerkmodel mit Kugelförmigen Poren und zylindrischen Verbindungselementen für beschleunigte Berechnungen der effktiven Transporteigenschaften von zweiphasigen Strukturelektrolyten.
        }
\end{figure}

Durch Multiplikation von Gleichung~\ref{eq:fick} mit dem Querschnitt $A$ ergibt sich die Massenflussrate $\dot{m}$.
\begin{equation}
\dot{m} = -D \cdot A(x) \frac{\partial c}{\partial x}
\end{equation}
Aus der Massenbilanz für jede Pore $i$  mit allen benachbarten Poren $ j \in \text{Nb}_i $ folgt
\begin{equation}
\sum_{j \in \text{Nb}_i} \dot{m}_{ij} = \sum_{j \in \text{Nb}_i} D \cdot S_{ij} (c_i - c_j).
\end{equation}
Dabei ist $S_{ij}$ der Formfaktor, welcher wie folgt definiert ist.
\begin{equation}
\frac{1}{S_{ij}} = \frac{1}{S^p_i} + \frac{1}{S^v_{ij}} + \frac{1}{S^p_j}
\end{equation}
Für kugelförmige Poren $p$ und die zylindrische Verbindung $v$ ergibt sich, der Formfaktor
\begin{align}
S^p_i &= \frac{\pi (3d_i^2 - 4l_{ij,i}^2)}{12l_{ij,i}} \\
S^p_j &= \frac{\pi (3d_j^2 - 4l_{ij,j}^2)}{12l_{ij,j}} \\
S^v_{ij} &= \frac{\pi d_{ij}^2}{4l_{ij}}
\end{align}

Die Umrechnung in ionische Leitfähigkeit erfolgt über die Nernst-Einstein-Gleichung.
\begin{equation}
D = \frac{kT}{nq^2} \sigma
\end{equation}
Analog ergibt sich für die modifizierte molare Flussrate $\dot{m}'$
\begin{equation}
\dot{m}' = \frac{nq^2}{kT} \dot{m}
\end{equation}
und
\begin{equation}
\sum_{j \in \text{Nb}_i} \dot{m}'_{ij} = \sum_{j \in \text{Nb}_i} \sigma \cdot S_{ij} (c_i - c_j) = 0
\end{equation}

Damit lassen sich die drei wichtigsten Größen für Porennetzwerke bestimmen:
\begin{enumerate}
    \item der effektive Diffusionskoeffizient
    \begin{equation}
    D_\text{eff} = \frac{\dot{m} \cdot L}{A \Delta c}
    \end{equation}
    \item die effektive Leitfähigkeit
    \begin{equation}
    \sigma_\text{eff} = \frac{\dot{m}' \cdot L}{A \Delta c}
    \end{equation}
    \item und die Tortuosität.
    \begin{equation}
    \tau = \epsilon \frac{D}{D_\text{eff}} = \epsilon \frac{\sigma}{\sigma_\text{eff}}
    \end{equation}
\end{enumerate}
Dabei ist die Porosität $\epsilon$ definiert durch das Verhältnis von Volumen flüssiger Phase $V_f$ zum Gesamtvolumen $V_{\text{gesamt}}$.
\begin{equation}
\epsilon = \frac{V_f}{V_{\text{gesamt}}} = \frac{V_p + V_v}{V_{\text{gesamt}}}
\end{equation}

\begin{figure}[!ht]
	%\raggedleft
		%\def\svgwidth{\columnwidth}
        \center
		\includegraphics[width=0.99\textwidth, angle=0]{RVE_generation.pdf}
		\caption{\label{fig:sphere_cylinder_model_RVE_generation}Die Generierung möglichst akurater Repräsentationen des Strukturelektrolytes beinhaltet (a) die experimenteller Bestimmung der Porenverteilung durch Gasabsorptionsmessungen verschiedener Oligomere (O1-O4), (b) die Konvertierung in eine kommulative Wahrscheinlichkeitsverteilung, (c) das zufällige Auswählen einer vorher festgelegten Anzahl an Poren, (d) die Vernetzung und Skalierung der Porenverbindungen und (e) die Simulation der Ionentransportes und vergleich mit den Experimentellen Ergebnissen.
        }
\end{figure}

Das Modell wurde mit experimentellen Porengrößenverteilungen validiert, die über Gasabsorptionsmessungen gewonnen wurden (Bild~\ref{fig:sphere_cylinder_model_RVE_generation}a). Dazu wurde die Verteilung auf das Intervall [0,1] normalisiert, um eine Wahrscheinlichkeitsverteilung für die zufällige Generierung zu erhalten (Bild~\ref{fig:sphere_cylinder_model_RVE_generation}b). Allerdings muss das Volumenverhältnis $r$ zwischen Verbindungs- und Porenvolumen vorgegeben werden, welches durch
\begin{equation}
r = \frac{V_v}{V_p}
\end{equation}
definiert ist.
Daraus ergibt sich das Porenvolumen alternativ als
\begin{equation}
V_f = V_p (1 + r).
\end{equation}
Durch die Wahl eines würfelförmigen repräsentativen Volumenelementes (RVE) können folgende Zusammenhänge für die Seitenlänge $L$ und die Querschnittsfläche $A$ aufgestellt werden.
\begin{align}
L &= \sqrt[3]{V_{\text{gesamt}}} \\
A &= L^2
\end{align}
Die Poren werden dann der Größe nach absteigend, in das RVE hineingelegt (Bild~\ref{fig:sphere_cylinder_model_RVE_generation}c). Befindet sich eine Pore am Rand muss, zum erhalt der Symmetrie, diese enstprechend dupliziert werden. Dabei wird ein KD-Tree-Algorithmus benutzt, um Porenkollisionen zu verhindern. Anschließend wurden die nächsten $n$ Nachbarn jeder Pore verbunden. Dabei wird als Ausgangsdurchmesser der Durchmesser der kleinsten Pore genommen und abschließend alle Verbindungsradien entsprechend skalliert\footnote{Das nicht-lineare Verhältnis macht ein iteratives Verfahren notwendig. In dieser Arbeit wurde dazu, dass Intervallhalbierungsverfahren benutzt.}, um dem vorher definierten Volumenverhältnis zu entsprechen (Bild~\ref{fig:sphere_cylinder_model_RVE_generation}d).
Zur Bestimmung der effektiven Größen wurde eine konstante molarere Flussrate an zwei gegenüberliegenden Seiten vorgegeben (Bild~\ref{fig:sphere_cylinder_model_RVE_generation}e). Aus den ermittelten Größen wurde der relative Fehler $\text{Err}_{\text{rel}}$ nach folgendem Prinzip bestimmt.
\begin{align}
    \text{Err}_{\text{rel}} &= \frac{\sigma_{\text{eff,PNM}} - \sigma_{\text{eff,exp}}}{\sigma_{\text{eff,PNM}}}\\
    &= \frac{D_{\text{eff,PNM}} - D_{\text{eff,exp}}}{D_{\text{eff,PNM}}}
\end{align}

\begin{figure}[!ht]
	%\raggedleft
		%\def\svgwidth{\columnwidth}
        \center
		\includegraphics[width=0.99\textwidth, angle=0]{convergence.pdf}
		\caption{\label{fig:sphere_cylinder_model_convergence}Der relative Fehler verschiedener Porennetzwerke mit (a) einer festen Anazahl an Verbindungen über verschieden Volumenverhältnise zwischen Verbindungs- und Porenvolumen, (b) Verschiedene Anzahl an verbunden Porennachbarn und (c)-(d) Konvergenzverhalten bei vermehrter Porenanzahl bei optimalen Volumenverhältnisen.
        }
\end{figure}

\input{Abbildungen/03_Modellierung/poren_network_results.tex}

Die Validierung der Methode wurde durch mehrere Studien mit verschiedenen Konfigurationen hinsichtlich Stichprobengröße und Modellparametern wurde für drei unterschiedliche bicontinuierliche oligomere Elektrolyte (O1\_50, O2\_40 und O4\_40) erbracht (Bild~\ref{fig:sphere_cylinder_model_RVE_generation}a)~\cite{Emilsson2023}. Für jede Konfiguration wurden auf Basis der gasadsorptiv gemessenen Porengrößenverteilung verschiedene Netzwerkmodelle erstellt. Die Ionentransporteigenschaft dieser Netzwerke wurde unter Verwendung der jeweiligen ionischen Leitfähigkeit der reinen Flüssigphase\footnote{O1: \SI{0,24}{\milli\siemens\per\centi\meter}, O2: \SI{0,026}{\milli\siemens\per\centi\meter} und O4: \SI{0,0026}{\milli\siemens\per\centi\meter}} simuliert.
Zur Analyse der Genauigkeit wurden die berechneten Leitfähigkeiten mit den experimentell gemessenen Werten verglichen (Tabelle~\ref{tab:pore_network_result}).

Für jede Konfiguration wurden 50 unterschiedliche Netzwerke mit zufälliger Porenverteilung generiert. In der ersten Untersuchung wurde der durchschnittliche relative Fehler für jede Substanz über verschiedene Volumenverhältnisse hinweg verfolgt (Bild~\ref{fig:sphere_cylinder_model_convergence}a). Das Volumenverhältnis mit dem geringsten durchschnittlichen Fehler wurde anschließend für eine Konvergenzstudie ausgewählt.

Diese Ergebnisse zeigt, dass der Fehlerbereich und der durchschnittliche Fehler mit zunehmender Netzwerkgröße kleiner werden. Bereits ab etwa 3000 Poren nähert sich der Fehler einem Wert unterhalb von \SI{10}{\percent} an und lässt sich mit steigender Porenzahl weiter reduzieren, siehe Bild~\ref{fig:sphere_cylinder_model_convergence}c-e.

Abschließend wurde die Konnektivität zwischen benachbarten Poren variiert. Dabei zeigte sich ein signifikanter Einfluss auf die Gesamtleitfähigkeit des Netzwerks: Eine höhere Konnektivität führte zu einer deutlichen Steigerung der effektiven Leitfähigkeit, sihe Bild~\ref{fig:sphere_cylinder_model_convergence}b.

Die Ergebnisse deuten darauf hin, dass die vorgeschlagene Methode die ionische Leitfähigkeit mit guter Genauigkeit modellieren kann, insbesondere bei zunehmender Porenzahl, was zu einer verbesserten Konvergenz führt. Allerdings ist nur die Porenkonnektivität direkt experimentell messbar. Die Beziehung zwischen Konnektivität und das Verhältnis von Hals- zu Porenvolumen — beides entscheidende Parameter zur Vorhersage der ionischen Leitfähigkeit — sind derzeit nicht direkt experimentell erfassbar. 
Allerdings könnten beide Parameter mithilfe von kleinmaßstäblichen RVE-Modellen abgeschätzt werden, die den Phasentrennungsprozess simulieren. 

Des Weiteren konnte auch ein Zusammenhang zwischen der Zusammensetzung des Elektrolyten und dem am besten passenden Verhältnis von Hals- zu Porendurchmesser gefunden werden. Ein möglicher Grund hierfür könnte sein, dass das zugrunde liegende Prinzip, den Halsdurchmesser anhand des kleineren Durchmessers der verbundenen Poren zu skalieren, nicht dem tatsächlichen physikalischen Verhalten entspricht.

Abschließend lassen sich mit diesem Ansatz nur schwer die effektiven Eigenschaften der festen Phase bestimmen, wie etwa die Steifigkeit. Hierfür benötigt es andere Methoden, wie etwa Walk-On-Stars~\cite{Sawhney2023a}.


\section{Parallele Berechnung des effektiven Diffusionskoeffizient und Steifigkeit für fraktale Geometrien mit der Walk-on-Stars Methode}

Walk-on-Stars (WoSt) ist eine netzfrei Methode zum Lösen linearer Differentialgleichungen, die mittels der Benutzung von Grafikkarten besonders gut parallelisiet werden kann und auch in Domainen funktioniert, die fraktale Strukturen mit vielen feinen Details aufweisen~\cite{Sawhney2023a}. Die mathematische Grundlage basiert dabei auf den Arbeiten von \textsc{Feyman} und \textsc{Kac}, die erstmal den Zusammenhang zwischen parabolischen partiellen Differentialgleichungen und storastischten Prozessen darstellten~\cite{Pascucci2024}. 
WoSt stellt dabei die Weiterentwicklung von Walk on Spheres (WoS) dar, welcher Brownische Teilchenbewegungen durch Zufallsbewegungen innerhalb von Kugeln annähert (Bild~\ref{fig:wost_method}a)~\cite{Sawhney2020}. WoSt basiert auf drei Kernmechnismen: die Bestimmung des größtmöglichen Sterngebietes\footnote{Ein (Teil-)Gebiet bei der alle Punkte von einem Punkt aus sichtbar sind.} für jeden Abfragepunkt, die Reflektion von Pfaden die durch Flächen mit Neumann Randbedingung durchlaufen würden und die Beednung und Aggregation der beendeten Pfade nach treffen auf eine Dirchlet-Randbedingung oder einer maximalen Anzahl an Schritten~\cite{Sawhney2023a}.

\begin{figure}[!ht]
	%\raggedleft
		%\def\svgwidth{\columnwidth}
        \center
		\includegraphics[width=0.99\textwidth, angle=0]{wost_method.pdf}
		\caption{\label{fig:wost_method}a) Brownschen Bewegung mit absorbierenden Dirichlet Randbedingungen und reflektierenden Neumann Randbedingung. b)-f) Ablauf der Walk-on-Star Methode zur Annäherung der Lösung einer linearen Differentialgleichung in einem Punkt. e) Berücksichtigung des Einflusses unterschiedlicher Materialkoeefizienten durch Abtastung an der Stelle $y_{k+1}$.
        }
\end{figure}

Vorrausgesetzt, die Geometrie des Strukturelektrolyten ist bekannt, kann diese Methode adaptiert werden um schnell eine sehr gute Annäherung an die Diffusionskoefizienten in der flüssigen Phase und eine durch die feste Phase erzeugte Steifigkeit zu bestimmen.

Ausgang der Annäherung der Verschiebung $\boldsymbol{u}$ in der festen Phase an der Stelle $\boldsymbol{x}$ ist dabei der folgende Zusammenhang.
\begin{equation}
    \boldsymbol{u}(\boldsymbol{x}) = \boldsymbol{E} \left( \int_{0}^{\tau} \boldsymbol{G}(\boldsymbol{x},\boldsymbol{X}_t) \boldsymbol{f}(\boldsymbol{X}_t)dt + \boldsymbol{G}(\boldsymbol{x},\boldsymbol{X}_{\tau}) \boldsymbol{g}(\boldsymbol{X}_{\tau}) \right)
\end{equation}
Dabi ist $\boldsymbol{G}(\boldsymbol{x},\boldsymbol{y})$ der elastische greensche Tensor\footnote{Häfig auch als Kelvinlösung oder Kelvinlet bezeichnet.}, $\tau$ die Endzeit beim Erreichen der Dirchlet Randbedingung und $\boldsymbol{g}$ die vorgegeben Verschiebung an dieser. Für isotope Medien ist greensche Tensor definiert\footnote{Die hier benutzte Version hat eine Singularität bei $r=0$. Diese kann zu bei kleinen Radien zu numerischen instabilitäten führen, weshalb in der Praxis oft regulierte Versionen verwendet werden~\cite{DeGoes2017,Chen2022b,Ringel2024}.} als~\cite{Lazar2014,Chen2022b}
\begin{equation}
    \boldsymbol{G}(\boldsymbol{x}, \boldsymbol{y}) = \frac{1}{4\pi \mu r} \begin{bmatrix}
        1-\frac {1}{2b}+\frac {1}{2b}\frac {x^2}{r^2} & {\frac {1}{2b}}{\frac {xy}{r^{2}}} & {\frac {1}{2b}}{\frac {xz}{r^{2}}}\\
        {\frac {1}{2b}}{\frac {yx}{r^{2}}} & 1-{\frac {1}{2b}}+{\frac {1}{2b}}{\frac {y^{2}}{r^{2}}} & {\frac {1}{2b}}{\frac {yz}{r^{2}}}\\
        {\frac {1}{2b}}{\frac {zx}{r^{2}}} & {\frac {1}{2b}}{\frac {zy}{r^{2}}} & 1-{\frac {1}{2b}}+{\frac {1}{2b}}{\frac {z^{2}}{r^{2}}}
    \end{bmatrix}
\end{equation}
wobei $\boldsymbol{r} = \boldsymbol{x} - \boldsymbol{y}$, $r = \lVert \boldsymbol{r} \rVert$, $a = 1-2 \nu$, $b = 2(1-\nu)$ und $\mu$ das Schubmodul ist, was für isotrope Materialien als 
\begin{equation}
    \mu = \frac{E}{2(1+\nu)}
\end{equation} definiert ist.
Unter der Annahme, dass außer an den Rändern keine Käfte im Körper auftreten und die Eigenschaften in der festen Phase ortsunabhängig sind, kann die Verschiebung an einem Punkt durch 
\begin{equation}
    \boldsymbol{u}(\boldsymbol{x}) = \frac{1}{N} \sum_{i=1}^{N} \sum_{K_i}^{k=1} \boldsymbol{G}(\boldsymbol{x}, \boldsymbol{X}_{i,k}) w_{i,k}
\end{equation}
angenähert werden~\cite{Kulkarni2003,Taylor2013,Chen2024b}. Wobei $K_i$ die Anzahl an Schritten der $i$-ten Stichprobe ist und $w_{i,k}$ die Wichtung darstellt. Die Wichtung bei der Reflektion durch Neumannflächen ist 
\begin{equation}
    w_{k,N} = \boldsymbol{t}(\boldsymbol{X}_{k+1}) = \boldsymbol{\sigma} \boldsymbol{n},
\end{equation}
während für Dirichlet Randbedingungen
\begin{equation}
    w_{k,D} = \boldsymbol{g}(\boldsymbol{X}_{k+1}) = \boldsymbol{u}
\end{equation} 
gilt und die Iteration beendet wird~\cite{Shia2000,Lazar2014,Sawhney2023a}.

Um die Steifikeit des RVEs mittels WoSt zu bestimmen wird in der würfelförmigen Domain $\Omega = [0,L] \times [0,L] \times [0,L]$ eine Dirichletbedingung am unteren Rand ($\boldsymbol{u}(x,y,0) = \boldsymbol{0}$) und eine Neumannbedingung am oberen Rand ($\boldsymbol{n} \boldsymbol{\sigma} = -p \boldsymbol{e}_z$) benutzt. Die Verschiebungen am oberen Rand können direkt und damit sehr effizient über WoSt approximiert werden. Die Gesamtverschiebung oben wird dann aus dem gemittelten Wert angenommen. Anschließend wird die Steifigkeit durch 
\begin{equation}
E = \frac{\Delta \sigma}{\Delta \varepsilon} = \frac{pL}{u_z(z=L)}  
\end{equation}
angenähert.

\begin{figure}[!ht]
	%\raggedleft
		%\def\svgwidth{\columnwidth}
        \center
		\includegraphics[width=0.99\textwidth, angle=0]{wost_results.pdf}
		\caption{\label{fig:wost_result}a) Simulation des Diffusionsverhaltens durch Walk-on-Stars. b) Simulation des Verformungsverhaltens. c-d) Konvergenzverhaltens des berrechneten Diffusions- und Steifigkeitsfehlers mit höherer Schrittmenge pro Pixel.
        }
\end{figure}

Analog kann WoSt benutzt werden, um in der flüssigen Phase die effektive Diffusion zu bestimmen.



\section{Automatisierte Generierung von repräsentativen Volumenelementen für zweiphasige Elektrolytsysteme aus Raster Elektronen Aufnahmen}

Sowohl die Porennetzwerkmethode, als auch die Annäherung durch Walk on Stars benötigt eine geometrische Rräsetnation der zweiphasige Strukturelektroyten. In den Erläuterungen zur Porennetzwerkmethode wurde bereits eine Möglichkeit aus Gasabsorptionsmessungen ein Ersatzmodel aus Kugeln und zylindirsichen Verbindungen zu genieren näher beschrieben. Während diese Methode auch für den WoSt Ansatz benutzt werden kann\footnote{Dazu die Domain innerhalb der Kugeln und Hälse für die fluid Phase wählen oder die außerhalb liegende, aber immer noch im RVE-Würfel liegende Domain für die feste Phase benutzen.} sind Gasabsorptionsmessungen in der Literatur nicht immer gegeben. Allerdings können aus Computertomographen Aufnahmen von ausreichender Auflösung benutzt werden, um mithilfe machinellen Lernens (ML) die Randbedingung der \textsc{Cahn-Hilliard}-Gleichungen, sowie die Stoppzeit abzuschätzen.
\begin{figure}[!ht]
	%\raggedleft
		%\def\svgwidth{\columnwidth}
        \center
		\includegraphics[width=0.99\textwidth, angle=0]{nn_metod_rve_se.pdf}
		\caption{\label{fig:nn_method_rve_se}Ausnutzung eines Convolutional Neural Network zur Abschätzung geeigneter Ausgangsparameter der \textsc{Cahn-Hilliard}-Gleichung.
        }
\end{figure}

Die Trainingsdaten bestehen aus Grauwert-REM-Scans, die in Schwarz-Weiß-Darstellung vorliegen, sowie den zugehörigen realen Abmessungen der Bildausschnitte in Nanometern. Zusätzlich werden experimentell bestimmte Diffusionskoeffizienten $D_i$ und elastische Steifigkeiten $E_i$ sowohl für die Einzelkomponenten als auch für den Verbundmaterial gemessen und den entsprechenden REM-Bildabschnitten zugeordnet.

Ein Convolutional Neural Network extrahiert aus den REM-Bildern textur- und strukturrelevante Merkmale. Diese Merkmale, kombiniert mit den realen Abmessungen, dienen als Eingabe für ein nachgeschaltetes Fully Connected Network, das die Parameter für die \textsc{Cahn-Hilliard}-Gleichung vorhersagt, namentlich die Randbindung $M$ und die Mobilität $L$, siehe Bild~\ref{fig:nn_method_rve_se}. %\citep{Zhang2021CHparameters, Müller2017MLforCH}.

\begin{figure}[!ht]
	%\raggedleft
		%\def\svgwidth{\columnwidth}
        \center
		\includegraphics[width=0.6\textwidth, angle=0]{training_progress.pdf}
		\caption{\label{fig:nn_method_result}Konvergenz der Verlustfunktion über den Trainingsprogress des neuronalen Netzwerkes.
        }
\end{figure}
Die \textsc{Cahn-Hilliard}-Gleichung wird mit einem Finite-Elemente Methode gelöst, wobei die Parameter $L(\mathbf{x})$ und $\varepsilon$ werden durch das ML-Modell bereitgestellt %\citep{Elliott1989CHFEM, Schneider2022FEMCH}.

Aus der zeitabhängigen Lösung $c(\mathbf{x},t)$ werden die Phasengrenzen als Isoflächen $c(\mathbf{x},t^*) = c_\mathrm{threshold}$ extrahiert. Dies erfolgt mit Hilfe von Marching Cubes~\cite{Lorensen1987}.

\begin{table}[ht!]
    \centering
    \caption{\label{tab:nn_method_rve_se_results}Von einem Neuronalen Netzwerk abgeschätzte \textsc{Cahn-Hilliard}-Parameter und resultierende RVE-Elemente basierend auf REM-Aufnahmen von Strukturelektrolyten.}
    \begin{tabularx}{\textwidth}{
        >{\centering\arraybackslash}m{0.2\textwidth}  % SEM column
        >{\centering\arraybackslash}m{0.08\textwidth} % M
        >{\centering\arraybackslash}m{0.08\textwidth} % λ
        >{\centering\arraybackslash}m{0.11\textwidth} % T_end
        >{\centering\arraybackslash}m{0.08\textwidth} % c0
        >{\centering\arraybackslash}m{0.08\textwidth} % L                                          % L
        >{\centering\arraybackslash}m{0.2\textwidth}  % RVE column
    }
    \toprule
    \textbf{REM-Aufnahme}
    & \textbf{M}
    & $\boldsymbol{\mathrm{\lambda}}$
    & $\boldsymbol{\mathrm{T_{end}}}$
    & $\boldsymbol{\mathrm{c_0}}$
    & $\boldsymbol{\mathcal{L}}$
    & \textbf{RVE}
    \\
    \midrule
    \makecell{\includegraphics[width=0.2\textwidth]{generated_rve_se/SEM_60DGEBA.png}\\60DEBA\footnotemark}
        & 0,94 & 0,013 & $\mathrm{5,5 \times 10^{-5}}$ & 0,92 & 0,97 
        & \includegraphics[width=0.22\textwidth]{generated_rve_se/RVE_spheres.png} \\
    \makecell{\includegraphics[width=0.2\textwidth]{generated_rve_se/SEM_50MTM57_2.3.png}\\50MTM57/2.3\footnotemark}
        & 3,22 & 0,061 & $\mathrm{36,0 \times 10^{-5}}$ & 0,94 & 0,98 
        & \includegraphics[width=0.22\textwidth]{generated_rve_se/RVE_Cahn_Hilbert.png} \\
    \makecell{\includegraphics[width=0.2\textwidth]{generated_rve_se/SEM_polyMIPE.png}\\polyMIPE\footnotemark}
        & 1,02 & 0,012 & $\mathrm{7,5 \times 10^{-5}}$ & 0,06 & 0,88 
        & \includegraphics[width=0.22\textwidth]{generated_rve_se/RVE_Template.png} \\
    \bottomrule
    \end{tabularx}\\
    %\noindent{\footnotesize{\textsuperscript{*} Gemessen gegenüber \ce{Li}/\ce{Li+}.}}
\end{table}

% WICHTIG: Die Reihenfolge muss exakt der in der Tabelle entsprechen!
\addtocounter{footnote}{-2} % Counter zurücksetzen, um die erste Markierung zu treffen
\footnotetext{Bezeichnet ein poröses Elektrolytsystem bestehend aus 60 Gew.-\% flüssigem Elektrolyten und einer festen Matrix aus \textit{Diglycidylether von Bisphenol A} (DGEBA), einem gängigen Epoxidharz, das hier mittels PIPS (\textit{Polymerisation-Induced Phase Separation}) hergestellt wurde.}
\stepcounter{footnote}
\footnotetext{Ein strukturelles Elektrolytsystem mit 50 Gew.-\% Elektrolytanteil. Die Matrix besteht aus dem trifunktionellen Epoxidharz \textit{MY0510}, dem Härter \textit{MNA} und dem Beschleuniger \textit{Tertiary amine} (zusammengefasst als MTM), infiltriert mit einem 2,3 M Lithium-Salz-Elektrolyten.}
\stepcounter{footnote}
\footnotetext{Steht für \textit{polymerized Medium Internal Phase Emulsion}. Im Gegensatz zu polyHIPE (High Internal Phase, $>74\%$ interne Phase) beschreibt polyMIPE ein poröses Polymer, das aus einer Emulsion mit einem mittleren Volumenanteil der inneren Phase (typischerweise zwischen 30 \% und 74 \%) synthetisiert wurde.}
Die extrahierte Geometrie definiert zwei Domänen: die Matrix- und die Flüssigphase. Auf jeder Domäne wird mittels linear-elastischer Finite-Elemente-Analyse die effektive Steifigkeit $E_\mathrm{eff}$ berechnet:
\begin{equation}
\nabla \cdot \bigl(\mathbf{C} : \nabla \mathbf{u}\bigr) = \mathbf{0}, 
\end{equation}
wobei $\mathbf{C}$ das Materialsteifigkeits-Tensor ist %\citep{Ciarlet2002FEM, Smith2018CompositeElasticity}. 
Parallel dazu wird die Diffusionsgleichung von \textsc{Fick}
zur Bestimmung des effektiven Diffusionskoeffizienten $D_\mathrm{eff}$ gelöst %\citep{Crank1979Diffusion, Kim2016MLDiffusion}.

Das gesamte Modell wird end-to-end trainiert, indem der Vergleich der berechneten $E_\mathrm{eff}$ und $D_\mathrm{eff}$ mit den experimentellen Werten in den Verlustfunktionsterm
\begin{equation}
    \mathcal{L} =  \frac{1}{1+\sqrt{\lVert \frac{E_\mathrm{eff}^\mathrm{pred} - E_\mathrm{eff}^\mathrm{exp}}{E_\mathrm{rein}^\mathrm{exp}}\rVert^2
+ \lVert \frac{D_\mathrm{eff}^\mathrm{pred} - D_\mathrm{eff}^\mathrm{exp}}{D_\mathrm{rein}^\mathrm{exp}}\rVert^2}}
\end{equation}
aufgenommen wird, siehe Bild~\ref{fig:nn_method_result}.  %\citep{Goodfellow2016DeepLearning, Bishop2006PatternRecognition}.

Die in Tabelle~\ref{tab:nn_method_rve_se_results} gezeigten Resultate verdeutlichen, dass das neuronale Netzwerk die Parameter der \textsc{Cahn-Hilliard}-Gleichung konsistent aus den jeweiligen REM-Aufnahmen ableiten kann. Besonders relevant ist dabei der Wert der Verlustfunktion $\mathcal{L}$, der durch die Funktionskonstruktion zwischen 0 und 1 liegt. Ein hoher Wert von $L$ entspricht einer sehr guten Übereinstimmung zwischen den experimentell bestimmten und den vom Modell vorhergesagten effektiven Materialeigenschaften. Alle drei untersuchten Strukturelektrolyte erreichen hohe $\mathcal{L}$-Werte zwischen 0,88 und 0,98, was die Robustheit des end-to-end Ansatzes bestätigt.

Die Parameter $M$ und $T_\mathrm{end}$ zeigen in allen Fällen eine deutliche Korrelation. Ein hoher Mobilitätskoeffizient $M$ führt zu einer schnelleren Koarsening-Dynamik, die sich häufig in einer größeren Phasenverbindung äußert. Da $T_\mathrm{end}$ direkt die Simulationsdauer bis zur Ausbildung der charakteristischen Mikrostruktur beschreibt, sind beide Größen weitgehend überbestimmt. Dies zeigt sich besonders beim System 50MTM57/2.3, das zugleich den höchsten Mobilitätswert und die längste Ausscheidungszeit aufweist. Die resultierende Morphologie besteht aus vielen großen Poren.

Der Wert von $c_0$ beeinflusst das Anfangsverhältnis von fester und flüssiger Phase. Ein niedriger Wert - wie bei polyMIPE - weist auf einen höheren Anteil der festen Phase hin, während hohe Werte, wie bei 60DGEBA und 50MTM57/2.3, eine stärker flüssigkeitsdominierte Ausgangsverteilung anzeigen. Dies wirkt sich maßgeblich auf die entstehenden Phasenvolumina im RVE aus und entspricht den beobachteten Strukturelementen.

Der Parameter $\lambda$ bestimmt die Feinheit der Grenzflächenauflösung und damit den Detaillierungsgrad der Mikrostruktur. Größere Werte wie bei 50MTM57/2.3 führen zu klarer ausgeprägten Grenzflächen, während kleinere Werte, wie bei polyMIPE, feinere und diffuser konturierte Strukturen erzeugen. Die Unterschiede in $\lambda$ spiegeln sich direkt in den generierten RVEs wider und stimmen qualitativ mit den REM-Beobachtungen überein.

Insgesamt zeigt die Tabelle, dass das Netzwerk in der Lage ist, sowohl grobe als auch feine strukturelle Merkmale der Eingangsbilder zuverlässig in physikalisch sinnvolle Modellparameter zu überführen. Die erzeugten RVE-Geometrien unterscheiden sich klar zwischen den Materialsystemen und repräsentieren die realen Mikrostrukturen überzeugend. Dies bestätigt, dass die ML-basierte Parametrisierung eine leistungsfähige Alternative zu experimentell schwer zugänglichen Methoden darstellt und eine konsistente Kopplung zwischen Bildgebung, physikalischem Modell und effektiven Materialeigenschaften ermöglicht.
