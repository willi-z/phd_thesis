%*******************************************************
%                       EINLEITUNG  
%*******************************************************
\chapter[Einleitung]{\label{sec:Einleitung}Einleitung}
% max 1 Seite
Angesichts der globalen Notwendigkeit zur Dekarbonisierung~\cite{MacDowell2017} rücken nachhaltige Mobilitätssysteme und die damit verbundenen Anforderungen an die Energieeffizienz in den Fokus der technologischen Entwicklung~\cite{Owusu2016}. Ein zentraler Hebel dieser Transformation liegt in der Substitution fossiler Energieträger durch elektrische Antriebssysteme~\cite{Sugiyama2012}. Zur Bereitstellung der elektrischen Energie haben sich Batterien aufgrund ihrer hohen spezifischen Energie gegenüber anderen Konzepten wie Brennstoffzellen oder Superkondensatoren~\cite{Winter2004, Hemmati2016, Salkuti2023} bereits in zahlreichen mobilen Applikationen etabliert. Dies umfasst sowohl straßengebundene Fahrzeuge~\cite{Huo2015, Donateo2015, Jochem2015} als auch unbemannte Flugsysteme sowie Drohnen~\cite{VincentWong2015, Boukoberine2019, Pham2022, Wang2022} und Anwendungen der mobilen Robotik~\cite{Hecht2023, Mikolajczyk2023, Ghobadpour2023, Wang2020}.


In der aktuellen Systemgestaltung stoßen diese elektrifizierten Antriebe jedoch an eine kritische Grenze der Leichtbaugüte: die systemimmanente Totlast der Energiespeicher~\cite{Foiadelli2018}. In konventionellen Architekturen sind Batterien als rein funktionale Komponenten ohne tragende Eigenschaften konzipiert~\cite{VanMierlo2007, Xu2022}. Aufgrund ihrer geringen mechanischen Belastbarkeit erfordern sie massive Schutzstrukturen und Gehäuse, um die strukturelle Integrität unter Crash- und Lastszenarien zu gewährleisten. Diese strikte Funktionstrennung zwischen Kraftfluss und Energiespeicherung unterbindet synergetische Effekte und führt bei aktuellen Batteriepacks zu einem Strukturmassenanteil von bis zu 40\,\%, der keinen direkten Beitrag zur Speicherkapazität leistet~\cite{Radu2021, Audi2022}. Dieses ungünstige Verhältnis von Wirkmasse zu Gesamtmasse generiert erhebliche Sekundärmassen und limitiert die Performanz moderner Leichtbaustrukturen signifikant~\cite{Armand2020, Schaefer2018, Cano2018, Goodenough2009}.

Ein vielversprechender Lösungsansatz zur Steigerung der Masseneffizienz ist der Übergang von der differentialen zur integralen Bauweise in Form von multifunktionalen Verbundstrukturen, sogenannten Strukturbatterien~\cite{Johannisson2018, Liu2009, Ekstedt2010, Wang2019, Zhao2020, Yin2020, Lutkenhaus2020, Fu2021, Jin2021, Kalnaus2021}. Diese Systeme realisieren eine konsequente Funktionsintegration, indem sie die elektrochemische Speicherung auf Interkalationsbasis direkt in die lasttragende Struktur integrieren~\cite{Wong2007, Carlson2013}. Neben der signifikanten Reduktion der Gesamtsystemmasse~\cite{Wetzel2004, Snyder2015, Carlstedt2020a, Asp2014, Johannisson2019} eröffnen Strukturbatterien neue Freiheitsgrade in der konstruktiven Gestaltung. Die dezentrale Integration ermöglicht eine beanspruchungsgerechte Gewichtsverteilung und die Platzierung unmittelbar am Verbraucher, wodurch zusätzliche Masseneinsparungen durch die Reduktion passiver Komponenten und kürzerer Leitungswege erzielt werden können~\cite{Thomas2004, Danzi2021, Moyer2020, Wang2020}.


Da sich dieser Forschungszweig in einem frühen Stadium befindet, konzentrierten sich bisherige Arbeiten primär auf eine begrenzte Anzahl an Werkstoffkombinationen und Zellchemien~\cite{Asp2019, Asp2024}. Es besteht daher ein erhebliches, bisher unerschlossenes Optimierungspotenzial in der Exploration neuartiger Materialsysteme~\cite{Asp2019, Snyder2006, Chen2024a}. Die vorliegende Dissertation adressiert dieses Defizit durch die Entwicklung computergestützter Analysen zur Identifizierung und Bewertung neuer, hochintegrativer Strukturbatterie-Systeme in laminarer Bauweise.

Die hier dargestellte Forschung wurde im Rahmen der Forschungsinitiative „Luftfahrtforschung und -technologie“ (LuFo-VI-2) des Bundesministeriums für Wirtschaft und Klimaschutz gefördert. Im Projekt „Entwicklung und Erprobung ultraleichter Verbundstrukturen mit integrierter elektrischer Speicherfunktion“ (ElViS) leistet diese Arbeit einen Beitrag zur Programmlinie „Disruptive Technologien und innovative Systeme“, um die methodischen Grundlagen für ein ökoeffizientes Fliegen der Zukunft zu schaffen~\cite{Scholz2018}.

%*************************************************************
%               Motivation und Zielstellung  
%*************************************************************
\section{\label{sec:Motivation_Zielstellung}Problemstellung und Zielsetzung}

Um zukünftig in der Satellitentechnik, Robotik oder Elektromobilität erfolgreich implementiert zu werden, müssen Strukturbatterien gegenüber einem Referenzsystem aus konventionellen Batterien und passiven Kohlenstofffaserverbunden einen signifikanten Massenvorteil erbringen~\cite{Wong2007, Carlson2013} (siehe Bild~\ref{fig:motivation}a). Das Erreichen dieses sogenannten Multifunktionalitätsbereichs setzt voraus, dass sowohl hohe Energiedichten als auch exzellente mechanische Kennwerte simultan realisiert werden~\cite{Snyder2015}. Während aktuelle Prototypen bereits beachtliche Steifigkeiten aufweisen, liegt ihre Energiedichte oft noch um den Faktor zehn unter der herkömmlicher Lithium-Ionen-Akkumulatoren~\cite{Asp2024}. Diese Diskrepanz limitiert ihr Einsatzfeld derzeit primär auf sekundäre Anwendungen oder Systeme mit geringem Leistungsbedarf (\textit{low energy applications}) und stellt eine erhebliche Markteintrittsbarriere dar~\cite{Snyder2006, Chen2024a} (siehe Bild~\ref{fig:motivation}b).

\begin{figure}[!ht]
    \centering
    \includegraphics[width=\textwidth, angle=0]{motivation.pdf}
    \caption{\label{fig:motivation} a) Technologische Einsatzgebiete von Strukturbatterien. b) Vergleich der mechanischen und elektrochemischen Eigenschaften aktueller Strukturspeicher mit kommerziellen Systemen.}
\end{figure}

Ein wesentliches Hemmnis für die technologische Weiterentwicklung liegt in der methodischen Ausrichtung der aktuellen Forschung, die sich maßgeblich auf zeit- und kostenintensive Experimente stützt. Der hohe apparative Aufwand~\cite{Fam2024, Duffner2020}, signifikante Sicherheitsrisiken im Umgang mit reaktiven Komponenten~\cite{Shirshova2021, Larsson2017} sowie fertigungsbedingte Varianzen~\cite{Siraj2023, Schnell2019} führen dazu, dass sich wissenschaftliche Untersuchungen bisher auf wenige Materialkombinationen konzentrieren. Es fehlt somit eine ganzheitliche Methodik, welche die vielfältigen Materialkombinationen hinsichtlich ihres kombinierten elektrochemischen und mechanischen Potenzials prädiktiv bewerten kann. Bestehende computergestützte Modelle bieten hierfür zwar eine Grundlage, erweisen sich jedoch aufgrund ihres hohen Detaillierungsgrads auf niedrigen Skalen als extrem rechenintensiv und benötigen eine Vielzahl schwer zu bestimmender Materialkennwerte~\cite{Plett2015, Carlstedt2022, Giessen2020}.

Das \textbf{Hauptziel (HZ) dieser Dissertation ist daher die Entwicklung einer computergestützten Berechnungsmethode zur belastungsgerechten Auslegung von Hochleistungsstrukturbatterien}. Diese modellgetriebene Alternative zum rein experimentellen Ansatz soll den Entwicklungsaufwand reduzieren und eine zielgenaue Materialvorauswahl für breite Anforderungsprofile ermöglichen (siehe Bild~\ref{fig:own_methode}).

\begin{figure}[!ht]
    \centering
    \includegraphics[width=\textwidth, angle=0]{methode.pdf}
    \caption{\label{fig:own_methode}Schematische Darstellung der entwickelten Methode zur Identifizierung anwendungsspezifischer Materialkombinationen.}
\end{figure}

Zur Realisierung dieses Hauptziels wird die Dissertation in vier Teilziele untergliedert, deren zeitlicher und inhaltlicher Aufbau grafisch in Bild~\ref{fig:thesis_structure} zusammengefasst ist:

\begin{figure}[!ht]
    \centering
    \includegraphics[width=\textwidth]{motivation.pdf}
    \caption{\label{fig:thesis_structure}Struktur der Arbeit und methodischer Ablauf zur Erreichung der Teilziele TZ 1 bis TZ 4.}
\end{figure}

Das erste Teilziel umfasst den \textbf{Aufbau eines wissenschaftlichen Verständnisses der gekoppelten Effekte in Strukturbatterien (TZ 1)}. Die fundierte Kenntnis der physikalischen Interaktionen zwischen mechanischen Lastspannungen und elektrochemischen Transportvorgängen bildet die Grundlage für jede Form der numerischen Abbildung. In der vorliegenden Arbeit erfolgt die Umsetzung von \textbf{TZ 1} zunächst in Kapitel~\ref{sec:soa} durch eine systematische Einführung in die relevanten Kenngrößen und die Materialvielfalt der Funktionskomponenten. Darauf aufbauend werden in Kapitel~\ref{sec:modelling_SB} die fundamentalen physikalischen Modelle der Mikroebene zusammengefasst und bewertet, um die Notwendigkeit einer Homogenisierung auf die makroskopische Ebene methodisch abzuleiten.

Darauf folgt die \textbf{Entwicklung einer skalenübergreifenden Simulationsmethodik zur Beschreibung der multiphysikalischen Effekte (TZ 2)}. Ziel ist es, die Komplexität der Mikrostruktur so zu reduzieren, dass sie für die Bauteilauslegung handhabbar wird, ohne die physikalische Genauigkeit zu verlieren. Hierauf wird in Kapitel~\ref{sec:sim_sbe} durch die Vorstellung verschiedener computergestützter Ansätze zur Charakterisierung der Strukturelektrolyte eingegangen, um aus experimentellen Daten effektive Parameter zu gewinnen. In Kapitel~\ref{sec:multi_scale} wird diese Methodik durch die Definition Repräsentativer Volumenelemente (RVE) in eine gesamtheitliche Simulation überführt, welche auch das multifunktionale Versagensverhalten berücksichtigt.

Ein integraler Bestandteil der methodischen Absicherung ist die \textbf{experimentelle Validierung der entwickelten Simulationsmodelle (TZ 3)}. Nur durch den direkten Vergleich zwischen Vorhersage und Messung kann die Verlässlichkeit der digitalen Methode sichergestellt werden. Die Umsetzung von \textbf{TZ 3} erfolgt dabei nicht isoliert, sondern sukzessive durch den Abgleich mit experimentellen Daten innerhalb der jeweiligen Entwicklungsschritte in den Kapiteln~\ref{sec:sim_sbe} bis~\ref{sec:lightweight_applications}.

Abschließend wird die \textbf{Auslegung von multifunktionalen Hochleistungsstrukturen als potenzielle Anwendungen für Strukturbatterien (TZ 4)} angestrebt. Dies dient dem Nachweis, dass die entwickelte Methode geeignet ist, reale Leichtbausysteme unter Berücksichtigung komplexer Lastprofile zu dimensionieren. Um dies zu demonstrieren, wird in Kapitel~\ref{sec:analytical_approach} zunächst ein schneller analytischer Vorauslegungsansatz abgeleitet. In Kapitel~\ref{sec:lightweight_applications} kulminiert die Arbeit schließlich in der belastungsgerechten Auslegung eines mobilen Laufroboters sowie eines Flugzeugsitzes, um den Weg von der Grundlagenforschung zur Systemintegration zu ebnen.


%**************************************************************
%                   LITERATURÜBERSICHT  
%**************************************************************
\section{\label{sec:Literaturübersicht}Literaturübersicht}
% 3-5 Seiten

\subsection*{Existierende Strukturbatteriekonzepte}

\begin{figure}[ht]
	%\raggedleft
		%\def\svgwidth{\columnwidth}
        \center
	\includegraphics[width=\textwidth, angle=0]{sb_types.pdf}
		\caption{\label{fig:sb_types}Vereinfachte Darstellung von a) konventionellen Li-Ionen Batterie, b) einer Strukturbatterie als multifunktionales Material und c) als multifunktionale Struktur.}
\end{figure}

Die erste multifunktionale Strukturbatterie wurde 2004 von \textsc{Wetzel} et al. im Army Research Laboratories (ARL) der USA entwickelt \cite{Wetzel2004, Snyder2006, Wong2007, Snyder2007}. Die Verbundmaterialien der Strukturbatterie  basierten auf Kohlenstofffasern als Anode, einer mit $\text{LiFePO}_\text{4}$ beschichteten Edelstahlkathode und einer Glasfasermatte als Separator \cite{Wong2007}. Dieser Aufbau zeigte bereits gute mechanische Eigenschaften. Allerdings konnten die elektrochemischen Eigenschaften wegen auftretender Kurzschlüsse nicht abschließend bestimmt werden.

2009 konzipierten \textsc{Liu} et al. \cite{Liu2009} die erste kurzfaserverstärkte Elektrode mit einem festen Polymerelektrolyt als Matrixmaterial. Allerdings konnte die faserverstärkte Elektrode nicht herstellt werden und keinen Feststoffelektrolyten mit ausreichender Ionenleitfähigkeit gefunden werden, weshalb schließlich auf ein gelbasiertes Elektrolyt mit festen und flüssigen Phasenanteilen umgeschwenkt wurde. Da aufgrund der besseren Ionenleitfähigkeit weniger Elektrolyt eingesetzt werden musste, konnte eine Energiedichte von 35~$\si{\watt \hour \per \kg}$ erreicht werden. Durch die fehlende strukturelle Verstärkung wurde allerdings nur eine geringe Zugsteifigkeit von 3~GPa erreicht.

Der Ansatz der Verwendung von Gelelektrolyten wurde von \textsc{Ekstedt} et al. verfolgt \cite{Ekstedt2010}, in das erstmals ein Kohlenstofffasergewebe als Elektrode einbettet wurde. Ähnlich wie in den Arbeiten von \textsc{Wetzel} et al. wurde auch hier ein glasfaserbasierter Separator und eine mit $\text{LiFePO}_\text{4}$ beschichtete, gewebte Aluminiumfasermatte als Kathode verwendet. Die resultierende Batterie zeichnete sich durch eine Zellspannung von 3,3~V aus. Allerdings wurden die mechanischen und elektrochemischen Eigenschaften nur theoretisch ermittelt.

Im Jahr 2011 präsentierten \textsc{Carlson} et al. \cite{Carlson2011} eine der ersten funktionierenden Strukturbatterien mit einer laminatartigen Faserverbundstruktur. Diese bestand aus einem IMS65-Kohlenstofffasergewebe als Anode, einem Gelelektrolyt, einem Glasfaserseparator und einer mit $\text{LiFePO}_\text{4}$ beschichteten gewebten Aluminiumfolie. Die speicherbare elektrische Energie betrug 0,0247~$\si{\watt \hour \per \kg}$, was ausreichte, um eine LED 70~s lang zu einem schwachen Leuchten zu bringen.

Zwei Jahre später wurden \textsc{Asp} et al. \cite{Asp2013US,Asp2013CN} zwei Patente zugesprochen, die einen Ansatz beschreiben, durch gezielte Funktionalisierung der Faseroberflächen jede Kohlenstofffaser zu einer Elektrode zu machen. Auch wenn die Patente seit 2017 nicht mehr verfolgt werden, wurde darauf aufbauend eine Batterie entwickelt, mit der eine Energiedichte von 10~$\si{\watt \hour \per \kg}$ erzielt wurde. Die theoretisch möglichen 175~$\si{\watt \hour \per \kg}$ bei einem gleichzeitigen Schubmodul von 1~GPa wurden jedoch noch nicht ansatzweise erreicht \cite{Leijonmarck2013, Carlson2013}.

2018 untersuchten \textsc{Meng} et al. \cite{Meng2018} erstmalig den Einsatz von vertikal ausgerichteten Kohlenstoffnanoröhrchen (CNT). Diese wurden für die Elektrode auf ein Edelstahlnetz aufgedampft. Anschließend wurde für die Anode $\text{NiO}_\text{x}$ durch einen elektrochemischen Ausscheidungsprozess auf die CNT-Edelstahlelektrode eingelagert. Mit dem gleichen Verfahren wurde $\text{FeO}_\text{x}$ für die Kathode aufgetragen. Die Strukturbatterie erreichte dabei eine Zugsteifigkeit von 7,0~GPa und eine Energiedichte von 1.4~$\si{\watt \hour \per \kg}$.

\textsc{Moyer} et al. \cite{Moyer2020} modifizierten den für Batterien typischen Pouchzellenansatz, indem das verpresste Kohlenstofffasergewebe gleichzeitig als Stromkollektor und Schutzfolie dient. Durch das anodenseitige Aufbringen von Graphit und $\text{LiFePO}_\text{4}$ auf der Kathode nehmen die Fasern jedoch nicht direkt am chemischen Prozess teil. Durch das bessere, als Interkalation bezeichnete, Ionen-Einlagerungsverhalten bei Graphit konnte eine hohe Energiedichte von 35~$\si{\watt \hour \per \kg}$ gemessen werden. Jedoch führte der Ansatz zu einer vergleichsweise geringen Zugsteifigkeit von 2~GPa.

\textsc{Thakur \& Dong} \cite{Thakur2020} stellten 2020 die erste 3D-gedruckte Strukturbatterie her. Mithilfe eines Koextrusionsprozesses konnte eine Kohlenstofffaser, die vorher mit einem festen Polymerelektrolyten beschichtet wurde, zusammen mit einem Li-gedopten Polylactide-Matrixmaterial aufgebracht werden. Nach dem Drucken der Elektrode wurde manuell eine Glasfasermatte und abschließend eine Aluminiumfolie als Kathode aufgebracht. Neben der Möglichkeit, neue Batteriegeometrien zu drucken, wurde auch durch die höhere Dichte an Aktivmaterial in Fasernähe eine vergleichsweise hohe Energiedichte von 24~$\si{\watt \hour \per \kg}$ erreicht. Allerdings konnte durch den geringen Faservolumenanteil eine niedrige Zugsteifigkeit von 0.29~GPa gemessen werden.

2021 präsentierte \textsc{Asp} et al. \cite{Asp2021} ein Design mit unidirektionalen Kohlenstofffasern als Anode, einem gewebten Glasfaserseparator und einer $\text{LiFePO}_\text{4}$-beschichteten Aluminiumplatte als Kathode. Durch ein verbessertes Herstellungsverfahren und eine günstige Faseranordnung konnte eine Zugsteifigkeit von 25~GPa und eine Zugfestigkeit von 300~MPa gemessen werden. Gleichzeitig erreichte die Strukturbatterie eine Energiedichte von 24~$\si{\watt \hour \per \kg}$. \textsc{Siraj} et al. \cite{Siraj2023} verbesserten zwei Jahre später den Infiltrationsprozess, wodurch sie bei annähernd gleichbleibenden mechanischen Eigenschaften die Energiedichte auf nahezu 41~$\si{\watt \hour \per \kg}$ verdoppeln konnten.

\begin{table}[ht]
    \centering
    \caption{Übersicht bisher entwickelter Strukturbatterien.}
    \begin{tabular}[t]{lccc}
    \toprule
    &Zugmodul~[GPa]&Energiedichte~[Wh/kg]&Referenz\\
    \midrule
    \textsc{Wong et al.}&8&/&\cite{Wong2007}\\
    \textsc{Liu et al.}&3&35& \cite{Liu2009}\\
    \textsc{Meng et al.}&7&4&\cite{Meng2018}\\
    \textsc{Moyer et al.}&35&2&\cite{Moyer2020}\\
    \textsc{Thakur et Dong}&0.29&24&\cite{Thakur2020}\\
    \textsc{Huang et al.}&9.2&43&\cite{Huang2020}\\
    \textsc{Asp et al.}&25&24&\cite{Asp2021} \\
    \textsc{Saraj et al.}&26&41&\cite{Siraj2023}\\
    \bottomrule
    \end{tabular}
\end{table}%

Die existierenden Studien an Strukturbatterien zeigen einen starken Fokus auf geschichtete Bauweisen, die sich an die Struktur von herkömmlichen Lithiumionenbatterien anlehnt, siehe Bild~\ref{fig:sb_types}a,b. Aus den bisherigen Ergebnissen leiten sich außerdem eine Reihe an Verbesserungspotenzialen für weitere Forschungen ab. Die größte Herausforderung stellt bislang das Erzielen einer möglichst guten Lithium-Interkalation in das Aktivmaterial der Elektrode dar. Das Gleiche gilt für die Lithiummigration zwischen den beiden Elektroden über den Strukturelektrolyt, welche möglichst unbehindert, bei gleichzeitiger Beibehaltung der mechanischen Steifigkeit und Festigkeit,  stattfinden muss~\cite{Asp2015}. Besonders zur weiteren Eröhung der Energiedichte, sind neue Ansätzte wie perforierte Stromabenehmer, Seperatorfreie Bauweisen durch Strukturelektorlyt oder hybride Bauweisen für bessere zellchemie denkbar, siehe Bild~\ref{fig:sb_types}c. Die Wichtigkeit solcher Verbesserungen wird dadurch verstärkt, dass die aktuell höchste erreichte Energiedichte von 41~$\si{\watt \hour \per \kg}$ am unteren Ende der benötigten Energiedichte, die für den Einsatz in z.B. Elektrofahrzeugen erforderlich ist, liegt. Diese Mindestgrenze der Energiedichte wurde allerdings 2022 von dem Forschungsteam um \textsc{Linde} am Deutschen Zentrum für Luft- und Raumfahrt (DLR) weiter angehoben auf 74~$\si{\watt \hour \per \kg}$, bei einem gleichzeitig Mindest-Zugsteifigkeit von 54~GPa und einer Mindestfestigkeit von 203~MPa, sowie einer Leistungsdichte von 376~W/kg ~\cite{Ishfaq2022}. Womit ein Durchbruch für Strukturbatterien im Sinne einer industriellen Anwendung noch aus steht.

\subsection*{Modellierung des gekoppelten elektrochemischen und mechanischen Verhaltens von Strukturbatterien}

Über jedes Material was in einer heutigen sicherheitsrelevanten Anwendung verwendet wird muss ausreichendes Wissen existieren um das Verhalten im Belastungsfall ausreichend vorherzusagen. Bei Strukturbatterien besteht die Herausforderung nicht sowohl die elektrochemsichen und mechansichen Prozesse einzeln zu verstehen, sondern auch dabei ihre gegenseitigen Wechselwirkungen zu berücksichtigen~\cite{Carlstedt2022a}. Diese Schwierigkeiten werden durch den hohen Multifunktionalitätsgrad von Material und Struktur zu einem .

Eine vergleichsweise einfache Beschreibung der elektrochemischen und thermischen Phänomenen kann durch äquivalente Ersatzschaltungen (\textit{engl.} equivalent circuit model, ECM) erfolgen \cite{Bavsic2022}. Bereits mit wenigen Elementen lassen sich Spannungsänderungen in Abhängigkeit vom Lade- und Entladeverhalten gut annähern \cite{YannLiaw2004}. Durch zusätzliche Erweiterungen lassen sich auch Alterung und der Einfluss zahlreicher thermischer Effekte berücksichtigen \cite{Hannan2017,Tran2021}. Aufgrund des geringen Rechenaufwands eignen sich diese Modelle besonders gut für zeitkritische Anwendungen wie die Lade- und Entladeregelung, siehe Abbildung~\ref{fig:battery_modelling_in_context}. Die Modellparameter müssen jedoch jedes Mal durch einen Fittingprozess für das jeweilige System bestimmt werden \cite{Tomasov2019}. Eine Verknüpfung der einzelnen Schaltelemente mit realphysikalischen Größen war bisher nur wenig erfolgreich \cite{Plett2015}.
\begin{figure}[ht]
	%\raggedleft
		%\def\svgwidth{\columnwidth}
        \center
	\includegraphics[width=\textwidth, angle=0]{batterie_modelling_approaches.pdf}
		\caption{\label{fig:battery_modelling_in_context}Übersicht der Batteriemodellierung im Kontext neuer Batterieentwicklungen.}
\end{figure}
Die physikalische Modellierung der elektrochemischen Phänomene wurde maßgeblich von \textsc{Doyle} \cite{Doyle1995,Doyle2003,Ceder2002}, \textsc{Fuller} \cite{Fuller2018,Takeuchi2008} und \textsc{Newman} \cite{Doyle1995,Newman2021} vorangetrieben. Das von ihnen entwickelte und nach ihnen benannte DFN-Modell (alternativ auch \textit{pseudo zwei dimensionale} (P2D) Modell genannt) \cite{Doyle1993} beschreibt die Prozesse auf der Makroskala und eignet sich daher sehr gut, um die Vorgänge auf Zellebene zu modellieren. Die benötigten Parameter können durch Experimente oder mithilfe von Simulationen auf niedrigeren Skalen wie der Dichtefunktionaltheorie (DFT) oder molekulardynamischen (MD) Simulationen bestimmt werden \cite{Chen2022}. Das DFN-Modell findet heute weitreichenden Einsatz und wurde seit seiner Veröffentlichung im Jahr 1993 zahlreich modifiziert und erweitert. Einige der bekanntesten Derivate sind das \textit{Single Particle Model} (SPM) \cite{Li2017} und das \textit{full homogenized macro-scale} (FHM) Modell \cite{Arunachalam2019}, welche beide darauf abzielen, die sehr rechenintensiven Differentialgleichungen in ihrer Komplexität zu reduzieren.

Die Kopplung mit thermischen Prozessen wurde bereits 1995 von \textsc{Pals et Newman} \cite{Pals1995,Pals1995a} begonnen und seitdem kontinuierlich weiterentwickelt \cite{Chen2005,Onda2006,Kim2013,Gao2021,Liu2023}. Auch der Einfluss der lithierungsbedingten Ausdehnung \cite{Bower2011,Yang2014,Roberts2014,Pereira2019,Mai2019,Li2020,Hoeschele2023}, Rissbildung \cite{Dionisi2017,Wang2020a,Pistorio2023} und Alterung \cite{RedondoIglesias2020} wurden in zahlreichen Studien untersucht. Zusätzlich existieren viele Studien, die sich einer vereinheitlichten Modellierung aller Effekte \cite{Wu2014,Kim2018,Liu2020,Yin2020} und auch einer Modellierung über mehrere Größenskalen hinweg widmen \cite{Liu2019,Li2020a,Katrasnik2021}.

\textsc{Carlstedt} erarbeitete stückweise in einer Reihe von Beiträgen eine Koppelung von Elektrochemie, Mechanik und Thermodynamik, um das Verhalten von Strukturbatterien zu beschreiben \cite{Carlstedt2019,Carlstedt2019a,Carlstedt2019b,Carlstedt2020,Carlstedt2020b,Carlstedt2022,Carlstedt2022a,Carlstedt2022b}. Auch wurde der Ansatz um den nicht-linearen Zusammenhang durch die Gestaltänderung von \textsc{Larsson} et al. \cite{Larsson2023} erweitert.

\subsection*{Modelierungsgetriebene Entwicklung von neuen Batterien und Strukturbatterien}

Die neu motivierte Forschung in bereits untersuchte und neue Batteriematerialien baut aktuell hauptsächlich auf theoretischen Überlegungen und experimentellen Prototypen auf. Um die Vielzahl an Effekten zu berücksichtigen und der enormen Anzahl an vielversprechenden Materialkombinationen Herr zu werden, argumentierten \textsc{Greenhalgh} \cite{Greenhalgh2024,Greenhalgh2024a} und \textsc{Asp} \cite{Asp2024} auf dem \textsc{1st Structural Power Research Showcase} in London, dass nur durch intensive Modellierungsarbeit diese Varianten voruntersucht und ausreichend eingeschränkt werden können. Der multiphysikalische Modellierungsansatz von \textsc{Carlstedt} wird zwar in diesem Zusammenhang oft erwähnt, allerdings sorgen Skalierungseffekte, wie etwa Defekte und Faser-Matrix-Interfaceeffekte, die besonders bei Oberflächenmodifikation von Kohlenstofffasern sowohl für deutlich andere elektrochemische als auch mechanische Eigenschaften des Verbundes sorgen, für große Ungenauigkeiten bei der Adaption \cite{Franco2019,Fam2024}. Hinzu kommt, dass die Modelle von Carlstedt mit zunehmender Komplexität immer mehr Materialparameter benötigen und bereits jetzt mehr als 20 teils aufwendig zu bestimmende Parameter pro Material erfordern \cite{Greenhalgh2024a}. Bis heute wurde dieser Ansatz daher vor allem zur nachträglichen Validierung und zur detaillierten Untersuchung von sensorischen und aktuatorischen Effekten von Strukturbatterien genutzt \cite{Carlstedt2023}.

Die einzige zurzeit existierende Veröffentlichung zur Vorhersage von Energiedichte und Zugsteifigkeit stammt ebenfalls von \textsc{Carlstedt} \cite{Carlstedt2018}. Jedoch wurde dazu ein in der Komplexität deutlich reduzierter Ansatz verwendet. Hinzu kommt, dass bei der Auswertung nur drei Varianten untersucht wurden, die sich in ihrer Elektrodendicke und dem Volumenanteil des Aktivmaterials unterschieden. Außerdem wurde der Einfluss des Elektrolyten nicht berücksichtigt.



% Dan Zenkert KTH Prof (Fasercehmie), 
% Dr Faye Smith OBE (Director at Avalon Consultancy Services)
% Peter Linde (DLR, CORCER Mitgleid)
% Milo Shaffer ( Professor of Materials Chemistry London)
% Natasha Shirshova (Lecturer in Engineering Materials at Durham University)
% Derrick Fam (Scientist at Institute of Materials Research and Engineering (IMRE), Adjunct Assistant Professor (NTU, MSE), Dy. Dir. Singapore Battery Consortium)
% Alexander Bismarck (Professor of Material Chemistry Wien)
% Madhavi Srinivasan (Professor at Nanyang Technological University Singapore)



   

