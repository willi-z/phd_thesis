\chapter{Abschließende Bemerkungen}

\section{Zusammenfassung und Bewertung}

Im Rahmen dieser Arbeit wurde eine computer­gestützte Auslegungsmethodik für Strukturbatterien entwickelt, die einen wesentlichen Beitrag zum Verständnis und zur praktischen Umsetzung multifunktionaler Energiespeicher leistet. Ein zentraler Schwerpunkt lag auf dem Aufbau eines wissenschaftlich fundierten Verständnisses der gekoppelten Effekte in Strukturbatterien. Dabei wurde herausgearbeitet, wie elektrochemische Prozesse, mechanische Lastpfade und thermische Einflüsse sich gegenseitig beeinflussen und gemeinsam das Gesamtverhalten dieser hochintegrierten Systeme bestimmen.

Auf dieser Grundlage konnte eine umfassende Simulationsmethodik erfolgreich entwickelt werden, die die gekoppelten elektrochemischen, mechanischen und thermischen Effekte über mehrere Skalenebenen hinweg abbildet. Die Methodik erlaubt es, sowohl material- und zellskalige Phänomene als auch das Verhalten kompletter Batteriestacks und struktureller Bauteile konsistent zu beschreiben. Durch diese multiskalige Betrachtung wird eine realitätsnahe Bewertung von Strukturbatterien bereits in frühen Entwicklungsphasen ermöglicht.

Die entwickelten Simulationsmodelle wurden anschließend erfolgreich experimentell validiert. Sowohl das mechanische Verhalten unter unterschiedlichen Belastungszuständen als auch die elektrochemische Performance konnten mit hoher Übereinstimmung zwischen Simulation und Experiment nachgewiesen werden. Diese Validierung unterstreicht die Aussagekraft und Zuverlässigkeit der vorgestellten Methodik.

Abschließend wurde die entwickelte Auslegungsmethodik erfolgreich auf die konkrete Auslegung von Strukturbatterien für den Einsatz in der mobilen Robotik angewendet. Die Ergebnisse zeigen, dass die Methodik nicht nur ein vertieftes Systemverständnis ermöglicht, sondern auch als praxisnahes Werkzeug zur gezielten Entwicklung und Optimierung von Strukturbatterien in realen Anwendungen dient. Insgesamt stellt die Arbeit damit einen wichtigen Schritt hin zu einer systematischen, simulationsbasierten Entwicklung multifunktionaler Energiespeicher dar.

\section{Ausblick}

Die in dieser Arbeit entwickelte computergestützte Auslegungsmethodik bildet eine belastbare Grundlage für die zukünftige Weiterentwicklung von Strukturbatterien, eröffnet jedoch zugleich mehrere weiterführende Forschungsrichtungen. Ein naheliegender nächster Schritt ist die Erweiterung der Simulationsmodelle um Alterungs- und Degradationsmechanismen, um die Langzeitstabilität und Lebensdauer von Strukturbatterien unter realen Betriebs- und Lastbedingungen vorhersagen zu können.

Darüber hinaus bietet die Methodik Potenzial für eine stärkere Kopplung mit effizienteren Optimierungsansätzen, um den großen Designraum multifunktionaler Materialien und Zellkonzepte effizient zu durchsuchen. Auch die Übertragung auf weitere Anwendungsfelder, etwa in der Luft- und Raumfahrt oder in der Elektromobilität, stellt ein vielversprechendes Einsatzgebiet dar.

Langfristig kann die Kombination aus experimentell validierter Simulation und anwendungsnaher Auslegung einen entscheidenden Beitrag dazu leisten, Strukturbatterien von einem Forschungskonzept zu einer robusten und breit einsetzbaren Schlüsseltechnologie für zukünftige Leichtbau- und Energiesysteme zu entwickeln.