\chapter{Abschließende Bemerkungen}

\section{Zusammenfassung und Bewertung}

Das Hauptziel der vorliegenden Dissertation war die Entwicklung einer computergestützten Berechnungsmethode zur belastungsgerechten Auslegung von Hochleistungsstrukturbatterien, um den experimentellen Aufwand zu reduzieren und den Weg in die Anwendung zu ebnen. Die Erreichung dieses Ziels lässt sich anhand der vier definierten Teilziele wie folgt zusammenfassen:

Verständnis gekoppelter Effekte (TZ 1): Durch die systematische Analyse der physikalischen Interaktionen konnte ein fundiertes Verständnis der Kopplung zwischen elektrochemischen Prozessen und mechanischen Lastpfaden erarbeitet werden. Es wurde nachgewiesen, dass die Leistungsfähigkeit von Strukturbatterien nicht allein durch die Summe der Einzelkomponenten, sondern maßgeblich durch die Grenzflächeneffekte und den Ionentransport unter mechanischer Spannung bestimmt wird. Die Identifikation dieser Schlüsselfaktoren bildete die notwendige Basis für die nachfolgende Modellierung.

Multiskalige Simulationsmethodik (TZ 2): Ein Kernstück der Arbeit ist die entwickelte skalenübergreifende Simulationsmethodik. Durch den Einsatz Repräsentativer Volumenelemente und effizienter Homogenisierungsverfahren wurde die Komplexität der Mikroebene so reduziert, dass sie für die makroskopische Bauteilauslegung handhabbar bleibt. Damit wurde die im Stand der Technik bestehende Lücke zwischen hochdetaillierten, rechenintensiven Mikro-Modellen und vereinfachten, rein empirischen Ansätzen geschlossen. Die Methodik erlaubt es nun, den weiten Designraum multifunktionaler Materialien effizient zu durchsuchen.

Experimentelle Validierung (TZ 3): Die Verlässlichkeit der entwickelten Modelle wurde durch eine sukzessive Validierung sichergestellt. Der Abgleich zwischen numerischer Vorhersage und experimentellen Daten zeigte über verschiedene Belastungsszenarien hinweg eine hohe Übereinstimmung, sowohl hinsichtlich der mechanischen Steifigkeit als auch der elektrochemischen Kapazität. Dies belegt, dass die Methodik trotz der notwendigen Komplexitätsreduktion die physikalische Realität präzise abbildet und somit als verlässliches Werkzeug für die virtuelle Materialentwicklung dienen kann.

Anwendung in multifunktionalen Strukturen (TZ 4): Die praktische Relevanz der Methode wurde schließlich durch die Auslegung realer Leichtbausysteme demonstriert. Am Beispiel eines mobilen Laufroboters sowie eines Flugzeugsitzes wurde aufgezeigt, dass durch die gezielte materialseitige Optimierung signifikante Masseneinsparungen gegenüber konventionellen Referenzsystemen möglich sind. Die Anwendung der Methodik ermöglichte dabei den Nachweis, dass der angestrebte Multifunktionalitätsbereich für reale Lastprofile bereits mit aktuellen Materialkombinationen erreichbar ist, sofern die geometrische Integration belastungsgerecht erfolgt.

Zusammenfassend wurde mit dieser Arbeit eine konsistente Kette von der grundlegenden Physik über die numerische Abbildung bis hin zur systemnahen Anwendung geschaffen. Damit leistet die Dissertation einen wesentlichen Beitrag, um Strukturbatterien von einem experimentellen Forschungsthema in eine computergestützte Ingenieursdisziplin zu überführen.


\section{Ausblick}

Die in dieser Arbeit entwickelte computergestützte Auslegungsmethodik bildet eine belastbare Grundlage für die zukünftige Weiterentwicklung von Strukturbatterien, zeigt jedoch zugleich spezifische Ansatzpunkte für weiterführende Forschungsaktivitäten auf.

Ein zentraler Aspekt, der im Rahmen der experimentellen Validierung als maßgeblicher Faktor für verbleibende Abweichungen zwischen Simulation und Messung identifiziert wurde, sind Fertigungsungenauigkeiten und prozessbedingte Varianzen. Lokale Schwankungen des Faservolumengehalts, Inhomogenitäten in der Matrixverteilung sowie fertigungsbedingte Geometrieabweichungen limitieren derzeit die prädiktive Güte rein deterministischer Modelle. Zukünftige Arbeiten sollten daher stochastische Ansätze und Sensitivitätsanalysen integrieren, um die methodische Robustheit gegenüber realen Fertigungstoleranzen zu erhöhen. 

In diesem Kontext bietet die Erweiterung der Methodik hin zu einem Digitalen Zwilling ein erhebliches Potenzial. Durch die Verknüpfung der Simulationsmodelle mit realen Prozessdaten aus der Fertigung sowie Sensordaten aus dem Betrieb könnten individuelle Bauteilzustände in Echtzeit abgebildet werden. Ein solcher Digitaler Zwilling würde nicht nur eine präzisere Vorhersage des multifunktionalen Verhaltens erlauben, sondern auch eine zustandsbasierte Überwachung ermöglichen, die den Sicherheitsanforderungen in der Luftfahrt oder Automobilindustrie Rechnung trägt.

Darüber hinaus eröffnen sich folgende vielversprechende Forschungsrichtungen:

\begin{itemize}
    \item \textbf{Langzeitstabilität und Alterung:} Die Integration von Degradationsmechanismen in die skalenübergreifende Simulation ist essenziell, um das Wechselspiel aus zyklischer mechanischer Last und elektrochemischer Alterung über die gesamte Lebensdauer zu verstehen.
    \item \textbf{KI-gestützte Optimierung:} Die Effizienz der entwickelten RVE-Modelle prädestiniert sie für die Kopplung mit Algorithmen des maschinellen Lernens. Dies würde es erlauben, den immensen Designraum neuartiger Materialkombinationen und Mikrostrukturen automatisiert zu explorieren.
    \item \textbf{Technologietransfer und Skalierung:} Die Übertragung der Methodik auf großformatige Primärstrukturen stellt den nächsten logischen Schritt dar, um die Vorteile der Multifunktionalität in komplexen Gesamtsystemen der Elektromobilität voll auszuschöpfen.
\end{itemize}

Zusammenfassend kann die Kombination aus physikalisch fundierter Simulation, der Berücksichtigung fertigungstechnischer Realitäten und der Evolution hin zu digitalen Abbildern den entscheidenden Beitrag dazu leisten, Strukturbatterien von einem Forschungskonzept zu einer robusten Schlüsseltechnologie zu entwickeln.