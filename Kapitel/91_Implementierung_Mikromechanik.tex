\chapter{Erweiterte Beschreibung der numerischen Implementierung in FEniCSx}
\label{app:fem_implementation}

Die numerische Umsetzung der mikroskaligen Modelle erfolgt vollständig in \texttt{FEniCSx} und nutzt dessen moderne UFL-Syntax sowie die PETSc-basierten linearen und nichtlinearen Löser. Die Implementierung ist in zwei logisch getrennte Simulationsschritte gegliedert: (i) die Erzeugung eines realistischen Zweiphasenfeldes mittels der Cahn-Hilliard-Gleichung und (ii) die anschließende monolithische Lösung des elektrochemisch-thermomechanischen Strukturbatteriemodells.

\section{Cahn-Hilliard-Preprocessing}

Die Cahn-Hilliard-Gleichung wird in operatorzerlegter Form gelöst:
\begin{align}
    \frac{\partial c}{\partial t} - \nabla \cdot (M \nabla \mu) &= 0,\\
    \mu - \frac{\partial f}{\partial c} + \lambda \nabla^2 c &= 0.
\end{align}
Für die Felder $c$ und $\mu$ werden kontinuierliche Lagrange-Elemente (P1 oder P2) verwendet. Die Zeitintegration erfolgt semi-implizit, beispielsweise mittels Eyre-stabilisierter Verfahren, um die Stabilität bei großen Zeitschritten sicherzustellen. Neumann-Randbedingungen $M \nabla \mu \cdot n = 0$ gewährleisten einen geschlossenen Stoffhaushalt. Das resultierende stationäre Phasenfeld $c(\boldsymbol{x})$ wird anschließend auf das Rechennetz der Strukturbatteriesimulation übertragen und dient dort zur Definition der Materialparameter in fester und flüssiger Phase.

\section{Monolithische Mehrfeldsimulation der Strukturbatterie}

Die Strukturbatteriesimulation umfasst die Gleichungen für Ladungserhalt, Massenerhalt, Wärmeleitung und lineare Elastizität. Aufgrund der starken Kopplung zwischen den Feldern (z.\,B. konzentrationsabhängige Elastizitätsmodule, temperaturabhängige Leitfähigkeiten, Joule-Heizung) wird das Gesamtsystem monolithisch formuliert. Die Funktionsräume bestehen aus einem blockstrukturierten gemischten Ansatz:


\[
(\phi_s, \phi_e, c_s, c_e, T_s, T_e, \boldsymbol{u}),
\]


wobei die mechanischen Verschiebungen $\boldsymbol{u}$ mit Lagrange-Elementen zweiter Ordnung diskretisiert werden, während die übrigen Felder mit P1- oder P2-Elementen abgebildet werden.

Die nichtlinearen Gleichungen werden mittels eines Newton-Krylov-Verfahrens (PETSc SNES) gelöst. Der Jacobian wird vollständig aus allen Teilgleichungen zusammengesetzt und blockweise vorkonditioniert, um die Konvergenz zu verbessern. Die Randbedingungen umfassen Dirichlet- und Neumann-Bedingungen für elektrische Potenziale, Konzentrationen und mechanische Verschiebungen. Die mechanische Gleichung wird ausschließlich in der Festkörperphase gelöst, während die Elektrolytphase spannungsfrei bleibt.

\section{Zeitintegration und Stabilität}

Für die Cahn-Hilliard-Simulation sowie für die Strukturbatteriesimulation werden adaptive Zeitschritte verwendet. Die Schrittweite wird anhand der Newton-Konvergenz, der Änderung der Feldgrößen und physikalisch motivierter Stabilitätskriterien (z.\,B. diffusionsähnliche CFL-Bedingungen) angepasst. Dies ermöglicht eine effiziente Abbildung sowohl schneller transiente Prozesse (z.\,B. Stromsprünge) als auch langsamer Diffusions- und Relaxationsvorgänge.

Die Kombination aus vorgelagerter Phasenfeldberechnung und monolithischer Mehrfeldsimulation erlaubt eine konsistente und numerisch robuste Abbildung der komplexen Kopplungen in Strukturbatterien und bildet die Grundlage für die in dieser Arbeit entwickelte skalenübergreifende Auslegungsmethodik.
