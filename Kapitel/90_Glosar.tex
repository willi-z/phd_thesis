\chapter{\label{sec:glosar}Glosar}

\begin{description}
    \item[Aktivitätskoeffizient ($f_{\pm}$)] Ein Korrekturfaktor in der Thermodynamik, der die Abweichung einer realen Elektrolytlösung vom idealen Verhalten beschreibt. Er berücksichtigt die interionischen Wechselwirkungen bei höheren Konzentrationen.
    
    \item[Asymmetrischer Ladungstransferkoeffizient ($\alpha$)] Der asymmetrische Ladungstransferkoeffizient $\alpha$ ergibt sich aus dem Verhältnis der Änderungen der Aktivierungsenergie der Reduktionsmittel ($\Delta E_{\text{a,red}}$) und der Gibbs-Energie der Oxidationsmittel ($\Delta G_0$): \begin{equation} \alpha = \left|\frac{\Delta E_{\text{a,red}}}{\Delta G_0}\right|, \end{equation} wodurch er definitionsgemäß im Bereich $0 < \alpha < 1$ liegt.

    \item[Butler-Volmer-Kinetik] Beschreibt den Zusammenhang zwischen dem elektrischen Strom an einer Elektrode und dem Elektrodenpotential (Überspannung). Sie ist das elektrochemische Äquivalent zu einem nichtlinearen Reibungs- oder Dämpfungsgesetz in der Mechanik und bestimmt die Rate des Ladungstransfers über die Grenzfläche.
    
    \item[Diffusionskoeffizient ($D$)] Ein Maß für die Geschwindigkeit, mit der Teilchen (z.\,B. Ionen) in einem Medium aufgrund von Konzentrationsgradienten wandern. Er ist ein zentraler Parameter in der Beschreibung des Massentransports in Elektrolyten und Festkörpern.

    \item[Hardcarbon-Slurry] Bezeichnung für eine viskose Suspension (Paste), die im Fertigungsprozess auf den Stromableiter (hier: PX-35 Kohlenstofffasergelege) appliziert wird. Der Slurry besteht primär aus Hardcarbon (nicht-graphitierbarer Kohlenstoff) als aktivem Anodenmaterial, einem leitfähigen Additiv (z. B. Carbon Black) und einem polymeren Binderschwerpunkt. Im Gegensatz zu Graphit weist Hardcarbon eine ungeordnete Schichtstruktur auf, die eine höhere mechanische Robustheit bei der Interkalation von Lithium-Ionen bietet. In Strukturbatterien ist die Grenzflächenhaftung des ausgehärteten Slurrys auf den Kohlenstofffasern kritisch für die Schubübertragung und die elektrische Kontaktierung; fertigungsbedingte Inhomogenitäten in der Slurry-Verteilung können zu lokalen Steifigkeitseinbrüchen und vorzeitigem strukturellem Versagen führen.

    \item[Hittorfsche Überführungszahl ($t_+^0$)] Gibt den Anteil des gesamten elektrischen Stroms an, der durch eine bestimmte Ionenart (meist das Kation) transportiert wird. Sie ist entscheidend für die Ausbildung von Konzentrationsgradienten im Elektrolyten.
    
    \item[Interkalation] Das reversible Einlagern von Ionen (z.\,B. Lithium-Ionen) in die Gitterstruktur eines Festkörpers (Wirtsgitter). In der Mechanik führt dies zu einer Volumenausdehnung, die analog zur thermischen Dehnung als \textit{interkalationsinduzierte Dehnung} modelliert wird.
    
    \item[Ionische Leitfähigkeit ($\gamma$)] Ein Maß für die Fähigkeit eines Elektrolyten, elektrischen Strom durch Ionenbewegung zu leiten. Sie ist das Analogon zur elektrischen Leitfähigkeit $\sigma$ in Metallen, hängt jedoch stark von der Konzentration und der Viskosität des Mediums ab.
    
    \item[P2D-Modell (Pseudo-Two-Dimensional)] Ein hocheffizientes Batteriemodell nach \textsc{Newman}. Es reduziert die komplexe Geometrie der Elektrode auf eine Dimension entlang der Schichtdicke, während die Diffusion im Inneren der Aktivmaterialpartikel als zusätzliche (pseudo-zweite) Dimension radial berechnet wird.
    
    \item[Repräsentatives Volumenelement (RVE)] Das kleinste Volumen eines heterogenen Materials (z.\,B. Faser-Matrix-Verbund), das die makroskopischen Eigenschaften des Gesamtsystems statistisch signifikant repräsentiert. In der Homogenisierung dient es dazu, effektive Materialparameter (wie den Elastizitätstensor oder die effektive Leitfähigkeit) zu bestimmen.
    
    \item[Strukturelektrolyt] Ein multifunktionales Material, das sowohl die Aufgabe des Ionentransports (Elektrolyt) als auch die der Lastübertragung (Matrix im Faserverbund) übernimmt. Meist handelt es sich um ein Zweiphasensystem aus einem steifen Polymer und einem ionisch leitfähigen Flüssigelektrolyten.
    
    \item[Tortuosität ($\tau$)] Ein dimensionsloser Faktor, der die Verlängerung des effektiven Transportweges in porösen oder komplexen Medien beschreibt. Sie beeinflusst die effektive Diffusions- und Leitfähigkeit in Materialien wie Elektrolyten oder Festkörpern.

    \item[Überspannung ($\eta$)] Die Differenz zwischen dem aktuellen Potential einer Elektrode unter Stromfluss und ihrem Gleichgewichtspotential ($U_{\text{ocp}}$). Sie stellt die "treibende Kraft" für die elektrochemische Reaktion dar, ist jedoch gleichzeitig mit energetischen Verlusten (Wärmebildung) verbunden.
\end{description}

