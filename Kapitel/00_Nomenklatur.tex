%\chapter*{Symbolverzeichnis}
%\label{sec:Symbolverzeichnis}
\addchap{Abkürzungs- und Symbolverzeichnis}
\markboth{Symbolverzeichnis}{Symbolverzeichnis}


%Einzug erste Spalte
\newlength{\TabulatorVZ} % definiert einen neuen Längenparameter
\settowidth{\TabulatorVZ}{$m$, $n$, $i$, $\imath$, ${\bar{\imath}}$, $\jmath$\quad} % Setzt den Längenparameter auf den Wert, den der Text hat
% Einzug Einheitenspalte
\newlength{\TabulatorEH} % definiert einen neuen Längenparameter
\setlength{\TabulatorEH}{\widthof{$\si{\celsius}$; $\si{\mole\per\metre\squared\per\second}$}}

\newlength{\TabulatorTX}
\setlength{\TabulatorTX}{\textwidth}
\addtolength{\TabulatorTX}{-\TabulatorVZ-\TabulatorEH-2\tabcolsep}

{\renewcommand*{\arraystretch}{1.2}%

\section*{Abkürzungen}

\begin{longtable}{@{}p{\TabulatorVZ}@{}p{\TabulatorTX+\TabulatorEH+2\tabcolsep}@{}}
FE								& Finite Elemente \\
FEM								& Finite"=Elemente"=Methode \\
PA/PA6						& Polyamid/Polyamid-6 \\
PEEK    	        & Polyetheretherketon \\
PP								& Polypropylen

\end{longtable}

\section*{Allgemeine Notation}

\begin{longtable}{@{}p{\TabulatorVZ}@{}p{\TabulatorTX+\TabulatorEH+2\tabcolsep}@{}}
a									& Skalar \\
\textbf{a}				& Tensor 1. Stufe (Vektor)
\end{longtable}

\section*{Lateinische Buchstaben}

\begin{longtable}{@{}p{\TabulatorVZ}@{}p{\TabulatorTX}p{\TabulatorEH}@{}}
	$A$					& Fläche					&$\si{\metre\squared}$\\
	$C$					& Capazität					&$\si{\ampere\s}$\\
	$\boldsymbol{C}$	& Elastizitätstensor		&$\si{\pascal}$\\
	$c$					& Konzentration				&$\si{\mole\per\metre\cubed}$\\
	$D$					& Diffusionskonstante		&$\si{\metre\squared\per\second}$\\
	$E$					& Elastizitätsmodul			&$\si{\pascal}$\\
	$F$					& Kraft						&$\si{\newton}$\\
	$F_{\text{K}}$		& Faraday-Konstante			&$\si{\coulomb\per\mole}$\\
	$f_{\pm}$			& Aktivitätskoeffizient		&$\si{\coulomb\per\mole}$\\
	$G$					& Schubmodul				&$\si{\pascal}$\\
	$h$					& Dicke						&$\si{\metre}$\\
	$\boldsymbol{i}$	& Stromdichte				&$\si{\ampere\per\metre\squared}$\\
	$j$					& molare Ionenflussdichte	&$\si{\mole\per\metre\squared\per\second}$\\
	$R_{\text{K}}$		& Unverselle-Gaskonstante	&$\si{\joule\per\mole\per\kelvin}$\\
	$T$					& Temperatur				&$\si{\kelvin}$\\
	$t^0_+$				& Hittorfsche Überführungszahl & - \\
	$t$					& Zeit						&$\si{\second}$\\
	$U_{\theta}$ 		& Elektrochemisches Standardpotenzial	& $\si{\volt}$\\
	$V$					& Volumen					&$\si{\cubic\metre}$
\end{longtable}

\section*{Griechische Buchstaben}
\begin{longtable}{@{}p{\TabulatorVZ}@{}p{\TabulatorTX}p{\TabulatorEH}@{}}
	$\alpha$				& asymetrischer Ladungungstransferkoeffizient & - \\
	$\boldsymbol{\alpha}$   & Ausdehungstensor				& $\si{\metre\per\kelvin}; \si{\metre\metre\cubed\per\mol}$\\
	$\boldsymbol{\varepsilon}$ & Dehnung						& -	\\
	$\sigma$				& elektrische Leitfähigkeit		& ??? \\
	$\boldsymbol{\sigma}$	& Mechanischer Spannungstensor	& $\si{\pascal}$ \\
	$\sigma_{\text{B,K}}$	& Boltzmann-Konstante 			& $\si{\joule\per\kelvin}$ \\
	$\nu$					& Poissonzahl					& -	\\
	$\rho$					& Dichte						& $\si{\kilo\per\metre\cubed}$
\end{longtable}

\section*{Indizes, Exponenten und mathematische Akzente}

\begin{longtable}{@{}p{\TabulatorVZ}@{}p{\TabulatorTX+\TabulatorEH+2\tabcolsep}@{}}
	$x_{\text{A}}$				& flächenbezogenes Maß\\
	$x_{\text{AM}}$				& Aktivmaterial\\
	$x_{\text{b}}$				& Binder\\
	$x_{\text{DS}}$				& Deckschicht\\
	$x_{\text{echem}}$			& elektro"=chemisch\\
	$x_{\text{E}}$				& Elektroylte\\
	$x_{\text{exp}}$			& experimentell\\
	$x_{\text{K}}$				& Konstante \\
	$x_{\text{l}}$				& leitende Phase\\
	$x_{\text{s}}$				& speichernde Phase\\
	$x_{\text{mech}}$			& mechanisch\\
	$x_n$						& Normal zur Oberfläche	\\
	$x_{-}$						& negative Elektrode \\
	$x_{\text{OCV}}$			& Gleichgewichtsspannung (\textit{engl.} open-circuit voltage) \\
	$x_{+}$						& positive Elektrode \\
	$x_{\text{th}}$				& thermisch \\
	$x_{\text{theor}}$			& theoretisch \\
	$\vec{x}^T$					& Transponierter Vektor	\\
	$\tilde{x}$				    & Effektivwert 	\\
	$\avrg{x}$					& Mittelwert
\end{longtable}

} % Ende arraystrecth


\clearpage