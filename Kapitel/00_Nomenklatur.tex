%\chapter*{Symbolverzeichnis}
%\label{sec:Symbolverzeichnis}
\addchap{Abkürzungs- und Symbolverzeichnis}
\markboth{Symbolverzeichnis}{Symbolverzeichnis}


%Einzug erste Spalte
\newlength{\TabulatorVZ} % definiert einen neuen Längenparameter
\settowidth{\TabulatorVZ}{$m$, $n$, $i$, $\imath$, ${\bar{\imath}}$, $\jmath$\quad} % Setzt den Längenparameter auf den Wert, den der Text hat
% Einzug Einheitenspalte
\newlength{\TabulatorEH} % definiert einen neuen Längenparameter
\setlength{\TabulatorEH}{\widthof{$\si{\celsius}$; $\si{\mole\per\metre\squared\per\second}$}}

\newlength{\TabulatorTX}
\setlength{\TabulatorTX}{\textwidth}
\addtolength{\TabulatorTX}{-\TabulatorVZ-\TabulatorEH-2\tabcolsep}

{\renewcommand*{\arraystretch}{1.2}%

\section*{Abkürzungen}

\begin{longtable}{@{}p{\TabulatorVZ}@{}p{\TabulatorTX+\TabulatorEH+2\tabcolsep}@{}}
FE								& Finite Elemente \\
FEM								& Finite"=Elemente"=Methode \\
PA/PA6						& Polyamid/Polyamid-6 \\
PEEK    	        & Polyetheretherketon \\
PP								& Polypropylen

\end{longtable}

\section*{Allgemeine Notation}

\begin{longtable}{@{}p{\TabulatorVZ}@{}p{\TabulatorTX+\TabulatorEH+2\tabcolsep}@{}}
a									& Skalar \\
\textbf{a}				& Tensor 1. Stufe (Vektor)
\end{longtable}

\section*{Lateinische Buchstaben}

\begin{longtable}{@{}p{\TabulatorVZ}@{}p{\TabulatorTX}p{\TabulatorEH}@{}}
	$A$					& Fläche					        &$\si{\metre\squared}$\\
	$b_{\text{K}}$		& Boltzmann-Konstante 			    &$\si{\joule\per\kelvin}$ \\
	$C$					& Kapazität					        &$\si{\ampere\second}$\\
	$\boldsymbol{C}$	& Elastizitätstensor (4. Stufe)		&$\si{\pascal}$\\
	$c$					& Konzentration				        &$\si{\mole\per\metre\cubed}$\\
	$D$					& Diffusionskoeffizient		        &$\si{\metre\squared\per\second}$\\
	$E$					& Elastizitätsmodul			        &$\si{\pascal}$\\
	$\boldsymbol{e}_z$  & Einheitsvektor in z-Richtung      & $-$ \\
	$\mathbb{E}$        & Erwartungswert                     & $-$ \\
	$F$					& Kraft						        &$\si{\newton}$\\
	$F_{\text{K}}$		& Faraday-Konstante			        &$\si{\coulomb\per\mole}$\\
	$f_{\pm}$			& Aktivitätskoeffizient		        &$-$\\
	$G$					& Schubmodul				        &$\si{\pascal}$\\
	$\boldsymbol{G}(\boldsymbol{x}, \boldsymbol{y})$ & elastischer Greenscher Tensor & $\si{\metre\per\newton}$ \\
	$h$					& Dicke						        &$\si{\metre}$\\
	$H$                 & isotroper Verfestigungsmodul      & $\si{\pascal}$ \\
	$\boldsymbol{i}$	& Stromdichte				        &$\si{\ampere\per\metre\squared}$\\
	$I$                 & elektrischer Strom                & $\si{\ampere}$ \\
	$j$					& molare Ionenflussdichte	        &$\si{\mole\per\metre\squared\per\second}$\\
    $M$                 & Mobilität (Cahn-Hilliard-Gleichung)   &$\si{\mole\squared\second\per\kilo\gram\per\metre\cubed}$\\
    $k$                 & Wärmeleitfähigkeit                &$\si{\watt\per\metre\per\kelvin}$\\
	$L$                 & charakteristische Länge / Dicke   & $\si{\metre}$ \\
	$p$                 & Druck						        &$\si{\pascal}$\\
	$r$					& Radius/Abstand					        &$\si{\metre}$\\
	$R$                 & Festigkeitskennwert               & $\si{\pascal}$ \\
	$R_{\text{K}}$		& Universelle Gaskonstante	        &$\si{\joule\per\mole\per\kelvin}$\\
	$T$					& Temperatur				        &$\si{\kelvin}$\\
	$t^0_+$				& Hittorfsche Überführungszahl      &$-$\\
	$t$					& Zeit						        &$\si{\second}$\\
	$\text{d}t$         & Zeitdifferential                  & $\si{\second}$ \\
	$\boldsymbol{t}$    & Oberflächenspannung (Traktion)    & $\si{\pascal}$ \\
	$\boldsymbol{u}$    & Verschiebungsvektor               & $\si{\metre}$ \\
	$U$					& elektrische Potenzial				&$\si{\volt}$\\
	$U_{\theta}$ 		& elektrochemisches Standardpotenzial	& $\si{\volt}$\\
	$V$					& Volumen					        &$\si{\cubic\metre}$\\
    $\boldsymbol{x}$    & Ortsvektor                        &$\si{\metre}$ \\
	$\boldsymbol{X}_t$  & stochastischer Prozess (Ort)      & $\si{\metre}$
\end{longtable}

\section*{Griechische Buchstaben}

\begin{longtable}{@{}p{\TabulatorVZ}@{}p{\TabulatorTX}p{\TabulatorEH}@{}}
	$\alpha$				& asymmetrischer Ladungstransferkoeffizient & $-$\\
	$\boldsymbol{\alpha}$   & Ausdehnungstensor				& $\si{\metre\per\kelvin}$ / $\si{\metre\metre\cubed\per\mole}$\\
	$\varphi$           	& Phasenanteil                  & $-$\\
	$\psi$					& Volumenanteil					& $-$\\
    $\boldsymbol{\varepsilon}$ & Dehnungstensor				& $-$\\
	$\bar{\varepsilon}^p$ 	& äquivalente plastische Dehnung  & $-$ \\
	$\kappa$				& elektrische Leitfähigkeit		& $\si{\siemens\per\metre}$\\
    $\gamma$				& ionische Leitfähigkeit		& $\si{\siemens\per\metre}$\\
    $\lambda$               & Gradientenenergiekoeffizient  & $\si{\joule\per\metre}$\\
    $\dot{\lambda}$     	& plastischer Multiplikator         & $-$ \\
	$\mu$                   & chemisches Potenzial          & $\si{\joule\per\mole}$\\
	$\nu$					& Poissonzahl					& $-$\\
    $\Omega$                & Gebietsdomäne                 & $-$\\
	$\rho$					& Dichte						& $\si{\kilo\gram\per\metre\cubed}$\\
	$\tau$ 					& Tortuosität & $-$ \\
	$\boldsymbol{\sigma}$	& mechanischer Spannungstensor	& $\si{\pascal}$ 
\end{longtable}

\section*{Indizes, Exponenten und mathematische Akzente}

\begin{longtable}{@{}p{\TabulatorVZ}@{}p{\TabulatorTX+\TabulatorEH+2\tabcolsep}@{}}
	$x_{\text{A}}$				& flächenbezogenes Maß\\
	$x_{\text{AM}}$				& Aktivmaterial\\
	$x_{\text{b}}$				& Binder\\
	$x_{\text{DS}}$				& Deckschicht\\
	$x_{\text{echem}}$			& elektro"=chemisch\\
	$x_{\text{E}}$				& Elektroylte\\
	$x_{\text{exp}}$			& experimentell\\
	$x_{\text{K}}$				& Konstante \\
	$x_{\text{l}}$				& leitende Phase\\
	$x_{\text{s}}$				& speichernde Phase\\
	$x_{\text{mech}}$			& mechanisch\\
	$x_n$						& Normal zur Oberfläche	\\
	$x_{-}$						& negative Elektrode \\
	$x_{\text{OCV}}$			& Gleichgewichtsspannung (\textit{engl.} open-circuit voltage) \\
	$x_{+}$						& positive Elektrode \\
	$x_{\text{th}}$				& thermisch \\
	$x_{\text{theor}}$			& theoretisch \\
	$\vec{x}^T$					& Transponierter Vektor	\\
	$\tilde{x}$				    & Effektivwert 	\\
	$\avrg{x}$					& Mittelwert
\end{longtable}

} % Ende arraystrecth


\clearpage