\chapter{\label{ch:anhang} Anhang}

\section{\label{ch:AudiEnergie}Bestimmung der Gesamtenergiedichte}

Die in der Audi Q4 e-tron Reihe eingebaute Lithium-Ionen-Batterie können maximale Energie von 82~kWh speichern, bei einer angegeben Masse des gesamten Batteriepacks von 500~kg folgt eine Energiedichte von 164~Wh/kg \cite{Radu2021,Audi2022}. Verglichen mit reinen Lithium-Ionen-Batterien, die eine Energiedichte von 265-280~Wh/kg \cite{Armand2020} aufweisen ist dies bereits eine Reduktion von $\approx 40$~\%. Dies bedeutet, dass fast die Hälfte der Masse nicht zur Energiespeicherung beiträgt. Auf Gesamtbauteilebene ergibt sich bei einem Leergewicht von ca. 2205~kg \cite{Audi2022} eine Gesamtenergiedichte von 37~kWh/kg, was einer Reduktion von ca. 86~\% entspricht.

\section{Begrifflichkeiten}

\subsection{Columbische Effizienz}
Die Coulumbische Effizienz (CE) ist eienr der meist benutzen Metriken um die intern Reaktion die eine. Die CE vom Zyklus n ist definiert als das Verhältnis der gemessene Kapazität während des Entladevorgangs $C_{Dch}(n)$ zum Kapaziät des vorherigen Beladungsvorganges $C_{Ch}(n)$ \cite{Tornheim2020}.
Die Formel
\begin{equation}
CE = \frac{C_{Dch}(n)}{C_{Ch}(n)}
\end{equation}
gilt dabei für Aufbauten, die in einem Entladenzustand zusammengebaut werden und daher zu erst Beladen werden müssen. Zellen die in einem beladenen Zustand gefertigt werden, wie etwa Lithium-Schwefel-Batterien beginnen allerdings zu erst mit einem Entladungszyklus. Die korrekte Formel lautet in einem solchen Fall
\begin{equation}
    CE = \frac{C_{Dch}(n+1)}{C_{Ch}(n)}.
\end{equation}

\subsection{Kapaziätserhalt}
Die Kapaziätserhalt (\textit{engl.} Capacity Retention) ist eine wichtige Metrik um den Anteil an Nebenreaktionen die zu einem Kaparzitätsverlust in Batterien führen zu bemessen. Sie ist definiert als das Verhältnis von Entladungskapazität des n+1 zykluses $C_{Dch}(n+1)$ und der des n-ten Zyklus $C_{Dch}(n)$ 
\begin{equation}
    CE = \frac{C_{Dch}(n+1)}{C_{Ch}(n)}.
\end{equation}
In einigen Fällen wird CR auch im Verhältnis zur intiiallen Entlade Kapazität bestimmt, also
\begin{equation}
    CE = \frac{C_{Dch}(n)}{C_{Ch}(1)}.
\end{equation}
Dieser Ansatz ist besonders dann hilfreich wenn die Langlebigkeit zu bestimmen \cite{Tornheim2020}.
Im Gegensatz zu CE ist CR meist relevanter Hersteller und Endnutzer.

\subsection{C-Raten}
Die Auflade- oder Entladerate einer Batterie wird oft in sogenanten C-Raten angegeben. Dabei meint 1~C, dass eine voll entladene/beladene Batterie in 1~h komplett aufgeladen/entladen wird. Bei einer doppelt so hohen C-Rate wird die Batterie folglich in halber Zeit also 30~min entladen bzw. aufgeladen. Bei einer halb so hohen Auflade- bzw. Entladerate (0.5C oder C/2) benötigt die Batterie 2~h um komplettt aufgeladen oder entladen zu werden.