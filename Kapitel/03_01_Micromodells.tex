% filepath: /home/williz/Promotion/Monografie/Kapitel/03_01_Micromodells.tex

\section{\label{sec:existing_micro_models}Mikroskalenmodelle}

Die auf Mikro- oder Partikelebene ablaufenden Prozesse sind grundsätzlich unabhängig davon, ob eine konventionelle oder eine Strukturbatterie betrachtet wird. Einige dieser Prozesse spielen in konventionellen Batterien jedoch nur eine untergeordnete Rolle und werden daher häufig vernachlässigt oder vereinfacht dargestellt~\cite{Carlstedt2020a}. Der Ionentransport stellt dabei den zentralen Prozess dar~\cite{Carlstedt2019b}. Nach \textsc{Newman} bestehen signifikante Unterschiede im Transportverhalten zwischen flüssigen und festen Phasen~\cite{Newman2021}. Da sowohl konventionelle als auch strukturelle Batterien mit zweiphasigen Elektrolyten einen Ionentransport durch beide Phasen ermöglichen, lässt sich ihr Verhalten in erster Näherung durch die folgenden fünf Differentialgleichungen, auch als \textsc{Doyle}-\textsc{Fuller}-\textsc{Newman}-Modell bezeichnet, beschreiben~\cite{Plett2015}.
\begin{enumerate}
    \item Ladungserhalt in homogenen Festkörpern
    \begin{equation}
        \nabla \cdot \boldsymbol{i}_{\text{s}} = \nabla \cdot \left( - \sigma \cdot \nabla \phi_{\text{s}} \right) = 0
    \end{equation}

    \item Massenerhalt in homogenen Festkörpern
    \begin{equation}
        \frac{\partial c_{\text{s}}}{\partial t}  = \nabla \cdot \left( D_{\text{s}} \nabla c_{\text{s}} \right) = 0
    \end{equation}

    \item Massenerhalt im homogenen Elektrolyt
    \begin{equation}
        \frac{\partial c_e}{\partial t} = \nabla \cdot \left( D_e   \nabla c_e \right) - \frac{\boldsymbol{i}_{\text{e}} \cdot    \nabla t_+^0}{F_{\text{K}}} - \nabla \cdot \left( c_{\text{e}} \boldsymbol{v}_0\right)
    \end{equation}

    \item Ladungserhalt im homogenen Elektrolyt
    \begin{equation}
        \nabla \cdot \boldsymbol{i}_{\text{e}} = \nabla \cdot \left(    - \kappa \nabla \phi_{\text{e}}  -\frac{2\kappa R_{\text{K}} T}{F_{\text{K}}} \left(  1+ \frac{\partial \ln f_\pm}{\partial \ln c_{\text{e}}}\right)   \left( t_+^0-1\right) \nabla \ln c_{\text{e}} \right) = 0
    \end{equation}

    \item Ionentransport zwischen fester und flüssiger Phase
    \begin{align}
        j &= \frac{i_0}{F}\left( \exp \left(\frac{\left(1-\alpha\right)  F}{RT}\eta \right) - \exp \left(-\frac{\alpha F_{\text{K}}}{R_{\text{K}} T}  \eta\right) \right)\\
        i_0 &= n F_{\text{K}} k_{0,\text{K}} \left(\prod_i c_{o,i}\right)^{1-\alpha} \left( \prod_i c_{r,i}\right)^\alpha\\
        \eta &= (\phi_{\text{s}}-\phi_{\text{e}}) - U_{\text{ocp}}
    \end{align}
\end{enumerate}
Dabei beschreibt $\boldsymbol{i}$ die Stromdichte, $\sigma$ die elektrische Leitfähigkeit des Materials, $\phi$ das elektrische Potenzial, $c$ die Konzentration der Ladungsträger, $D$ den effektiven Diffusionskoeffizienten, $\boldsymbol{t}^0_+$ die Hittorfsche Überführungszahl der Kationen bezogen auf das Elektrolytsystem, $F_{\text{K}}$ die Faraday-Konstante, $\boldsymbol{v}_0$ die Geschwindigkeit des Elektrolyten, $\kappa$ die ionische Leitfähigkeit, $R_{\text{K}}$ die ideale Gaskonstante, $T$ die Temperatur, $f_{\pm}$ den mittleren molaren Aktivitätskoeffizienten, $j$ die molare Flussdichte der Ionen, $i_0$ die Austauschstromdichte\footnote{Vereinfacht sich für Lithium und Natrium zu: $i_0 = F_{\text{K}} k_{0,\text{K}}  c_e^{1-\alpha} (c_{s,\text{max}} - c_{s,e})^{1-\alpha} c_{s,e}^\alpha$}, $\eta$ das Reaktionsüberpotenzial, $k_{0,\text{K}}$ die effektive Reaktionsratenkonstante, $U_{\text{ocp}}$ das Open-Circuit-Potenzial (Leerlaufspannung) und $\alpha$ den asymmetrischen Ladungstransferkoeffizienten. Letzterer ist durch das Verhältnis aus Änderung der Aktivierungsenergie der Reduktionsmittel ($\Delta E_{\text{a,red}}$) und Änderung der Gibbs-Energie der Oxidationsmittel ($\Delta G_0$) definiert:
\begin{equation}
        \alpha = \left|\frac{\Delta E_{\text{a,red}}}{\Delta G_0}\right|
\end{equation}
und ist dadurch auf den Wertebereich $0 < \alpha < 1$ beschränkt.

Neben dem Ladungstransport beeinflussen auch die Temperaturentwicklung sowie die Entstehung mechanischer Spannungen das Systemverhalten. Die Temperaturverteilung in der festen und flüssigen Phase wird dabei durch die Dichte $\rho$, die spezifische Wärmekapazität $c_\text{P}$, die Wärmeleitfähigkeit $\lambda$ sowie den elektrischen Strom bestimmt~\cite{Gao2021,Katrasnik2021}.
\begin{align}
    \rho_{\text{s}} c_{\text{P,s}} \frac{\partial T_{\text{s}}}{\partial t} &= \nabla \cdot (\lambda_{\text{s}} \nabla T_{\text{s}}) - \boldsymbol{i}_{\text{s}} \cdot \nabla \phi_{\text{s}}\\
    \rho_{\text{e}} c_{\text{P,e}} \frac{\partial T_{\text{e}}}{\partial t} &= \nabla \cdot (\lambda_{\text{e}} \nabla T_{\text{e}}) - \boldsymbol{i}_{\text{e}} \cdot \nabla \phi_{\text{e}}
\end{align}

Mechanischen Spannungen kommt insbesondere im Kontext von Strukturbatterien eine zentrale Rolle zu~\cite{Carlstedt2020b}. Auch bei konventionellen Batterien werden sie als ein entscheidender Faktor für bestimmte Alterungsmechanismen berücksichtigt~\cite{Mueller2019}. Dabei können mechanische Spannungen ausschließlich in der Festkörperphase auftreten~\cite{Kaliaperumal2021,Berg2022}. Für statische und rein mechanische Problemstellungen folgt ihre Beschreibung durch die lokale Impulsbilanz:
\begin{equation}\label{eq:stress_gov}
    -\nabla \cdot \boldsymbol{\sigma} + f = \boldsymbol{0}.
\end{equation}
Für kleine Deformationen und homogene Werkstoffe kann das Deformationsverhalten durch das \textsc{Hook}sche Gesetz beschrieben werden:
\begin{equation}\label{eq:stress_material}
    \boldsymbol{\sigma} = \boldsymbol{C} \boldsymbol{\varepsilon}_{mech}
\end{equation}
Der Elastizitätstensor $\boldsymbol{C}$ wird im Kontext von Strukturbatterien in Abhängigkeit vom Material als isotrop\footnote{z.\,B. Metallelektrode, Aktivmaterial, Polymerphase},
\begin{align}
\boldsymbol{C}^{-1}_{\text{iso}} &= 
\begin{bmatrix}
    \frac{1}{E} & -\frac{\nu}{E} & -\frac{\nu}{E} & 0 & 0 & 0 \\
    -\frac{\nu}{E}& \frac{1}{E} & -\frac{\nu}{E} & 0 & 0 & 0 \\
    -\frac{\nu}{E} & -\frac{\nu}{E} & \frac{1}{E} & 0 & 0 & 0 \\
    0 & 0 & 0 & \frac{2(1+\nu)}{E} & 0 & 0 \\
    0 & 0 & 0 & 0 & \frac{2(1+\nu)}{E} & 0 \\
    0 & 0 & 0 & 0 & 0 & \frac{2(1+\nu)}{E} \\
\end{bmatrix}
\end{align}
transversal-isotrop\footnote{z.\,B. einzelne Kohlenstofffaser},
\begin{align}
\boldsymbol{C}^{-1}_{\text{trans}} &= 
\begin{bmatrix}
    \frac{1}{E_{1}} & -\frac{\nu_{12}}{E_{1}} & -\frac{\nu_{13}}{E_{1}} & 0 & 0 & 0 \\
    -\frac{\nu_{12}}{E_{1}}& \frac{1}{E_{2}} & -\frac{\nu_{23}}{E_{2}} & 0 & 0 & 0 \\
    -\frac{\nu_{13}}{E_{1}} & -\frac{\nu_{23}}{E_{2}} & \frac{1}{E_{2}} & 0 & 0 & 0 \\
    0 & 0 & 0 & \frac{2(1+\nu_{23})}{E_{2}} & 0 & 0 \\
    0 & 0 & 0 & 0 & \frac{1}{G_{31}} & 0 \\
    0 & 0 & 0 & 0 & 0 & \frac{1}{G_{12}} \\
\end{bmatrix}
\end{align}
oder orthotrop\footnote{z.\,B. Kohlenstofffasergewebe, Glasfaserseparator}:
\begin{align}
\boldsymbol{C}^{-1}_{\text{ortho}} &= 
\begin{bmatrix}
    \frac{1}{E_{1}} & -\frac{\nu_{12}}{E_{1}} & -\frac{\nu_{13}}{E_{1}} & 0 & 0 & 0 \\
    -\frac{\nu_{12}}{E_{1}}& \frac{1}{E_{2}} & -\frac{\nu_{23}}{E_{2}} & 0 & 0 & 0 \\
    -\frac{\nu_{13}}{E_{1}} & -\frac{\nu_{23}}{E_{2}} & \frac{1}{E_{3}} & 0 & 0 & 0 \\
    0 & 0 & 0 & \frac{1}{G_{23}} & 0 & 0 \\
    0 & 0 & 0 & 0 & \frac{1}{G_{31}} & 0 \\
    0 & 0 & 0 & 0 & 0 & \frac{1}{G_{12}} \\
\end{bmatrix}
\end{align}
beschrieben.

Besonders bei den Materialien, die als Interkalationsort dienen, haben Untersuchungen von \textsc{Duan}~\cite{Duan2021} gezeigt, dass die Elastizitätsmodule $E_i$, mit $i \in [1,2,3]$, näherungsweise linear von der Ionenkonzentration abhängig sind:
\begin{equation}
    E_i(c_{s}) = E_{i,0} + \frac{c_{s}}{c_{s,1}} (E_{i,1} - E_{i,0}).
\end{equation}

Die Gesamtdehnung $\boldsymbol{\varepsilon}$ ergibt sich dabei aus der Summe der elektrochemischen, thermischen und mechanischen Dehnungsanteile:
\begin{equation}\label{eq:strain_total}
    \boldsymbol{\varepsilon} = \boldsymbol{\varepsilon}_{echem} +\boldsymbol{\varepsilon}_{th} + \boldsymbol{\varepsilon}_{mech}
\end{equation}
und wird direkt aus dem Verschiebungsfeld $u$ bestimmt:
\begin{equation}\label{eq:strain_total_displacement}
    \boldsymbol{\varepsilon} = \frac{1}{2}\left[\left(\nabla u\right)^T + \left(\nabla u\right)\right].
\end{equation}
Die thermischen und elektrochemischen Dehnungsanteile hängen von den jeweiligen Ausdehnungskoeffizienten $\boldsymbol{\alpha}$ linear von der Veränderung der Temperatur beziehungsweise Konzentration ab:
\begin{align}
    \boldsymbol{\varepsilon}_{echem} &= \boldsymbol{\alpha}_{echem} \left(c_{\pm}-c_{\pm,0}\right),\\
    \boldsymbol{\varepsilon}_{th}  &= \boldsymbol{\alpha}_{th}\left( T - T_0\right).
\end{align}
\begin{figure}[!ht]
    %\raggedleft
        %\def\svgwidth{\columnwidth}
        \center
        \includegraphics[width=1.0\textwidth, angle=0]{cahn-hilliard.pdf}
        \caption{\label{fig:cahn-hilliard}a) Zufallsverteilte Ausgangswerte, b) Funktion $f(c)$ zur Separation der Konzentrationen, c)--d) Ergebnisse mit $M=0{,}2$, $\lambda = 0{,}5$}
\end{figure}
Die aus den Gleichungen abgeleitete mikroskalige Modellierung kann eingesetzt werden, um Halbzellen mit Geometrien im vergleichbaren Größenspektrum zu analysieren~\cite{Plett2015}. Zur realitätsnahen Simulation einer Strukturbatteriezelle aus zwei Fasern ist es jedoch erforderlich, auch die Struktur des Zweiphasen-Elektrolyten adäquat abzubilden~\cite{Tu2020}. Die zugrunde liegende Geometrie ergibt sich aus dem Prozess der Phasenseparierung, welcher durch die \textsc{Cahn-Hilliard}-Gleichung beschrieben werden kann~\cite{Carolan2015,Grant1993}.
\begin{align}
    \frac{\partial c}{\partial t} - \nabla \cdot M \left( \nabla \left( \frac{df}{dc} - \lambda \nabla^2 c\right) \right) &= 0 \text{ in }\Omega\\
    M\left( \nabla \left( \frac{df}{dc} - \lambda \nabla^2 c \right)\right) \cdot \boldsymbol{n} &= 0 \text{ auf }\partial\Omega
\end{align}
Als Ausgangspunkt wird dabei häufig eine zufallsbasierte Konzentrationsverteilung $c(\boldsymbol{x})$ genommen, siehe Bild~\ref{fig:cahn-hilliard}a, die nach Konvention den folgenden Zusammenhang zum Phasenanteil $\varepsilon$ aufweist:
\begin{equation}
    \varepsilon \hat{=} \frac{1}{\left| \Omega \right|}\int_\Omega c(\boldsymbol{x}) \partial \boldsymbol{x}.
\end{equation}
Dabei wird die Phasenseparation der Konzentration $c$\footnote{Nach Konvention gehören Konzentrationswerte kleiner 0{,}5 zur ersten Phase, während Werte größer 0{,}5 der zweiten Phase zugeordnet werden.}, siehe Bild~\ref{fig:cahn-hilliard}c--d, allein durch zwei Parameter $f$\footnote{Eine in $c$ nicht-konvexe Polynomfunktion 4. Grades.}, siehe Bild~\ref{fig:cahn-hilliard}b, und $M$ beschrieben. Da die \textsc{Cahn-Hilliard}-Gleichung jedoch eine Differentialgleichung vierter Ordnung ist, führt dies in der schwachen Formulierung zu Ortsableitungen zweiter Ordnung, was mit Standard-Lagrange-Elementen nicht direkt lösbar ist. Eine häufig verwendete Herangehensweise zur Lösung dieses Problems besteht darin, die Gleichung mittels Operatorzerlegung umzuformulieren:
\begin{align}
    \frac{\partial c}{\partial t} - \nabla \cdot M \nabla \mu &= 0 \text{ in }\Omega\\
    \mu - \frac{\partial f}{\partial c} + \lambda \nabla^2 c &= 0 \text{ in }\Omega
\end{align}

Das resultierende Mikroskalenmodell besteht aus einer einfasrigen Kohlenstofffanode und einer mit LFP beschichteten Kohlenstofffkathode, die durch einen Strukturelektrolyt verbunden sind, siehe Bild~\ref{fig:micro_model}a. Dabei befinden sich die Poren des Strukturelektrolyten im Nanometerbereich, während die typischen Partikelgrößen der LFP-Komponenten und Kohlenstofffasern $1~\mu\text{m}$ beziehungsweise $10~\mu\text{m}$ betragen~\cite{Chaudhary2024a,Huson2014}. Dies erfordert ein äußerst feines Rechennetz\footnote{Hier $180 \times 180 \times 640 = 20\,736\,000$ Elemente.}, um die relevanten Mikrostrukturen adäquat abzubilden, siehe Bild~\ref{fig:micro_model}b.
\begin{figure}[!ht]
    %\raggedleft
        %\def\svgwidth{\columnwidth}
        \center
        \includegraphics[width=1.0\textwidth, angle=0]{micro_model.pdf}
        \caption{\label{fig:micro_model}a) Simulation einer Zweifaser-Batterie aus einer Kohlenstofffaser als Anode und einer mit LFP beschichteten Kohlenstofffaser als Kathode, b) Blockvernetzung und Zuweisung der Domänen für die gekoppelte FE-Simulation, c) Der angelegte Strom als treibende Randbedingung über die Zeit, d) Die elektrische Spannung und Stromdichte über die Zeit, e) Die gemittelte Temperatur über die Zeit. Die Lithiumkonzentration (f), die mechanische Spannung (g) und die Temperaturverteilung (h) bei $t = 2000~\text{s}$.}
\end{figure}
In Kombination mit den nichtlinearen Differentialgleichungen und den vielfältigen physikalischen Kopplungen resultiert daraus ein komplexes Simulationsmodell\footnote{Um die Parallelisierbarkeit von Blocknetzen möglichst gut auszunutzen, werden alle benötigten Parameter allen Knoten zugewiesen. Bereiche, die nicht an den jeweiligen Prozessen teilnehmen, bekommen dafür um mehrere Größenordnungen größere beziehungsweise kleinere Parameter. Außerdem verhindert dieser Ansatz Singularitäten in der Matrix, die sonst bei isolierten Bereichen entstehen können.}. Die Simulation eines vollständigen Entlade- und Beladevorgangs wird dadurch sehr rechenintensiv\footnote{Berechnungsserver der HTWK unter Ausnutzung von zwei integrierten AMD EPYC 7F753 CPUs mit einer Taktfrequenz von $2{,}95~\text{GHz}$ und jeweils 32 Rechenkernen.} (siehe Bild~\ref{fig:micro_model}).