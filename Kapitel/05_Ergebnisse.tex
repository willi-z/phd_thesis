\chapter{Identifizierung geeigneter Materialkombinationen für verschiedene Anforderungsbereiche}
\section{\label{sec:digitalisation}Erstellung einer Materialdatenbank für Strukturbatterien}

%Das hergeleitet Modell zur multi-physikalischen Beschreibung der Strukturbatterie auf der Mikroskala benötigt in der kompletten ausführung 18 zu bestimmende Parameter für jede faserbasierte Elektrode, 11 Parameter pro Elektrolytesystem und faserbasierten Separator, und zwei Interaktionskoeffizienten für jede Kombination an Elektrode und Elektrolyte. Hinzukommen 10 Parameter die für transversal Isotropematerialien, die als Pouchbag und damit nicht an der Reaktion teilnehmen.

Zur systematischen Identifizierung von geeigneten Materialkombinationen von Strukturbatterien wurde eine umfassende Materialdatenbank aufgebaut, welche die relevanten elektrochemischen und mechanischen Eigenschaften sämtlicher beteiligter Schichten abbildet. Ziel dieser Datenbank ist es, eine konsistente und zugleich flexible Grundlage für die entwickleten Aulegungsmethodik zu schaffen und damit eine tiefgehende Analyse des Materialparameterraums zu erlauben.

Insgesamt wurden elf unterschiedliche Anodenmaterialien sowie sechs verschiedene Kathodenmaterialien identifiziert und in die Datenbank aufgenommen. Für diese elektrochemisch aktiven Schichten werden als minimale geometrische und physikalische Grundgrößen die Schichtdicke $t$, die spezifische Kapazität $C_\mathrm{s}$ sowie die Dichte $\rho$ gespeichert. Abhängig von der angenommenen Materialsymmetrie werden darüber hinaus unterschiedliche Sätze an mechanischen und elektrischen Materialparametern berücksichtigt. 

Im Fall isotroper Materialannahmen umfasst die Datenbank die spezifische elektrische Leitfähigkeit $\kappa$, das Elastizitätsmodul $E$, die Querkontraktionszahl $\nu$ sowie die Streckgrenze $R_\mathrm{m}$. Für transversal-isotrope Materialien werden jeweils zwei unabhängige Werte der elektrischen Leitfähigkeit ($\kappa_\parallel$, $\kappa_\perp$), zwei Elastizitätsmoduln ($E_\parallel$, $E_\perp$), zwei Schubmoduln ($G_\parallel$, $G_\perp$) sowie zwei Querkontraktionszahlen ($\nu_\parallel$, $\nu_\perp$) gespeichert. Ergänzt wird dieser Satz durch fünf Festigkeitskoeffizienten, welche zur Parametrisierung des Tsai-Wu-Versagenskriteriums dienen. Für orthotrope Materialien erweitert sich dieser Parametersatz auf drei Elastizitätsmoduln ($E_1$, $E_2$, $E_3$), drei Schubmoduln ($G_{12}$, $G_{23}$, $G_{13}$), drei Querkontraktionszahlen ($\nu_{12}$, $\nu_{23}$, $\nu_{13}$) sowie ebenfalls fünf Festigkeitskoeffizienten.

Optional wurde für die Anoden und Kathoden zusätzlich die interkalationsbedingte Ausdehnung $\varepsilon_\mathrm{el.chem}$ bestimmt, um volumetrische Änderungen infolge des Lade- und Entladevorgangs in der mechanischen Simulation zu berücksichtigen.

Neben den elektrochemisch aktiven Schichten wurden weiterhin vier Folienmaterialien sowie fünf verschiedene Separatormaterialien in die Datenbank integriert. Für diese Schichten werden als minimale Parameter die Dicke $t$ und die Dichte $\rho$ gespeichert. Die mechanischen Kennwerte werden, analog zu den Elektrodenmaterialien, in isotroper, transversal-isotroper oder orthotroper Form abgelegt. Während bei Folienmaterialien primär mechanische Eigenschaften betrachtet werden, kann für Separatormaterialien optional zusätzlich ein effektiver Diffusionskoeffizient $D_\mathrm{eff}$ hinterlegt werden, um den Stofftransport im Elektrolyten realistisch zu erfassen.

Für den flüssigen Elektrolyten selbst werden als minimale Eigenschaften der Diffusionskoeffizient $D$, die Dichte $\rho$ sowie das notwendige Volumen pro Batteriekapazität $V/C$ in die Datenbank aufgenommen. Darüber hinaus wurden fünf unterschiedliche Polymermaterialien als mögliche Matrizes für einen strukturellen Elektrolyten definiert. Für diese Polymere werden insbesondere die Dichte, das Elastizitätsmodul sowie die Streckgrenze als Mindestparameter gespeichert, da diese maßgeblich das mechanische Gesamtverhalten der Strukturbatterie beeinflussen.

Aus der Kombination der elf Anoden-, sechs Kathoden-, vier Folien-, fünf Separator-, eines flüssigen Elektrolyten sowie der fünf Polymermatrizes ergeben sich insgesamt
\begin{equation}
N_\mathrm{Komb} = 11 \cdot 6 \cdot 4 \cdot 5 \cdot 1 \cdot 5 = 6600
\end{equation}
potenzielle Materialkombinationen für eine Strukturbatterie. 
%Diese große Variantenvielfalt unterstreicht die Notwendigkeit einer strukturierten Datenhaltung und einer automatisierten Auswahl- und Bewertungssystematik. Die entwickelte Materialdatenbank bildet somit die Grundlage für alle weiterführenden Simulations-, Optimierungs- und Risikobewertungsprozesse im Rahmen dieser Arbeit.


\section{Ashby Logic}
Basierend auf den vorliegenden Erkenntnissen wurde eine systematischen Untersuchung von insgesamt 6600 Materialkombinationen für Batterie- und Strukturbatteriekonzepte vollzogen. Die Kombinationen wurden mithilfe der eigens entwickelten Software Quintus analysiert, welche auf Basis der zuvor beschriebenen Methoden Batteriestacks hinsichtlich relevanter mechanischer Eigenschaften für Strukturbatterien sowie ihrer Energiedichte bewertet. In den Abbildungen sind die durch Quintus ermittelten Ergebnisse als Scatterplots dargestellt. Zusätzlich ist jeweils ein hellgrauer Hintergrundbereich eingezeichnet, der mögliche Abweichungen berücksichtigt, welche aus experimentellen Verifikationen der Modellannahmen resultieren.

\begin{figure}[!ht]
	%\raggedleft
		%\def\svgwidth{\columnwidth}
        \center
		\includegraphics[width=0.8\textwidth, angle=0]{quintus_tensile_energy.pdf}
		\caption{\label{fig:quintus_tensile_energy}Konvergenz der Verlustfunktion über den Trainingsprogress des neuronalen Netzwerkes.
        }
\end{figure}
Bild~\ref{fig:quintus_tensile_energy} stellt die Zugsteifigkeit in GPa auf der y-Achse in Abhängigkeit von der gravimetrischen Energiedichte in Wh/kg auf der x-Achse dar. Jeder Punkt repräsentiert eine konkrete Materialkombination. Der in der Abbildung markierte Multifunktionalitätsbereich kennzeichnet jene Kombinationen, die gegenüber einer konventionellen Funktionstrennung, bestehend aus einem klassischen Batteriestack, der in ein Kohlenstofffasergewebe einlaminiert ist, einen funktionalen Mehrwert bieten. In dieser Abbildung konnten insgesamt 1321 Kombinationen identifiziert werden, die innerhalb dieses Multifunktionalitätsbereichs liegen.

\begin{figure}[!ht]
	%\raggedleft
		%\def\svgwidth{\columnwidth}
        \center
		\includegraphics[width=0.8\textwidth, angle=0]{quintus_bending_energy.pdf}
		\caption{\label{fig:quintus_bending_energy}Konvergenz der Verlustfunktion über den Trainingsprogress des neuronalen Netzwerkes.
        }
\end{figure}
In Bild~\ref{fig:quintus_bending_energy} ist analog dazu die Biegesteifigkeit in MPa gegenüber der Energiedichte in Wh/kg aufgetragen. Auch hier sind die Ergebnisse der Quintus-Simulationen als Scatterplot dargestellt und durch einen hellgrauen Bereich ergänzt, der experimentell beobachtete Streuungen abbildet. Der Multifunktionalitätsbereich ist ebenfalls hervorgehoben und umfasst in diesem Fall 1143 Materialkombinationen, die einen Mehrwert gegenüber der konventionellen Lösung aufweisen.

In beiden Abbildungen ist zu erkennen, dass die Datenpunkte nicht homogen verteilt sind, sondern sich lokale Ansammlungen bilden. Diese Cluster lassen sich jeweils auf bestimmte Typen von Materialien und Konstruktionsprinzipien zurückführen.

Die Analyse von Bild~\ref{fig:quintus_tensile_energy} zeigt eine klare Gruppierung der Materialkombinationen in mehrere charakteristische Klassen. Diese lassen sich grob in reine konventionelle Elektroden, Kombinationen aus Kohlenstofffaser- und konventionellen Elektroden, Aluminiumpouchbags mit reinen Kohlenstofffaserelektroden sowie Kohlenstofffaserpouchbags mit reinen Kohlenstofffaserelektroden einteilen. Ein zentraler Trend ist dabei der Zielkonflikt zwischen mechanischer Leistungsfähigkeit und Energiedichte: Ein höherer Kohlenstoffanteil, realisiert etwa durch Kohlenstofffaserpouchbags und entsprechende Elektroden, führt zu einer signifikanten Erhöhung der Zugsteifigkeit. Gleichzeitig geht dieser Zugewinn jedoch mit einer Reduktion der Energiedichte einher. Umgekehrt weisen Kombinationen mit überwiegend konventionellen Elektroden tendenziell höhere Energiedichten auf, erreichen jedoch geringere Zugsteifigkeiten. Diese Beobachtung verdeutlicht den fundamentalen Trade-off zwischen struktureller Funktionalität und elektrochemischer Leistungsfähigkeit.

Bild~\ref{fig:quintus_bending_energy} offenbart ein ähnliches Grundverhalten, wobei sich die Gruppen hier nach anderen Kriterien differenzieren lassen. Die Kombinationen lassen sich unter anderem in Kohlenstofffaserpouchbags mit Strukturelektrolyt, Kohlenstofffaserpouchbags mit flüssigem Elektrolyt, konventionelle Elektroden mit Strukturelektrolyt, konventionelle Elektroden mit  flüssigem Elektrolyt, sowie hybride Kombinationen aus Kohlenstofffaserelektroden und konventionellen Elektroden einteilen. Im Vergleich zu Bild~\ref{fig:quintus_tensile_energy} wird deutlich, dass die Wahl des Pouchbags einen besonders starken Einfluss auf die Biegesteifigkeit besitzt. Ein Wechsel zu Kohlenstofffaserpouchbags führt hier zu einer deutlichen Steigerung der Biegesteifigkeit, ohne einen nachteiligen Effekt auf die Energiedichte zu verursachen.

Darüber hinaus zeigt sich, dass der Einsatz von Strukturelektrolyten eine signifikante Erhöhung der Biegesteifigkeit bewirkt. Dieser mechanische Vorteil geht jedoch mit einer starken Reduktion der Energiedichte einher, was die Notwendigkeit einer sorgfältigen Abwägung zwischen strukturellen und energetischen Anforderungen unterstreicht. Insgesamt bestätigen beide Abbildungen, dass multifunktionale Strukturbatterien zwar ein hohes Potenzial zur Gewichts- und Funktionsintegration bieten, ihre optimale Auslegung jedoch maßgeblich von der gezielten Auswahl und Kombination der verwendeten Materialien abhängt.


