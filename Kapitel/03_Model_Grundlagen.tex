\chapter{\label{sec:modelling_SB}Grundlagen und Herleitung mikroskaliger Modellgleichungen für Strukturbatterien}

Die Entwicklung einer belastungsgerechten, multiphysikalischen Auslegungsmethodik für Strukturbatterien erfordert ein präzises Verständnis der maßgeblichen mikroskaligen Prozesse. Im Mittelpunkt stehen die grundlegenden Gleichungen, die das elektrochemische, thermische und mechanische Verhalten der Funktionsmaterialien beschreiben und den theoretischen Rahmen der weiteren Modellierung bilden.
Zugleich wird deutlich, dass eine direkte Simulation der Mikroskala aufgrund der geometrischen Komplexität und der Vielzahl gekoppelter physikalischer Mechanismen nur eingeschränkt praktikabel ist. Die systematische Darstellung der Mikrophysik schafft daher die Grundlage für eine spätere Reduktion der Modellkomplexität und die Überführung in makroskopische Beschreibungen.
Auf dieser Basis lassen sich die folgenden skalenübergreifenden Modellierungsansätze konsistent entwickeln und fachlich einordnen.

\section{\label{sec:existing_micro_models}Mikroskalenmodelle}

Die mikroskalige Beschreibung bildet die physikalische Grundlage für das Verständnis der in Strukturbatterien ablaufenden elektrochemischen, thermischen und mechanischen Prozesse. Obwohl viele dieser Vorgänge sowohl in konventionellen als auch in strukturellen Batterien auftreten, gewinnen sie im Kontext tragender Energiespeicher aufgrund der zusätzlichen mechanischen Belastungen und der damit verbundenen Kopplungseffekte eine deutlich höhere Relevanz. %Die folgenden Abschnitte führen die zentralen mikroskaligen Gleichungen ein, die das Verhalten der Funktionsmaterialien bestimmen und damit die Basis für die spätere Homogenisierung und makroskalige Modellbildung bilden.

\subsection{Elektrochemische Grundgleichungen}

Der Ionentransport stellt den dominierenden Prozess auf der Mikroskala dar~\cite{Carlstedt2019b}. Nach \textsc{Newman} unterscheiden sich die Transportmechanismen in flüssigen und festen Phasen signifikant~\cite{Newman2021}. Da sowohl konventionelle als auch strukturelle Batterien zweiphasige Elektrolyte nutzen, lässt sich ihr Verhalten in erster Näherung durch das etablierte \textsc{Doyle}-\textsc{Fuller}-\textsc{Newman}-Modell beschreiben~\cite{Plett2015}. Dieses umfasst die folgenden Erhaltungsgleichungen:

\begin{enumerate}
    \item \textbf{Ladungserhalt im Festkörper}
    \begin{equation}
        \nabla \cdot \boldsymbol{i}_{\text{s}} = \nabla \cdot \left( - \sigma \nabla \phi_{\text{s}} \right) = 0
    \end{equation}

    \item \textbf{Massenerhalt im Festkörper}
    \begin{equation}
        \frac{\partial c_{\text{s}}}{\partial t} = \nabla \cdot \left( D_{\text{s}} \nabla c_{\text{s}} \right)
    \end{equation}

    \item \textbf{Massenerhalt im Elektrolyten}
    \begin{equation}
        \frac{\partial c_e}{\partial t} = \nabla \cdot (D_e \nabla c_e) - \frac{\boldsymbol{i}_{\text{e}} \cdot \nabla t_+^0}{F_{\text{K}}} - \nabla \cdot (c_{\text{e}} \boldsymbol{v}_0)
    \end{equation}

    \item \textbf{Ladungserhalt im Elektrolyten}
    \begin{equation}
        \nabla \cdot \boldsymbol{i}_{\text{e}} = \nabla \cdot \left( - \kappa \nabla \phi_{\text{e}} - \frac{2\kappa R_{\text{K}} T}{F_{\text{K}}} \left(1 + \frac{\partial \ln f_\pm}{\partial \ln c_{\text{e}}}\right)(t_+^0 - 1)\nabla \ln c_{\text{e}} \right) = 0
    \end{equation}

    \item \textbf{Ionentransport zwischen fester und flüssiger Phase}
    \begin{align}
        j &= \frac{i_0}{F}\left[\exp\left(\frac{(1-\alpha)F}{RT}\eta\right) - \exp\left(-\frac{\alpha F_{\text{K}}}{R_{\text{K}}T}\eta\right)\right],\\
        i_0 &= n F_{\text{K}} k_{0,\text{K}} \left(\prod_i c_{o,i}\right)^{1-\alpha} \left(\prod_i c_{r,i}\right)^\alpha,\\
        \eta &= (\phi_{\text{s}} - \phi_{\text{e}}) - U_{\text{ocp}}.
    \end{align}
\end{enumerate}

Die Parameter umfassen unter anderem die Stromdichte $\boldsymbol{i}$, die elektrische Leitfähigkeit $\sigma$, das elektrische Potenzial $\phi$, die Konzentration der Ladungsträger $c$, den Diffusionskoeffizienten $D$, die Hittorfsche Überführungszahl $t_+^0$, die Faraday-Konstante $F_{\text{K}}$, die Elektrolytgeschwindigkeit $\boldsymbol{v}_0$, die ionische Leitfähigkeit $\kappa$, der asymmetrische Ladungstransferkoeffizient $\alpha$ die ideale Gaskonstante $R_{\text{K}}$, die Temperatur $T$ sowie den Aktivitätskoeffizienten $f_{\pm}$.

\subsection{Thermische und mechanische Kopplung}

Neben dem Ladungs- und Stofftransport beeinflussen auch Temperaturentwicklung und mechanische Spannungen das Verhalten der Batterie. Die Temperaturverteilung in fester und flüssiger Phase ergibt sich aus der Wärmeleitungsgleichung, ergänzt um Joule-Heizterme:
\begin{align}
    \rho_{\text{s}} c_{\text{P,s}} \frac{\partial T_{\text{s}}}{\partial t} &= \nabla \cdot (\lambda_{\text{s}} \nabla T_{\text{s}}) - \boldsymbol{i}_{\text{s}} \cdot \nabla \phi_{\text{s}},\\
    \rho_{\text{e}} c_{\text{P,e}} \frac{\partial T_{\text{e}}}{\partial t} &= \nabla \cdot (\lambda_{\text{e}} \nabla T_{\text{e}}) - \boldsymbol{i}_{\text{e}} \cdot \nabla \phi_{\text{e}}.
\end{align}

Mechanische Spannungen treten ausschließlich in der Festkörperphase auf~\cite{Kaliaperumal2021,Berg2022} und sind insbesondere für Strukturbatterien von zentraler Bedeutung~\cite{Carlstedt2020b}. Die lokale Impulsbilanz lautet:
\begin{equation}\label{eq:stress_gov_final}
    -\nabla \cdot \boldsymbol{\sigma} + f = \boldsymbol{0}.
\end{equation}
Für kleine Deformationen und homogene Werkstoffe gilt das lineare elastische Materialgesetz nach \textsc{Hooke}:
\begin{equation}\label{eq:stress_material_final}
    \boldsymbol{\sigma} = \boldsymbol{C}\,\boldsymbol{\varepsilon}_{\text{mech}}.
\end{equation}

Der Elastizitätstensor $\boldsymbol{C}$ wird je nach Materialklasse isotrop, transversal-isotrop oder orthotrop formuliert. Für interkalationsaktive Materialien zeigen Untersuchungen von \textsc{Duan}~\cite{Duan2021}, dass die Elastizitätsmodule $E_i$ linear von der Ionenkonzentration abhängen:
\begin{equation}
    E_i(c_s) = E_{i,0} + \frac{c_s}{c_{s,1}}(E_{i,1} - E_{i,0}).
\end{equation}

Die Gesamtdehnung setzt sich aus mechanischen, thermischen und elektrochemischen Anteilen zusammen:
\begin{equation}
    \boldsymbol{\varepsilon} = \boldsymbol{\varepsilon}_{\text{mech}} + \boldsymbol{\varepsilon}_{\text{th}} + \boldsymbol{\varepsilon}_{\text{echem}},
\end{equation}
wobei die Dehnung aus dem Verschiebungsfeld $u$ folgt:
\begin{equation}
    \boldsymbol{\varepsilon} = \tfrac{1}{2}\left[(\nabla u)^T + \nabla u\right].
\end{equation}

\subsection{Phasenseparation und Mikrostrukturabbildung}

Für eine realitätsnahe Abbildung des Zweiphasen-Elektrolyten wird die Mikrostruktur mittels der \textsc{Cahn-Hilliard}-Gleichung generiert~\cite{Carolan2015,Grant1993}. Diese beschreibt die zeitliche Entwicklung einer Konzentrationsverteilung $c(\boldsymbol{x})$, aus der feste und flüssige Domänen abgeleitet werden:
\begin{align}
    \frac{\partial c}{\partial t} - \nabla \cdot M \left( \nabla \left( \frac{df}{dc} - \lambda \nabla^2 c\right) \right) &= 0,\\
    M\left( \nabla \left( \frac{df}{dc} - \lambda \nabla^2 c \right)\right) \cdot \boldsymbol{n} &= 0.
\end{align}

\begin{figure}[!ht]
    \center
    \includegraphics[width=1.0\textwidth, angle=0]{cahn-hilliard.pdf}
    \caption{\label{fig:cahn-hilliard}a) Zufallsverteilte Ausgangswerte, b) Funktion $f(c)$ zur Separation der Konzentrationen, c)--d) Ergebnisse mit $M=0{,}2$, $\lambda = 0{,}5$}
\end{figure}

Aus einer zufälligen Anfangsverteilung entsteht durch Phasenseparation eine charakteristische Mikrostruktur, die anschließend als Grundlage für die Zuordnung der Materialparameter dient. Die Operatorzerlegung
\begin{align}
    \frac{\partial c}{\partial t} - \nabla \cdot M \nabla \mu &= 0,\\
    \mu - \frac{\partial f}{\partial c} + \lambda \nabla^2 c &= 0
\end{align}
ermöglicht eine numerisch stabile Lösung mit Standard-Lagrange-Elementen.

\subsection{Mikroskalige Simulation einer Zweifaser-Strukturbatterie}

Zur Veranschaulichung der zuvor hergeleiteten mikroskaligen Gleichungen wurde ein vollständiges Mehrfeldmodell einer Zweifaser-Strukturbatteriezelle realisiert. Das Modell umfasst eine einfasrige Kohlenstofffaseranode sowie eine mit LFP beschichtete Kohlenstofffaserkathode, die durch einen Strukturelektrolyten getrennt sind, siehe Bild~\ref{fig:micro_model}. Die Poren des Elektrolyten liegen im Nanometerbereich, während die charakteristischen Partikelgrößen der LFP-Komponenten und Kohlenstofffasern im Bereich von $1~\mu\text{m}$ bzw. $10~\mu\text{m}$ liegen~\cite{Chaudhary2024a,Huson2014}. Die Abbildung dieser Skalen erfordert ein sehr feines Rechennetz\footnote{Hier $180 \times 180 \times 640 = 20\,736\,000$ Elemente.}, wodurch die Simulation eines vollständigen Lade- und Entladezyklus sehr rechenintensiv wird\footnote{Berechnungsserver der HTWK mit zwei AMD EPYC 7F753 CPUs, $2{,}95~\text{GHz}$, jeweils 32 Kerne.}.

\begin{figure}[!ht]
    \center
    \includegraphics[width=1.0\textwidth, angle=0]{micro_model.pdf}
    \caption{\label{fig:micro_model}a) Simulation einer Zweifaser-Batterie aus einer Kohlenstofffaser als Anode und einer mit LFP beschichteten Kohlenstofffaser als Kathode, b) Blockvernetzung und Zuweisung der Domänen für die gekoppelte FE-Simulation, c) Der angelegte Strom als treibende Randbedingung über die Zeit, d) Die elektrische Spannung und Stromdichte über die Zeit, e) Die gemittelte Temperatur über die Zeit. Die Lithiumkonzentration (f), die mechanische Spannung (g) und die Temperaturverteilung (h) bei $t = 2000~\text{s}$.}
\end{figure}


Die numerische Umsetzung erfolgt in FEniCSx/Dolfinx und folgt einem zweistufigen Ansatz. Zunächst wird die Mikrostruktur des Zweiphasen-Elektrolyten in einem vorgelagerten Simulationslauf mittels der operatorzerlegten \textsc{Cahn-Hilliard}-Gleichung berechnet. Die Felder $c$ und $\mu$ werden mit kontinuierlichen Lagrange-Elementen diskretisiert, und die zeitliche Integration erfolgt semi-implizit, um Stabilität bei großen Zeitschritten sicherzustellen. Das resultierende stationäre Phasenfeld definiert die festen und flüssigen Domänen und wird anschließend auf das Rechennetz der Strukturbatteriesimulation übertragen.

Im zweiten Schritt wird das elektrochemisch-thermomechanische Modell als monolithisch gekoppeltes Mehrfeldproblem formuliert. Die Funktionsräume umfassen elektrische Potenziale, Konzentrationen, Temperaturen und mechanische Verschiebungen, die gemeinsam in einem blockstrukturierten Ansatz gelöst werden. Die nichtlinearen Kopplungen, insbesondere die Butler-Volmer-Kinetik, konzentrationsabhängige Elastizitätsmodule und Joule-Heizung, werden mittels eines Newton-Krylov-Verfahrens behandelt. Die Randbedingungen für Potenziale, Konzentrationen und Verschiebungen werden entsprechend den physikalischen Schnittstellen formuliert. Eine adaptive Zeitschrittsteuerung stellt sicher, dass sowohl schnelle transiente Vorgänge als auch langsame Diffusionsprozesse numerisch stabil und effizient erfasst werden.

Das realisierte Modell dient somit als konsistenter numerischer Demonstrator der zuvor eingeführten mikroskaligen Gleichungen und bildet die Grundlage für die in den folgenden Kapiteln dargestellten Homogenisierungs- und Skalierungsansätze.


\section{\label{sec:homogenisation}Homogenisierung von Mikroskalenmodellen}

Die Modellierung der einzelnen physikalischen Prozesse ist auf der Mikroskala häufig einfacher umzusetzen~\cite{Plett2015}. Mithilfe mikroskaliger Modelle lassen sich die Einflüsse der Geometrie, Verteilung und Clusterbildung präzise ermitteln~\cite{Newman2021}. Aufgrund der hohen Komplexität, die mit den verschiedenen Skalenbereichen einhergeht, ist der damit verbundene Berechnungsaufwand jedoch zu groß, um eine Vielzahl von Zellen effizient zu simulieren~\cite{Liu2019}. Daher sind makroskalige Modelle erforderlich, welche den Rechenaufwand durch Homogenisierung und geeignete Modellvereinfachungen deutlich reduzieren~\cite{Plett2015}. Darüber hinaus bestehen Abweichungen durch Skalierungseffekte sowie durch die richtige Abbildung der untersuchten Mikrostruktur und durch nicht hinreichend bestimmte Materialkennwerte.

Ein häufig verwendeter Ansatz stellt die Mittelung der physikalischen Eigenschaften über ein repräsentatives Volumenelement~(RVE) dar~\cite{Burow2016,Arunachalam2019,Li2020}. Die dazugehörigen mathematischen Grundlagen basieren auf drei Volumenmittelungstheoremen~\cite{Gray1977}.
\begin{enumerate}
    \item Volumenmittelung für ein skalares Feld $\psi$ 
    \begin{equation}
        \varepsilon_{\alpha} \overline{\nabla \psi_{\alpha}} = \nabla \left(\varepsilon_{\alpha} \bar{\psi}_{\alpha} \right) + \frac{1}{V} \iint_{A_{\alpha \beta(\boldsymbol{x},t)}}\psi_{\alpha} \hat{\boldsymbol{n}}_{\alpha} \,\mathrm{d}A,
    \end{equation}
    \item Volumenmittelung für ein Vektorfeld $\boldsymbol{\psi}$
    \begin{equation}
        \varepsilon_{\alpha} \overline{\nabla \cdot \boldsymbol{\psi}_{\alpha}} = \nabla \cdot \left(\varepsilon_{\alpha} \bar{\boldsymbol{\psi}}_{\alpha} \right) + \frac{1}{V} \iint_{A_{\alpha \beta(\boldsymbol{x},t)}}\boldsymbol{\psi}_{\alpha} \cdot \hat{\boldsymbol{n}}_{\alpha} \,\mathrm{d}A,
    \end{equation}
    \item Volumenmittelung für die zeitliche Änderung eines skalaren Feldes $\psi$ 
    \begin{equation}
        \varepsilon_{\alpha} \overline{\left[\frac{\partial \psi_{\alpha}}{\partial t}\right]} = \frac{\partial \left(\varepsilon_{\alpha} \bar{\psi}_{\alpha} \right)}{\partial t} - \frac{1}{V} \iint_{A_{\alpha \beta(\boldsymbol{x},t)}}\psi_{\alpha} \boldsymbol{v}_{\alpha \beta} \cdot \hat{\boldsymbol{n}}_{\alpha} \,\mathrm{d}A.
    \end{equation}
\end{enumerate}
Dabei beschreibt $\bar{\psi}_{\alpha}$ bzw. $\bar{\boldsymbol{\psi}}_{\alpha}$ die intrinsische Mittelung über Phase $\alpha$. Diese Mittelung wird nur über das von Phase $\alpha$ eingenommene Volumen\footnote{Hier als Zwei-Phasen-System mit der zweiten Phase $\beta$ betrachtet.} ermittelt. Die intrinsische Mittelung bietet gegenüber einer klassischen Mittelung $\langle \psi_{\alpha} \rangle$, die sich auf das Volumen des gesamten Gebiets bezieht, größere Flexibilität und Wiederverwendbarkeit\footnote{Intrinsische Werte können wegen der Unabhängigkeit vom Phasenanteil für beliebige Phasenanteile wiederverwendet werden.}. Mittels des Volumenanteils $\varepsilon_{\alpha}$
\begin{equation}
    \varepsilon_{\alpha} = \frac{V_{\alpha}(\boldsymbol{x},t)}{V} 
\end{equation}
können die beiden Mittelungsarten ineinander umgewandelt werden:
\begin{equation}
    \langle \psi_{\alpha} \rangle = \varepsilon_{\alpha} \bar{\psi}_{\alpha}.
\end{equation}

Mithilfe der drei Volumenmittelungstheoreme lassen sich die folgenden vier Gleichungen herleiten~\cite{Doyle1995}.
\begin{enumerate}
    \item Volumengemittelte Näherung des Ladungserhalts in der festen Phase der porösen Elektrode
    \begin{equation}
        \nabla \cdot \left(\sigma_{\text{eff}} \nabla \hat{\phi}_{s} \right) = a_s F_{\text{K}} \hat{j},
    \end{equation}
    \item Volumengemittelte Näherung des Ladungserhalts in der Elektrolytphase der porösen Elektrode
    \begin{equation}
        \nabla \cdot \left(\kappa_{\text{eff}} \nabla \hat{\phi}_e + \kappa_{D, \text{eff}} \nabla \ln \hat{c}_e\right) + a_s F_{\text{K}} \hat{j} = 0,
    \end{equation}
    \item Volumengemittelte Näherung des Massenerhalts in der Elektrolytphase der porösen Elektrode
    \begin{equation}
        \frac{\partial \left(\varepsilon_e \hat{c}_e \right)}{\partial t} = \nabla \cdot \left(D_{e,\text{eff}}\nabla\hat{c}_e\right) + a_s (1+t^0_+) \hat{j},
    \end{equation}
    \item Volumengemittelte Näherung der mikroskopischen Butler-Volmer-Beziehung für den Ionenphasenwechsel
    \begin{equation}
        \hat{j} = j(c_{s,e},\hat{c}_e,\hat{\phi}_s,\hat{\phi}_e).
    \end{equation}
\end{enumerate}

Analog lassen sich für die mechanische Spannung und die Temperatur die folgenden Zusammenhänge aufstellen.
\begin{enumerate}
    \item Homogenisierung der mechanischen Spannung
    \begin{equation}
    \boldsymbol{\sigma} = \boldsymbol{C}_{\text{eff}} \boldsymbol{\varepsilon}_{\text{mech}},
    \end{equation}
    \item Volumengemittelte Darstellung der Temperaturentwicklung
    \begin{equation}
        \frac{\partial (\rho c_{\text{P}} T)}{\partial t} = \nabla \cdot (\lambda \nabla T) + q.
    \end{equation}
\end{enumerate}

Die eingeführte Wärmequelle $q$ kann dabei aus den folgenden fünf Beiträgen zusammengesetzt werden~\cite{Plett2015}.
\begin{enumerate}
    \item Irreversible Wärmeentstehung durch chemische Reaktionen
    \begin{equation}
        q_i = a_{\text{s}} F_{\text{K}} \hat{j}_j \eta_{j},
    \end{equation}
    \item Reversible Wärmebildung durch Entropieänderung
    \begin{equation}
    q_{r} = a_{\text{s}} F_{\text{K}} \hat{j}_j T \frac{\partial U_{\text{ocp},j}}{\partial T},
    \end{equation}
    \item Joule-Wärme im Feststoff
    \begin{equation}
    q_{s} = \sigma_{\text{eff}}(\nabla\hat{\phi}_{\text{s}} \cdot \nabla\hat{\phi}_{\text{s}}),
    \end{equation}
    \item Joule-Wärme im Elektrolyt
    \begin{equation}
        q_{e} = \kappa_{\text{eff}}(\nabla\hat{\phi}_{\text{e}} \cdot \nabla\hat{\phi}_{\text{e}}) + \kappa_{D,\text{eff}} (\nabla \ln \hat{c}_e \cdot \nabla \hat{\phi}_{\text{e}}),
    \end{equation}
    \item Wärmeentstehung durch Kontaktwiderstände\footnote{$q_c$ gilt nur für die Elektrodenfläche und ist daher auf die Einheitsfläche bezogen; die anderen Terme sind auf das Einheitsvolumen bezogen.}
    \begin{equation}
        q_{c} = i_{\text{app}}^2 R_{\text{Kontakt}}.
    \end{equation}
\end{enumerate}

\begin{figure}[!ht]
    \center
    \includegraphics[width=0.8\textwidth, angle=0]{carlstedt.pdf}
    \caption{\label{fig:carlstedt}a) Beispielhafte Darstellung der untersuchten Kohlenstofffaser-Strukturbatterie und der LFP-Zelle nach~\cite{Carlstedt2022b}, b) zweidimensionales Modell zur Durchführung der FEM-Simulation, c) Zeitverlauf des angelegten Stroms als treibende Randbedingung, d) elektrische Spannung und Stromdichte im zeitlichen Verlauf sowie die Lithiumkonzentration zu den Zeitpunkten $t_1 = 2000\,\text{s}$ und $t_2 = 6000\,\text{s}$, e) gemittelte Temperatur über die Zeit sowie Temperaturverteilungen bei $t_1$ und $t_2$, f) mechanische Spannungskomponenten $\sigma_{11}$ und $\sigma_{22}$ zu den Zeitpunkten $t_1$ und $t_2$.}
\end{figure}

Angelehnt an Arbeiten von \textsc{Carlstedt}~\cite{Carlstedt2022b}\footnote{Die Materialwerte, Geometrie und Randbedingungen wurden der Arbeit entnommen, um einen Vergleich zu ermöglichen.} können diese Gleichungen bereits verwendet werden, um das Verhalten ganzer Zellen zu beschreiben\footnote{Hier: eine Kohlenstofffaser-LFP-Zelle} (Bild~\ref{fig:carlstedt}). Die Zelle durchläuft dabei einen Entlade- und Ladezyklus innerhalb von 2,2\,h. Die Simulationszeit betrug 34,6\,h auf einem Berechnungsserver der HTWK\footnote{Unter voller Ausnutzung von zwei eingebauten CPUs der Marke AMD EPYC 75F3 mit einer Taktrate von 2,95\,GHz und jeweils 32 Kernen.}. Der hohe Rechenaufwand bereits für einen Ladezyklus macht diesen Ansatz jedoch ungeeignet, um eine Vielzahl an Varianten und größere, mehrzellige Batteriesysteme auszulegen.

Durch Ermittlung effektiver physikalischer Eigenschaften werden die Inhomogenitäten auf der Mikroskala durch ein Kontinuum auf der Makroskala beschrieben~\cite{Plett2024}. Die Genauigkeit dieses Ansatzes hängt jedoch stark von den zu betrachtenden Längenskalen ab~\cite{Plett2015}. Lokal erhöhte Porendichten oder ähnliche inhomogene Effekte lassen sich nur aufwendig berücksichtigen~\cite{Mei2019}. Bei der Analyse deutlich größerer Skalen als die Inhomogenitäten zeigen diese Modelle hingegen eine höhere Effizienz und ausreichende Genauigkeit~\cite{Plett2015}. 

Um die Berechnungszeit weiter zu reduzieren, kann aufgrund der Butler-Volmer-Randbedingung keine Volumenmittelung für die Massenerhaltung in der festen Phase\footnote{Die Materialien, die als Interkalationsort dienen.} verwendet werden~\cite{Plett2015}. Durch Geometrievereinfachungen lassen sich jedoch Freiheitsgrade reduzieren und zusätzlicher Rechenaufwand vermeiden. Im Kontext von Strukturbatterien ist der interkalationsaktiv teilnehmende Bereich partikel- oder faserförmig und kann durch Kugeln bzw. Zylinder approximiert werden~\cite{Newman2021}. Daraus ergeben sich die nachfolgenden Gleichungen:
\begin{enumerate}
    \item Spezialfall Massenerhalt in kugelförmigen Festkörpern
    \begin{equation}
        \label{eq:diffusion_sphere}
    \frac{\partial c_{\text{s}}}{\partial t} = \frac{1}{r^2} \frac{\partial}{ \partial r} \left[ D_{\text{s}} r^2 \frac{\partial c_{\text{s}}}{\partial r}\right],
    \end{equation}
    \item Spezialfall Massenerhalt in zylindrischen Festkörpern
    \begin{equation}
        \label{eq:diffusion_cylinder}
    \frac{\partial c_{\text{s}}^{\pm}}{\partial t} = \frac{1}{r} \frac{\partial}{ \partial r} \left[ D_{\text{s}} r \frac{\partial c_{\text{s}}}{\partial r}\right] + \frac{\partial}{ \partial z}\left[D_{\text{s}}  \frac{\partial c_{\text{s}}}{\partial z}\right].
    \end{equation}
\end{enumerate}
Dabei ist für viele Szenarien die Verteilung der Konzentration in $z$-Richtung näherungsweise konstant~\cite{Wang2020c}. In diesem Fall kann der Massenerhalt in zylindrischen Festkörpern weiter vereinfacht werden:
\begin{equation}
    \frac{\partial c_{\text{s}}^{\pm}}{\partial t} = \frac{1}{r} \frac{\partial}{ \partial r} \left[ D_{\text{s}} r \frac{\partial c_{\text{s}}}{\partial r}\right].
\end{equation}
In beiden Fällen lässt sich das Interkalationsverhalten durch die Randbedingungen
\begin{align}
    \left.\frac{\partial c_{\text{s}}^{\pm}}{\partial r}\right\vert_{r=0} &= 0, \\
    \left.\frac{\partial c_{\text{s}}^{\pm}}{\partial r}\right\vert_{r=R_{\text{p,s}}^{\pm}} &= -\frac{1}{ D_{\text{s}}^\pm} j_{n}^{\pm}(x,t),
\end{align}
darstellen, wobei im Falle einer stromgesteuerten Be- und Entladung
\begin{equation}
j_{n}^{\pm}(t) = \mp \frac{I(t)}{F a^{\pm} L^{\pm}}
\end{equation}
ist~\cite{Plett2015}.

\begin{figure}[!ht]
    \center
    \includegraphics[width=0.99\textwidth, angle=0]{p2d_model.pdf}
    \caption{\label{fig:p2d_model}a) Vereinfachung und Überführung einer NMC-Zelle zu einem 2D-Modell für die FEM-Berechnung, b) elektrische Spannung über mehrere durch den Strom geprägte Lade- und Entladezyklen, c) Temperaturverlauf während der Zyklen, d) maximale und minimale mechanische Spannung über den betrachteten Zeitraum.}
\end{figure}

Die daraus folgenden zweidimensionalen Modelle\footnote{Eine Dimension in Dicken-/Höhenrichtung und eine weitere in Radialrichtung der Partikel oder Fasern.} (Bild~\ref{fig:p2d_model}) gelten als die effizientesten physikalisch basierten Batteriemodelle. Mit diesen lassen sich mehrere Zyklen über 65\,h in unter 43\,min simulieren\footnote{Unter voller Ausnutzung von zwei eingebauten CPUs der Marke AMD EPYC 75F3 mit einer Taktrate von 2,95\,GHz und jeweils 32 Kernen.}. Die Genauigkeit dieser Modelle ist dabei hoch und zeigt meist Abweichungen von unter 0,5~\%~\cite{Pistorio2023}. Wie in anderen Modellen werden die schwer zu bestimmenden kinetischen Parameter häufig durch Anpassung an die ersten Zyklenverläufe identifiziert, wobei als Startwerte Literaturwerte verwendet werden~\cite{Sauerteig2018,Shui2023}. Eine einheitliche Bestimmung und ein konsistenter Austausch dieser Parameter zwischen verschiedenen Modellen ist aufgrund der unterschiedlichen Modellannahmen jedoch schwierig~\cite{Madani2018}. Für eine breite Werkstoff- bzw. Komponentenauswahl im Sinne einer Vorauslegung von Strukturbatterien sind diese Modelle aufgrund der hohen Anzahl zu bestimmender Parameter oft ungeeignet~\cite{Li2022}.
