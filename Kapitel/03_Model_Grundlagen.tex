\chapter{\label{sec:modelling_SB}Grundlagen und Herleitung mikroskaliger Modellgleichungen für Strukturbatterien}

Die Entwicklung einer belastungsgerechten, multiphysikalischen Auslegungsmethodik für Strukturbatterien erfordert ein präzises Verständnis der maßgeblichen mikroskaligen Prozesse. Den theoretischen Rahmen bilden die mikroskaligen Modellgleichungen, welche das elektrochemische, thermische und mechanische Verhalten der Funktionsmaterialien beschreiben. 
Da eine direkte Auflösung der Mikroskala numerisch nicht praktikabel ist, erfolgt mittels Homogenisierungsverfahren zunächst die Überführung in eine makroskopische Kontinuumsbeschreibung. Auf dieser Basis lassen sich gezielte geometrische Vereinfachungen ableiten, die die Komplexität der ursprünglichen Mikrostruktur reduzieren und recheneffiziente, skalenübergreifende Modellierungsansätze ermöglichen.

\section{\label{sec:existing_micro_models}Mikroskalenmodelle}

Die mikroskalige Beschreibung bildet die physikalische Grundlage für das Verständnis der in Strukturbatterien ablaufenden elektrochemischen, thermischen und mechanischen Prozesse. Obwohl viele dieser Vorgänge sowohl in konventionellen als auch in strukturellen Batterien auftreten, gewinnen sie im Kontext tragender Energiespeicher aufgrund der zusätzlichen mechanischen Belastungen und der damit verbundenen Kopplungseffekte eine deutlich höhere Relevanz. %Die folgenden Abschnitte führen die zentralen mikroskaligen Gleichungen ein, die das Verhalten der Funktionsmaterialien bestimmen und damit die Basis für die spätere Homogenisierung und makroskalige Modellbildung bilden.

\subsection{Elektrochemische Grundgleichungen}

Der Ionentransport stellt den dominierenden Prozess auf der Mikroskala dar~\cite{Carlstedt2019b}. Nach \textsc{Newman} unterscheiden sich die Transportmechanismen in flüssigen und festen Phasen signifikant~\cite{Newman2021}. Da sowohl konventionelle als auch strukturelle Batterien zweiphasige Elektrolyte nutzen, lässt sich ihr Verhalten in erster Näherung durch das etablierte \textsc{Doyle}-\textsc{Fuller}-\textsc{Newman}-Modell beschreiben~\cite{Plett2015}. Dieses umfasst die folgenden Erhaltungsgleichungen:

\begin{enumerate}
    \item \textbf{Ladungserhalt im Festkörper}
    \begin{equation}
        \nabla \cdot \boldsymbol{i}_{\text{s}} = \nabla \cdot \left( - \sigma \nabla U_{\text{s}} \right) = 0
    \end{equation}

    \item \textbf{Massenerhalt im Festkörper}
    \begin{equation}
        \frac{\partial c_{\text{s}}}{\partial t} = \nabla \cdot \left( D_{\text{s}} \nabla c_{\text{s}} \right)
    \end{equation}

    \item \textbf{Massenerhalt im Elektrolyten}
    \begin{equation}
        \frac{\partial c_e}{\partial t} = \nabla \cdot (D_e \nabla c_e) - \frac{\boldsymbol{i}_{\text{e}} \cdot \nabla t_+^0}{F_{\text{K}}} - \nabla \cdot (c_{\text{e}} \boldsymbol{v}_0)
    \end{equation}

    \item \textbf{Ladungserhalt im Elektrolyten}
    \begin{equation}
        \nabla \cdot \boldsymbol{i}_{\text{e}} = \nabla \cdot \left( - \gamma \nabla U_{\text{e}} - \frac{2\gamma R_{\text{K}} T}{F_{\text{K}}} \left(1 + \frac{\partial \ln f_\pm}{\partial \ln c_{\text{e}}}\right)(t_+^0 - 1)\nabla \ln c_{\text{e}} \right) = 0
    \end{equation}

    \item \textbf{Ionentransport zwischen fester und flüssiger Phase}
    \begin{align}
        j &= \frac{i_0}{F}\left[\exp\left(\frac{(1-\alpha)F}{RT}\eta\right) - \exp\left(-\frac{\alpha F_{\text{K}}}{R_{\text{K}}T}\eta\right)\right],\\
        i_0 &= n F_{\text{K}} k_{0,\text{K}} \left(\prod_i c_{o,i}\right)^{1-\alpha} \left(\prod_i c_{r,i}\right)^\alpha,\\
        \eta &= (U_{\text{s}} - U_{\text{e}}) - U_{\text{ocp}}.
    \end{align}
\end{enumerate}

Die Parameter umfassen unter anderem die Stromdichte $\boldsymbol{i}$, die elektrische Leitfähigkeit $\sigma$, das elektrische Potenzial $U$, die Konzentration der Ladungsträger $c$, den Diffusionskoeffizienten $D$, die Hittorfsche Überführungszahl $t_+^0$, die Faraday-Konstante $F_{\text{K}}$, die Elektrolytgeschwindigkeit $\boldsymbol{v}_0$, die ionische Leitfähigkeit $\gamma$, der asymmetrische Ladungstransferkoeffizient $\alpha$ die ideale Gaskonstante $R_{\text{K}}$, die Temperatur $T$ sowie den Aktivitätskoeffizienten $f_{\pm}$.

\subsection{Thermische und mechanische Kopplung}

Neben dem Ladungs- und Stofftransport beeinflussen auch Temperaturentwicklung und mechanische Spannungen das Verhalten der Batterie. Die Temperaturverteilung in fester und flüssiger Phase ergibt sich aus der Wärmeleitungsgleichung, ergänzt um Joule-Heizterme:
\begin{align}
    \rho_{\text{s}} c_{\text{P,s}} \frac{\partial T_{\text{s}}}{\partial t} &= \nabla \cdot (\gamma_{\text{s}} \nabla T_{\text{s}}) - \boldsymbol{i}_{\text{s}} \cdot \nabla U_{\text{s}},\\
    \rho_{\text{e}} c_{\text{P,e}} \frac{\partial T_{\text{e}}}{\partial t} &= \nabla \cdot (\gamma_{\text{e}} \nabla T_{\text{e}}) - \boldsymbol{i}_{\text{e}} \cdot \nabla U_{\text{e}}.
\end{align}

Mechanische Spannungen treten ausschließlich in der Festkörperphase auf~\cite{Kaliaperumal2021,Berg2022} und sind insbesondere für Strukturbatterien von zentraler Bedeutung~\cite{Carlstedt2020b}. Die lokale Impulsbilanz lautet:
\begin{equation}\label{eq:stress_gov}
    -\nabla \cdot \boldsymbol{\sigma} + f = \boldsymbol{0}.
\end{equation}
Für kleine Deformationen und homogene Werkstoffe gilt das lineare elastische Materialgesetz nach \textsc{Hooke}:
\begin{equation}\label{eq:stress_material}
    \boldsymbol{\sigma} = \boldsymbol{C}\,\boldsymbol{\varepsilon}_{\text{mech}}.
\end{equation}

Der Elastizitätstensor $\boldsymbol{C}$ wird je nach Materialklasse isotrop, transversal-isotrop oder orthotrop formuliert. Für interkalationsaktive Materialien zeigen Untersuchungen von \textsc{Duan}~\cite{Duan2021}, dass die Elastizitätsmodule $E_i$ linear von der Ionenkonzentration abhängen:
\begin{equation}
    E_i(c_s) = E_{i,0} + \frac{c_s}{c_{s,1}}(E_{i,1} - E_{i,0}).
\end{equation}

Die Gesamtdehnung setzt sich aus mechanischen, thermischen und elektrochemischen Anteilen zusammen:
\begin{equation}
    \boldsymbol{\varepsilon} = \boldsymbol{\varepsilon}_{\text{mech}} + \boldsymbol{\varepsilon}_{\text{th}} + \boldsymbol{\varepsilon}_{\text{echem}},
\end{equation}
wobei die Dehnung aus dem Verschiebungsfeld $u$ folgt:
\begin{equation}\label{eq:strain_total_displacement}
    \boldsymbol{\varepsilon} = \tfrac{1}{2}\left[(\nabla u)^T + \nabla u\right].
\end{equation}

\subsection{Phasenseparation und Mikrostrukturabbildung}

Für eine realitätsnahe Abbildung des Zweiphasen-Elektrolyten wird die Mikrostruktur mittels der \textsc{Cahn-Hilliard}-Gleichung generiert~\cite{Carolan2015,Grant1993}. Diese beschreibt die zeitliche Entwicklung einer Konzentrationsverteilung $c(\boldsymbol{x})$, aus der feste und flüssige Domänen abgeleitet werden:
\begin{align}
    \frac{\partial c}{\partial t} - \nabla \cdot M \left( \nabla \left( \frac{df}{dc} - \lambda \nabla^2 c\right) \right) &= 0,\\
    M\left( \nabla \left( \frac{df}{dc} - \lambda \nabla^2 c \right)\right) \cdot \boldsymbol{n} &= 0.
\end{align}

\begin{figure}[!ht]
    \center
    \includegraphics[width=1.0\textwidth, angle=0]{cahn-hilliard.pdf}
    \caption{\label{fig:cahn-hilliard}a) Zufallsverteilte Ausgangswerte, b) Funktion $f(c)$ zur Separation der Konzentrationen, c)--d) Ergebnisse mit $M=0{,}2$, $\lambda = 0{,}5$}
\end{figure}

Aus einer zufälligen Anfangsverteilung entsteht durch Phasenseparation eine charakteristische Mikrostruktur, die anschließend als Grundlage für die Zuordnung der Materialparameter dient (Bild~\ref{fig:cahn-hilliard}). Die Operatorzerlegung
\begin{align}
    \frac{\partial c}{\partial t} - \nabla \cdot M \nabla \mu &= 0,\\
    \mu - \frac{\partial f}{\partial c} + \lambda \nabla^2 c &= 0
\end{align}
ermöglicht eine numerisch stabile Lösung mit Standard-Lagrange-Elementen.

\subsection{Mikroskalige Simulation einer Zweifaser-Strukturbatterie}

Zur Veranschaulichung der zuvor hergeleiteten mikroskaligen Gleichungen wurde ein vollständiges Mehrfeldmodell einer Zweifaser-Strukturbatteriezelle realisiert. Das Modell umfasst eine einfasrige Kohlenstofffaseranode sowie eine mit LFP beschichtete Kohlenstofffaserkathode, die durch einen Strukturelektrolyten getrennt sind, siehe Bild~\ref{fig:micro_model}. Die Poren des Elektrolyten liegen im Nanometerbereich, während die charakteristischen Partikelgrößen der LFP-Komponenten und Kohlenstofffasern im Bereich von $1~\mu\text{m}$ bzw. $10~\mu\text{m}$ liegen~\cite{Chaudhary2024a,Huson2014}. Die Abbildung dieser Skalen erfordert ein sehr feines Rechennetz\footnote{Hier $180 \times 180 \times 640 = 20\,736\,000$ Elemente.}, wodurch die Simulation eines vollständigen Lade- und Entladezyklus sehr rechenintensiv wird\footnote{Berechnungsserver der HTWK mit zwei AMD EPYC 7F753 CPUs, $2{,}95~\text{GHz}$, jeweils 32 Kerne.}.

\begin{figure}[!ht]
    \center
    \includegraphics[width=1.0\textwidth, angle=0]{micro_model.pdf}
    \caption{\label{fig:micro_model} a) Aufbau der Zweifaser-Strukturbatteriezelle, b) FE-Netz und Domänenzuweisung, c) angelegtes Stromprofil, d) resultierende elektrische Spannung und Stromdichte, e) gemittelte Temperatur, f) Lithiumkonzentration, g) mechanische Spannung, h) Temperaturverteilung zum Zeitpunkt $t = 2000\,\text{s}$.}
\end{figure}


Die numerische Umsetzung erfolgt in \texttt{FEniCSx} und folgt einem zweistufigen Ansatz. Zunächst wird die Mikrostruktur des Zweiphasen-Elektrolyten in einem vorgelagerten Simulationslauf mittels der operatorzerlegten \textsc{Cahn-Hilliard}-Gleichung berechnet. Die Felder $c$ und $\mu$ werden mit kontinuierlichen Lagrange-Elementen diskretisiert, und die zeitliche Integration erfolgt semi-implizit, um Stabilität bei großen Zeitschritten sicherzustellen. Das resultierende stationäre Phasenfeld definiert die festen und flüssigen Domänen und wird anschließend auf das Rechennetz der Strukturbatteriesimulation übertragen.

Im zweiten Schritt wird das elektrochemisch-thermomechanische Modell als monolithisch gekoppeltes Mehrfeldproblem formuliert. Die Funktionsräume umfassen elektrische Potenziale, Konzentrationen, Temperaturen und mechanische Verschiebungen, die gemeinsam in einem blockstrukturierten Ansatz gelöst werden. Die nichtlinearen Kopplungen, insbesondere die Butler-Volmer-Kinetik, konzentrationsabhängige Elastizitätsmodule und Joule-Heizung, werden mittels eines Newton-Krylov-Verfahrens behandelt. Die Randbedingungen für Potenziale, Konzentrationen und Verschiebungen werden entsprechend den physikalischen Schnittstellen formuliert. Eine adaptive Zeitschrittsteuerung stellt sicher, dass sowohl schnelle transiente Vorgänge als auch langsame Diffusionsprozesse numerisch stabil und recheneffizient erfasst werden.

Die Simulationsergebnisse zeigen die erwarteten qualitativen Trends: Die Zellspannung folgt dem vorgegebenen Stromprofil, die Lithiumkonzentration verschiebt sich während der Entladung in Richtung der Kathode, und die mechanischen Spannungen entstehen in Abhängigkeit der lokalen Interkalation. Auch die Temperaturentwicklung bleibt, wie für kleine Zellen typisch,c nahezu konstant. Damit bestätigt das Modell die korrekte Umsetzung der zuvor beschriebenen Gleichungen. Gleichzeitig wird deutlich, dass die mikroskalige Vollfeldsimulation für praktische Auslegungsaufgaben ungeeignet ist. Trotz moderater Lasten und kurzer Simulationsdauer entstehen Rechenzeiten im Bereich vieler Stunden. Die Kombination aus nichtlinearen Kopplungen, hoher räumlicher Auflösung und monolithischer Lösung macht das Verfahren extrem rechenintensiv. 
Das realisierte Modell dient somit als konsistenter numerischer Demonstrator der zuvor eingeführten mikroskaligen Gleichungen und bildet die Grundlage für die in den folgenden Kapiteln dargestellten Homogenisierungs- und Skalierungsansätze.


\section{\label{sec:homogenisation}Homogenisierung mikroskaliger Modellgleichungen}

Die physikalisch exakte Abbildung einzelner Prozesse ist auf der Mikroskala oft mit vergleichsweise geringem Modellierungsaufwand möglich~\cite{Plett2015}. Mikroskalige Modelle erlauben es, den Einfluss der Geometrie, der Phasenverteilung und der Clusterbildung präzise zu determinieren~\cite{Newman2021}. Aufgrund der signifikanten Skalenunterschiede und der hohen geometrischen Komplexität ist der damit verbundene Rechenaufwand jedoch zu groß, um vollständige Batteriezellen oder Systeme in praktikabler Zeit zu simulieren~\cite{Liu2019}. Daher ist der Übergang zu makroskaligen Modellen erforderlich, die den Rechenaufwand durch Homogenisierung und gezielte Modellvereinfachungen reduzieren~\cite{Plett2015}.

\subsection{Mathematische Grundlagen der Volumenmittelung}

Der Übergang von der Mikro- zur Makroskala basiert auf der Annahme, dass Inhomogenitäten auf der kleinen Skala durch ein effektives Kontinuum auf der großen Skala beschrieben werden können~\cite{Plett2024}. Ein etablierter Ansatz ist die Mittelung physikalischer Größen über ein repräsentatives Volumenelement (RVE)~\cite{Burow2016,Arunachalam2019,Li2020}. Die mathematische Basis bilden drei zentrale Volumenmittelungstheoreme~\cite{Gray1977}:

\begin{enumerate}
    \item \textbf{Volumenmittelung eines skalaren Feldes} $\psi$:
    \begin{equation}
        \varphi_{\alpha} \overline{\nabla \psi_{\alpha}} = \nabla \left(\varphi_{\alpha} \bar{\psi}_{\alpha} \right) + \frac{1}{V} \iint_{A_{\alpha \beta}(\boldsymbol{x},t)}\psi_{\alpha} \hat{\boldsymbol{n}}_{\alpha} \,\mathrm{d}A
    \end{equation}
    \item \textbf{Volumenmittelung eines Vektorfeldes} $\boldsymbol{\psi}$:
    \begin{equation}
        \varphi_{\alpha} \overline{\nabla \cdot \boldsymbol{\psi}_{\alpha}} = \nabla \cdot \left(\varphi_{\alpha} \bar{\boldsymbol{\psi}}_{\alpha} \right) + \frac{1}{V} \iint_{A_{\alpha \beta}(\boldsymbol{x},t)}\boldsymbol{\psi}_{\alpha} \cdot \hat{\boldsymbol{n}}_{\alpha} \,\mathrm{d}A
    \end{equation}
    \item \textbf{Volumenmittelung der zeitlichen Änderung}:
    \begin{equation}
        \varphi_{\alpha} \overline{\left[\frac{\partial \psi_{\alpha}}{\partial t}\right]} = \frac{\partial \left(\varphi_{\alpha} \bar{\psi}_{\alpha} \right)}{\partial t} - \frac{1}{V} \iint_{A_{\alpha \beta}(\boldsymbol{x},t)}\psi_{\alpha} \boldsymbol{v}_{\alpha \beta} \cdot \hat{\boldsymbol{n}}_{\alpha} \,\mathrm{d}A
    \end{equation}
\end{enumerate}

Hierbei beschreibt $\bar{\psi}_{\alpha}$ die intrinsische Mittelung über die Phase $\alpha$. Im Gegensatz zur klassischen Mittelung $\langle \psi_{\alpha} \rangle$, die sich auf das gesamte Volumen bezieht, bietet die intrinsische Mittelung eine höhere Flexibilität bei variierenden Phasenanteilen $\varphi_{\alpha}$, da die Kennwerte unabhängig vom Volumenanteil definiert werden können. Die Umrechnung erfolgt über den Zusammenhang $\langle \psi_{\alpha} \rangle = \varphi_{\alpha} \bar{\psi}_{\alpha}$~\cite{Doyle1995}.

\subsection{Makroskopische Bilanzgleichungen}

Durch Anwendung der Volumenmittelungstheoreme auf die mikroskaligen Erhaltungssätze lassen sich die makroskopischen Gleichungen für poröse Elektroden herleiten~\cite{Doyle1995}. Diese beschreiben die effektiven Transportvorgänge und Reaktionen innerhalb eines homogenisierten Kontinuums, wobei die mikroskalige Geometrie über effektive Parameter und Volumenanteile berücksichtigt wird. Die zentralen Gleichungen umfassen:

\begin{enumerate}
    \item \textbf{Volumengemittelter Ladungserhalt (Festkörper)}
    \begin{equation}
        \nabla \cdot \left(\sigma_{\text{eff}} \nabla \hat{U}_{\text{s}} \right) = a_{\text{s}} F_{\text{K}} \hat{j}
    \end{equation}
    \item \textbf{Volumengemittelter Ladungserhalt (Elektrolyt)}
    \begin{equation}
        \nabla \cdot \left(\gamma_{\text{eff}} \nabla \hat{U}_{\text{e}} + \gamma_{\text{D,eff}} \nabla \ln \hat{c}_{\text{e}}\right) + a_{\text{s}} F_{\text{K}} \hat{j} = 0
    \end{equation}
    \item \textbf{Volumengemittelter Massenerhalt (Elektrolyt)}
    \begin{equation}
        \frac{\partial \left(\varepsilon_{\text{e}} \hat{c}_{\text{e}} \right)}{\partial t} = \nabla \cdot \left(D_{\text{e,eff}}\nabla\hat{c}_{\text{e}}\right) + a_{\text{s}} (1-t^0_+) \hat{j}
    \end{equation}
    \item \textbf{Homogenisierte Butler-Volmer-Beziehung}
    \begin{equation}
        \hat{j} = j(c_{\text{s,e}},\hat{c}_{\text{e}},\hat{U}_{\text{s}},\hat{U}_{\text{e}})
    \end{equation}
\end{enumerate}

Für die thermische und mechanische Kopplung werden analoge Homogenisierungsansätze verwendet. Während die mechanische Spannung über den effektiven Steifigkeitstensor $\boldsymbol{\sigma} = \boldsymbol{C}_{\text{eff}} \boldsymbol{\varepsilon}_{\text{mech}}$ beschrieben wird, ergibt sich die Temperaturentwicklung aus der volumengemittelten Wärmeleitungsgleichung:
\begin{equation}
    \frac{\partial (\rho c_{\text{P}} T)}{\partial t} = \nabla \cdot (\lambda \nabla T) + q
\end{equation}

Die Wärmequelle $q$ integriert dabei die maßgeblichen physikalischen Verlustmechanismen und setzt sich aus den folgenden fünf Beiträgen zusammen~\cite{Plett2015}:

\begin{enumerate}
    \item \textbf{Irreversible Wärmeentstehung} durch chemische Reaktionen:
    \begin{equation}
        q_{\text{i}} = a_{\text{s}} F_{\text{K}} \hat{j}_j \eta_{j}
    \end{equation}
    \item \textbf{Reversible Wärmebildung} durch Entropieänderung:
    \begin{equation}
        q_{\text{r}} = a_{\text{s}} F_{\text{K}} \hat{j}_j T \frac{\partial U_{\text{ocp},j}}{\partial T}
    \end{equation}
    \item \textbf{Joule-Wärme im Feststoff}:
    \begin{equation}
        q_{\text{s}} = \gamma_{\text{eff}}(\nabla\hat{U}_{\text{s}} \cdot \nabla\hat{U}_{\text{s}})
    \end{equation}
    \item \textbf{Joule-Wärme im Elektrolyt}:
    \begin{equation}
        q_{\text{e}} = \gamma_{\text{eff}}(\nabla\hat{U}_{\text{e}} \cdot \nabla\hat{U}_{\text{e}}) + \gamma_{\text{D,eff}} (\nabla \ln \hat{c}_{\text{e}} \cdot \nabla \hat{U}_{\text{e}})
    \end{equation}
    \item \textbf{Wärmeentstehung durch Kontaktwiderstände}\footnote{$q_{\text{c}}$ wirkt auf die Elektrodenfläche; die übrigen Terme beziehen sich auf das Einheitsvolumen.}:
    \begin{equation}
        q_{\text{c}} = i_{\text{app}}^2 R_{\text{Kontakt}}
    \end{equation}
\end{enumerate}

\subsection{Numerische Implementierung und Analyse nach Carlstedt}

Die praktische Anwendung der theoretisch hergeleiteten Volumenmittelung wird zunächst anhand einer direkten numerischen Implementierung untersucht. Dieser Ansatz folgt methodisch der Vorgehensweise von \textsc{Carlstedt}~\cite{Carlstedt2022b}, um die Homogenisierungstheorie ohne zusätzliche geometrische Annahmen über die Partikelform auf eine vollständige Zelle zu übertragen.

Die numerische Umsetzung erfolgt, analog zu den Mikroskalenmodellen (Abschnitt~\ref{sec:existing_micro_models}), mithilfe der FEniCS-Plattform unter Nutzung von \textsc{Dolfinx}. Dabei werden für das resultierende System aus partiellen Differentialgleichungen vergleichbare iterative Solver und Vorkonditionierer eingesetzt. Ein wesentlicher Unterschied liegt jedoch in der reduzierten Dimensionalität: Während das Mikromodell eine vollflächige 3D-Vernetzung erfordert, nutzt dieses Modell zweidimensionale, achsensymmetrische Elemente (siehe Bild~\ref{fig:carlstedt}). Die komplexen Faser-Matrix-Grenzflächen werden hierbei durch ein homogenisiertes Kontinuum ersetzt, dessen effektive Materialeigenschaften über die Volumenanteile $\varepsilon_{\alpha}$ definiert sind. 

Trotz der methodischen Anlehnung an \textsc{Carlstedt} weist das hier implementierte Gleichungssystem signifikante Unterschiede auf. Während viele von \textsc{Carlstedt}-Gleichungen linearisiert wurden, bleiben in diesem Ansatz die physikalischen Nichtlinearitäten der Elektrochemie (insbesondere der Butler-Volmer-Kinetik) vollständig erhalten~\cite{Carlstedt2022a, Carlstedt2022b}. Zudem wird das Transportverhalten im Festkörper explizit über den konzentrationsabhängigen Diffusionskoeffizienten $D_{\text{s}}$ beschrieben, statt, auf eine Beschreibung mittels Mobilitätstensor zurückzugreifen~\cite{Carlstedt2020b,Carlstedt2019a}. Dies erlaubt eine präzisere Abbildung der transienten Konzentrationsgradienten, erhöht jedoch die numerische Komplexität.

\begin{figure}[!ht]
    \center
    \includegraphics[width=0.8\textwidth, angle=0]{carlstedt.pdf}
    \caption{\label{fig:carlstedt}a) Beispielhafte Darstellung der untersuchten Kohlenstofffaser-Strukturbatterie und der LFP-Zelle nach~\cite{Carlstedt2022b}, b) zweidimensionales Modell zur Durchführung der FEM-Simulation, c) Zeitverlauf des angelegten Stroms als treibende Randbedingung, d) elektrische Spannung und Stromdichte im zeitlichen Verlauf sowie die Lithiumkonzentration zu den Zeitpunkten $t_1 = 2000\,\text{s}$ und $t_2 = 6000\,\text{s}$, e) gemittelte Temperatur über die Zeit sowie Temperaturverteilungen bei $t_1$ und $t_2$, f) mechanische Spannungskomponenten $\sigma_{11}$ und $\sigma_{22}$ zu den Zeitpunkten $t_1$ und $t_2$.}
\end{figure}

Wissenschaftlich kritisch ist die Diskrepanz zwischen physikalischer Detailtiefe und verbleibendem numerischem Aufwand zu bewerten. Zwar bildet das Modell das globale Zellverhalten (Bild~\ref{fig:carlstedt}d, e) präzise ab, jedoch bleibt die Recheneffizienz trotz des Wechsels auf 2D-Elemente ein Hindernis. Die Rechenzeit für einen 2,2\,h-Zyklus betrug etwa 34,6\,h auf einem HPC-Serverknoten\footnote{Zwei AMD EPYC 75F3 CPUs, 2,95\,GHz, jeweils 32 Kerne.}. Dies resultiert primär aus dem Erhalt der Nichtlinearitäten und der Tatsache, dass die Diffusion weiterhin über die vollen räumlichen Koordinaten der 2D-Domäne berechnet wird. 

Zudem führt die Volumenmittelung zu einem Verlust lokaler Informationen: Die mechanischen Spannungen (Bild~\ref{fig:carlstedt}f) stellen lediglich gemittelte Kontinuumswerte dar, wodurch lokale Spitzen an den Faser-Matrix-Grenzflächen geglättet werden. Das Modell beweist somit die Validität der Homogenisierung, verdeutlicht aber gleichzeitig die Notwendigkeit für die weitere Komplexitätsreduktion im P2D-Ansatz, um für iterative Optimierungsprozesse praktikabel zu sein.

\section{Geometrische Vereinfachung und P2D-Ansatz}

Um die für iterative Optimierungsschleifen und Systemauslegungen notwendige Recheneffizienz zu erreichen, wird das Modell durch gezielte geometrische Idealisierungen der aktiven Phasen zum sogenannten Pseudo-Zweidimensionalen Modell (P2D) erweitert~\cite{Newman2021}. Während im zuvor beschriebenen Carlstedt-Modell die Konzentrationen noch über die vollen Raumkoordinaten der 2D-Domäne berechnet werden, abstrahiert der P2D-Ansatz die Geometrie der interkalationsaktiven Bereiche (Fasern oder Partikel) innerhalb des homogenisierten Kontinuums.

Hierbei werden die Kohlenstofffasern oder Aktivmaterialpartikel als idealisierte geometrische Körper betrachtet. Die Diffusion im Festkörper wird dadurch von der makroskopischen Transportrichtung entkoppelt und auf ein lokales, eindimensionales Problem in radialer Richtung reduziert (siehe Bild~\ref{fig:p2d_model}). Für die Bestimmung der Lithium-Konzentration in den Aktivmaterialien der Anode und Kathode ergeben sich daraus die beiden spezifischen Massenerhaltungsgleichungen:

\begin{enumerate}
    \item Kugelförmige Partikel\footnote{z. B. NMC/LFP}: 
    \begin{equation}
        \frac{\partial c_{\text{s}}}{\partial t} = \frac{1}{r^2} \frac{\partial}{ \partial r} \left[ D_{\text{s}} r^2 \frac{\partial c_{\text{s}}}{\partial r}\right]
    \end{equation}
    \item Zylindrische Fasern\footnote{z. B. Kohlenstofffasern}: 
    \begin{equation}
        \label{eq:diffusion_cylinder_p2d}
        \frac{\partial c_{\text{s}}^{\pm}}{\partial t} = \frac{1}{r} \frac{\partial}{ \partial r} \left[ D_{\text{s}} r \frac{\partial c_{\text{s}}}{\partial r}\right]
    \end{equation}
\end{enumerate}
Durch diese Reduktion der Dimensionalität des Diffusionsprozesses sinkt die Anzahl der zu lösenden Freiheitsgrade deutlich, ohne die wesentlichen physikalischen Transportmechanismen zu vernachlässigen.

\begin{figure}[!ht]
    \center
    \includegraphics[width=0.99\textwidth, angle=0]{p2d_model.pdf}
    \caption{\label{fig:p2d_model}a) Vereinfachung und Überführung einer NMC-Zelle zu einem 2D-Modell für die FEM-Berechnung, b) elektrische Spannung über mehrere durch den Strom geprägte Lade- und Entladezyklen, c) Temperaturverlauf während der Zyklen, d) maximale und minimale mechanische Spannung über den betrachteten Zeitraum.}
\end{figure}

Das in Abbildung~\ref{fig:p2d_model} dargestellte multiphysikalische Modell wurde ebenfalls in \texttt{FEniCSx} implementiert. Die vorliegende Implementierung ermöglicht die Simulation mehrerer Ladezyklen über einen Realzeitraum von 65\,h in einer Rechenzeit von weniger als 43\,min\footnote{Unter Verwendung von zwei CPUs des Typs AMD EPYC 75F3 mit jeweils 32 Kernen.}. Dies entspricht im Vergleich zum direkten Homogenisierungsansatz nach \textsc{Carlstedt} einer Beschleunigung um mehrere Größenordnungen. Trotz der vorgenommenen Modellvereinfachungen bleibt die Approximationsgüte hoch. In der Literatur werden für diesen Ansatz meist Abweichungen von unter 0,5\,\% gegenüber höherdimensionalen Modellen berichtet~\cite{Pistorio2023}. Sofern die Materialparameter präzise identifiziert werden können, stellt der P2D-Ansatz somit ein effizientes Werkzeug für die angestrebte skalenübergreifende Simulation dar.