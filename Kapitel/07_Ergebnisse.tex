\chapter{Identifizierung geeigneter Materialkombinationen für verschiedene Anforderungsbereiche}
\section{\label{sec:digitalisation}Erstellung einer Materialdatenbank für Strukturbatterien}

%Das hergeleitet Modell zur multi-physikalischen Beschreibung der Strukturbatterie auf der Mikroskala benötigt in der kompletten ausführung 18 zu bestimmende Parameter für jede faserbasierte Elektrode, 11 Parameter pro Elektrolytesystem und faserbasierten Separator, und zwei Interaktionskoeffizienten für jede Kombination an Elektrode und Elektrolyte. Hinzukommen 10 Parameter die für transversal Isotropematerialien, die als Pouchbag und damit nicht an der Reaktion teilnehmen.

Zur systematischen Identifizierung von geeigneten Materialkombinationen von Strukturbatterien wurde eine umfassende Materialdatenbank aufgebaut, welche die relevanten elektrochemischen und mechanischen Eigenschaften sämtlicher beteiligter Schichten abbildet. Ziel dieser Datenbank ist es, eine konsistente und zugleich flexible Grundlage für die entwickleten Aulegungsmethodik zu schaffen und damit eine tiefgehende Analyse des Materialparameterraums zu erlauben\footnote{Nähere Erläuterungen zu dieser Verfahrensweise werden in einer eignen Veröffentlichung ausgibieger erläutert~\cite{zschiebsch2024multifunctional}.}.

Insgesamt wurden elf unterschiedliche Anodenmaterialien sowie sechs verschiedene Kathodenmaterialien identifiziert und in die Datenbank aufgenommen. Für diese elektrochemisch aktiven Schichten werden als minimale geometrische und physikalische Grundgrößen die Schichtdicke $t$, die spezifische Kapazität $C_\mathrm{s}$ sowie die Dichte $\rho$ gespeichert. Abhängig von der angenommenen Materialsymmetrie werden darüber hinaus unterschiedliche Sätze an mechanischen und elektrischen Materialparametern berücksichtigt. 

Im Fall isotroper Materialannahmen umfasst die Datenbank die spezifische elektrische Leitfähigkeit $\kappa$, das Elastizitätsmodul $E$, die Querkontraktionszahl $\nu$ sowie die Streckgrenze $R_\mathrm{m}$. Für transversal-isotrope Materialien werden jeweils zwei unabhängige Werte der elektrischen Leitfähigkeit ($\kappa_\parallel$, $\kappa_\perp$), zwei Elastizitätsmoduln ($E_\parallel$, $E_\perp$), zwei Schubmoduln ($G_\parallel$, $G_\perp$) sowie zwei Querkontraktionszahlen ($\nu_\parallel$, $\nu_\perp$) gespeichert. Ergänzt wird dieser Satz durch fünf Festigkeitskoeffizienten, welche zur Parametrisierung des \textsc{Tsai-Wu}-Versagenskriteriums dienen. Für orthotrope Materialien erweitert sich dieser Parametersatz auf drei Elastizitätsmoduln ($E_1$, $E_2$, $E_3$), drei Schubmoduln ($G_{12}$, $G_{23}$, $G_{13}$), drei Querkontraktionszahlen ($\nu_{12}$, $\nu_{23}$, $\nu_{13}$) sowie ebenfalls fünf Festigkeitskoeffizienten.

Optional wurde für die Anoden und Kathoden zusätzlich die interkalationsbedingte Ausdehnung $\varepsilon_\mathrm{el.chem}$ bestimmt, um volumetrische Änderungen infolge des Lade- und Entladevorgangs in der mechanischen Simulation zu berücksichtigen.

Neben den elektrochemisch aktiven Schichten wurden weiterhin vier Folienmaterialien sowie fünf verschiedene Separatormaterialien in die Datenbank integriert. Für diese Schichten werden als minimale Parameter die Dicke $t$ und die Dichte $\rho$ gespeichert. Die mechanischen Kennwerte werden, analog zu den Elektrodenmaterialien, in isotroper, transversal-isotroper oder orthotroper Form abgelegt. Während bei Folienmaterialien primär mechanische Eigenschaften betrachtet werden, kann für Separatormaterialien optional zusätzlich ein effektiver Diffusionskoeffizient $D_\mathrm{eff}$ hinterlegt werden, um den Stofftransport im Elektrolyten realistisch zu erfassen.

Für den flüssigen Elektrolyten selbst werden als minimale Eigenschaften der Diffusionskoeffizient $D$, die Dichte $\rho$ sowie das notwendige Volumen pro Batteriekapazität $V/C$ in die Datenbank aufgenommen. Darüber hinaus wurden fünf unterschiedliche Polymermaterialien als mögliche Matrizes für einen strukturellen Elektrolyten definiert. Für diese Polymere werden insbesondere die Dichte, das Elastizitätsmodul sowie die Streckgrenze als Mindestparameter gespeichert, da diese maßgeblich das mechanische Gesamtverhalten der Strukturbatterie beeinflussen.

Aus der Kombination der elf Anoden-, sechs Kathoden-, vier Folien-, fünf Separator-, eines flüssigen Elektrolyten sowie der fünf Polymermatrizes ergeben sich insgesamt
\begin{equation}
N_\mathrm{Komb} = 11 \cdot 6 \cdot 4 \cdot 5 \cdot 1 \cdot 5 = 6600
\end{equation}
potenzielle Materialkombinationen für eine Strukturbatterie. 
%Diese große Variantenvielfalt unterstreicht die Notwendigkeit einer strukturierten Datenhaltung und einer automatisierten Auswahl- und Bewertungssystematik. Die entwickelte Materialdatenbank bildet somit die Grundlage für alle weiterführenden Simulations-, Optimierungs- und Risikobewertungsprozesse im Rahmen dieser Arbeit.


\section{Parameterstudien und Aufbereitung in Ashby-Diagrammen}

Basierend auf dem hergeleiteten Simulationsansatz wurde eine systematische Untersuchung von insgesamt 6600 Materialkombinationen für Batterie- und Strukturbatteriekonzepte durchgeführt. Die Kombinationen wurden mithilfe der eigens entwickelten Software \texttt{Quintus}\footnote{\url{https://github.com/willi-z/quintus}}~\cite{zschiebsch2024computational} analysiert. Die Software generiert aus den Materialdaten verschiedene Batteriestacks\footnote{Zur Übereinstimmung mit den ElViS-Ergebnissen werden hier nur jene Kombinationen berücksichtigt, die aus drei Anodenlagen, zwei Kathodenlagen, vier Separatoren und zwei Pouchfolien bestehen.}, welche hinsichtlich relevanter mechanischer Eigenschaften für Strukturbatterien sowie ihrer Energiedichte bewertet werden. In den Abbildungen sind die durch \texttt{Quintus} ermittelten Ergebnisse als Scatterplots dargestellt. Zusätzlich ist jeweils ein hellgrauer Hintergrundbereich eingezeichnet, der die möglichen Abweichungen\footnote{Grundlage für die elliptische Darstellung ist die Annahme, dass die Versuche einer Normalverteilung folgen, die durch Skalierungseffekte eine elliptische Form annehmen. Die Maße und Ausrichtung der Ellipse werden so bestimmt, dass die experimentellen Werte innerhalb des Bereichs liegen; zudem liegt die Vorhersage erwartungsgemäß am Rand dieses Bereichs.} berücksichtigt, welche aus experimentellen Verifikationen der Modellannahmen resultieren.


\begin{figure}[!ht]
        \centering
        \includegraphics[width=0.78\textwidth, angle=0]{quintus_bending_energy.pdf}
        \caption{\label{fig:quintus_bending_energy}Energiedichte und Biegesteifigkeit für die mit \texttt{Quintus} bestimmten Batterievarianten. Zusätzlich sind die experimentell bestimmten Datenpunkte der Strukturbatterie aus dem ElViS-Vorhaben sowie einer konventionellen Li-Ionen-Batterie dargestellt.}
\end{figure}

\begin{table}[!ht]
    \centering
    \caption{\label{tab:combinations}Materialzusammensetzung und Kennwerte von getesteten und identifizierten vielversprechenden Batterievarianten.}
    \footnotesize
    \setlength{\tabcolsep}{2pt} 
    \begin{tabularx}{\textwidth}{l >{\raggedright\arraybackslash}X >{\raggedright\arraybackslash}X >{\raggedright\arraybackslash}X >{\raggedright\arraybackslash}X c c c c}
        \toprule
        \textbf{Name} & 
        \textbf{Anode} & 
        \textbf{Kathode} &
        \makecell{\textbf{Sepa-}\\\textbf{rator}} &
        \makecell{\textbf{Elek-}\\\textbf{trolyt}} & 
        \makecell{\textbf{Pouch-}\\ \textbf{folie}} & 
        \makecell{\textbf{Energie-}\\\textbf{dichte}\\ \textbf{[\unit{\watt\hour\per\kilo\gram}]}} & 
        \makecell{\textbf{Biege-}\\\textbf{steifigkeit}\\ \textbf{[\unit{\mega\pascal}]}} & 
        \makecell{\textbf{Zug-}\\\textbf{steifigkeit}\\ \textbf{[\unit{\giga\pascal}]}} \\
        \midrule
        \textbf{ElViS}
            & PX-35 Gewebe, \ce{Cu}-Primer, Hardcarbon
            & 2x \ce{NMC622}, \ce{Al}
            & Celgard 2400 (PP)
            & LP30 + KYNAR
            & \ce{Al}
            & 43 & 152 & 13\textsuperscript{*} \\
        \addlinespace
        \makecell[tl]{\textbf{Li-Ionen-}\\\textbf{Batterie}}
            & 2x Graphit, \ce{Cu}
            & 2x \ce{NMC622}, \ce{Al}
            & Celgard 2400 (PP)
            & LP30
            & \ce{Al}
            & 75 & 47 & 4\textsuperscript{*} \\
        \addlinespace
        \textbf{A}
            & Graphit/ Faser/ Cellulose
            & T300-Gewebe, \ce{Li2S_{n}}-Katholyt
            & GF-Gewebe
            & PEO
            & T300-Lam.
            & 9\textsuperscript{*} & 287\textsuperscript{*} & 56\textsuperscript{*} \\
        \addlinespace
        \textbf{B}
            & T300-Faser, \ce{Li}-Metall
            & T300-Gewebe, \ce{Li2S_{n}}-Katholyt
            & GF-Gewebe
            & \ce{LiPF6} in \ce{DMC}:\ce{EC}
            & T300-Lam.
            & 70\textsuperscript{*} & 272\textsuperscript{*} & 25\textsuperscript{*} \\
        \addlinespace
        \textbf{C}
            & IWS \ce{Li}-Metall
            & \ce{NMC} auf \ce{Al} (doppelt)
            & Whatman GF/C
            & \ce{LiClO4}, \ce{EC}/\ce{PC}, \ce{PAN}
            & CF-PEAK
            & 106\textsuperscript{*} & 104\textsuperscript{*} & 11\textsuperscript{*} \\
        \addlinespace
        \textbf{D}
            & IWS \ce{Li}-Metall
            & IWS Slurry E-CS-1591
            & Whatman GF/C
            & \ce{LiPF6} in \ce{DMC}:\ce{EC}
            & \ce{Al}
            & 165\textsuperscript{*} & 3\textsuperscript{*} & 0,4\textsuperscript{*} \\
        \addlinespace
        \textbf{E}
            & Graphit/ Faser/ Cellulose
            & T300-Gewebe, \ce{Li2S_{n}}-Katholyt
            & GF-Gewebe
            & \ce{LiDFOB}/ \ce{LiBF4} in PVDF-HFP
            & T300-Lam.
            & 39\textsuperscript{*} & 224\textsuperscript{*} & 54\textsuperscript{*} \\
        \addlinespace
        \textbf{F}
            & T300-Faser, \ce{Li}-Metall
            & T300-Gewebe, \ce{Li2S_{n}}-Katholyt
            & GF-Gewebe
            & LP30 + Additive
            & \ce{Al}
            & 73\textsuperscript{*} & 252\textsuperscript{*} & 29\textsuperscript{*} \\
        \addlinespace
        \textbf{G}
            & T300-Faser, \ce{Li}-Metall
            & IWS Slurry E-CS-1591
            & Whatman GF/C
            & \ce{LiPF6} in \ce{DMC}:\ce{EC}
            & \ce{Al}
            & 121\textsuperscript{*} & 12\textsuperscript{*} & 13,5\textsuperscript{*} \\
        \bottomrule
    \end{tabularx}
    \raggedright
    \noindent{\footnotesize{\textsuperscript{*} berechneter Wert.}}
\end{table}

In Bild~\ref{fig:quintus_bending_energy} ist die Biegesteifigkeit in \unit{\mega\pascal} gegenüber der Energiedichte in \unit{\watt\hour\per\kilo\gram} aufgetragen. Der in der Abbildung markierte Multifunktionalitätsbereich kennzeichnet jene Kombinationen, die gegenüber einer konventionellen Funktionstrennung — bestehend aus einem klassischen Batteriestack, der in ein Kohlenstofffasergewebe einlaminiert ist — einen funktionalen Mehrwert bieten. Dieser Bereich umfasst in diesem Fall 1143 Materialkombinationen. Tabelle~\ref{tab:combinations} listet die experimentell untersuchten sowie als besonders vielversprechend identifizierten Materialzusammensetzungen und Kennwerte auf.

\begin{figure}[!ht]
        \centering
        \includegraphics[width=0.78\textwidth, angle=0]{quintus_tensile_energy.pdf}
        \caption{\label{fig:quintus_tensile_energy}Die mit \texttt{Quintus} bestimmten Batterievarianten für Zugsteifigkeit und Energiedichte dargesteltl. Zur Referenz sind bekannte Strukturbatterien aus der Literatur sowie die berechneten Werte für die ElViS- und Li-Ionen-Batterie eingezeichnet.}
\end{figure}

Bild~\ref{fig:quintus_tensile_energy} stellt die Zugsteifigkeit in \unit{\giga\pascal} auf der y-Achse in Abhängigkeit von der gravimetrischen Energiedichte in \unit{\watt\hour\per\kilo\gram} auf der x-Achse dar. Auch hier sind die Ergebnisse der \texttt{Quintus}-Simulationen als Scatterplot dargestellt. In dieser Abbildung konnten insgesamt 1321 Kombinationen identifiziert werden, die innerhalb des Multifunktionalitätsbereichs liegen.

In beiden Abbildungen ist zu erkennen, dass die Datenpunkte nicht homogen verteilt sind, sondern lokale Ansammlungen bilden. Diese Cluster lassen sich jeweils auf bestimmte Materialklassen und Konstruktionsprinzipien zurückführen. Die Analyse von Bild~\ref{fig:quintus_bending_energy} zeigt eine klare Gruppierung in mehrere charakteristische Klassen: Kohlenstofffaser-Pouchbags mit Strukturelektrolyt, Kohlenstofffaser-Pouchbags mit flüssigem Elektrolyt, konventionelle Elektroden mit Strukturelektrolyt, konventionelle Elektroden mit flüssigem Elektrolyt sowie hybride Kombinationen. Im Vergleich zu Bild~\ref{fig:quintus_tensile_energy} wird deutlich, dass die Wahl des Pouchbags einen besonders starken Einfluss auf die Biegesteifigkeit besitzt. Ein Wechsel zu Kohlenstofffaser-Pouchbags führt hier zu einer deutlichen Steigerung der Biegesteifigkeit, ohne die Energiedichte nachteilig zu beeinflussen.

Darüber hinaus zeigt sich, dass der Einsatz von Strukturelektrolyten eine signifikante Erhöhung der Biegesteifigkeit bewirkt. Dieser mechanische Vorteil geht jedoch mit einer starken Reduktion der Energiedichte einher, was die Notwendigkeit einer sorgfältigen Abwägung zwischen strukturellen und energetischen Anforderungen unterstreicht. Insgesamt bestätigen beide Abbildungen, dass multifunktionale Strukturbatterien ein hohes Potenzial zur Gewichts- und Funktionsintegration bieten, ihre optimale Auslegung jedoch maßgeblich von der gezielten Auswahl der Materialien abhängt.

Bild~\ref{fig:quintus_tensile_energy} offenbart ein ähnliches Grundverhalten, wobei sich die Gruppen nach anderen Kriterien differenzieren: reine konventionelle Elektroden, Kombinationen aus Kohlenstofffaser- und konventionellen Elektroden sowie Varianten mit Kohlenstofffaser-Pouchbags. Ein zentraler Trend ist dabei der Zielkonflikt zwischen mechanischer Leistungsfähigkeit und Energiedichte: Ein höherer Kohlenstofffaseranteil führt zu einer signifikanten Erhöhung der Zugsteifigkeit, geht jedoch gleichzeitig mit einer Reduktion der Energiedichte einher.

Zudem zeigt sich, dass neben einer optimalen Materialkombination auch die Qualität der Herstellung einen erheblichen Einfluss auf die erreichbaren Kennwerte hat. Dies wird deutlich im Vergleich zu den Arbeiten von \textsc{Asp}\cite{Asp2021} und \textsc{Siraj}\cite{Siraj2023}, bei denen der wesentliche Unterschied in den verbesserten Herstellungsprozessen liegt.




