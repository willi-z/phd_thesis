\section{\label{sec:homogenisation}Homogenisierung von Mikroskalenmodellen}

Die Modellierung der einzelnen physikalischen Prozesse ist auf der Mikroskala häufig einfacher umzusetzen~\cite{Plett2015}. Mithilfe mikroskaliger Modelle lassen sich die Einflüsse der Geometrie, Verteilung und Clusterbildung präzise ermitteln~\cite{Newman2021}. Aufgrund der hohen Komplexität, die mit den verschiedenen Skalenbereichen einhergeht, ist der damit verbundene Berechnungsaufwand jedoch zu groß, um eine Vielzahl von Zellen effizient zu simulieren~\cite{Liu2019}. Daher sind makroskalige Modelle erforderlich, welche den Rechenaufwand durch Homogenisierung und geeignete Modellvereinfachungen deutlich reduzieren~\cite{Plett2015}. Darüber hinaus bestehen Abweichungen durch Skalierungseffekte sowie durch die richtige Abbildung der untersuchten Mikrostruktur und durch nicht hinreichend bestimmte Materialkennwerte.

Ein häufig verwendeter Ansatz stellt die Mittelung der physikalischen Eigenschaften über ein repräsentatives Volumenelement~(RVE) dar~\cite{Burow2016,Arunachalam2019,Li2020}. Die dazugehörigen mathematischen Grundlagen basieren auf drei Volumenmittelungstheoremen~\cite{Gray1977}.
\begin{enumerate}
    \item Volumenmittelung für ein skalares Feld $\psi$ 
    \begin{equation}
        \varepsilon_{\alpha} \overline{\nabla \psi_{\alpha}} = \nabla \left(\varepsilon_{\alpha} \bar{\psi}_{\alpha} \right) + \frac{1}{V} \iint_{A_{\alpha \beta(\boldsymbol{x},t)}}\psi_{\alpha} \hat{\boldsymbol{n}}_{\alpha} \,\mathrm{d}A,
    \end{equation}
    \item Volumenmittelung für ein Vektorfeld $\boldsymbol{\psi}$
    \begin{equation}
        \varepsilon_{\alpha} \overline{\nabla \cdot \boldsymbol{\psi}_{\alpha}} = \nabla \cdot \left(\varepsilon_{\alpha} \bar{\boldsymbol{\psi}}_{\alpha} \right) + \frac{1}{V} \iint_{A_{\alpha \beta(\boldsymbol{x},t)}}\boldsymbol{\psi}_{\alpha} \cdot \hat{\boldsymbol{n}}_{\alpha} \,\mathrm{d}A,
    \end{equation}
    \item Volumenmittelung für die zeitliche Änderung eines skalaren Feldes $\psi$ 
    \begin{equation}
        \varepsilon_{\alpha} \overline{\left[\frac{\partial \psi_{\alpha}}{\partial t}\right]} = \frac{\partial \left(\varepsilon_{\alpha} \bar{\psi}_{\alpha} \right)}{\partial t} - \frac{1}{V} \iint_{A_{\alpha \beta(\boldsymbol{x},t)}}\psi_{\alpha} \boldsymbol{v}_{\alpha \beta} \cdot \hat{\boldsymbol{n}}_{\alpha} \,\mathrm{d}A.
    \end{equation}
\end{enumerate}
Dabei beschreibt $\bar{\psi}_{\alpha}$ bzw. $\bar{\boldsymbol{\psi}}_{\alpha}$ die intrinsische Mittelung über Phase $\alpha$. Diese Mittelung wird nur über das von Phase $\alpha$ eingenommene Volumen\footnote{Hier als Zwei-Phasen-System mit der zweiten Phase $\beta$ betrachtet.} ermittelt. Die intrinsische Mittelung bietet gegenüber einer klassischen Mittelung $\langle \psi_{\alpha} \rangle$, die sich auf das Volumen des gesamten Gebiets bezieht, größere Flexibilität und Wiederverwendbarkeit\footnote{Intrinsische Werte können wegen der Unabhängigkeit vom Phasenanteil für beliebige Phasenanteile wiederverwendet werden.}. Mittels des Volumenanteils $\varepsilon_{\alpha}$
\begin{equation}
    \varepsilon_{\alpha} = \frac{V_{\alpha}(\boldsymbol{x},t)}{V} 
\end{equation}
können die beiden Mittelungsarten ineinander umgewandelt werden:
\begin{equation}
    \langle \psi_{\alpha} \rangle = \varepsilon_{\alpha} \bar{\psi}_{\alpha}.
\end{equation}

Mithilfe der drei Volumenmittelungstheoreme lassen sich die folgenden vier Gleichungen herleiten~\cite{Doyle1995}.
\begin{enumerate}
    \item Volumengemittelte Näherung des Ladungserhalts in der festen Phase der porösen Elektrode
    \begin{equation}
        \nabla \cdot \left(\sigma_{\text{eff}} \nabla \hat{\phi}_{s} \right) = a_s F_{\text{K}} \hat{j},
    \end{equation}
    \item Volumengemittelte Näherung des Ladungserhalts in der Elektrolytphase der porösen Elektrode
    \begin{equation}
        \nabla \cdot \left(\kappa_{\text{eff}} \nabla \hat{\phi}_e + \kappa_{D, \text{eff}} \nabla \ln \hat{c}_e\right) + a_s F_{\text{K}} \hat{j} = 0,
    \end{equation}
    \item Volumengemittelte Näherung des Massenerhalts in der Elektrolytphase der porösen Elektrode
    \begin{equation}
        \frac{\partial \left(\varepsilon_e \hat{c}_e \right)}{\partial t} = \nabla \cdot \left(D_{e,\text{eff}}\nabla\hat{c}_e\right) + a_s (1+t^0_+) \hat{j},
    \end{equation}
    \item Volumengemittelte Näherung der mikroskopischen Butler-Volmer-Beziehung für den Ionenphasenwechsel
    \begin{equation}
        \hat{j} = j(c_{s,e},\hat{c}_e,\hat{\phi}_s,\hat{\phi}_e).
    \end{equation}
\end{enumerate}

Analog lassen sich für die mechanische Spannung und die Temperatur die folgenden Zusammenhänge aufstellen.
\begin{enumerate}
    \item Homogenisierung der mechanischen Spannung
    \begin{equation}
    \boldsymbol{\sigma} = \boldsymbol{C}_{\text{eff}} \boldsymbol{\varepsilon}_{\text{mech}},
    \end{equation}
    \item Volumengemittelte Darstellung der Temperaturentwicklung
    \begin{equation}
        \frac{\partial (\rho c_{\text{P}} T)}{\partial t} = \nabla \cdot (\lambda \nabla T) + q.
    \end{equation}
\end{enumerate}

Die eingeführte Wärmequelle $q$ kann dabei aus den folgenden fünf Beiträgen zusammengesetzt werden~\cite{Plett2015}.
\begin{enumerate}
    \item Irreversible Wärmeentstehung durch chemische Reaktionen
    \begin{equation}
        q_i = a_{\text{s}} F_{\text{K}} \hat{j}_j \eta_{j},
    \end{equation}
    \item Reversible Wärmebildung durch Entropieänderung
    \begin{equation}
    q_{r} = a_{\text{s}} F_{\text{K}} \hat{j}_j T \frac{\partial U_{\text{ocp},j}}{\partial T},
    \end{equation}
    \item Joule-Wärme im Feststoff
    \begin{equation}
    q_{s} = \sigma_{\text{eff}}(\nabla\hat{\phi}_{\text{s}} \cdot \nabla\hat{\phi}_{\text{s}}),
    \end{equation}
    \item Joule-Wärme im Elektrolyt
    \begin{equation}
        q_{e} = \kappa_{\text{eff}}(\nabla\hat{\phi}_{\text{e}} \cdot \nabla\hat{\phi}_{\text{e}}) + \kappa_{D,\text{eff}} (\nabla \ln \hat{c}_e \cdot \nabla \hat{\phi}_{\text{e}}),
    \end{equation}
    \item Wärmeentstehung durch Kontaktwiderstände\footnote{$q_c$ gilt nur für die Elektrodenfläche und ist daher auf die Einheitsfläche bezogen; die anderen Terme sind auf das Einheitsvolumen bezogen.}
    \begin{equation}
        q_{c} = i_{\text{app}}^2 R_{\text{Kontakt}}.
    \end{equation}
\end{enumerate}

\begin{figure}[!ht]
    \center
    \includegraphics[width=0.8\textwidth, angle=0]{carlstedt.pdf}
    \caption{\label{fig:carlstedt}a) Beispielhafte Darstellung der untersuchten Kohlenstofffaser-Strukturbatterie und der LFP-Zelle nach~\cite{Carlstedt2022b}, b) zweidimensionales Modell zur Durchführung der FEM-Simulation, c) Zeitverlauf des angelegten Stroms als treibende Randbedingung, d) elektrische Spannung und Stromdichte im zeitlichen Verlauf sowie die Lithiumkonzentration zu den Zeitpunkten $t_1 = 2000\,\text{s}$ und $t_2 = 6000\,\text{s}$, e) gemittelte Temperatur über die Zeit sowie Temperaturverteilungen bei $t_1$ und $t_2$, f) mechanische Spannungskomponenten $\sigma_{11}$ und $\sigma_{22}$ zu den Zeitpunkten $t_1$ und $t_2$.}
\end{figure}

Angelehnt an Arbeiten von \textsc{Carlstedt}~\cite{Carlstedt2022b}\footnote{Die Materialwerte, Geometrie und Randbedingungen wurden der Arbeit entnommen, um einen Vergleich zu ermöglichen.} können diese Gleichungen bereits verwendet werden, um das Verhalten ganzer Zellen zu beschreiben\footnote{Hier: eine Kohlenstofffaser-LFP-Zelle} (Bild~\ref{fig:carlstedt}). Die Zelle durchläuft dabei einen Entlade- und Ladezyklus innerhalb von 2,2\,h. Die Simulationszeit betrug 34,6\,h auf einem Berechnungsserver der HTWK\footnote{Unter voller Ausnutzung von zwei eingebauten CPUs der Marke AMD EPYC 75F3 mit einer Taktrate von 2,95\,GHz und jeweils 32 Kernen.}. Der hohe Rechenaufwand bereits für einen Ladezyklus macht diesen Ansatz jedoch ungeeignet, um eine Vielzahl an Varianten und größere, mehrzellige Batteriesysteme auszulegen.

Durch Ermittlung effektiver physikalischer Eigenschaften werden die Inhomogenitäten auf der Mikroskala durch ein Kontinuum auf der Makroskala beschrieben~\cite{Plett2024}. Die Genauigkeit dieses Ansatzes hängt jedoch stark von den zu betrachtenden Längenskalen ab~\cite{Plett2015}. Lokal erhöhte Porendichten oder ähnliche inhomogene Effekte lassen sich nur aufwendig berücksichtigen~\cite{Mei2019}. Bei der Analyse deutlich größerer Skalen als die Inhomogenitäten zeigen diese Modelle hingegen eine höhere Effizienz und ausreichende Genauigkeit~\cite{Plett2015}. 

Um die Berechnungszeit weiter zu reduzieren, kann aufgrund der Butler-Volmer-Randbedingung keine Volumenmittelung für die Massenerhaltung in der festen Phase\footnote{Die Materialien, die als Interkalationsort dienen.} verwendet werden~\cite{Plett2015}. Durch Geometrievereinfachungen lassen sich jedoch Freiheitsgrade reduzieren und zusätzlicher Rechenaufwand vermeiden. Im Kontext von Strukturbatterien ist der interkalationsaktiv teilnehmende Bereich partikel- oder faserförmig und kann durch Kugeln bzw. Zylinder approximiert werden~\cite{Newman2021}. Daraus ergeben sich die nachfolgenden Gleichungen:
\begin{enumerate}
    \item Spezialfall Massenerhalt in kugelförmigen Festkörpern
    \begin{equation}
        \label{eq:diffusion_sphere}
    \frac{\partial c_{\text{s}}}{\partial t} = \frac{1}{r^2} \frac{\partial}{ \partial r} \left[ D_{\text{s}} r^2 \frac{\partial c_{\text{s}}}{\partial r}\right],
    \end{equation}
    \item Spezialfall Massenerhalt in zylindrischen Festkörpern
    \begin{equation}
        \label{eq:diffusion_cylinder}
    \frac{\partial c_{\text{s}}^{\pm}}{\partial t} = \frac{1}{r} \frac{\partial}{ \partial r} \left[ D_{\text{s}} r \frac{\partial c_{\text{s}}}{\partial r}\right] + \frac{\partial}{ \partial z}\left[D_{\text{s}}  \frac{\partial c_{\text{s}}}{\partial z}\right].
    \end{equation}
\end{enumerate}
Dabei ist für viele Szenarien die Verteilung der Konzentration in $z$-Richtung näherungsweise konstant~\cite{Wang2020c}. In diesem Fall kann der Massenerhalt in zylindrischen Festkörpern weiter vereinfacht werden:
\begin{equation}
    \frac{\partial c_{\text{s}}^{\pm}}{\partial t} = \frac{1}{r} \frac{\partial}{ \partial r} \left[ D_{\text{s}} r \frac{\partial c_{\text{s}}}{\partial r}\right].
\end{equation}
In beiden Fällen lässt sich das Interkalationsverhalten durch die Randbedingungen
\begin{align}
    \left.\frac{\partial c_{\text{s}}^{\pm}}{\partial r}\right\vert_{r=0} &= 0, \\
    \left.\frac{\partial c_{\text{s}}^{\pm}}{\partial r}\right\vert_{r=R_{\text{p,s}}^{\pm}} &= -\frac{1}{ D_{\text{s}}^\pm} j_{n}^{\pm}(x,t),
\end{align}
darstellen, wobei im Falle einer stromgesteuerten Be- und Entladung
\begin{equation}
j_{n}^{\pm}(t) = \mp \frac{I(t)}{F a^{\pm} L^{\pm}}
\end{equation}
ist~\cite{Plett2015}.

\begin{figure}[!ht]
    \center
    \includegraphics[width=0.99\textwidth, angle=0]{p2d_model.pdf}
    \caption{\label{fig:p2d_model}a) Vereinfachung und Überführung einer NMC-Zelle zu einem 2D-Modell für die FEM-Berechnung, b) elektrische Spannung über mehrere durch den Strom geprägte Lade- und Entladezyklen, c) Temperaturverlauf während der Zyklen, d) maximale und minimale mechanische Spannung über den betrachteten Zeitraum.}
\end{figure}

Die daraus folgenden zweidimensionalen Modelle\footnote{Eine Dimension in Dicken-/Höhenrichtung und eine weitere in Radialrichtung der Partikel oder Fasern.} (Bild~\ref{fig:p2d_model}) gelten als die effizientesten physikalisch basierten Batteriemodelle. Mit diesen lassen sich mehrere Zyklen über 65\,h in unter 43\,min simulieren\footnote{Unter voller Ausnutzung von zwei eingebauten CPUs der Marke AMD EPYC 75F3 mit einer Taktrate von 2,95\,GHz und jeweils 32 Kernen.}. Die Genauigkeit dieser Modelle ist dabei hoch und zeigt meist Abweichungen von unter 0,5~\%~\cite{Pistorio2023}. Wie in anderen Modellen werden die schwer zu bestimmenden kinetischen Parameter häufig durch Anpassung an die ersten Zyklenverläufe identifiziert, wobei als Startwerte Literaturwerte verwendet werden~\cite{Sauerteig2018,Shui2023}. Eine einheitliche Bestimmung und ein konsistenter Austausch dieser Parameter zwischen verschiedenen Modellen ist aufgrund der unterschiedlichen Modellannahmen jedoch schwierig~\cite{Madani2018}. Für eine breite Werkstoff- bzw. Komponentenauswahl im Sinne einer Vorauslegung von Strukturbatterien sind diese Modelle aufgrund der hohen Anzahl zu bestimmender Parameter oft ungeeignet~\cite{Li2022}.
