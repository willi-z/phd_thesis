
\section{\label{sec:homogenisation}Homogenisierung von Mikroskalenmodellen}

Die Modellierung der einzelnen physikalischen Prozesse ist auf der Mikroskala häufig einfacher umzusetzen~\cite{Plett2015}. Mithilfe mikroskaliger Modelle lassen sich die Einflüsse der Geometrie, Verteilung und Clusterbildung präzise ermitteln~\cite{Newman2021}. Aufgrund der hohen Komplexität, die mit den verschiedenen Skalenbereichen einhergeht, ist der damit verbundene Berechnungsaufwand jedoch zu groß, um eine Vielzahl von Zellen effizient zu simulieren~\cite{Liu2019}. Daher sind makroskalige Modelle erforderlich, welche den Rechenaufwand durch Homogenisierung und geeignete Modellvereinfachungen deutlich reduzieren~\cite{Plett2015}. Darüberhinaus bestehen Abweichungen durch Skalierungseffekte sowi der richtigen Abbildung der untersuchten Mikostruktur und der efolgreichen Materialkennwerte.

Ein häufig verwendeter Ansatz stellt dabei die Mittelung der physikalischen Eigenschaften über ein repräsentatives Volumenelement~(RVE) dar~\cite{Burow2016,Arunachalam2019,Li2020}. Die dazu mathematischen Grundlagen basieren auf drei Volumenmittlungstheoremen~\cite{Gray1977}.
\begin{enumerate}
    \item Volumenmittlung für ein skalares Feld $\psi$ 
    \begin{equation}
        \varepsilon_{\alpha} \overline{\nabla \psi_{\alpha}} = \nabla \left(\varepsilon_{\alpha} \bar{\psi}_{\alpha} \right) + \frac{1}{V} \iint_{A_{\alpha \beta(\boldsymbol{x},t)}}\psi_{\alpha} \hat{\boldsymbol{n}}_{\alpha} \text{d}A,
    \end{equation}
    \item Volumenmittlung für ein Vektorfeld $\boldsymbol{\psi}$
    \begin{equation}
        \varepsilon_{\alpha} \overline{\nabla \cdot \boldsymbol{\psi}_{\alpha}} = \nabla \cdot \left(\varepsilon_{\alpha} \bar{\boldsymbol{\psi}}_{\alpha} \right) + \frac{1}{V} \iint_{A_{\alpha \beta(\boldsymbol{x},t)}}\boldsymbol{\psi}_{\alpha} \cdot \hat{\boldsymbol{n}}_{\alpha} \text{d}A,
    \end{equation}
    \item Volumenmittlung für die zeitliche Änderung eines skalaren Feldes $\psi$ 
    \begin{equation}
        \varepsilon_{\alpha} \overline{\left[\frac{\partial \psi_{\alpha}}{\partial t}\right]} = \frac{\partial \left(\varepsilon_{\alpha} \bar{\psi}_{\alpha} \right)}{\partial t} - \frac{1}{V} \iint_{A_{\alpha \beta(\boldsymbol{x},t)}}\psi_{\alpha} \boldsymbol{v}_{\alpha \beta} \cdot \hat{\boldsymbol{n}}_{\alpha} \text{d}A.
    \end{equation}
\end{enumerate}
Dabei beschreibt $\bar{\psi}_{\alpha}$ bzw. $\bar{\boldsymbol{\psi}}_{\alpha}$ die intrinsischen Mittelung über Phase $\alpha$. Diese Art der Mittelung wird nur über das von Phase $\alpha$ eingenommene Volumen\footnote{Hier als zwei Phasensystem mit der zweiten Phase $\beta$ betrachtet.} ermittelt. Die intrinische Mittelung erlaubt gegenüber einer klassischen Mittelung $\langle \psi_{\alpha} \rangle$, welcher auf das Volumen des gesamten Gebietes bezogen ist, eine größere Flexibilität und Wiederverwendbarkeit\footnote{Intrinsiche Werte können durch die Unabhängigkeit zum Phasenanteile für jeden belieben Phasenanteil wiederverwendet werden.}. Mittels des Volumenanteils $\varepsilon_{\alpha}$
\begin{equation}
    \varepsilon_{\alpha} = \frac{V_{\alpha}(\boldsymbol{x},t)}{V} 
\end{equation}
können die beiden Mittelungsarten in einander umgewandelt werden.
\begin{equation}
    \langle \psi_{\alpha} \rangle = \varepsilon_{\alpha} \bar{\psi}_{\alpha}
\end{equation}

Mithilfe der drei Volumenmittelungstheoreme lassen sich die folgenden vier Gleichungen herleiten~\cite{Doyle1995}.
\begin{enumerate}
    \item Volumengemittelte Annäherung des Ladungserhaltes in der festen Phase der porösen Elektrode
    \begin{equation}
        \nabla \cdot \left(\sigma_{\text{eff}} \nabla \hat{\phi}_{s} \right) = a_s F_{\text{K}} \hat{j},
    \end{equation}
    \item Volumengemittelte Annäherung des Ladungserhaltes in der Elektrolytphase der porösen Elektrode
    \begin{equation}
        \nabla \cdot \left(\kappa_{\text{eff}} \nabla \hat{\phi}_e + \kappa_{D, \text{eff}} \nabla ln \hat{c}_e\right) + a_s F_{\text{K}} \hat{j} = 0,
    \end{equation}
    \item Volumengemittelte Annäherung des Massenerhaltes in der Elektrolytphase der porösen Elektrode
    \begin{equation}
        \frac{\partial \left(\varepsilon_e \hat{c}_e \right)}{\partial t} = \nabla \cdot \left(D_{e,\text{eff}}\nabla\hat{c}_e\right) + a_s (1+t^0_+) \hat{j},
    \end{equation}
    \item Volumengemittelte Annäherung der mikroskopischen Butler-Volmer Beziehung für den Ionenphasenwechsel
    \begin{equation}
        \hat{j} = j(c_{s,e},\hat{c}_e,\hat{\phi}_s,\hat{\phi}_e).
    \end{equation}
\end{enumerate}

Analog lassen sich für die mechansiche Spannung und die Temparatur die folgenden Zusammenhänge aufstellen.
\begin{enumerate}
    \item Homogenisierung der mechansichen Spannung
    \begin{equation}
    \boldsymbol{\sigma} = \boldsymbol{C}_{\text{eff}} \boldsymbol{\varepsilon}_{\text{mech}},
    \end{equation}
    \item Volumengemittelte Annäherung der Temperatur
    \begin{equation}
        \frac{\partial (\rho c_{\text{P}} T)}{\partial t} = \nabla \cdot (\lambda \nabla T) + q.
    \end{equation}
\end{enumerate}


Der eingeführte Wärmequelle $q$ kann dabei aus den folgenden fünf Quellen zusammengesetz werden~\cite{Plett2015}.
\begin{enumerate}
    \item Irreversible Wärmeentstehung durch $j$ chemische Reaktionen
    \begin{equation}
        q_i = a_{\text{s}} F_{\text{K}} \hat{j}_j \eta_{j},
    \end{equation}
    \item Reversible Wärmebildung durch Veränderung der Entropie
    \begin{equation}
    q_{r} = a_{\text{s}} F_{\text{K}} \hat{j}_j \eta_{j} T \frac{\partial U_{\text{ocp},j}}{\partial T},
    \end{equation}
    \item Joule-Wärmeentstehung durch Gradient des elektrischen Potenzials im Feststoff
    \begin{equation}
    q_{s} = \sigma_{\text{eff}}(\nabla\hat{\phi}_{\text{s}} \cdot \nabla\hat{\phi}_{\text{s}}),
    \end{equation}
    \item Joule-Wärmeentstehung durch Gradient des elektrischen Potenzials im Elektrolyt
    \begin{equation}
        q_{e} = \kappa_{\text{eff}}(\nabla\hat{\phi}_{\text{e}} \cdot \nabla\hat{\phi}_{\text{e}}) + \kappa_{D,\text{eff}} (\nabla ln \hat{c}_e \cdot \nabla \hat{\phi}_{\text{e}}),
    \end{equation}
    \item Warmeentstehung durch Kontaktwiderstände\footnote{$q_c$ gilt nur für die Elektodenfläche und ist daher bezogen auf die Einheitsfläche und nicht wie die anderen Therme auf das Einheitsvolumen}
    \begin{equation}
        q_{c} = i_{\text{app}}^2 R_{\text{Kontakt}}.
    \end{equation}
\end{enumerate}

\begin{figure}[!ht]
	%\raggedleft
		%\def\svgwidth{\columnwidth}
        \center
		\includegraphics[width=0.8\textwidth, angle=0]{carlstedt.pdf}
		\caption{\label{fig:carlstedt}a) Beispielhafte Darstellung der untersuchten Kohlenstofffaser-Strukturbatterie und der LFP-Zelle nach~\cite{Carlstedt2022b}, b) zwei-dimensionales Modell zur Durchführung der FEM-Simulation, c) Zeitverlauf des angelegten Stroms als treibende Randbedingung, d) elektrische Spannung und Stromdichte im zeitlichen Verlauf sowie die Lithiumkonzentration zu den Zeitpunkten $t_1 = 2000\,\text{s}$ und $t_2 = 6000\,\text{s}$, e) gemittelte Temperatur über die Zeit sowie Temperaturverteilungen bei $t_1$ und $t_2$, f) mechanische Spannungskomponenten $\sigma_{11}$ und $\sigma_{22}$ zu den Zeitpunkten $t_1$ und $t_2$
        }
\end{figure}

Angelehnt an Arbeiten von \textsc{Carlstedt}~\cite{Carlstedt2022b}\footnote{Die Materialwerte und Geometrie sowie die Randbedingungen wurden aus der Arbeit entnommen, um einen Vergleich zu ermöglichen.} können diese Gleichungen bereits verwendet werden, um das Verhalten ganzer Zellen zu beschreiben\footnote{Hier eine Kohlenstofffaser-LFP-Zelle} (Bild~\ref{fig:carlstedt}). Die Zelle durchläuft dabei einen Entlade- und Ladezyklus innerhalb von 2,2 h. Die Simulationszeit betrug dabei 34,6 h auf einem Berechnungsserver der HTWK\footnote{Unter voller Ausnutzung von zwei eingebauten CPUs der Marke AMD EPYC 75F3 mit einer Taktrate von 2,95 GHz und jeweils 32 Kernen}. Der hohe Berechnungsaufwand bereits für einen Ladezyklus macht diesen Ansatz jedoch ungeeignet, um eine Vielzahl an Varianten und größere, mehrzellige Batteriesysteme auszulegen.

Durch Ermittlung effektiver physikalischer Eigenschaften werden dabei die Inhomogenitäten auf der Mikroskala durch ein Kontinuum auf der Makroskala beschrieben~\cite{Plett2024}. Die Genauigkeit dieses Ansatzes hängt jedoch stark von den zu betrachtenden Dimensionen ab~\cite{Plett2015}. Durch die Beschreibung als Kontinuum können Einflüsse wie etwa eine lokal höhere Porendichte nur aufwendig berücksichtigt werden~\cite{Mei2019}. Bei der Analyse deutlich größerer Skalen als der Inhomogenitäten zeigen diese Modelle hingegen eine höhere Genauigkeit~\cite{Plett2015}. 

Um die Berechnungszeit weiter zu reduzieren, kann wegen der Butler-Volmer-Randbedingung keine Volumenmittelung für die Massenerhaltung in der festen Phase\footnote{Die Materialien, die als Interkalationsort dienen} verwendet werden~\cite{Plett2015}. Jedoch können durch Geometrievereinfachungen Freiheitsgrade reduziert und zusätzlicher Berechnungsaufwand vermieden werden. Im Kontext von Strukturbatterien ist der an der Interkalation aktiv teilnehmende Teil partikel- oder faserförmig, welcher durch Kugeln oder Zylinder approximiert werden kann~\cite{Newman2021}. Daraus leiten sich die nachfolgenden Glecihungen ab:
\begin{enumerate}
    \item Spezialfall Massenserhalt in kugelförmigen Festkörpern
    \begin{equation}
        \label{eq:diffusion_sphere}
    \frac{\partial c_{\text{s}}}{\partial t} = \frac{1}{r^2} \frac{\partial}{ \partial r} \left[ D_{\text{s}} r^2 \frac{\partial c_{\text{s}}}{\partial r}\right],
    \end{equation}
    \item Spezialfall Massenserhalt in zylindrischen Festkörpern
    \begin{equation}
        \label{eq:diffusion_cylinder}
    \frac{\partial c_{\text{s}}^{\pm}}{\partial t} = \frac{1}{r} \frac{\partial}{ \partial r} \left[ D_{\text{s}} r \frac{\partial c_{\text{s}}}{\partial r}\right] + \frac{\partial}{ \partial z}\left[D_{\text{s}}  \frac{\partial c_{\text{s}}}{\partial z}\right]
    \end{equation}.
\end{enumerate}
Dabei ist für viele Szenarien die Verteilung der Konzentration in $z$-Richtung näherungsweise gleich~\cite{Wang2020c}. In diesem Fall kann der Massenerhalt in zylindirschen Festkörpern weiter zu 
\begin{equation}
    \frac{\partial c_{\text{s}}^{\pm}}{\partial t} = \frac{1}{r} \frac{\partial}{ \partial r} \left[ D_{\text{s}} r \frac{\partial c_{\text{s}}}{\partial r}\right]
\end{equation}
vereinfacht werden.
In beiden Fällen lässt sich das Interkalationsverhalten durch die beiden Randbedingungen
\begin{align}
    \left.\frac{\partial c_{\text{s}}^{\pm}}{\partial r}\right\vert_{r=0} &= 0 \\
    \left.\frac{\partial c_{\text{s}}^{\pm}}{\partial r}\right\vert_{r=R_{\text{p,s}}^{\pm}} &= -\frac{1}{ D_{\text{s}}^\pm} j_{n}^{\pm}(x,t)
\end{align}
darstellen, wobei im Falle einer Stromgesteuerten Be- und Entladung
\begin{equation}
j_{n}^{\pm}(t) = \mp \frac{I(t)}{F a^{\pm} L^{\pm}}
\end{equation}
ist~\cite{Plett2015}.

\begin{figure}[!ht]
	%\raggedleft
		%\def\svgwidth{\columnwidth}
        \center
		\includegraphics[width=0.99\textwidth, angle=0]{p2d_model.pdf}
		\caption{\label{fig:p2d_model}a) Vereinfachung und Überführung von einer NMC-Zelle zu einem 2D-Modell für die FEM-Berechnung, b) die el. Spannung über mehrere durch den Strom geprägte sich steigernde Lade- und Entladezyklen, c) der Temperaturverlauf während der Zyklen, d) die maximale und minimale mechanische Spannung über den zu betrachtenden Zeitraum
        }
\end{figure}

Die daraus folgenden zweidimensionalen Modelle\footnote{Eine Dimension in Dicken-/Höhenrichtung und eine weitere in Radialrichtung der Partikel oder Fasern.} (Bild~\ref{fig:p2d_model}) gelten als die effizientesten physikalisch basierten Batteriemodelle. Mit diesen lassen sich mehrere Zyklen über 65,h in unter 43\,min simulieren\footnote{Unter voller Ausnutzung von zwei eingebauten CPUs der Marke AMD EPYC 75F3 mit einer Taktrate von 2,95\,GHz und jeweils 32 Kernen}. Die Genauigkeit dieser Modelle ist dabei sehr hoch und zeigt meist Abweichungen von unter 0,5~\%~\cite{Pistorio2023}. Jedoch werden, wie auch in anderen Modellen, die nur schwer zu bestimmenden kinetischen Parameter durch Annäherung während der ersten Zyklenverläufe bestimmt, wobei als Startwerte für die Optimierung Literaturwerte verwendet werden~\cite{Sauerteig2018,Shui2023}. Eine Möglichkeit, diese Parameter einheitlich zu bestimmen und zwischen verschiedenen Modellen auszutauschen, ist aufgrund der teilweise unterschiedlichen Beziehungen schwierig~\cite{Madani2018}. Für eine breite Werkstoff- bzw. Komponentenauswahl im Sinne einer Vorauslegung von Strukturbatterien sind diese Modelle aufgrund ihrer hohen Parameteranzahl, welche oft nicht direkt bestimmt werden können, ungeeignet~\cite{Li2022}.

