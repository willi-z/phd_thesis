\chapter{Herleitung Analytischer Ansätze für perspektivische Vorauslegungen}
\section{\label{sec:improve_elchem}Analytische Vorhersage der Energiedichte}
% \begin{itemize}
%     \item Diffusionskoeffizient wird durch equivalente Schaltung ermittelt, die % konstanten Wert vorraussetzen
%     \item Diffusionskoeeffizeint ist eigentlich stark von der Lithierung abhängig
%     \item aufwendig zu ermitteln
%     \item außerdem abweichungen durch Bildung Elektrolyteinterface
%     \item daher für vorhersagen ist die benutzung eher ungeeignet
%     \item für Batterien ist Energidichte wichtiger als Leisungsdichte
%     \item Lösung quasistatische Be- und Entladung, also warten bis vorher
%     \item dies reduziert die vereinfacht die oberen Gleichungen enorm
% \end{itemize}

Die präzise elektrochemische Modellierung von Lithium-Ionen-Zellen erfordert in der Regel die Berücksichtigung eines diffusionsabhängigen Stofftransportes im aktiven Material. Traditionell wird der Diffusionskoeffizient über eine äquivalente elektrische Schaltung angenähert, die einen konstanten, d.h. lithierungsunabhängigen Wert voraussetzt. Diese Annahme führt jedoch zu erheblichen Vereinfachungen, da der tatsächliche Diffusionskoeffizient stark von der lokalen Lithierung abhängt und zudem im Betrieb Temperaturänderungen sowie Alterungsmechanismen, insbesondere die Ausbildung der SEI, berücksichtigt werden müssten. Die exakte Ermittlung eines dynamischen, lithierungsabhängigen Diffusionskoeffizienten ist erfahrungsgemäß rechenaufwendig und experimentell schwer zugänglich. Für schnelle, echtzeitfähige Vorhersagemodelle ist dieser Ansatz daher nur bedingt geeignet.

Für viele Anwendungen im Bereich der elektrochemischen Energiespeicher ist jedoch nicht primär die Leistungsdichte, sondern vielmehr die Energiedichte der Batterie maßgebend. Dieser Fokus ermöglicht es, den Modellierungsaufwand erheblich zu reduzieren. Eine praktikable Lösung besteht darin, Be- und Entladeprozesse quasistatisch zu betrachten. Dabei wird nach jedem kleinen Lade- oder Entladeschritt eine Relaxationsphase angenommen, in der sich das Konzentrationsfeld im aktiven Material vollständig ausgleicht. Die Diffusionsprozesse müssen somit nicht explizit zeitaufgelöst simuliert werden, was die oberen elektrochemischen Gleichungen drastisch vereinfacht und den Berechnungsaufwand deutlich reduziert.Auf dieser Grundlage lassen sich die relevanten Kapazitäts-, Massen- und Energiekennwerte des Batteriestacks durch folgende vereinfachte Beziehungen beschreiben.

Ein zentraler Schritt zur Bewertung der maximal verfügbaren Energiedichte eines elektrochemischen Systems ist die Bestimmung der Oberflächenkapazität der Zelle. Die Oberflächenkapazität beschreibt die umgesetzte Stoffmenge pro aktiver Elektrodenfläche und ist damit direkt an die stöchiometrisch erreichbare Lithiumaufnahme der Elektroden gekoppelt. Da eine Lithium-Ionen-Zelle stets aus einer Anode und einer Kathode besteht, die über den Elektrolyten Lithiumionen austauschen, bestimmt die jeweils limitierende Elektrode die insgesamt nutzbare Kapazität. 

Genau dieses limitierende Verhalten wird durch die folgende Beziehung beschrieben
\begin{equation}
    C_{\text{A, Zelle}} = \min \left( C_{\text{A, -}} , C_{\text{A, +}}\right).
\end{equation}

Hierbei stehen $C_{\text{A, -}}$ und $C_{\text{A, +}}$ für die spezifischen Oberflächenkapazitäten der negativen  bzw. positiven  Elektrode. Die $\min$-Funktion reflektiert, dass eine Elektrode stets vor der anderen ihre maximal mögliche Lithierung bzw. Delithierung erreicht. Sobald die Kapazität der limitierenden Elektrode ausgeschöpft ist, kann der elektrochemische Prozess nicht weitergeführt werden, unabhängig davon, ob die Gegenelektrode theoretisch noch Ladung aufnehmen oder abgeben könnte. 

Die Wahl der minimalen Kapazität stellt somit sicher, dass das Modell physikalisch konsistent bleibt und die tatsächliche, durch die Zellchemie begrenzte Leistungsfähigkeit korrekt beschreibt. Dieser Ansatz ist insbesondere für vereinfachte Modelle essenziell, da er ohne detaillierte Konzentrationsprofile oder komplexe Diffusionsberechnungen auskommt und dennoch die reale Einschränkung durch die elektrochemisch schwächere Elektrode akkurat abbildet.

Da in der Praxis ausschließlich identische Zellen innerhalb eines Stacks verschaltet werden, addieren sich deren Einzelkapazitäten linear, sodass die Gesamtkapazität des Stacks einfach als Produkt aus der Anzahl der Zellen und der Oberflächenkapazität einer Einzelzelle beschrieben werden kann
\begin{equation}
    C_{\text{A, Stack}} = n_{\text{Zellen}} \cdot C_{\text{A, Zelle}}.
\end{equation}

Die Masse des im Stack enthaltenen Elektrolyten lässt sich direkt aus der Gesamtoberflächenkapazität ableiten, da pro umgesetzter Kapazität ein charakteristisches Elektrolytvolumen $V_{\text{C,E}}$ benötigt wird. Multipliziert man dieses volumenbezogene Kapazitätsmaß mit der Gesamtoberflächenkapazität des Stacks sowie der Elektrolytdichte $\rho_{\text{E}}$, ergibt sich die Gesamtmasse des erforderlichen Elektrolyten
\begin{equation}
    m_{\text{A, Stack, E}} = C_{\text{A, Stack}} \cdot V_{\text{C,E}} \cdot \rho_{\text{E}}.
\end{equation}

Durch Summierung der Masse des Elektrolytens und Einzelschichten folgt kann die Gesamtmasse des Batteriestacks $m_{\text{A, Stack}}$ bestimmt werden 
\begin{equation}
    m_{\text{A, Stack}} = m_{\text{A, Stack, E}} + \sum_{i}^{n_{\text{Schichten}}} m_{\text{A,i}}.
\end{equation}

Während die zuvor hergeleiteten Größen auf die elektrochemisch aktive Fläche bezogen sind, erfordert die Bewertung der Energiedichte eine Normierung auf die Gesamtmasse des Stacks. Erst durch diese Umrechnung lässt sich beurteilen, wie viel Kapazität pro Masseeinheit tatsächlich bereitgestellt werden kann. Die massenbezogene Kapazität ergibt sich daher als Quotient aus der gesamten Oberflächenkapazität des Stacks und seiner Gesamtmasse
\begin{equation}
    C_{\text{m, Stack}} = \frac{C_{\text{A, Stack}} }{ m_{\text{A, Stack}}}.
\end{equation}

Daraus ergibt sich die gravimetrische Energiedichte des Stacks unmittelbar aus der massenbezogenen Kapazität und der nutzbaren Zellspannung. Durch Multiplikation dieser beiden Größen erhält man den spezifischen Energieinhalt des Systems.
\begin{equation}
    \Gamma_{\text{Stack}} = C_{\text{m, Stack}} \cdot \left(U_{+} - U_{-}\right)
\end{equation}

% Simultaneously Coupled Mechanical-Electrochemical- Thermal Simulation of Lithium-Ion Cells
Darüber hinaus können Kurzschluss- und Versagensmechanismen weiterhin mit reduzierten Modellen abgebildet werden,
\begin{align}
    R_{\text{Kurz}} &= A_{\text{Kurz}} \sum_{i} \frac{1}{K_i}\\
    A_{\text{Kurz}} &= \sum_{i}^{n_{\text{Versagen}}} A_{i}
\end{align}
wobeit $K_i$ die effektive Leitfähigkeit bzw. Permeabilität des jeweiligen Kurzschluss- oder Versagenspfades beschreibt~\cite{Zhang2016}.

Die Zellspannung ergibt sich unter Berücksichtigung vereinfachter Konzentrationsüberspannungen sowie des Kurzschlusswiderstandes~\cite{Daigle2013} zu
\begin{equation}
    V_{\text{Zelle}} = U_{+} - U_{-} + \frac{RT}{F} \ln \left( \frac{1-x}{x}\right) - i_{app} R_{\text{Kurz}}.
\end{equation}
Das thermische Verhalten kann in der quasistatischen Betrachtung ebenfalls reduziert werden, da reversible Wärmequellen dominieren und dissipative Beiträge klein sind. Damit ergibt sich
\begin{equation}
    \rho v c_p \frac{\partial T}{\partial t} = i_{app}\left(V_{\text{Zelle}} - U_{+} + U_{-} + i_{app} R_{Kurz} \right) -q.
\end{equation}
Da beim quasistatischen Be- und Entladen keine nennenswerten Konzentrationsgradienten und damit auch keine irreversiblen Relaxationsprozesse auftreten, können wärmeerzeugende Nebenreaktionen vernachlässigt werden. Somit entfallen dissipative Wärmequellen vollständig, und es gilt in guter Näherung von
\begin{equation}
    q = 0.
\end{equation}
Durch diese Annahmen lassen sich die elektrochemischen und thermischen Teilmodelle signifikant vereinfachen, wodurch der Gesamtberechnungsaufwand stark reduziert wird, ohne die Aussagekraft für energiedichteorientierte Anwendungen wesentlich einzuschränken.


\section{\label{sec:improve_mech}Analytische Bestimmung des Verformungsverhaltens von Strukturbatterien unter Berücksichtigung verschiederner Elektrolytarten}
Unter der Annahme, dass alle Einzelschichten bei der Bestimmung der Zugsteifigkeit auf beiden Seiten in der Klemmung mit aufgenommen werden und keiner Vordehnung der Einzelschichten sind die Dehnungen in Zugrichtungen für alle Schichten gleich.
\begin{equation}
    \varepsilon_{x,ges} = \varepsilon_{x,i}\\
\end{equation}

Der Struktur von konventionellen Batterien oder Strukturbatterie mit Gel oder flüssigem Elektrolytsystemen kann vereinfacht als Schichtung, lastentragende Materialien betrachtet werden, in deren Zwischnraum eine nicht-lastentragenden Substanz in Form eines Flüssigen oder Gelartigen Zustandes infiltriert wurde.
Die einzelnen Schichten sind nicht direkt mit einander verbundnen und halten einzig durch den Druck der durch die äußere Pouchfolie aneinander. Unter der Annahme, dass die Sichten sich lückenlos anschmiegen ist davon aus zugehen, dass die Krümmung $\kappa$ mit
\begin{equation}
    \kappa = \frac{1}{r} = \frac{M_y}{E I_y}
\end{equation}
in jeder Schicht gleichgroßt ist.
\begin{equation}
    \kappa = \kappa_1 = \kappa_2 = \dots = \kappa_i = \dots = \kappa_n
\end{equation}
Des Weiteren folgt aus dem Momentengleichgewicht, dass das außen angreifende Biegemoment $M_{b}$ gleich der Summe der Schnittmomente in den Einzelschichten sein muss.
\begin{equation}
    M_{b} = \sum_{i}^{n}M_{y,i}
\end{equation}
Unter Annahme von rechticken Querschnitten mit Breite $b_i$ und Höhe $h_i$ und der Annhame, dass alle Elektroden näherungsweise gleich Breit sind, also $b_i = b$ gilt, folgt für die Belastung einer Einzelschicht durch das Moment $M_i$:
\begin{align}
    M_{b} &= M_i \sum_{k}^{n}\frac{E_k I_{yy,k}}{E_i I_{yy,i}}\\
    M_{b} &= M_i \frac{\sum_{k}^{n} E_k h_k^3}{E_i h_i^3}\\
    M_i &= M_{b} \frac{ E_i h_i^3} { \sum_{k}^{n}E_k h_k^3}
\end{align}
Durch einsetzen Einzelschichtbelastung in die Formel zur Bestimmung der Biegespannung erhält man einen Zusammenhang zwischen Einzelschichtspannung und Biegemomentenbelastung:
\begin{align}
    \sigma_{b,i} &= \frac{M_y,i}{I_{yy}/h_i} \\
    \sigma_{b,i} &= 12 \frac{ M_y,i}{b h_i^2}\\
    \sigma_{b,i} &= 12 \frac{M_{b} E_i h_i^3}{b h_i^2 \sum_{k}^{n}E_k h_k^3}\\
    \sigma_{b,i} &= 12 \frac{M_{b} E_i h_i}{b \sum_{k}^{n}E_k h_k^3}
\end{align}

Für die Bestimmung der Durchbiegung $u$ beim 3-Punkt-Biegeversuch kann 
unter der
\begin{equation}
\frac{\frac{\partial^2 u(x)}{\partial x^2}}{\left(1 + \left(\frac{\partial u(x)}{\partial x} \right)^2 \right)^{3/2}} = -\frac{M_y}{E I_{yy}}
\end{equation}
Diese Gleichung kann für kleine Verformungen, so dass $(\frac{\partial u(x)}{\partial x})^2 \ll 1$ durch die folgende Näherung ersetzt werden.
\begin{equation}
    \frac{\partial^2 u(x)}{\partial x^2} \approx -\frac{M_y(x)}{E I_{yy}}
\end{equation}

Unter der Annhame kleiner Verformung und konstantem Querschnitt und Steifigkeit lässt sich die Durchbiegung infolge der Kraft F durch folgende Gleichung annähern.
\begin{align}
    u_\text{flüssig} (x) &= \frac{F L^3}{48 \sum_{k}^{n} E_k I_{yy,k}} \left[ 3 \frac{x}{L} - 4\left(\frac{x}{L}\right)^3 \right] \text{für} \; 0 \leq x \leq L/2 \\
    u_\text{max,flüssig} (x = L/2) &= \frac{FL^3}{48 \sum_{k}^{n} E_k I_{yy,k}} \label{eq:bending_sbe_0}
\end{align}



An dieser Stelle ist zu bemerken, dass für Spezialfall wo alle $n$ Schichten gleich dick sind und aus dem gleichen Material bestehen, die Spannung sich wie folgend ergibt.
\begin{equation}
    \sigma = \sigma_i = \frac{12 M_{b}}{n b h^2},
\end{equation}
Für diesen Spezialfall ergibt sich die maximale Druchbiegung $u_\text{max,flüssig} $ als
\begin{equation}
    u_\text{max,flüssig}  = \frac{L^3 Q}{4 n b h^3 E} = \frac{L^2 \sigma}{6 h E}.
\end{equation}


Da die bisherigen Annäherung vorallem bei konventionellen Batterien mit nichttragenden Elektrolytschichten eine bessere Gültigkeit besitzen ist auch über den Fall zu sprechen wo die Schichten durch das Elektrolyt fest verbunden sind, wie es vorrangig bei Strukturelektrolyten der Fall ist. Die resultierende Laminatstruktur kann beliebig viele Lagen mit unterschiedlichen Orientierungen aufweisen und wird durch die klassische Laminattheorie (CLT) beschrieben~\cite{Carlstedt2018}. Dabei wird angenommen, dass das Material linear-elastisch ist und kleine Durchbiegungen auftreten.


In der CLT wird die Dehnung über der Dicke $z$ des Laminats beschrieben durch
\begin{equation}
\boldsymbol{\varepsilon}(z) = \boldsymbol{\varepsilon}^0 + z\,\boldsymbol{\kappa},
\end{equation}
wobei $\boldsymbol{\varepsilon}^0$ die Dehnungen in der Mittelfläche und $\boldsymbol{\kappa}$ die Krümmungen sind. Die Beziehung zwischen den Schnittgrößen (Kräfte und Momente pro Breite) und den Dehnungen/Krümmungen ergibt sich aus der ABD-Steifigkeitsmatrix:
\begin{equation}
\begin{pmatrix}
\mathbf{N} \\
\mathbf{M}
\end{pmatrix}
=
\begin{pmatrix}
\mathbf{A} & \mathbf{B} \\
\mathbf{B} & \mathbf{D}
\end{pmatrix}
\begin{pmatrix}
\boldsymbol{\varepsilon}^0 \\
\boldsymbol{\kappa}
\end{pmatrix},
\end{equation}
wobei $\mathbf{A}$ die Membransteifigkeit, $\mathbf{D}$ die Biegesteifigkeit und  $\mathbf{B}$ die Kopplung zwischen Membran- und Biegebeanspruchung des Laminats beschreibt.

Für reine Biegung, in der keine Normalkräfte wirken, gilt $\mathbf{N} = \mathbf{0}$. Daraus folgt
\begin{equation}
\mathbf{A}\boldsymbol{\varepsilon}^0 + \mathbf{B}\boldsymbol{\kappa} = \mathbf{0}
\quad\Rightarrow\quad
\boldsymbol{\varepsilon}^0 = -\mathbf{A}^{-1} \mathbf{B} \boldsymbol{\kappa}.
\end{equation}
Durch Einsetzen in die Momentengleichung ergibt
\begin{equation}
\mathbf{M} = \mathbf{B} \boldsymbol{\varepsilon}^0 + \mathbf{D} \boldsymbol{\kappa} = 
\left( \mathbf{D} - \mathbf{B} \mathbf{A}^{-1} \mathbf{B} \right) \boldsymbol{\kappa}.
\end{equation}

Die Größe
\begin{equation}
\mathbf{D}^* = \mathbf{D} - \mathbf{B} \mathbf{A}^{-1} \mathbf{B}.
\end{equation}
wird als effektive Biegesteifigkeit eines unsymmetrischen Laminats bezeichnet~\cite{Jones2018}. 
Damit ergibt sich der Zusammenhang
\begin{equation}
\boldsymbol{\kappa} = (\mathbf{D}^*)^{-1} \mathbf{M}.
\end{equation}

Für reine Biegung um die $x$-Achse gilt $M_x \ne 0$, reduziert sich das Gleichungssystem zu
\begin{equation}
\kappa_x = \frac{M_x}{D^*_{11}}.
\end{equation}

Unter einer mittigen Krafteinwirkung $F$ auf einen Balken ergibt sich ein Momentenverlauf von
\begin{equation}
M_x(x) = \frac{F}{2}x \quad\text{für } 0 \le x \le \frac{L}{2}.
\end{equation}

Somit ergibt sich die Krümmung zu:
\begin{equation}
\kappa_x(x) = \frac{F x}{2 D^*_{11}}.
\end{equation}

Nach zweimaliger Integration gemäß der Euler-Bernoulli-Theorie entsteht die maximale Durchbiegung in der Balkenmitte zu
\begin{equation}\label{eq:bending_sbe_100}
u_\text{max,fest} = \frac{F L^3}{48 D^*_{11}}, \quad
D^*_{11} = D_{11} - [\mathbf{B} \mathbf{A}^{-1} \mathbf{B}]_{11}.
\end{equation}

Für einen symmetrischen Laminataufbau ($\mathbf{B} = \mathbf{0}$) vereinfacht sich $D^*_{11} = D_{11}$, und die Durchbiegung wird:
\begin{equation}
u_\text{max,fest} = \frac{F L^3}{48 D_{11}}.
\end{equation}

Die Dehnung in $x$-Richtung in einer Schicht bei Höhe $z$ ergibt sich zu
\begin{equation}
\varepsilon_x(z) = \varepsilon_x^0 + z\,\kappa_x.
\end{equation}

\begin{figure}[!ht]
	%\raggedleft
		%\def\svgwidth{\columnwidth}
        \center
		\includegraphics[width=0.99\textwidth, angle=0]{bending_pre_tests.pdf}
		\caption{\label{fig:bending_pre_tests}Zur Entwicklung einer vereinfachten Bestimmung des Verformungsverhaltens wurden \textbf{a)} Probekörper aus 11 Schichten einer $100~\mu m$ dicken Aluminiumfolie hergestellt, welche \textbf{b)} flächig mit Verschiedenen Anteilen an PVDF und Luft zusammengefügt wurden. Die \textbf{c-d)} 3-Punkt-Biegeversuche zeigten \textbf{e)} wurden mit den äqivalenten Kraft-Verschiebungskurven der entwickelten Methode anschließend verglichen.}
\end{figure}

Zur bestimmung der Schichtspannungen wird diese globale Dehnung in die lokale Koordinatenrichtung jeder Schicht transformiert\footnote{je nach Faserwinkel $\theta_i$} und mittels Anwendung des materialgesetzes bestimmt
\begin{equation}
\boldsymbol{\sigma}_i = \mathbf{Q}^{(i)}\, \boldsymbol{\varepsilon}_i.
\end{equation}

Dabei ist $\mathbf{Q}^{(i)}$ die Steifigkeitsmatrix der $i$-ten Schicht im lokalen Koordinatensystem. Die Spannungsverteilung ist innerhalb jeder Schicht linear, aber mit unterschiedlichen Steigungen, jedoch abhängig von Materialparametern und Faserorientierung.

Die hergeleiteten Gleichungen beschreiben das 3-Punkt-Biegungsverhalten eines aus mehreren Schichten aufgebauten Stacks, wobei sowohl der Fall rein aufeinanderliegender Lagen (\ref{eq:bending_sbe_0}) als auch der einer lasttragend über das Elektrolyt verbundenen Struktur (\ref{eq:bending_sbe_100}) berücksichtigt wird. Da Strukturelektrolyte in der Regel aus einer Mischung aus fester und flüssiger Phase bestehen, wurde zur Abschätzung des intermediären mechanischen Verhaltens eine Reihe von Probekörpern (80\,mm $\times$ 10\,mm) gefertigt, bestehend aus elf 0,1\,mm dicken Aluminiumfolien (vgl. Bild~\ref{fig:bending_pre_tests}a).

Durch Vergleich des Verformungsverhaltens von glatten und angerauten Aluminiumfolien, die ohne PVDF auf einander gelegt wurden, konnte gezeigt werden, dass die vernachlässigte Oberflächenrauheit keinen relevanten Einfluss auf das Biegeverhalten ausübt. Wie in Bild~\ref{fig:bending_pre_tests}e dargestellt, ist der Unterschied vernachlässigbar. 

Wegen der besseren Adhäsion wurden anschließend die glatten Aluminiumfolien durch verschiedene poröse PVDF-Folien miteinander verbunden (Bild~\ref{fig:bending_pre_tests}b). Aus den Versuchen ließ sich ableiten, dass sich das resultierende maximale Durchbiegungsverhalten als lineare Kombination der beiden Grenzfälle beschreiben lässt. Diese Überlagerung kann über den Phasenvolumenanteil $\varphi$ der flüssigen Phase formuliert werden zu
\begin{equation}
    u_\text{max} = \varphi \cdot u_{\text{max,flüssig}} + (1-\varphi) \cdot u_{\text{max,fest}}.
\end{equation}