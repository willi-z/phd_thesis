\chapter{Entwicklung einer effizienten multiskaligen Auslegungsmethodik von Strukturbatterien}
Die Modelle, die sich aus den Arbeiten von \textsc{Carlstedt}, \textsc{Doyle}, \textsc{Newman}, \textsc{Fuller} und \textsc{Plett} ergeben, sind mit einem hohen Detailgrad versehen. Dieser Detailgrad erlaubt eine hohe physikalische Präzision, gleichzeitig führt er jedoch zu einer großen Anzahl an Parametern, die aufwendig bestimmt werden müssen. Daraus ergibt sich das Problem, dass es in bestimmten Konstellationen schneller und günstiger ist, direkt Experimente mit allen Materialkombinationen durchzuführen, statt zuerst alle benötigten Material- und Interaktionsparameter zu bestimmen. Daher wird für eine möglichst optimale Entwicklungsstrategie in Kapitel~\ref{sec:efficent_development} ein entsprechendes Auswahl- und Bewertungsrahmenwerk entwickelt. Mithilfe dieses Rahmens wird dann in Kapitel~\ref{sec:reduction_and_parallelization} eine gezielte Reduktion weniger signifikanter Einflussfaktoren vorgenommen. Um die Einsatzfähigkeiten verschiedener Strukturbatterien automatisiert bewerten zu können, wird in Kapitel~\ref{sec:automated_failure} eine Versagens- und Risikoabschätzung vorgestellt. Abschließend werden in Kapitel~\ref{sec:validation} die simulierten Ergebnisse dieses Ansatzes mit experimentellen Ergebnissen verglichen.

\section{\label{sec:efficent_development}Konzeption einer effizienten Entwicklung von Strukturbatterien}
Für den effektiven Einsatz der Modelle zur Vorhersage mechanischer und elektrochemischer Eigenschaften wurden zahlreiche Anforderungen an die Modellierung zusammengestellt:
\begin{itemize}
    \item Modellierung basierend auf physikalischen Prozessen (kein reines Fitting),
    \item geringe Anzahl an Materialparametern und keine Einführung neuer, schwer bestim-mbarer Größen,
    \item ausreichend präzise für Vergleichbarkeit zwischen Ergebnissen,
    \item schnelle Berechnungen (keine Wochenlaufzeiten).
\end{itemize} 

Diese Anforderungen beruhen auf der Annahme, dass experimentell gewonnene Ergebnisse das reale Verhalten des untersuchten Objekts abbilden. Daraus folgt, dass Simulationsergebnisse diesem idealerweise möglichst nahekommen, jedoch stets ungenauer sind, solange experimentelle Messfehler vernachlässigt werden können~\cite{Morris2024}. Der mit hochwertigen Experimenten verbundene Aufwand ($k_{\mathrm{exp}}$) hinsichtlich Material- und Zeitkosten ist in vielen Fällen deutlich höher als der Aufwand für eine Computersimulation ($k_{\mathrm{sim}}$):
\begin{equation}
    k_{\mathrm{exp}} \gg k_{\mathrm{sim}}.
\end{equation}
Für einen rein experimentellen Ansatz, der jede mögliche Materialkombination ($n_{\mathrm{Kombis}}$) einer bestimmten Anzahl an experimentellen Bestimmungen ($n_{\mathrm{exp,Bestimmungen}}$) unterzieht, ergibt sich der gesamte Aufwand ($k_{\mathrm{exp, gesamt}}$) etwa als
\begin{align}
    k_{\mathrm{exp, gesamt}} &\approx k_{\mathrm{exp}} \cdot n_{\mathrm{Kombis}} \cdot n_{\mathrm{exp,Bestimmungen}}.
\end{align}
Der schnelle Anstieg des experimentellen Aufwands bei vielen Kombinationen kann zu hohen zeitlichen und finanziellen Kosten führen. Insbesondere im Kontext von Strukturbatterien liegt der Einzelaufwand $k_{\mathrm{exp}}$ für Präparation, Zusammenbau und Validierung einzelner Zellen häufig im Bereich mehrerer Monate.

Um viele potenzielle Materialkombinationen zu testen, sind simulationsbasierte Modelle daher unverzichtbar. Dabei muss jedoch sichergestellt werden, dass der Aufwand zur Bestimmung der Materialkennwerte und die Gesamtrechenzeit der Modellierung ($k_{\mathrm{sim, gesamt}}$) nicht den reinen experimentellen Aufwand übersteigen. Eine grobe Abschätzung des Gesamtaufwands für eine Simulationsstrategie lautet:
\begin{align}
    k_{\mathrm{sim, gesamt}}
    &= k_{\mathrm{sim}} \cdot n_{\mathrm{Kombis}} \cdot n_{\mathrm{Rechnungen}} \nonumber \\
    &\quad + \sum_{m=1}^{n_{\mathrm{Material}}} \Big( n_{\mathrm{exp, Bestimmung,m}} \cdot k_{\mathrm{exp}} + n_{\mathrm{lit, Bestimmung,m}} \cdot k_{\mathrm{lit}} \Big).
\end{align}
Für eine große Zahl an Materialkandidaten erhöht die Menge der experimentell zu bestimmenden Größen den Simulationsaufwand schnell. Unter Abwägung dieser Aufwände stellt ein mehrstufiges Verfahren oft einen guten Kompromiss zwischen Aussagegenauigkeit und Bestimmungsaufwand dar: Zunächst grobe, schnelle Simulationen zur Vorauswahl, dann detaillierte Simulationen für aussichtsreiche Kandidaten und abschließend gezielte Experimente zur Validierung und Feinkalibrierung.

\section{\label{sec:reduction_and_parallelization}Entkopplung und Parallelisierungsstrategie zur Reduktion des Simulationsaufwandes von Strukturbatterien}
\begin{figure}[!ht]
    \center
    \includegraphics[width=0.99\textwidth, angle=0]{simulation_model.pdf}
    \caption{\label{fig:homogenisation}Mehrskalige Homogenisierung der Strukturbatterie durch Abstraktion der Geometrie.}
\end{figure}
Strukturbatterien neigen aufgrund des etwa hundertfach geringeren effektiven Diffusionskoeffizienten von Kohlenstofffasern im Vergleich zu Graphit dazu, bei schnellen Lade- und Entladezyklen ineffizient zu werden~\cite{Uchida1996,Kim2021,Johansen2024}. Dies führt zu vergleichsweise langsamen Auf- und Entladeprozessen. In diesem Bereich verlieren insbesondere temperatureinflussabhängige Effekte an Einfluss auf das Gesamtverhalten und können in erster Näherung vernachlässigt werden~\cite{Carlstedt2019a,Carlstedt2018}. Auch die meisten Interaktionskoeffizienten zwischen den unterschiedlichen physikalischen Domänen haben nur geringen Einfluss und fallen oft mehrere Größenordnungen kleiner aus\footnote{Der Einfluss der Interkalation auf das Elastizitätsmodul liegt unter 1\,\%\cite{Carlstedt2019}.}~\cite{Carlstedt2022b}. Zwar existieren sowohl chemische als auch mechanische Wechselwirkungen; diese können jedoch aufgrund sicherheitstechnischer Restriktionen praktisch kaum experimentell untersucht und damit nur begrenzt validiert werden~\cite{Asp2024}. Aus diesem Grund erscheint eine gezielte Entkopplung der gekoppelten Feldgrößen gerechtfertigt und zweckmäßig.

Die für die betrachteten Belastungsfälle relevante Pouch-Folienbauform kann weiter vereinfacht werden. Insbesondere die in der realen Zelle vorhandenen Siegelstellen in der Mitte des Batteriestacks werden im abstrahierten Modell weggelassen, siehe Bild~\ref{fig:homogenisation}. Da die Diffusion durch den Elektrolyten im Vergleich zu den mechanisch induzierten Längenänderungen sehr schnell erfolgt, kann die Kopplung zwischen Diffusion und geometrischer Deformation ebenfalls vernachlässigt werden. 

\begin{figure}[!ht]
    \center
    \includegraphics[width=0.7\textwidth, angle=0]{bending.pdf}
    \caption{\label{fig:bending_electroylte_tests}Validierung des 3-Punkt-Biegeversuchs: a) visueller Vergleich von Simulation und Experiment, b) Kraftverlauf in Abhängigkeit der Durchbiegung für Pouchzellen mit und ohne Elektrolyt.}
\end{figure}

Da der Fokus dieser Arbeit primär auf den Batterieeffekten und nicht auf möglichen sensorischen Eigenschaften liegt, und da diese, wie in \cite{Carlstedt2023} gezeigt, selbst bei Anregung nur sehr kleine Ströme ausbilden, kann die Rückkopplung mechanischer Spannungen auf die elektrochemische Funktionalität entkoppelt werden. Infolge dieser Reduktion der Kopplungen kann die Gesamtverformung der Strukturbatterie als Summe zweier getrennter Beiträge beschrieben werden: der rein mechanisch verursachten Ausdehnung und der elektrochemisch bedingten Volumenänderung.

\begin{figure}[!ht]
    \center
    \includegraphics[width=0.65\textwidth, angle=0]{simulation_electro_chem.pdf}
    \caption{\label{fig:simulation_electro_chem}Reduktion des Rechenaufwands der elektrochemischen Simulation durch Modellreduktion und Homogenisierung.}
\end{figure}

Für die mechanische Simulation wird ausgenutzt, dass der gesamte Zellstapel bei einem 3-Punkt-Biegeversuch lediglich entlang der Biegelinie unterschiedliche Verschiebungen erfährt, während sich die Elemente in der dazu senkrechten Richtung aufgrund der vorhandenen Symmetrie ähnlich verhalten. Daher genügt es, einen dünnen, repräsentativen Streifen zu betrachten, der anteilig belastet wird. Zusätzlich können die einzelnen Schichten des Stapels durch die Kombination mehrerer repräsentativer Volumenelemente als Blockelemente beschrieben werden. Nichtlineare Eigenschaften dieser Volumenelemente können an ausgewählten Stellen ausgewertet und anschließend mittels Interpolation ohne zusätzlichen Rechenaufwand rekonstruiert werden. Eine vergleichende Studie des Deformationsverhaltens eines Batteriestacks mit und ohne Elektrolyt zeigte dabei gute Übereinstimmung mit den simulierten Ergebnissen, siehe Bild~\ref{fig:bending_electroylte_tests}.

Für die elektrochemische Simulation wird eine analoge Strategie verfolgt: Es wird nur ein Streifen der Batterie betrachtet, wobei jeder Zellstapel als Kombination mehrerer, auf jeweils ein Partikel reduzierter Ersatzschaltungen repräsentiert wird, siehe Bild~\ref{fig:simulation_electro_chem}. Dieser Ansatz basiert auf einem von \textsc{Moura et al.} modifizierten Single-Particle-Modell~\cite{Moura2017}. Als Basis dienen die Diffusionsgleichungen (\ref{eq:diffusion_sphere}) und (\ref{eq:diffusion_cylinder}) für die Partikel in den Elektroden sowie die Transportgleichung im Elektrolyten für die Bereiche Anode ($-$), Separator (sep) und Kathode ($+$):
\begin{equation}
    \frac{\partial c_{e,j}}{\partial t} = \frac{\partial}{\partial x}\left[\frac{D_{e}^{\text{eff}}(c_{e,j})}{\varepsilon_{e,j}} \frac{\partial c_{e,j}}{\partial x} (x,t)\right]-\operatorname{sign}(j)\frac{1-t_c^0}{\varepsilon_{e,j} F L_j} I(t),
\end{equation}
mit $j \in \{-,\text{sep},+\}$ und
\begin{equation}
    \operatorname{sign}(j) = \begin{cases}
        -1 & j = - ,\\
         0 & j = \text{sep},\\
         1 & j = +.
    \end{cases}
\end{equation}
Die Differentialgleichungen sind über folgende Randbedingungen verbunden:
\begin{align}
\frac{\partial c_{e,-}}{\partial x} (0_-,t) &= \frac{\partial c_{e,+}}{\partial x} (0_+,t) = 0,\\
D_{e,\text{eff},-}\frac{\partial c_{e,-}}{\partial x} (L_-,t) &= D_{e,\text{eff,sep}} \frac{\partial c_{e,\text{sep}}}{\partial x} (0_{\text{sep}},t),\\
D_{e,\text{eff,sep}}\frac{\partial c_{e,\text{sep}}}{\partial x} (L_{\text{sep}},t) &= D_{e,\text{eff},+} \frac{\partial c_{e,+}}{\partial x} (L_+,t),\\
c_e(L_-,t) &= c_e(0_{\text{sep}}, t),\\
c_e(L_{\text{sep}},t) &= c_e(L_+, t).
\end{align}
Die effektiven Eigenschaften werden, wie bereits im mechanischen Simulationsteil, durch Homogenisierung der zugrunde liegenden Mikrostruktur bestimmt. Auf diese Weise lässt sich der numerische Aufwand erheblich reduzieren, während die maßgeblichen physikalischen Effekte weiterhin mit ausreichender Genauigkeit erfasst werden.

\section{\label{sec:automated_failure}Versagensanalyse für Strukturbatterien und Risikoeinschätzung}
Mithilfe des entwickelten Modells kann der lokale Spannungszustand
\begin{equation}
\boldsymbol{\sigma} =
\begin{bmatrix}
\sigma_{xx} & \tau_{xy} & \tau_{xz} \\
\tau_{xy}   & \sigma_{yy} & \tau_{yz} \\
\tau_{xz}   & \tau_{yz}   & \sigma_{zz}
\end{bmatrix}
\end{equation}
für alle relevanten Materialschichten der Strukturbatterie bestimmt werden. Aufbauend darauf wird eine automatisierte Versagensanalyse durchgeführt, um kritische Bereiche zu identifizieren und das mit einem Schichtversagen verbundene Gesamtrisiko der Batterie zu bewerten.

\subsection{Versagenskriterien und plastisches Fließverhalten der Einzelschichten}
Für isotrope Materialien, etwa metallische Stromableiter oder Teile der Polymerhülle, wird das von-Mises-Kriterium~\cite{Hill1998} verwendet. Die Vergleichsspannung lautet
\begin{equation}
\sigma_\mathrm{v} = \sqrt{\frac{1}{2}
\left [
(\sigma_{xx}-\sigma_{yy})^2 +
(\sigma_{yy}-\sigma_{zz})^2 +
(\sigma_{zz}-\sigma_{xx})^2
\right ]
+ 3\left ( \tau_{xy}^2 + \tau_{yz}^2 + \tau_{xz}^2 \right )}.
\end{equation}
Versagen bzw. plastisches Fließen tritt ein, wenn
\begin{equation}
\sigma_\mathrm{v} \ge \sigma_\mathrm{y}
\end{equation}
gilt. Für ideal plastisches Verhalten ist die Fließspannung konstant und entspricht der Anfangsfließgrenze $\sigma_{\mathrm{y}0}$.

Zur realistischeren Beschreibung metallischer Werkstoffe wird ein isotrop verfestigendes Materialmodell verwendet. Die Fließspannung wächst mit zunehmender plastischer Deformation:
\begin{equation}
\sigma_\mathrm{y} = \sigma_{\mathrm{y}0} + H \, \bar{\varepsilon}^p ,
\end{equation}
wobei $H$ der isotrope Verfestigungsmodul und $\bar{\varepsilon}^p$ die äquivalente plastische Dehnung ist. Diese ergibt sich inkrementell aus
\begin{equation}
\Delta \bar{\varepsilon}^p = \sqrt{\frac{2}{3} \, \Delta \boldsymbol{\varepsilon}^p : \Delta \boldsymbol{\varepsilon}^p},
\end{equation}
und wird über den Belastungsverlauf aufsummiert:
\begin{equation}
\bar{\varepsilon}^{p}_{n+1} = \bar{\varepsilon}^{p}_{n} + \Delta \bar{\varepsilon}^{p}.
\end{equation}
Die Fließbedingung lautet damit
\begin{equation}
f(\boldsymbol{\sigma}, \bar{\varepsilon}^p) =
\sigma_\mathrm{v} - \left( \sigma_{\mathrm{y}0} + H \, \bar{\varepsilon}^p \right) \le 0.
\end{equation}
Bei Überschreitung ($f>0$) tritt plastische Deformation auf; die plastische Dehnungsrate wird nach dem assoziativen Fließgesetz berechnet:
\begin{equation}
\dot{\boldsymbol{\varepsilon}}^p
= \dot{\lambda}
\frac{\partial f}{\partial \boldsymbol{\sigma}}
= \dot{\lambda} \frac{3}{2} \frac{\boldsymbol{s}}{\sigma_\mathrm{v}},
\end{equation}
mit dem Deviatorspannungstensor
\begin{equation}
\boldsymbol{s} = \boldsymbol{\sigma} - \frac{1}{3}\operatorname{tr}(\boldsymbol{\sigma}) \boldsymbol{I}
\end{equation}
und dem plastischen Multiplikator $\dot{\lambda}$.

Für anisotrope, faserverstärkte Werkstoffe, insbesondere die strukturellen Verstärkungslagen, wird das Tsai-Wu-Kriterium verwendet. Es lautet:
\begin{equation}
F_1 \sigma_1 + F_2 \sigma_2 + F_{11} \sigma_1^2 + F_{22} \sigma_2^2
+ 2F_{12} \sigma_1 \sigma_2 + F_{66} \tau_{12}^2 \geq 1,
\end{equation}
wobei $\sigma_1$ und $\sigma_2$ die Normalspannungen in Faserrichtung und quer zur Faserrichtung sowie $\tau_{12}$ die Schubspannung in der Materialebene darstellen. Die Koeffizienten $F_i$ und $F_{ij}$ werden aus den Zug- und Druckfestigkeiten in Faserrichtung ($R^\text{z}_{||}$, $R^\text{d}_{||}$) sowie quer zur Faser ($R^\text{z}_{\bot}$, $R^\text{d}_{\bot}$) und der Schubfestigkeit ($R_{||\bot}$) bestimmt:
\begin{align}
F_1 &= \frac{1}{R^\text{z}_{||}} - \frac{1}{R^\text{d}_{||}}, &
F_{11} &= \frac{1}{R^\text{z}_{||} R^\text{d}_{||}}, \\
F_2 &= \frac{1}{R^\text{z}_{\bot}} - \frac{1}{R^\text{d}_{\bot}}, &
F_{22} &= \frac{1}{R^\text{z}_{\bot} R^\text{d}_{\bot}}, \\
F_{66} &= \frac{1}{R_{||\bot}^2}, &
F_{12} &\approx -\frac{1}{2}\sqrt{F_{11} F_{22}}.
\end{align}

\subsection{Verknüpfte Sicherheitsrisiken}
\begin{table}[ht]
    \centering
    \caption{\label{tab:failure_modes}Überischt des mit Versagens der Einzelschichten verknüpften Sichheitsrisikos.}
    \begin{tabularx}{\textwidth}{lXXX}
    \toprule
    &\makecell{Pouchfolienversagen\\\includegraphics[width=0.2\textwidth]{failure_modes/failure_mode_pouch.png}}
    &\makecell{Elektrodenversagen\\\includegraphics[width=0.2\textwidth]{failure_modes/failure_mode_electrode.png}}
    &\makecell{Separatorversagen\\\includegraphics[width=0.2\textwidth]{failure_modes/failure_mode_separator.png}}
    \\
    \midrule
    Funktion
        & Funktionsversagen der gesamten Batterie durch austrocknen
        & Leistungsverlust der Zelle
        & Funktionsversagen der Zelle, je nach Verschaltung auch der gesamten Batterie
    \\
    Brandgefahr
        & kein Risiko
        & kein Risiko
        & Flammenbildung durch Überhitzung
    \\
    Gesundheit
        & hohes Risiko durch austretendes Elektrolyt
        & kein Risiko
        & kein zusätzliches Risiko
    \\
    \bottomrule
    \end{tabularx}\\
    %\noindent{\footnotesize{\textsuperscript{*} Gemessen gegenüber \ce{Li}/\ce{Li+}.}}
\end{table}%

Wesentlich kritischer als das lokale Versagen einer einzelnen Schicht ist das damit verbundene Sicherheitsrisiko für die gesamte Strukturbatterie und ihre Umgebung. Ein rein mechanisches Versagen kann, abhängig von der betroffenen Lage, direkte Auswirkungen auf Funktionalität, Sicherheit und Umweltverträglichkeit haben. Tabelle~\ref{tab:failure_modes} gibt einen Überblick über die wichtigsten Versagensarten und deren Konsequenzen.

Besonders kritisch ist das Versagen des Separators, da dies zu einem internen Kurzschluss mit lokaler Überhitzung und im Extremfall zu thermischem Durchgehen führen kann. Ein Versagen der Pouchfolie kann zum Austreten des Elektrolyten und damit zu Funktionsverlust und Gesundheitsrisiko führen. Ein Versagen der Elektroden führt primär zu Leistungsverlust und verringerter Kapazität.

Im Rahmen der automatisierten Versagensanalyse wird nicht nur ein binärer Schadenszustand (intakt/versagt) bewertet, sondern zusätzlich eine qualitative Risikoklassifikation vorgenommen. Diese erlaubt eine Zuordnung von Simulationsergebnissen zu sicherheitsrelevanten Zuständen und dient als Grundlage für das strukturelle Batterie-Design unter Sicherheitsaspekten.

\section{\label{sec:validation}Validierung der Eigenschaftsvorhersagen}
Die Validierung der theoretischen Modelle erfolgt durch Vergleich mit experimentellen Daten eines Schichtverbunds aus neun Lagen, der in einer Stapelanordnung vier funktionale Einzelzellen (Anode–Separator–Kathode) realisiert. Als Referenz dient ein System mit Graphitanode auf Kupferfolie, NMC622-Kathode auf Aluminium und Celgard-2400-Separator mit flüssigem LP30-Elektrolyten. Die zu validierende Strukturbatterie basiert auf einem PX-35 Kohlenstofffasergelege mit Kupfer-Primer und Hardcarbon-Beschichtung; die mechanische Integrität wurde durch Modifikation des Elektrolyten mit KYNAR FLEX 28 erhöht.

Die experimentelle Charakterisierung erfolgte mittels elektrochemischer Zyklierungsversuche und 3-Punkt-Biegeversuchen. Die elektrische Prüfung wurde galvanostatisch durchgeführt, die Stromraten wurden schrittweise von $C/100$ bis $2C$ variiert, um Ratenfähigkeit und Energiedichte über bis zu 35 Zyklen zu erfassen. Parallel wurde die mechanische Tragfähigkeit im 3-Punkt-Biegeversuch bestimmt; die Proben wurden mit konstanter Traversengeschwindigkeit belastet, um die resultierende Kraftaufnahme in Abhängigkeit von der Durchbiegung zu dokumentieren.

\begin{figure}[!ht]
    \center
    \includegraphics[width=0.8\textwidth, angle=0]{electrical_sim_final.pdf}
    \caption{\label{fig:electrical_sim_final}Vergleich der experimentellen und simulierten gravimetrischen Energiedichte über 40 Zyklen bei variierenden Entladeraten für Referenz- und Strukturbatterie.}
\end{figure}

In Bild~\ref{fig:electrical_sim_final} erreicht die Referenzzelle eine Energiedichte von ca. 75,0 [Wh/kg], während die Strukturbatterie aufgrund zusätzlicher passiver Masse der strukturellen Komponenten etwa 43 [Wh/kg] erreicht. Die simulierten Vorhersagen zeigen gute Übereinstimmung mit den experimentellen Daten über das betrachtete Entladespektrum ($C/10$ bis $2C$).

\begin{figure}[!ht]
    \center
    \includegraphics[width=0.99\textwidth, angle=0]{mech_sim_final.pdf}
    \caption{\label{fig:mech_sim_final}Kraft-Durchbiegungs-Diagramm: experimentelle Validierung und korrigierte Simulation im 3-Punkt-Biegeversuch.}
\end{figure}

Die mechanische Validierung (Bild~\ref{fig:mech_sim_final}) zeigt: Die Referenzzelle erreicht eine maximale Kraftaufnahme von ca. 4,5 [N], die Strukturbatterie etwa 9,6 [N]. Geringe Abweichungen im Post-Peak-Bereich der Strukturbatterie lassen sich auf lokal begrenzte Versagensmechanismen im Hardcarbon-Slurry zurückführen, die über die globale Modellierung hinausgehen.

Zur Optimierung der Übereinstimmung wurde ein gezieltes Parameterfitting durchgeführt: Modellparameter wurden anhand der Messdaten der ersten Zyklen (elektrisch) sowie der linearen Phase der Dehnung (mechanisch) kalibriert. Dadurch können fertigungsbedingte Toleranzen (z. B. Schichtdickenvariationen oder Infiltrationsqualität des Elektrolyten) kompensiert werden.

\begin{figure}[!ht]
    \center
    \includegraphics[width=0.69\textwidth, angle=0]{plasticity.pdf}
    \caption{\label{fig:plasticity}Post-mortem-Analyse nach dem 3-Punkt-Biegeversuch im Vergleich zu simulierten Ergebnissen der plastischen Verformung: a) erste Schicht, b) siebte (mittlere) Schicht, c) fünfzehnte (letzte) Schicht.}
\end{figure}

Die postmortale Analyse zeigt gute Übereinstimmung zwischen experimentell beobachteten Deformationsmechanismen und den modellbasierten Vorhersagen. Plastische Verformungszonen in den einzelnen Schichten lassen sich in Lokalisation und Ausprägung mit den Prognosen korrelieren, was die Tauglichkeit des verwendeten plastischen Materialmodells bestätigt.
