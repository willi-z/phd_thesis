\chapter{Stand der Forschung}

Im folgenden Kapitel wird ein grundlegendes Verständnis für die Funktionsweise von Strukturbatterien vermittelt werden. Außerdem werden die Besonderheiten gegenüber konventionellen Batterien oder faserverstärkten Verbundwerkstoffen erläutert. Dazu werden die wichtigsten Eigenschaften und ihre Ermittlungsverfahren erläutert und Rolle der Einzelkomponenten im Zusammenhang der Materialauswahl näher erklärt. Anschließend werden aktuelle Entwicklungsansätze diskutiert und abschließend die ungelösten Herausforderungen mit den aktuellen Methoden näher analysiert.

\section{Grundlagen der Strukturbatterie}
Strukturbatterien sind 

\section{Wichtigsten Eigenschaften und ihre Ermittlungsverfahren}
%\subsection{Interkalation}
\subsection{Gesamtkapazität}
\subsection{Energiedichte}
\subsection{Elektrische Spannung}
\subsection{Zyklenverhalten}
\subsection{Steifigkeit und Festigkeit}
%\subsection{Mechanische Spannung}
\subsection{Multifunktionale Effizienz}

\section{Materialauswahl}

Jedes  Material in einer Strukturbatterie erfüllt mehrere Aufgaben gleichzeitig. Die am meisten benutzte Untergliederung teilt die Materialien nach ihrer elektrochemischen Rolle ein.


\subsection{Anode}
Die Anode sollte ein niedriges elektrochemisches Potenzial und eine schnelle Interkalation für eine möglichst hohe Energiedicht und Leistungsdichte aufweisen. Zusätzlich profitieren Strukturbatterien sehr von Anoden mit hohen Festigkeits- und Steifigkeitswerten.

Die Verwendung von Kohlenstoff in Lithium-Ionen Batterien wurde erstmalig von \textsc{Yoshino} \cite{Yoshino1986} 1886 veröffentlicht der für diesen Durchbuch 2019 den Nobelpreis erhielt.
Heute ist Kohlenstoff eines der meist benutzen Material in wiederaufladbaren nicht-aqueous Batterien \cite{Ahmad2021}. Am meist verbreitesten ist dabei die Kombination von Grafit als Anode und einer Kathode aus Phosphat, welche eine maximale Energiedichte von 200-250~$\si{\watt \hour \per \kg}$. 
Es gibt zwei Arten von Kohlenstoff die in der Lage sind Ionen einzulagern: geordenten und ungeordenten \cite{Ghosh2024}.

Geordenter Kohlenstoff sind Materialien mit einer weitreichenden Ordnung ud hoger Kristallinität. Die Ordnung kann sich dabei auf eine Achse (CNTs), eine Ebene (Graphen) oder den Raum (Grafit) begrenzen \cite{Wang2021}.

Grafit hat eine hoch-kristaline Struktur und besitzt eine weit reichende Ordnung. Die $\text{sp}^\text{2}$-hybritisierten Graphenelagen sind in lang der c-Achse gestappelt und folgen entweder die hexagona AB-Sequenz oder dei rhombohedrale ABC-Folge. Die binden $\pi$-Orbitale ermögliche eine gute Leitfähigkeit von $10^3$-$10^4$~$\si{\siemens \per \cm}$ in Ebenenrichtung. Die Graphenebenen ahben einen Abstand von 3,35~$\si{\angstrom}$ entlang der c-Achse und werden nur durch relativ schwache van der Waals Kräfte (16-17~$\si{\kJ \per \mol}$) zusammen gehalten. Der relative hohe Abstand und die schwachen Bindungskräfte machen es einfach, damit kleine Atome wie Lithium oder Kalium sich zwischen den Ebenen einlagern können \cite{Wang2021}.

Der Interkalationsprozess läuft dabei in vier Stufen ab, was sich im Potenzialverlauf erkennen lässt. Das Lithium-Ion wird dabei zwischen zwei benachbarten Graphene-Ebenen eingelager, wobei jedes Lithium-Ion den niedrigsten Energiezustand einnimmt, der im Zentrum eines hexagonalen Kohlenstoffring existiert \cite{Sole2014,Weng2023}. Allerdings können, Lithium-Ionen sich nicht durch die Grapheneschichten hindurchtunneln, wehalb sie die Transportbewegung zwischen die Schichten nur entlang von Gitterdefekten möglich ist \cite{Nishidate2005}. Die Einlagerungsgeschwindigkeit ist dabei nicht konstant und kann während jeder Stufe um teilweise das tausendfache einbrechen \cite{Levi1997}. Dieses Verhalten kommt nach \textsc{Aurbach et al.} durch die Bildung von Lithiumklustern zwischen den beiden Graphenschichten zu stande, welche die Diffusion weiterer Lithiumionen am Anfang einer neuen Phase verhindern \cite{Markevich2005}.  Das maximale Einlagerungsmenge ist mit der $\text{LiC}_\text{6}$-Konfugration erreicht, bei der zwischen jeder Grafitschicht alle möglichen Plätze beleget wurden. Die Menge an eingelagerten Lithiumion entspricht dabei  einer theoretischen spezischen Kapazität von 372~$\si{\mA \hour \per \g}$ \cite{Winter1998}. 
Eine weitere wichtige Eigenschaft ist die relative hohe Dichte von >2~$\si{\g \per \cm \cubed}$, was dabei hilft möglichst viel Aktivmaterial in kleinem Raum zu haben um kleine Batterien mit einer hohen Energiedichte zu erzeugen.

Seit seiner Entdeckung in 2004 \cite{Novoselov2004} ist Graphene zunehmend in den Fokus der Batterieforschung geraten mit einer theoretischen Kapazität von >1000~$\si{\mA \hour \per \g}$ hoher mechansicher Zugfestigkeit von $\approx$130~$\si{\GPa}$ und einer Zugsteifigkeit von $\approx$1~$\si{\tera \Pa}$ stellt es ein ideales Material für den Einsatz in Strukturbatterien da \cite{Novoselov2012}. Jedoch konnte das Material bisher nur im Labormaßstab und nur in unzureichenden Mengen syntehtisiert werden. Auch ist bisher umstritten, wie die Einlagerung von Lithium bei Graphene genau abläuft, was je nachdem die theoretische Kapazität noch stark noch oben oder unten korrigiert. Bisherige Experimente mit zweilagigen Graphene kommen zu unterschiedlichen Ergebnissen. \textsc{Ji et al.} beobachtete einen Mechanismus der auf einen ähnlichen Prozess wie bei Grafit vermuten lässt, während \textsc{Kühne et al.} sugenante super-dichte Lithiumeinlagerung zwischen den beiden Grapheneschichten gemessen haben will. Derzeitig geht die Produktion von Graphene nicht über den Labormaßstab hinaus und bleibt daher für den Einsatz in Strukturbatterien bis auf weiteres ungeeignet.

CNTs sind geordente 1D Kohlenstoffstrukturen, welche 1991 von \textsc{Iijima} \cite{Iijima1991} erstmalig entdeckt wurden. Diese zylindrischen Formen des Kohlenstoffes haben einen Durchmesser von 1-20~$\si{\nano\metre}$  und meist ein hohes Längen-zu-Durchmesser-Verhältnis, mit der höchsten bisher dokumentierten Länge von 55~$\si{\centi\metre}$ von \textsc{Zhang et al.} \cite{Zhang2013}. CNTs werden meist durch ihrer Schichtanzahl in SWCNT und MWCNT unterschieden. Darüber hinaus können SWCNTs je nach Winkel des graphenähnlichen Gitters im Mantels gegenüber der Zylinderachse metalische oder Halbleiterähnliche Eigenschaften aufweisen. 
SWCNTs und MWCNTs besitzen hohe spezifische Oberflächen (1300~$m^2/g$), eine sehr hohe elektische Leitfähigkeit (5000~$\si{\siemens \per \cm}$) und eine hohe Ionenleitfähigkeit von (>100000~$\si{\cm \squared \per \V \per \s}$) \cite{Xu2011,Uetani2014,Charlier2007}.

Ungeordneter Kohlenstoff hat keine weitreichende periodischen Strukturen in Ebenen oder Stapelrichtung. Sie bestehen hauptsächlich aus zufällig ausgerichten sp2 grafitischen Mikrobereichen und Verknüpfungen durch sp3 hybridiserte Kohlenstoffatome in amorphen Gebieten. Der Anteil der sp3-Verknüpfungen bestimmt ob eine Grafitisierung bei Temperaturen bis zu 3000~$\si{\degreeCelsius}$ möglich ist. Dies führt zu einer Unterscheidung in sogenannten harten oder weichen Kohlenstoff. 

Bei weichem oder grafitisierenden Kohlenstoff kann aufgrund der geringen Anzahl von sp3 Verknüfungen immer noch eine thermisch bedingten Mobilität der Kohlenstoffschichten erfolgen, was bei einer Wärmebehandlung von 1500-3000~$\si{\degreeCelsius}$ unter Sauerstoffausschluss(Pyrolyse) zu einer Umwandlung zu Grafit führt. Ein weitverbreiter Ansatz zu Herstellung von weichem Kohlenstoff ist die thermische Zersetzung von verschiednen organsichen Precursorn in einer inerten Atmosphare bei hohen Temperaturen (1000-1700 $\si{\degreeCelsius}$) (Karbonisierung). Besonders eigenen sich hierbei pyrolytische aromatische Verbindungen, wie etwa Pech, Benzol, Petrolkoks, Polyvinylacetat und Polyvinylchlorid \cite{Wang2021}. Die Wahl des sogenannten Precursormaterials und Prozessparameter haben maßgeblichen Einfluss auf chemische Strukut, die wiederum die Eigenschaften von ungeordenter Kohlenstoff bestimmt. Besonders entscheident ist hierbei der Kristallinitätsgrad oder Grafitisierungsgrad, welcher u.a. durch Ramanspektroskopie bestimmt werden kann \cite{Yu2014}. Die micro-kristalinen Grafitberiche haben dabei ein ein ähnliches Einlagerungsverhalten als Grafit. Die kleinere Menge an grafitischen Strukturen sorgt jedoch das die Ionenspeicherkapazität bei einer langsamen Beladung (C/10) von grafitischen Kohlenstoff nur etwa 250~$\si{\mA \hour \per \g}$ (Grafit 372~$\si{\mA \hour \per \g}$) erreicht. Jedoch ist die Einlagerung deutlich schneller, was bei höheren Beladungs- und Entladungstests (10C) zu einer drei Mal höheren Kapazität (weicher Kohelenstoff 90$\si{\mA \hour \per \g}$ und Grafit 25 $\si{\mA \hour \per \g}$) führt \cite{Schroeder2014}. Auch zeigt grafitisierender kohlenstoff im Gegensatz zu Grafit keine Einbrüche im Diffionsverhalten was für dafür spricht, dass die Einlagerung stufenlos erfolgt. sich bei höherenführen die felxibleren Graphenebnene und höhere Menge an Diffusionswegen durch eine höhere Anzahl an vorleigenden Gitterdefekten zu einem besseren Einlagerungsverhalten bei  Allerdings beleiben auch in den weniger geordneten Strukturen  mehr Lithiumionen gefangen, weshalb die CE während des ersten Zykluses für weichen Kohlenstoff nur bei etwa 72~\% (Grafit 82~\%) liegt. Jedoch ist nachdem Prelithierungspozess die CE auch hier bei über 99~\% \cite{Schroeder2014}. 

Harter oder nicht-grafitisierender Kohlenstoff lässt sich selbst bei hohen Karbonisierungstemperaturen (<3000~$\si{\degreeCelsius}$) nicht in Grafit umwandeln. Meist wird dieser aus der Karbonisierung von Precursoren mit wenigen aromatischen Strukturen, wie etwa Zuker, Holzkohle, Cellulose und Kokosnussschalen gewonnen \cite{Wang2021}. Die komplexeren Organischen Strukturen der Precursors sorgen, dass nach der Kabronisierung eine signifikante Anzahl an kleineren Poren und Risse in der Mikrostruktur verbleiben, die einen schnellen Zugang zu den Interkaltationsbereichen erlauben und für eine hohe aktive Oberfläche sorgen \cite{Liu2019a}. Graphenschichten $\approx$0.4~$\si{\nm}$ die ungeordente Microstuktur 
CR von 85~\% nach hundertausend Zyklen \cite{Cao2014}.

% Eine CAG ist ein hartcarbon?

%Eines der am frühsten und immer noch am weitverbreitesten Aktivematerialien anodenseitig ist Grafit. Zwischen den Grafitschichten können Lithiumionen eingelagert werden. In herkömmlichen monofunktionalen Batterien werden oft dünne Kupferfolien mit einer Grafitpartikelbeschichtung verwendet. Die zusätzliche Additive in der Pulvermischung halten die Partikel zusammen und sorgen für einen geringen Widerstand beim Transport der Elektronen zur Kupferelektrode. Die Bindungen zwischen den Partikeln sind jedoch sehr schwach und tragen nicht zur Steigerung der mechanischen Eigenschaften bei \cite{Chen2024}. Außerdem ~mAh/gsorgt die Ausdehnung infolge von Lithierung mit der Zeit für Risse durch die mit der Zeit der Leitungswiderstand steigt, was einer von vielen beobachten Alterungsmechanismen von Batterien ist \cite{Xiong2020}.

Die begrenzte Kapazität, langsame Diffusionskinetik, geringe mechanische Eigenschaften, sind einige der Faktoren die Untersuchungen Kohlenstoff-Nanostrukturen und andere Morphologien bewegen.

Damit ist einer der vielversprechesten Kandidaten für lastentragenden Anoden die Kohlenstofffaser. Ca. 96~\% aller Fasern weltweit werden aus Polyacrylnitril (PAN) hergestellt, die Restlichen werden aus Precursorn wie Pech, Rayon oder Lignin gewonnen \cite{Das2016}. Kohlenstofffasern besitzen im Allgemeinen hohe Festigkeits- und Steifigkeitswerten. Jedoch haben Wahl des sogenanten Precursormaterials, sowie die Verfahrensparameter während des Spinnens, Stabilisierens und Karbonisierens einen entscheidenten Einfluss auf die Struktur der Faser, was sich wiederum signifikant in den mechanischen, elektrischen und elektrochemischen Eigenschaften bemerkbar macht \cite{Newcomb2015}.
Verallgemeinert lässt sich feststellen, dass ein höherer Anteil an kristallien Grafitstrukturen in der Faser zu einer höheren Steifigkeit, Festigkeit, sowie thermische und elektrischen Leitfähigkeit verbessert. Jedoch sind 


Bereits unbehandelt können bis zu XXX mol/g eingelagert werden, was einer Flächenkapazität von etwa XXX entspricht. Nebenzahlreichen Prozessparametern, wie etwa Verstreckung und Temperatur während der XXX ist das Precursor-Material von entscheidenter Bedeutung. Untersuchungen von XXX haben gezeigt, dass PAN basierte Kohlestofffasern zur Skinbildung neigen, Pech basierte wiederum nicht. Die Skinbildungen erhoht die Steifigkeit der Faser, jedoch stellt diese auch eine Diffusionsbarriere dar, weshalb PAN basierte Fasern einen deutlich geringeren Diffusionskoeffizienten als PEEK basierte Fasern aufweisen. Allerdings sind auch die amorphen Strukturen in PITCH basierten Fasern deutlich weniger ausgeprägt, als bei PAN, was die Beladungsmenge deutlich reduziert.
% https://www.sciencedirect.com/science/article/pii/S1359835X16303451#b0235


\begin{table}[ht]
    \centering
    \caption{Übersicht bisher entwickelter Strukturbatterien.}
    \begin{tabular}[t]{lcccc}
    \toprule
    &Entlade Kapazität\textsuperscript{*} [mAh/g]
    &CR [\%] % Capacity Retention
    &$\text{D}_{\text{Li}}$ %[$\text{cm^2/s}$]
    &Ref.\\
    \midrule
    Graphit
        &356-372
        &98
        &$10^{-7}-10^{-6}$ ($10^{-11}$\textsuperscript{,K})
        &\cite{Persson2010,Wang2021,Olutogun2024}\\
    Graphen
        &770/1115
        %&100
        &90
        &$7 \times 10^{-5}$
        &\cite{Zhu2014,Wang2017,Kuehne2017}\\
    Kohlenstofff Nanoröhren
        &1115
        &90
        &$10^{-14}-10^{-11}$
        &\cite{Maurin1999,Zhao2000,Meunier2002,Shin2002,Nishidate2005,Schauerman2012}\\
    Harter Kohlenstoff
        &200-600 % 0.2C
        %802-1063 lade capacitität
        % 27.9-47.3 lade/entlade effizienz / Columbic Efficiency
        &72-90 % nach 50 Zyklen
        &$10^{-9}$-$10^{-8}$
        &\cite{Fujimoto2010,Bridges2012,Yang2012}\\
    Karbon Aerogel
        &349-570,2
        &31,9-97%(836.9-570.2)/836.9
        &n.a.
        &\cite{Yang2015,Pham2024,Li2022a}\\
    T300
        &91
        &46,5 % (170-91)/170
        &$10^-12-10^-11$
        &\cite{Uchida1996,Kjell2011,Johansen2022}\\
    T300 unbeschichtet
        &130
        &62,9 %(350-130)/350
        &$10^-12-10^-11$
        &\cite{Uchida1996,Kjell2011,Johansen2022}\\
    T800
        &98
        &42,4 % (170-98)/170
        &n.a.
        &\cite{Kjell2011,Johansen2022,Johansen2024}\\
    T800 unbeschichtet
        &112
        &42,3 %(194-112)/194
        &n.a.
        &\cite{Kjell2011,Johansen2022,Johansen2024}\\
    IMS65
        &108
        &34,9 %(166-108)/166
        &$10^{-8}-10^{-6}$
        &\cite{Kjell2011}\\
    IMS65 unbeschichtet
        &177
        &52,3 %(360-177)/350
        & $10^{-8}-10^{-6}$
        &\cite{Kjell2011,Kjell2013}\\
    \bottomrule
    \end{tabular}
    \noindent{\footnotesize{\textsuperscript{1} gemessen nach min. 10 Zyklen.}}
    \noindent{\footnotesize{\textsuperscript{K} Korngrenze.}}
    \noindent{\footnotesize{\textsuperscript{*} Die Abkürzung nicht auffindbar (n.a.) wurde benutzt.}}
\end{table}%

Conversion/alloying metalle wie etwa Silicium erreichen zwar Energiedichten >250Wh/kg jedoch ist ihre Aufnahme von Lithium mit großen Volumenänderungen verbunden, welche die Zyklenzahl drastisch reduziert.

\subsection{Kathode}

On the other hand, phosphate-based intercalation cathodes (LiFePO4 and LiMn1-xFexPO4) are the safest choice for high-power batteries. The robust phosphate framework undergoes minimal volume changes during de/lithiation, offers faster ionic diffusion, and does not release oxygen when damaged \cite{Ling2021}. However the material fails drastically in the absence of a conductive coating due to its poor electronic conductivity ($10^{\text{-}9}-10^{\text{-}11}$ S cm-1).

\subsection{Elektrolyte}
ionenleitfähigkeit und Festigkeit
Elektrolyteinterface
Sicherheitsaspekt bei zweiphasigen Elektrolyten (neg: Brennbarkeit Ionicliquid, pos: durch Wärme schmilzt bei Thermoplasitischen Systemen und verklept die Poren, was einen weiteren Ladungsaustausch bei Kurzschluss verhindern kann)
\subsection{Separator}

\begin{table}[h!]
    \caption{Properties of different types of separators}
    \label{tab_separator_comp}
    %\begin{adjustwidth}{-\extralength}{0cm}
    \newcolumntype{C}[1]{>{\hsize=#1\hsize\centering\arraybackslash}X}%
    \begin{tabularx}{\textwidth}{
    %C{0.6}
    C{1} 
    C{1.8} 
    C{0.8} 
    C{0.8} 
    C{0.8} 
    C{0.6}
    }
        \toprule
        \textbf{Separatortyp}
        &\textbf{Separatormaterial} 
        &\textbf{Ionische Leit- fähigkeit\textsuperscript{*} (mS/cm)} 
        &\textbf{Zug- steifigkeit\textsuperscript{*} (GPa)}
        & \textbf{Festigkeit\textsuperscript{*} (MPa)}
        &\textbf{Ref.} \\
        \midrule
        %\legendsep{c0}&
        Glassfaser&Glassfaser&1.13&21
        &325
        &\cite{Deka2017}\\
        %\midrule
        \addlinespace
        %\legendsep{c10}&
        Polymer&RF/PLA&110&0.3271
        &15.2
        &\cite{Vargun2020}\\
        %\midrule
        \addlinespace
        %Gel polymer electrolyte&$\mathrm{PVA/KOH/K_3[Fe(CN)_6]}$&45.56&n.a.&n.a.&\cite{maHighPerformanceSolidstate2014}\\
        %%\midrule
        %\legendsep{c4}&
        Feststoff- elektrolyt&$\mathrm{PEGDGE/TETA/EMIBF_4}$&0.2&26
        &350
        &\cite{Hubert2022, Choi2022}\\
        %\midrule
        \addlinespace
        %\multirowcell{2}{\legendsep{c6}}&
        \multirowcell{2}{Keramik}
            &$\mathrm{PVDF/PPG/LiCl/CaTiO_3}$&n.a.&1.2
            &65
            &\cite{Alvarez‐Sanchez2019}\\
            &$\mathrm{PVB/Al_2O_3NW}$&13.5&n.a.
            &30
            &\cite{Liu2020a}\\
        %\midrule
        \addlinespace
        %Diode-like polymer electrolyte&PVP/PEI/SWCNT&n.a.&n.a.&n.a.&\cite{chowdhurySupercapacitorsElectricalGates2019}\\
        %%\midrule
        %Ceramic&NPs/PTFE/SiC&n.a.&n.a.&1.3&\cite{qinCeramicBasedSeparatorHighTemperature2018,zhaoInorganicCeramicFiber2017}\\
        %%\midrule
        %Tree-leave&Quercus rubra&n.a.&n.a.&n.a.&\cite{chenTrashTreasureFallen2022,wangMechanicalCharacteristicsTypical2010}\\
        %%\midrule
        %Eggshell membrane&Eggshell membrane&3.8&n.a.&6.59&\cite{yuUsingEggshellMembrane2012}\\
        %%\midrule
        %\legendsep{c8}&
        Cellulose&MCC/AMIM-Cl&298.6&5.43
        &71.71
        &\cite{Ahankari2022, Xu2020}\\
        %%\midrule
        %Graphene oxide&Graphene oxide paper&n.a.&n.a.&n.a.&\cite{shulgaSupercapacitorsGrapheneOxide2015,comptonTuningMechanicalProperties2012}\\
        %%\midrule
        %Metal-organic framework&Metal-organic framework&n.a.&n.a.&n.a.&\cite{mengMetalOrganicFrameworks2015,bundschuhMechanicalPropertiesMetalorganic2012}\\
        \bottomrule
    \end{tabularx}
    %\end{adjustwidth}
    \noindent{\footnotesize{\textsuperscript{*} The abbreviation not available (n.a.) is used.}}
\end{table}

\subsection{Pouchfolie}
Herkömmliche Pouchzellen sind mit einer kunststoffbeschichteten Aluminiumhülle vor Umwelteinflüssen geschützt. Insbesondere verhindert diese das Feuchtigkeit in die Batterie eindringt und giftige oder brennbare Stoffe aus der Batterie entweichen können. Außerdem ermöglichen die guten mechanischen und Wärmeleiteigenschaften der Alumiumfolie eine geringe Gesamtmasse und eine effizientere Temperaturregulierung der Zellen. Eine zunehmend wichtiger werdende Aufgabe, die allerdings noch nicht hinreichend erfüllt, wird ist das Aufbirngen einen äußeren Zelldruckes.
In mehrere Studien konnte gezeigt werden, dass durch einen hohen externen Druck die Kontaktierung zwischen Elektrode und Elektrolyte verbessert wird, was einen besseren Ionen- und Elektronentransport bewirkt. Außerdem können ungewünschte Nebenreaktionen unterdrückt werden, wie etwa Gasbildung und Dendritwachstum, was den Lithiumverlust beim Laden und Entladen reduziert und somit dem Kapazitätsverlust entgegenwirkt und das Batterieleben verlängert \cite{Mussa2018,Mueller2019,Sakamoto2019}.
Besonders Batterien mit Feststoffelektrolyten benötigen einen deutlich höherer Druck um den Kontakt zwischen Elektrode und Elektrolyte zu gewährleisten \cite{Boaretto2021}. Jedoch existiert zurzeit noch keine zufriedenstellende Lösung. Zwar wird bereits bei der Herstellung mittels verpressen der Elektroden ein gewisser Druck realisiert, allerdings können größere Drücke damit nicht appliziert werden oder über längere Zeit aufrechterhalten werden \cite{Garayt2023}. Daher wird oft versucht durch eine externen Einspannung auf Systemebene diesen Druck aufzubringen. Jedoch entsteht durch die innere Reibung der Batterien kein gleichmäßiger Druckverlauf, was dazuführt, dass äußere Zellen stärker belastet werden und weiter innen liegende Zellen kaum von dem äußeren Druck profitieren. Auch haben höhere Ausgleichsdrücke, dass Problem, dass diese eine höhere Anstrengung für das Gesamtpaket darstellen, was zu dickeren Materialien und damit einer niedrigeren Gesamtenergiedichte führt.
Einzig die Knopfzellen, die durch eine integrierte Feder einen definierten Druck auf eine, im Verhältnis zur Pouchzelle, deutlich kleinere Fläche auswirkt ist die einzige bekannte Lösung zu diesem Problem. Hinzukommt, dass auch hier der Massenanteil von Gehäuse zu Zelle deutliche höher ist als bei Pouchzellen.

Für Strukturbatterien sind bisher keine Alternativen zum herkömmlichen Aluminiumpouchfolie untersucht wurden \cite{Ye2024}. Jedoch gibt es viele Gruppen die ihre Strukturbatterien mit Pouchfolie zusätzlich in einen kohlefaserverstärkten Kunststoff einbetten \cite{Pattarakunnan2020,Asp2021}. 


\section{Aktuelle Ansätze zur Entwicklung und Auslegung von Strukturbatterien}

\section{Ungelöste Herausforderungen in der Entwicklung von Strukturbatterien}
Erstellung Bild siehe Kommentar in .tex Datei
%Bild in Inkscape erzeugt und als SVG sowie pdf_tex speichern (Speichern unter -> .pdf -> Text in PDF weglassen und LaTex Datei erstellen). 


\begin{figure}[h]
	%\raggedleft
		%\def\svgwidth{\columnwidth}
	\def\svgscale{0.98}
		\input{testbild.pdf_tex} 
		\caption{\label{fig:testbild}Testbild erzeugt mit Inkscape}
\end{figure}

Bild \ref{fig:testbild} %\cite{Dannemann.Kucher_et.al_AppliedSciences_2018}  

Verwendung Package SIUNITX %(siehe Datei latex_package_readme_siunitx.pdf)   

Anzugsdrehmoment von $M_{\textnormal{a}}=\SI{1.1}{\newtonmetre}$

von \SI{1}{\kilo\hertz} bis \SI{15}{\kilo\hertz}

mittlere Temperaturänderung von $\left\langle \Delta T_{\textnormal{p}}\right\rangle(t)<\SI{0.5}{\degreeCelsius}$

Masse von $m_{\textnormal{p}}=\SI[separate-uncertainty]{0.884 (15)}{\gram}$

(siehe Abschnitt \ref{ch:anhang})