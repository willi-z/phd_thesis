\chapter{\label{sec:basics_sb}Grundlagen faserbasierter Strukturbatterien}

Im folgenden Kapitel wird ein grundlegendes Verständnis für die Funktionsweise von Strukturbatterien vermittelt. Außerdem werden deren Besonderheiten im Vergleich zu konventionellen Batterien und faserverstärkten Verbundwerkstoffen erläutert. Dazu werden die wichtigsten Eigenschaften dargestellt sowie die Rolle der einzelnen Komponenten im Zusammenhang mit der Materialauswahl näher erklärt. Abschließend werden aktuelle Entwicklungsansätze diskutiert und die ungelösten Herausforderungen unter Nutzung der aktuellen Methoden eingehend analysiert.

\section{Funktionsweise und Bauweisen} 
%Strukturbatterien sind Batterien, die mechanisch belastbar sind und damit auch zur strukturellen Integrität beitragen können.

Strukturbatterien ermöglichen die Speicherung von elektrischer Energie durch reversible elektrochemische Prozesse~\cite{Newman2021}. Dabei werden unterschiedliche Bindungsenergien bei der Einlagerung von ionischen Ladungsträgern ausgenutzt~\cite{Ashcroft2021}. Auf diesem Prinzip basierende Systeme werden in der Literatur auch "`Shuttle-Clock"'-, "`Rocking-Chair"'- oder "`Swing"'-Zellen bezeichnet~\cite{Ohzuku1993,Tarascon1993,Bittihn1993}. Der Einlagerungsprozess selbst wird meist als "`Interkalation"' bezeichnet~\cite{Eichinger1976}. Die Materialien, in denen der Interkalationsprozess stattfindet, nehmen dadurch eine aktive Rolle im Prozess ein, wodurch sich der Name "`Aktivmaterial"' ableitet (Bild~\ref{fig:battery_function}). Häufig sind die Aktivmaterialien für die Anode Graphit und für die Kathode eine Kombination verschiedener Metalloxide (\ce{MO_2}).
\begin{figure}[!ht]
    \center
	\includegraphics[width=0.9\textwidth, angle=0]{battery_function.pdf}
	\caption{\label{fig:battery_function}Darstellung des skalenübergreifenden Ladungsaustausches einer faserbasierten Strukturbatterie mit Graphitstrukturen in den Kohlenstofffasern als Interkalationsort für die Anode und Phasenumwandlungen im Metalloxid als Bindungssystem auf Kathodenseite}
\end{figure}
Beim Entladen der Batteriezelle wandern die Ladungsträger von der Anode zur Kathode. An beiden Elektroden kommt es zum Ladungsaustausch, der mithilfe der nachfolgenden Redox-Gleichungen beschrieben werden kann~\cite{Goodenough2013}. 
\begin{align}
	\ce{C_6 + x Li^+ + x e^- &<=> Li_xC_6},\\ 
	\ce{LiMO_2 - x Li^+ - x e^- &<=> Li_{1-x}MO_2} 
\end{align} 
Der Einlagerungsprozess erlaubt im Vergleich zur alleinigen elektrostatischen Speicherung, wie etwa bei Kondensatoren, eine signifikant höhere Energiedichte, wodurch größere Energiemengen bei gleicher Masse oder Volumen gespeichert werden können~\cite{Newman2021}.

Entscheidend für die Funktionsfähigkeit ist hierbei, dass der Transport der Ionen durch den Elektrolyt und den Separator erfolgt. Die Elektronen können jedoch nur entlang des Stromkollektors geleitet werden und werden daher außerhalb der Zelle entlang geleitet~\cite{Plett2015} (Bild~\ref{fig:battery_function}). Die Bewegung der Elektronen ist dabei im Vergleich zum Ionentransport deutlich schneller, weshalb die Ent- und Beladungsgeschwindigkeit einzig von der Ionenmobilität begrenzt wird~\cite{Plett2024}.

Im Falle großer Ströme, wie etwa bei einem Kurzschluss, finden viele Ladungsumwandlungsreaktionen gleichzeitig statt~\cite{Berg2022}. Die dabei entstehende Wärme kann zum elektrochemischen Versagen oder zum Brennen der Zelle führen~\cite{Wang2019a}. Mögliche Ursachen für Kurzschlüsse sind Herstellungsfehler, Dendritwachstum oder mechanische Belastungen, bei denen die Elektroden durch das Versagen des Separators in direkten Kontakt kommen~\cite{Kaliaperumal2021,Berg2022}. Typische Beispiele hierfür sind Penetration durch Nägel oder Biegung aufgrund unsachgemäßer Lagerung~\cite{Beard2019}.

Darüber hinaus führt die mit den mechanischen Belastungen einhergehende Rissbildung zu einer Erhöhung des inneren elektrischen Widerstandes, was einen Verlust an Speichereffizienz zur Folge hat und daher vermieden werden sollte~\cite{Plett2024}. Die Maßnahmen, um eine Verbesserung der strukturellen Eigenschaften zu erreichen, lassen sich dabei in zwei Kategorien einteilen~\cite{Jin2023} (Bild~\ref{fig:sb_type_scales}):
\begin{enumerate}[label=(\roman*)]
	\item Einbettung bestehender Batterien in tragende Strukturen und
	\item Multifunktionale Materialien mit Fähigkeiten zur Energiespeicherung und Lastenaufnahme.
\end{enumerate}
\begin{figure}[!ht]
    \center
	\includegraphics[width=\textwidth, angle=0]{sb_type_scales.pdf}
	\caption{\label{fig:sb_type_scales}Bauweisen von verschiedenen strukturtragenden Batteriesystemen: a) Bessere strukturelle Eigenschaften durch Neugestaltung des Batteriepaketes~\cite{Tesla2020}. b) Verstärkung durch Einlaminierung bestehender Pouchbagsysteme~\cite{Pattarakunnan2020}. c) Strukturbatterie als Ergebnis eines geschichteten Aufbaus verschiedener strukturverstärkenden Materialien~\cite{Asp2015}. d) Faserbatteriedesign durch mehrere dünne Materialbeschichtungen der Fasern~\cite{Thakur2020}}
\end{figure}

Kategorie (i) beinhaltet dabei alle Maßnahmen auf Zellebene, die meist durch Integration konventioneller Batterien zwischen verstärkenden Außenlagen aus Metall oder Kohlenstofffasern~\cite{Galos2020,Attar2020} umgesetzt werden, um eine versagensrelevante Belastung der Batterien zu verhindern~\cite{Beard2019}. Dadurch liegt weiterhin eine Funktionstrennung von Trag- und Energiespeicherfunktion vor, welche das Masseeinsparungspotenzial stark limitiert~\cite{Jin2023}. Allerdings ist dieser Ansatz sehr kompatibel mit bestehenden Produktionsprozessen und findet daher bereits Einsatz in der Massenproduktion von Batterien für elektrische Autos~\cite{Tesla2020}.

Bei Kategorie (ii) werden auf der Materialebene multifunktionale Materialien entwickelt, die sowohl Energie speichern als auch mechanische Beanspruchungen aufnehmen können~\cite{Asp2014}. Dadurch besteht ein größeres Potenzial für Masseeinsparungen, jedoch ist die Auslegung und Entwicklung solcher Materialien durch die zahlreichen Wechselwirkungen deutlich komplexer~\cite{Jin2023,Carlstedt2018}.
„Die Optimierung struktureller Batterien erfordert eine Synergie aus elektrochemischem Fachwissen und Leichtbauprinzipien. Hierfür müssen sowohl die spezifischen Eigenschaften der Komponenten als auch bewährte Materialien aus beiden Fachbereichen evaluiert werden~\cite{Asp2019}.

\section{\label{sec:Ermittlungsverfahren}Charakterisierung von Strukturbatterien}

Das Interesse an Strukturbatterien kommt aus zwei verschiedenen Fachbereichen: (1) der konventionellen Batterieforschung und (2) dem funktionsintegrativen Leichtbau. Beide Bereiche betrachten diese Technologie im Kontext unterschiedlicher Referenzsysteme. Daher erfolgt die Bewertung der Leistungsfähigkeit einer Strukturbatterie durch Vegleich mit aktuellen Hochleistungsbatterien, im Kontext von elektrochemischer Leistungsfähigkeit, als auch mit mechanische Hochleistungsmaterialien, wie etwa Kohlenstofffasern, im Kontext von Steifigkeit- und Festigkeitsbetrachtungen~\cite{O_Brien2011,Snyder2015}. Dies erfordert eine einheitliche Beschreibung der Strukturbatterieeigenschaften, um die Kennzahlen beider Fachdisziplinen systematisch miteinander zu vergleichen zu können. Darüber hinaus stellen perspektivische Anwendungen, wie der Automobilbereich~\cite{Martins2021,Pistoia2010}, die Luftfahrt~\cite{Ishfaq2022} und wachsende ökologische Bedenken~\cite{EU2023}, ein breites Spektrum an Anforderungen, das den Charakterisierungsaufwand weiter erhöht. 
Zusätzliche Komplexität entsteht durch die Heterogenität nicht-standardisierter Charakterisierungsmethoden, was eine objektive Bewertung neuer Forschungsansätze behindert~\cite{Greenhalgh2023,Zschiebsch2024}. In diesem Kapitel soll daher ein Überblick über die wichtigsten Größen gegeben werden.
Die Brennbarkeit~\cite{Shirshova2021,Shirshova2024}, \ce{CO2}-Produktion und Recyclingfähigkeit~\cite{Heyadati2024} sind zwar zukünftig von wachsender Bedeutung, sollen aber im Rahmen dieser Arbeit nicht betrachtet werden.

\subsection{Elektrochemische Charakterisierung}

%\subsection{Interkalation}
\subsubsection*{Ladungszustand}
Ein wesentliches Merkmal von Batterien ist die Interkalation von Ionen in das Aktivmaterial. Je nach Aktivmaterial steht dabei eine bestimmte Menge an Einlagerungsplätzen zur Verfügung. Der Ladungszustand (SOC\footnote{\textit{engl.} state of charge}) beschreibt den Anteil von aufgenommen zur maximal möglichen Menge an aufnehmbaren Ladungsträgern~\cite{Plett2015}.

Daraus folgt, dass bei einem SOC von 1 bzw. 100~\% keine weiteren Ladungsträger mehr aufgenommen werden können, und wenn alle Ionen das Aktivmaterial verlassen, der SOC gegen 0~\% fällt. Alternativ wird der Beladungszustand auch durch die stöchiometrische Größe $x$ angegeben, die besonders häufig im Kontext der Beschreibung von Zwischenzuständen (z.B. \ce{Li_xMn2O4} oder \ce{Li_{1-x}C}) benutzt wird~\cite{Newman2021}. Der SOC hat maßgeblichen Einfluss auf das elektrochemische Potenzial (Bild~\ref{fig:battery_voltage}), die Volumenausdehnung (Tabelle~\ref{tab:volume_change}) und die Diffusionsgeschwindigkeiten~\cite{Plett2024}. Der SOC kann jedoch nicht direkt gemessen werden und wird stattdessen über die Elektrodenspannung ermittelt~\cite{Newman2021}.

\begin{table}[ht]
    \centering
    \caption{Volumenänderungen bo der Lithiierung für gängige Aktivmaterialien.}
    \begin{tabular}[t]{lccc}
        \toprule
        Material& Vöumenänderung $\Delta$V/V\textsubscript{0}&Verlauf&Quelle\\
        \midrule
        Graphit & +10\% - +13\% & nicht-linear & \cite{Qi2010,Woodford2012}\\
        NMC111 &-2,4\%&nicht-linear& \cite{Yabuuchi2005}\\
        NMC422 &+2,4\%&nicht-linear& \cite{Ma2007}\\
        LCO &-1,9\% & linear & \cite{Reimers1992}\\
        NCA &-1,6\% & nicht-linear& \cite{Itou2005}\\
        LFP &+6,5\% & linear & \cite{Padhi1997}\\
        LMO &+6,6\% & linear & \cite{Christensen2006}\\
        \bottomrule
    \end{tabular}
\end{table}

\subsubsection*{Elektrische Spannung}

Die elektrische Spannung ist neben dem Strom, der direkt mit der Bewegung der Ladungsträger verbunden ist, eine der wichtigsten Größen zur Beschreibung des Zustandes in einer Batteriezelle~\cite{Beard2019}. 
\begin{figure}[ht]
    \center
	\includegraphics[width=0.9\textwidth, angle=0]{uocp.pdf}
	\caption{\label{fig:battery_voltage}Spannung über den stöchiometrischen Anteil der Beladung einer Lithiumionenbatterie, sowie die Anteile der negativen (Graphit) und positiven (Manganoxid) Elektrode (angelehnt an~\cite{Newman2021}).}
\end{figure}
Im Kontext der Elektrochemie ist die elektrische Spannung ein Maß für das elektrochemische Potenzial einer Elektrode. Ein Potenzial kann jedoch nur durch den Vergleich zu einem Referenzpotenzial, das dann oft als Nullpotenzial bezeichnet wird, gemessen werden. In der Batterieforschung wird hierbei häufig die Referenzspannung gegen reines Lithium gemessen~\cite{Newman2021}. Entscheidend ist hierbei, dass diese Referenzspannung bei einer Elektrode nicht konstant ist, da sich durch das Ein- und Auslagern das chemische Potenzial verändert. Der resultierende Spannungsverlauf ist abhängig von der chemischen Struktur, wodurch Aufschlüsse auf das Einlagerungsverhalten und damit verbundene Phasenumwandlungen gibt~\cite{Plett2015}, siehe Bild~\ref{fig:battery_voltage}.

Damit Batterien in der Anwendung sich nicht selbst entladen werden können, ist es wichtig, dass die Zellspannung\footnote{häufig auch als nominale Spannung bezeichnet} mit zunehmender Entladung stetig sinkt~\cite{Newman2021}~(Bild~\ref{fig:battery_voltage}). Dabei ist zu beachten, dass an der Elektrode das Potenzial immer dann sinkt, wenn sich diese auflädt\footnote{Anstieg des SOC}~\cite{Newman2021}. Da in einer Batterie durch den Ladungsaustausch immer eine Elektrode aufgeladen wird, während die andere entladen wird, kann der stetige Spannungsabfall während der Entladung sichergestellt werden, wenn die Referenzspannung einer Elektrode immer größer ist als die der anderen~\cite{Plett2024}. Aus dem stetigen Verlauf der Zellspannung leiten sich die Abschaltspannung\footnote{minimale erlaubte Spannung, die den entladenen Zustand markiert}~\cite{Plett2015} und die Leerlaufspannung\footnote{Spannung, bei der kein Strom fließt, markiert damit den Gleichgewichtszustand} ab~\cite{Newman2021}.

\subsubsection*{C-Raten}
Die Auflade- oder Entladerate einer Batterie wird oft in sogenannten C-Raten angegeben. Dabei bedeutet 1~C, dass eine vollständig entladene/geladene Batterie in 1~h komplett aufgeladen/entladen wird. Bei einer doppelt so hohen C-Rate wird die Batterie folglich in der Hälfte der Zeit entladen bzw. aufgeladen. Bei einer halb so hohen Auflade- bzw. Entladerate (C/2) benötigt die Batterie 2~h zur vollständigen Auf- oder Entladung. Die Wahl der C-Rate ist vor allem bei der Messung der Kapazität von Bedeutung. Je höher die C-Rate, desto geringer ist die gemessene Kapazität. Die Stärke des Kapazitätsabfalls wird durch eine Reihe von Faktoren, wie etwa Übergangsverhalten, Form und Art der chemischen Struktur der Elektrode bestimmt~\cite{Plett2015,Beard2019}.


\subsubsection*{Kapazität, Coulombeffizienz und Kapazitätserhalt}
Die Kapazität einer Elektrode beschreibt, wie viele Ladungsträger eingelagert oder entfernt werden können. Besonders in den ersten Zyklen besteht ein großer Unterschied zwischen Auflade- $\text{C}_{\text{aufl}}$ und Entladekapazität $\text{C}_{\text{entl}}$, weshalb die Kapazität für beide Prozesse getrennt bestimmt wird~\cite{Plett2015}.
Zur Ermittlung der Kapazität wird ein konstanter Auflade- $\text{I}_\text{C,aufl}$ oder Entladestrom $\text{I}_\text{C,entl}$ an eine Zelle aus Elektrode und Referenzelektrode, meist aus Lithiummetall, angelegt und die Zeit $\Delta \text{T}_\text{aufl}$ gemessen, die für die komplette Auf- bzw. Entladung benötigt wird. Da die Kapazität das zeitliche Integral des Stromes ist, kann die Bestimmung auf die folgenden Formeln vereinfacht werden~\cite{Newman2021}:
\begin{align}
	\text{C}_{\text{aufl}} &= \text{I}_\text{C,aufl} \cdot \Delta \text{T}_\text{aufl},\\
	\text{C}_{\text{entl}} &= \text{I}_\text{C,entl} \cdot \Delta \text{T}_\text{entl}.
\end{align}

%komplett entladen oder in der die Spannung von Abschalt- bzw. Leerspannung $\text{U}_{\text{leer}}$ zur vorher ermittelten maximal Spannung $\text{U}_{\text{voll}}$ benötigt. Die Aufladekapazit stellt dabei das Produkt aus Durch Umkehrung des Prozesses lässt sich die Entladekapazität bestimmen 

Bei der Entwicklung neuer Batteriematerialien wird die Kapazität meist auf die Masse des am Einlagerungsprozess teilnehmenden Materials (Aktivmaterial) normiert. %Diese spezifische Kapazität hat dann die Einheit [$\si{\A \hour \per \g}$]. Im Kontext der Batterieentwicklung wird allerdings die Kapazität je Elektrodenfläche [$\si{\A \hour \per \cm\squared}$] häufig angegeben.


%\subsection*{Columbische Effizienz}
Die Coulombeffizienz (CE) dient zur Bewertung der internen Batteriereaktionen. Die CE des Zyklus \( n \) ist definiert als das Verhältnis der gemessenen Kapazität während des Entladevorgangs \( C_{Dch}(n) \) zur Kapazität des vorherigen Beladungsvorgangs \( C_{Ch}(n) \) \cite{Tornheim2020}.
Die Formel
\begin{equation}
CE = \frac{C_{Dch}(n)}{C_{Ch}(n)}
\end{equation}
gilt dabei für Aufbauten, die in einem Entladenzustand zusammengebaut werden und daher zuerst beladen werden müssen. Zellen, die in einem beladenen Zustand gefertigt werden, wie etwa Lithium-Schwefel-Batterien, beginnen allerdings zuerst mit einem Entladungszyklus. Die korrekte Formel lautet in einem solchen Fall
\begin{equation}
    CE = \frac{C_{Dch}(n+1)}{C_{Ch}(n)}.
\end{equation}

%\subsection{Kapaziätserhalt}
Der Kapazitätserhalt (CR\footnote{\textit{engl.} Capacity Retention}) dient zur Bemessung des Anteils an Nebenreaktionen, die zu einem Kapazitätsverlust in Batterien führen. Dieser ist definiert als das Verhältnis von Entladungskapazität des $(n+1)$-ten Zyklus $C_{Dch}(n+1)$ und der des $n$-ten Zyklus $C_{Dch}(n)$:
\begin{equation}
    CR = \frac{C_{Dch}(n+1)}{C_{Dch}(n)}.
\end{equation}
In einigen Fällen wird CR auch im Verhältnis zur initialen Entlade-Kapazität bestimmt,
\begin{equation}
    CR = \frac{C_{Dch}(n)}{C_{Dch}(1)}.
\end{equation}
Dieser Ansatz ist besonders dann hilfreich, wenn die Langlebigkeit von Batterien zu quantifizieren ist \cite{Tornheim2020}.
Im Gegensatz zu der CE ist der CR meist relevanter für Hersteller und Endnutzer.

\subsubsection*{Energiedichte}
Die gravimetrische Energiedichte bzw. spezifische Energie bemisst, wie viel Energie pro eingesetzter Masse gespeichert werden kann. Alternativ wird mittels der volumetrischen Energiedichte die Menge an speicherbarer Energie pro Volumen angegeben. Beide Kenngrößen sind wichtige Größen, die bei der Entwicklung neuer Speichertechnologien nach Möglichkeit gesteigert werden sollen~\cite{Plett2015}.

Eine der größten Schwierigkeiten beim Umgang mit angegebenen Energiedichten aus der Literatur besteht in dem Umstand, dass oft bei der Masse oder dem Volumen auf verschiedene Komponenten Bezug genommen wird~\cite{Son2021}. Im Allgemeinen unterscheiden sich die Angaben darin, ob die Werte im Kontext von (1) Materialentwicklung, (2) Elektrodenentwicklung oder (3) Zellentwicklung
erhoben wurden (Tabelle~\ref{tab:energy_densities}).
So wird bei der Forschung an neuen Aktivmaterialien die spezifische Energie auf die eingesetzte Masse an Aktivmaterial bezogen~\cite{Son2021}. Im Bereich neuer Elektroden wird die Speicherkapazität entweder auf die Elektrodenmasse oder, im Falle einer Oberflächenbeschichtung, meist nur auf die Masse der Beschichtung normiert~\cite{Greenhalgh2023}. In Publikationen zu neuen Zellen gibt es noch mehr Varianten der massenbasierten Normierung. Je nach Autor finden sich hier Angaben in Referenz zur Masse der gebauten Knopfzelle oder der gebauten Pouchzelle~\cite{Akimoto1998,Liu2018}. Darüber hinaus gibt es bei Pouchzellen Varianten mit einer einzelnen Zelle oder einer mehrlagigen Ausführung~\cite{Schmuch2018}. Außerdem werden in manchen Publikationen die Mantelmaterialien herausgerechnet~\cite{Greenhalgh2023}.
\begin{table}[ht]
    \centering
    \caption{\label{tab:energy_densities}Spezifische Energie und Energiedichte für eine representative \ce{LiCoO2} (LCO) Kathode und eine Grafitanode in verscheidenen Referenzsystemen.\cite{Son2021}}
    \begin{tabular}[t]{lccccc}
    \toprule
    \multirow{2}{*}{}
    &\multirow{1}{*}{Materialebene} % \textsuperscript{*}
    &\multirow{1}{*}{Elektrodenebene}
    &\multicolumn{2}{c}{Zellebene}
    \\ \cmidrule{2-5}
    &\makecell{Aktivmaterial\\\includegraphics[width=0.125\textwidth]{CathodeMaterials/LiC6.png}\vspace{-1em}}
    &\makecell{Elektrode\\\includegraphics[width=0.1\textwidth]{EnergyDensitiesScales/Electrode.png}\vspace{-1em}}
    &\makecell{Knopfzelle\\\includegraphics[width=0.1\textwidth]{EnergyDensitiesScales/CoinCell.png}\vspace{-1em}}
    &\makecell{Pouchzelle\\(1 Ah)\\\includegraphics[width=0.1\textwidth]{EnergyDensitiesScales/PouchCell.png}\vspace{-1em}}
    \\
    \midrule
    \makecell{Spezifische Energie\\ $\left[ \si{\watt \hour \per \kg} \right]$} & 627 & 514 & 5 & 260\\
    \makecell{Energiedichte\\ $\left[ \si{\watt \hour \per \liter} \right]$} & 3166 & 1527 & 19 & 414\\
    \bottomrule
    \end{tabular}
    %\noindent{\footnotesize{\textsuperscript{*} Die Abkürzung nicht auffindbar (n.a.) wurde benutzt.}}
\end{table}%
Aufgrund der mangelnden Standardisierung bei der Publikation von Energiedichten ist die Evaluierung und Benchmarking-Analyse neuer Strukturbatteriesysteme mit einem erheblichen analytischen Aufwand verbunden~\cite{Greenhalgh2023, Zschiebsch2024}.

%\subsection{Zyklenverhalten}
\subsection{Mechanische Eigenschaften}
Während mechanische Kennwerte wie Steifigkeit und Festigkeit in der konventionellen Batterieforschung bisher eine untergeordnete Rolle spielten~\cite{Chen2024a} und meist erst im Kontext von Sicherheitsbetrachtungen zum mechanischen Versagen Berücksichtigung fanden~\cite{Zhu2023}, bilden sie bei Strukturbatterien das zentrale Forschungsmerkmal~\cite{Asp2021}.

Die mechanische Charakterisierung dieser Systeme steht jedoch vor erheblichen methodischen Hürden. Zur Ermittlung fundamentaler Kenngrößen wie der Zugsteifigkeit und -festigkeit werden zwar etablierte Standards wie ISO 527-4 und ISO 527-5 herangezogen~\cite{ISO527-4-2023, ISO527-5-2021, Xu2022, Liu2022a}. Die Übertragbarkeit dieser Normen auf multifunktionale Batteriesysteme ist jedoch aufgrund ihres ausgeprägten anisotropen Verhaltens~\cite{Carlstedt2020b} und der hohen Sprödigkeit~\cite{Kalnin1982} limitiert. Erschwerend kommt hinzu, dass die mechanische Antwort eine starke Abhängigkeit von extrinsischen Faktoren (Temperatur~\cite{Carlstedt2019a}, Feuchtigkeit~\cite{Kosfeld2023}) sowie vom elektrochemischen Ladezustand (SoC) aufweist~\cite{Qi2010, Duan2021}. Dieser mehrdimensionale Parameterraum führt bei einer umfassenden Materialklassifizierung zu einem exponentiell steigenden Prüf- und Ressourcenaufwand.

Besonders kritisch ist die Charakterisierung unter komplexen Lastpfaden. Während das Deformationsverhalten unter Druck bisher nur in Einzelfällen systematisch untersucht wurde~\cite{Liu2022a, DiMauro2023}, greift die Forschung häufig auf den Drei-Punkt-Biegeversuch (ISO 178) zurück~\cite{AriefBudiman2022, Keshavarzi2022}. Dieser liefert zwar effiziente Kennwerte zur Tragfähigkeit, wie sie für Anwendungen in Drohnenstrukturen benötigt werden~\cite{Hollinger2019}, ist jedoch für die präzise Werkstoffmodellierung unzureichend. Die inhärente Überlagerung von Zug-, Druck- und Schubspannungen im Biegeversuch erschwert die Extraktion reiner Materialkennwerte, die für fortgeschrittene Versagensmodelle (z. B. nach \textsc{Hooke}~\cite{Atanackovic2000}, \textsc{Cuntze}~\cite{Cuntze2004} oder \textsc{Puck}~\cite{Puck2004}) zwingend erforderlich sind~\cite{Saba2019}.

\subsection{Multifunktionale Effizienz}

Die einheitenlose sogenannte multifunktionale Effizienz ist ein Maß, um zu bewerten, ob ein Vorteil durch den Einsatz von multifunktionalen Lösungen gegenüber einem kombinierten Einsatz von monofunktionalen Komponenten entsteht~\cite{Johannisson2020}.
Der von \textsc{Snyder} et al.~\cite{Snyder2015} im Jahr 2011 veröffentlichte Ansatz beschreibt die multifunktionale Effizienz als Summe der mechanischen und elektrochemischen Effizienz.
\begin{equation}
	\eta_{\text{mutli}} = \eta_{\text{mech}} + \eta_{\text{elchem}}= \frac{E}{E_{\text{ref}}} + \frac{\Gamma}{\Gamma_\text{ref}} 
\end{equation}
Dabei wird häufig die mechanische Effizienz $\eta_{\text{mech}}$ durch das Verhältnis der Zugsteifigkeiten ($E/E_{\text{ref}}$) von System und Referenz und $\eta_{\text{mech}}$ durch das Verhältnis der Energiedichten $\Gamma$ angenähert.
Diese Herangehensweise erlaubt eine vereinfachte Betrachtung der sonst komplexen multidisziplinären Optimierung und ist nach \textsc{Ashby}~\cite{Ashby2000} eine der möglichen Optimierungsstrategien für Materialdesign und -auswahl. In der Strukturbatterieforschung hat sich der Bewertungsansatz mittels multifunktionaler Effizienz weitgehend durchgesetzt~\cite{O_Brien2011,Freund2018}.

Im Kontext von Strukturspeichern wird für als Referenzsystem für die mechanische Effizienz ein UD-Kohlenstofffasergelege und für die elektrochemischen Anteil eine kommerzielle Li-Ionenbatterie, die beide allerdings an neue Entwicklungen stets anzupassen sind~\cite{Sha2021}. Ein multifunktionaler Effizienzwert größer oder gleich eins bedeutet, dass durch den Einsatz eine Reduktion der Gesamtmasse gegenüber der monofunktionalen Lösung erreicht wird~\cite{Snyder2015}.

\section{Materialauswahl für Komponenten von Strukturbatterien}

Obwohl Materialien in Strukturbatterien multifunktional ausgelegt sind, orientiert sich deren Einteilung meist an der klassischen elektrochemischen Funktion. Der folgende Abschnitt beleuchtet dieses Spannungsfeld, indem zunächst die allgemeinen Materialanforderungen dargelegt und anschließend die spezifisch eingesetzten Werkstoffe für Anode, Kathode, Elektrolyt, Separator und Pouchfolie analysiert werden.


\subsection{Anode}
Die Anode sollte ein niedriges elektrochemisches Potenzial und eine schnelle Interkalation für eine möglichst hohe Energiedichte und Leistungsdichte aufweisen. Insbesondere bei Strukturbatterien sind außerdem Materialien mit hohen Festigkeits- und Steifigkeitswerten interessant.

Die Verwendung von Kohlenstoff in Lithium-Ionen-Batterien wurde erstmals von \textsc{Yoshino} \cite{Yoshino1986} im Jahr 1986 veröffentlicht.
Heute ist Kohlenstoff eines der meistbenutzten Materialien in wiederaufladbaren nicht-wässrigen Batterien \cite{Ahmad2021}. Am weitesten verbreitet ist dabei die Kombination von Graphit als Anode und einer Kathode aus Phosphat, welche eine maximale Energiedichte von 200 bis 250~$\si{\watt \hour \per \kg}$ erreicht. 
Für die elektrochemischen Eigenschaften von Kohlenstoff ist eine Unterteilung in geordneten und ungeordneten Kohlenstoff von besonderer Bedeutung~\cite{Ghosh2024}, siehe Bild~\ref{fig:carbon_types}.

\begin{figure}[!ht]
	%\raggedleft
		%\def\svgwidth{\columnwidth}
        \center
		%\input{Abbildungen/02_SoA/electrolyte_data/}
		\includegraphics[width=\textwidth, angle=0]{carbon_types.pdf}
	%\includegraphics[width=\textwidth, angle=0]{bicontinous_electrolyte.pdf}
		\caption{\label{fig:carbon_types}Unterteilung der Kohlenstoffarten nach \textsc{Ghosh} und \textsc{Wang}~\cite{Kundu2020,Wang2021,Liu2022b,Ghosh2024}.}
\end{figure}

Geordneter Kohlenstoff ist ein Sammelbegriff für sp\textsuperscript{2}-hybridisierte Kohlenstoffverbindungen\footnote{Dieser Begriff leitet sich aus dem Orbitalmodell ab, das die Elektronenhülle eines Atoms beschreibt. Für Kohlenstoffverbindungen sind vor allem die Hybridisierungsformen sp\textsuperscript{2}, die zu ebenen Strukturen führt, und sp\textsuperscript{3}, die räumliche bzw. tetraedrische Verbindungen ergibt, von Bedeutung.} mit einer weitreichenden Ordnung und damit verbundenen mit hoher Kristallinität. Die Ordnung kann dabei entlang (1) einer Achse (CNTs), (2) einer Ebene (Graphen) oder (3) aller drei Raumkoordinaten (Graphit) existieren~\cite{Wang2021}.

CNTs ssind hochgeordnete Kohlenstoffstrukturen mit ausgeprägter eindimensionaler Morphologie, welche erstmals 1991 von \textsc{Iijima}~\cite{Iijima1991} beschrieben wurden. Diese zylindrischen Formen des Kohlenstoffs haben einen Durchmesser von 1 bis 20~$\si{\nano\metre}$ und meist ein hohes Verhältnis von Länge zu Durchmesser, mit der höchsten bisher dokumentierten Länge von 55~$\si{\centi\metre}$. CNTs werden meist durch ihre Schichtanzahl in SWCNT\footnote{\textit{engl.} Single Wall Carbon Nano Tube} und MWCNT\footnote{\textit{engl.} Multi Wall Carbon Nano Tube} unterschieden. Darüber hinaus können SWCNTs, je nach Winkel des graphenähnlichen Gitters im Mantel gegenüber der Zylinderachse, metallische oder halbleiterähnliche Eigenschaften aufweisen. 
SWCNTs und MWCNTs besitzen hohe spezifische Oberflächen von etwa 1300~$\si{\m^2\per g}$, eine sehr hohe elektrische Leitfähigkeit von bis zu 5000~$\si{\siemens \per \cm}$ und eine hohe Ionenleitfähigkeit von über 100000~$\si{\cm \squared \per \V \per \s}$ \cite{Xu2011,Uetani2014,Charlier2007}.

Seit seiner Entdeckung im Jahr 2004 \cite{Novoselov2004} ist Graphen zunehmend in den Fokus der Batterieforschung geraten. Mit einer theoretischen Kapazität von über 1000~$\si{\mA \hour \per \g}$, einer hohen mechanischen Zugfestigkeit von annähernd 130~$\si{\GPa}$ und einer Zugsteifigkeit von ungefähr 1~$\si{\tera \Pa}$ stellt es ein ideales Material für den Einsatz in Strukturbatterien dar \cite{Novoselov2012}. Jedoch konnte das Material bisher nur im Labormaßstab und in unzureichenden Mengen synthetisiert werden~\cite{Shams2015}. Auch ist bisher umstritten, wie die Einlagerung von Lithium bei Graphen genau abläuft, was einen starken Einfluss auf die theoretische Kapazität hat~\cite{Safie2023, Singh2024}. Bisherige Experimente mit zweilagigem Graphen kommen zu unterschiedlichen Ergebnissen. \textsc{Ji} et al. beobachteten einen Mechanismus, der auf ein ähnliches Verhalten wie bei Graphit schließen lässt~\cite{Ji2019}, während \textsc{Kühne} et al. sogenannte superdichte Lithiumeinlagerungen zwischen den beiden Graphenschichten beschreiben~\cite{Kuehne2017}. Derzeit geht die Produktion von Graphen nicht über den Labormaßstab hinaus und es bleibt daher für den Einsatz in Strukturbatterien bis auf Weiteres ungeeignet~\cite{Zhu2014, Singh2024}.


Graphit zeichnet sich durch eine hochkristalline Struktur mit ausgeprägter Fernordnung aus. Die Basis bilden hexagonal angeordnete Kohlenstoffatome, die innerhalb einzelner Ebenen, den sogenannten Graphenschichten, über kovalente $\text{sp}^2$-Hybridorbitale fest miteinander verbunden sind. Jedes Kohlenstoffatom steuert dabei ein Elektron zu einem delokalisierten $\pi$-System bei, dessen freie Ladungsträger eine hohe elektrische Leitfähigkeit von $10^3$ bis $10^4$\,$\si{\siemens\per\cm}$ innerhalb der Schichtebenen ermöglichen~\cite{Dutta1953}. 

Die Stapelung dieser Ebenen erfolgt senkrecht dazu entlang der kristallographischen $c$-Achse (siehe Bild~\ref{fig:carbon_types}), wobei die Abfolge der Graphenschichten zwischen einer hexagonalen AB- und einer rhomboedrischen ABC-Sequenz variieren kann\footnote{Die Abfolge der Graphenschichten ist für die hexagonale AB-Sequenz und die rhomboedrische ABC-Folge unterschiedlich.}~\cite{Inagaki2014}. Im Gegensatz zu den starken Bindungen innerhalb der Ebenen werden die Schichten untereinander lediglich durch schwache Van-der-Waals-Kräfte mit einem Wert von 16 bis 17\,$\si{\kJ\per\mol}$ zusammengehalten~\cite{Xu2012}. Dies resultiert in einem vergleichsweise großen Schichtabstand von 0,335\,$\si{\nano\meter}$ entlang der $c$-Achse, was die Interkalation kleinerer Atome, wie z.\,B. Lithium oder Kalium, maßgeblich erleichtert~\cite{Wang2021}.

\begin{figure}[!ht]
	%\raggedleft
		%\def\svgwidth{\columnwidth}
        \center
		%\input{Abbildungen/02_SoA/electrolyte_data/}
		\includegraphics[width=\textwidth, angle=0]{graphite_intercalation.pdf}
	%\includegraphics[width=\textwidth, angle=0]{bicontinous_electrolyte.pdf}
		\caption{\label{fig:graphite_intercalation}Der Interkalationsprozess von Lithiumionen in Graphit unter Andeutung der vorwiegenden Gitterverzerrungen~\cite{Nishidate2005,Markevich2005}.}
\end{figure}

Der Interkalationsprozess läuft dabei in mehreren Stufen ab, was sich im Potenzialverlauf erkennen lässt. Das Lithium-Ion ($\text{Li}^{+}$) wird dabei zwischen zwei benachbarten Graphenschichten eingelagert, wobei jedes $\text{Li}^{+}$ den niedrigsten Energiezustand einnimmt, der im Zentrum eines hexagonalen Kohlenstoffrings existiert \cite{Sole2014,Weng2023} (Bild~\ref{fig:graphite_intercalation}). Allerdings können $\text{Li}^{+}$ sich nicht durch die Graphenschichten hindurch bewegen, weshalb diese Transportbewegung nur durch Gitterdefekte möglich ist \cite{Nishidate2005}. Die Einlagerungsgeschwindigkeit zwischen den Schichten ist dabei nicht konstant und kann während jeder Stufe um teilweise das Tausendfache schwanken \cite{Levi1997}. Dieses Verhalten kommt nach \textsc{Aurbach} et al. durch die Bildung von Lithium-Clustern zwischen den beiden Graphenschichten zustande, welche die Diffusion weiterer $\text{Li}^{+}$ am Anfang einer neuen Phase verhindern \cite{Markevich2005}. Die maximale Einlagerungsmenge ist mit der $\text{LiC}_\text{6}$-Konfiguration erreicht, bei der zwischen jeder Graphitschicht alle möglichen Plätze belegt sind. Die Menge an eingelagerten $\text{Li}^{+}$ entspricht dabei einer theoretischen spezifischen Kapazität von 372~$\si{\mA \hour \per \g}$ \cite{Winter1998}. 
Eine weitere wichtige Eigenschaft ist die relativ hohe Dichte von über 2~$\si{\g \per \cm \cubed}$, wodurch möglichst viel Aktivmaterial in einem geringen Volumen untergebracht und damit kleine Batterien mit einer hohen Energiedichte erzeugt werden können.

Im Gegensatz zum geordneten Kohlenstoff werden unter ungeordnetem Kohlenstoff alle Strukturen zusammengefasst, die durch amorphe sp\textsuperscript{3}-hybridisierte Bereiche die weitreichende periodische Struktur in zufällig ausgerichteten sp\textsuperscript{2}-graphitischen Mikrobereichen zergliedern~\cite{Inagaki2014}. Dabei entscheidet der Graphitisierungsgrad\footnote{Anteil an sp\textsuperscript{2}-hybridisierten Bereichen. Dieser lässt sich u.a. durch Ramanspektroskopie bestimmen~\cite{Yu2014}.}, ob eine Graphitisierung bei Temperaturen bis zu 3000~$\si{\degreeCelsius}$ in einer inerten Atmosphäre möglich ist\footnote{Dieser Prozess wird auch als Pyrolyse bezeichnet~\cite{Kim2017a}.}~\cite{Inagaki2014}. Dies führt zu einer Unterscheidung in sogenannten weichen bzw. graphitisierenden und harten bzw. nicht-graphitisierenden Kohlenstoff, siehe Bild~\ref{fig:carbon_types}.

Bei weichem bzw. graphitisierendem Kohlenstoff reduzieren die amorphen, $\text{sp}^3$-hybridisierten Bereiche die Ionenspeicherkapazität bei geringen Beladungsraten (C/10). Während Graphit eine theoretische Kapazität von 372\,$\si{\mA\hour\per\g}$ aufweist, erreicht graphitischer Kohlenstoff unter diesen Bedingungen lediglich etwa 250\,$\si{\mA\hour\per\g}$~\cite{Schroeder2014}. Ein wesentlicher Vorteil liegt jedoch in der deutlich beschleunigten Einlagerungskinetik: Bei hohen Belastungen ab 10C erzielt weicher Kohlenstoff mit 90\,$\si{\mA\hour\per\g}$ eine mehr als dreimal höhere Kapazität als Graphit mit nur 25\,$\si{\mA\hour\per\g}$~\cite{Schroeder2014}. Zudem zeigt graphitisierender Kohlenstoff im Gegensatz zu Graphit keine Einbrüche im Diffusionsverhalten, was auf einen stufenlosen Einlagerungsmechanismus hindeutet~\cite{Huajun2007}.Ein Nachteil der weniger geordneten Strukturen ist jedoch der erhöhte Anteil irreversibel gebundener Lithium-Ionen ($\text{Li}^+$). Infolgedessen liegt die CE im ersten Zyklus für weichen Kohlenstoff bei nur etwa 72\,\%, während Graphit 82\,\% erreicht~\cite{Schroeder2014}. Diese Verluste treten primär in der Initialphase auf; nach einer ausreichenden Prelithiierung\footnote{Verfahren zur Beladung der Anode mit Lithium vor dem regulären Betrieb, um Initialverluste auszugleichen.} steigt die CE auch für graphitisierenden Kohlenstoff auf über 99\,\%~\cite{Schroeder2014}.Die Herstellung von weichem Kohlenstoff erfolgt häufig durch thermische Zersetzung organischer Precursoren unter inerter Atmosphäre bei Temperaturen zwischen 1000 und 1700\,$\si{\degreeCelsius}$\footnote{Dieser Prozess wird in der Fachliteratur als Karbonisierung bezeichnet.}~\cite{Ghosh2024,Kim2017a}. Als Ausgangsstoffe eignen sich besonders aromatische Verbindungen wie Pech, Benzol oder Petrolkoks sowie Polymere wie Polyvinylacetat und Polyvinylchlorid~\cite{Wang2021}.

Harter bzw. nicht-graphitisierender Kohlenstoff wird primär durch die Karbonisierung von Precursoren mit geringem Anteil aromatischer Strukturen gewonnen, wie beispielsweise Zucker, Cellulose oder biogenen Reststoffen wie Kokosnussschalen~\cite{Wang2021}. Die komplexen organischen Vernetzungen der Ausgangsmaterialien verhindern eine weitreichende Ordnung während des thermischen Prozesses, sodass eine signifikante Anzahl an Mikroporen und Defekten in der Struktur verbleibt~\cite{Liu2019a}. Diese Fehlstellen resultieren in einer hohen aktiven Oberfläche und ermöglichen einen schnellen Zugang zu den ungeordneten graphitischen Domänen, in denen die Interkalation stattfindet~\cite{Fujimoto2010}.
Ein herausragendes Merkmal von nicht-graphitisierenden Kohlenstoffen ist ihre exzellente Zyklenstabilität; so konnte eine CR von 85\,\% selbst nach einhunderttausend Zyklen nachgewiesen werden~\cite{Cao2014}. Allerdings führt der geringe Graphitisierungsgrad im Vergleich zu weichem Kohlenstoff oder Graphit zu einer deutlich reduzierten Kapazität, die mit etwa 53\,$\si{\mA\hour\per\g}$ bei 1C und 48\,$\si{\mA\hour\per\g}$ bei 20C quantifiziert wird~\cite{Sun2017}.

% Eine CAG ist ein hartcarbon?

%Eines der am frühsten und immer noch am weitverbreitesten Aktivematerialien anodenseitig ist Graphit. Zwischen den Graphitschichten können Lithiumionen eingelagert werden. In herkömmlichen monofunktionalen Batterien werden oft dünne Kupferfolien mit einer Graphitpartikelbeschichtung verwendet. Die zusätzliche Additive in der Pulvermischung halten die Partikel zusammen und sorgen für einen geringen Widerstand beim Transport der Elektronen zur Kupferelektrode. Die Bindungen zwischen den Partikeln sind jedoch sehr schwach und tragen nicht zur Steigerung der mechanischen Eigenschaften bei \cite{Chen2024}. Außerdem ~mAh/gsorgt die Ausdehnung infolge von Lithiierung mit der Zeit für Risse durch die mit der Zeit der Leitungswiderstand steigt, was einer von vielen beobachten Alterungsmechanismen von Batterien ist \cite{Xiong2020}.

%Die begrenzte Kapazität, langsame Diffusionskinetik, geringe mechanische Eigenschaften, sind einige der Faktoren die Untersuchungen Kohlenstoff-Nanostrukturen und andere Morphologien bewegen.
\begin{figure}[!ht]
	%\raggedleft
		%\def\svgwidth{\columnwidth}
        \center
		%\input{Abbildungen/02_SoA/electrolyte_data/}
		\includegraphics[width=0.90\textwidth, angle=0]{fiber_comparison.pdf}
	%\includegraphics[width=\textwidth, angle=0]{bicontinous_electrolyte.pdf}
		\caption{\label{fig:fiber_comp}(1) Vergleich der Entladungskapazität nach dem ersten Zyklus für PAN-basierte und Pech-basierte Fasern im Bezug auf a) Zugmodul und b) Zugfestigkeit der Fasern. (2) Vergleich der a) Entladekapazität und b) Coulombeffizienz für beschichtete und unbeschichtete PAN-basierte Fasern~\cite{Kjell2011,Snyder2009a,Hagberg2016,Ye2024}.}
\end{figure}
Kohlenstofffasern sind einer der vielversprechendsten Kandidaten für lasttragende Anoden~\cite{Greenhalgh2024a}. Ungefähr 96~\% aller Fasern weltweit werden aus Polyacrylnitril (PAN) hergestellt~\cite{Das2016}. Die restlichen werden aus Precursorn wie Pech, Rayon oder Lignin gewonnen. Kohlenstofffasern besitzen im Allgemeinen hohe Festigkeits- und Steifigkeitswerte sowie eine elektrisch gut leitende Oberfläche, die mit 0,2~$\si{\metre\squared\per\g}$ für konventionelle Kohlenstofffasern zu klein für Batterieanwendungen ist~\cite{Qian2013,Senokos2023}. Durch verschiedene Oberflächenmodifikationen kann diese allerdings auf über 200~$\si{\metre\squared\per\g}$ gesteigert werden~\cite{Zenkert2024}. Jedoch haben die Wahl des sogenannten Precursormaterials sowie die Verfahrensparameter während des Spinnens, Stabilisierens und Karbonisierens einen entscheidenden Einfluss auf die Struktur der Faser~\cite{Newcomb2015}. Diese hat wiederum signifikanten Einfluss auf die mechanischen, elektrischen und elektrochemischen Eigenschaften.

Verallgemeinert lässt sich feststellen, dass ein höherer Anteil an kristallinen Graphitstrukturen in der Faser zu einer höheren Steifigkeit, Festigkeit sowie thermischen und elektrischen Leitfähigkeit führt~\cite{Zenkert2024}. Jedoch ist die Kapazität von 150~$\si{\mA\per\g}$ (C/10) bei diesen hochmoduligen Fasern, wie etwa M60J, deutlich geringer als bei Fasern mit niedrigem Kristallinitätsanteil, wie etwa T800 mit 265~$\si{\mA\per\g}$ und IMS65 mit 358~$\si{\mA\per\g}$ \cite{Fredi2018} (Bild~\ref{fig:fiber_comp}). Die geringere Kapazität kommt durch die relativ großen, sich wie ein Mantel um die Faser ausbildenden Kristallstrukturen und turbostatischen Graphitstrukturen zustande, die einen radialen Ionentransport stark behindern~\cite{Zenkert2024,He2021}. Bei Fasern mit weniger ausgeprägter Graphitkristallausbildung bieten die zahlreichen Gitterdefekte, ähnlich wie bei nicht-graphitisierenden Kohlenstoff, genug Zugang für die $\text{Li}^{+}$, um sich bei kleineren Beladungsraten vollständig einlagern zu können \cite{Fredi2018}. Dies deckt sich mit Beobachtungen, dass sich Lithium zunächst in den ungeordneten, amorphen Bereichen einlagert und erst bei höherer Beladung auch die graphitischen Strukturen besetzt werden \cite{Fang2022}. Wie auch bei graphitischen Kohlenstoffen verlieren Kohlenstofffasern einen großen Teil ihrer Ladungsträger während des ersten Zyklus \cite{Jacques2013} (Bild~\ref{fig:fiber_comp}). Jedoch bleibt die CE auch nach zehn Zyklen bei einem Wert über 99,9~\% \cite{Hagberg2016}, was bedeutet, dass der weitere Beladungs- und Entladungsprozess nahezu verlustfrei ist. Allerdings hat die Einlagerung von Ionen auch zur Folge, dass sich die mechanischen Eigenschaften der Fasern ändern. Dabei verdoppelte sich der E-Modul quer zur Faserrichtung im lithiierten Zustand und geht während der Delithiierung nahezu vollständig auf die Werte im Ursprungszustand zurück~\cite{Duan2021}. Für den E-Modul in Faserrichtung ergeben sich keine Veränderungen. Weiterführende Zugversuche im lithiierten und delithiierten Zustand zeigten außerdem, dass die Zugfestigkeit während der Lithiierung um 25-30~\% zurückgeht und selbst nach der Entladung um 5-10~\% geringer ist als im ursprünglichen Zustand \cite{Jacques2012}. Versuche mit verschiedenen Lithiierungsgraden konnten dabei eine direkte Abhängigkeit zur Zugfestigkeit feststellen \cite{Jacques2014}, was darauf schließen lässt, dass die durch die Einlagerung verursachten Dehnungen im Material maßgeblich den Festigkeitsverlust beeinflussen \cite{Zenkert2024}. Der Festigkeitsverlust im Zusammenhang mit einer multifunktionalen Nutzung muss damit zwar unbedingt berücksichtigt werden, spielt aber besonders bei Anwendungen mit hoher geforderter Steifigkeit eine untergeordnete Rolle, da eine weitere Degradation der Fasern nicht beobachtet wurde \cite{Zenkert2024}.

\begin{table}[ht]
    \centering
    \caption{Übersicht bisher entwickelter Strukturbatterien.}
    \begin{tabular}[t]{lcccc}
    \toprule
    &Entlade Kapazität\textsuperscript{*} [mAh/g]
    &CR [\%] % Capacity Retention
    &$\text{D}_{\text{Li}}$ %[$\text{cm^2/s}$]
    &Ref.\\
    \midrule
    Graphit
        &356-372
        &98
        &$10^{-7}-10^{-6}$ ($10^{-11}$\textsuperscript{,K})
        &\cite{Persson2010,Wang2021,Olutogun2024}\\
    Graphen
        &770/1115
        %&100
        &90
        &$7 \times 10^{-5}$
        &\cite{Zhu2014,Wang2017,Kuehne2017}\\
    Kohlenstofff Nanoröhren
        &1115
        &90
        &$10^{-14}-10^{-11}$
        &\cite{Maurin1999,Zhao2000,Meunier2002,Shin2002,Nishidate2005,Schauerman2012}\\
    Harter Kohlenstoff
        &200-600 % 0.2C
        %802-1063 lade capacitität
        % 27.9-47.3 lade/entlade effizienz / Columbic Efficiency
        &72-90 % nach 50 Zyklen
        &$10^{-9}$-$10^{-8}$
        &\cite{Fujimoto2010,Bridges2012,Yang2012}\\
    Karbon Aerogel
        &349-570,2
        &31,9-97%(836.9-570.2)/836.9
        &n.a.
        &\cite{Yang2015,Pham2024,Li2022a}\\
    T300
        &91
        &46,5 % (170-91)/170
        &$10^-12-10^-11$
        &\cite{Uchida1996,Kjell2011,Johansen2022}\\
    T300 unbeschichtet
        &130
        &62,9 %(350-130)/350
        &$10^-12-10^-11$
        &\cite{Uchida1996,Kjell2011,Johansen2022}\\
    T800
        &98
        &42,4 % (170-98)/170
        &n.a.
        &\cite{Kjell2011,Johansen2022,Johansen2024}\\
    T800 unbeschichtet
        &112
        &42,3 %(194-112)/194
        &n.a.
        &\cite{Kjell2011,Johansen2022,Johansen2024}\\
    IMS65
        &108
        &34,9 %(166-108)/166
        &$10^{-8}-10^{-6}$
        &\cite{Kjell2011}\\
    IMS65 unbeschichtet
        &177
        &52,3 %(360-177)/350
        & $10^{-8}-10^{-6}$
        &\cite{Kjell2011,Kjell2013}\\
    \bottomrule
    \end{tabular}
    \noindent{\footnotesize{\textsuperscript{1} gemessen nach min. 10 Zyklen.}}
    \noindent{\footnotesize{\textsuperscript{K} Korngrenze.}}
    \noindent{\footnotesize{\textsuperscript{*} Die Abkürzung nicht auffindbar (n.a.) wurde benutzt.}}
\end{table}%

Abschließend sei erwähnt, dass Umwandlungsmetalle wie etwa Silizium zwar Energiedichten größer als 250~$\si{\watt \hour \per \kg}$ erreichen, jedoch ist der Interkalationsprozess mit großen Volumenänderungen verbunden, wodurch sich die Zyklenzahl drastisch reduziert~\cite{Gayet2009, Pereira2019}. Die großen Dehnungsunterschiede und die geringe mechanische Belastbarkeit machen diese Art von Anodenmaterial daher uninteressant für den Einsatz in Strukturbatterien~\cite{Javaid2018}.

\subsection{Kathode}

Kohlenstofffasern spielen nicht nur bei faserbasierten Anoden für Strukturbatterien eine entscheidende Rolle~\cite{Ye2024}. Im Kontext von Faserkathoden dienen sie jedoch nicht als Aktivmaterial, sondern übernehmen hauptsächlich die Aufgabe des Stromkollektors, wodurch Materialien wie Aluminium effektiv substituiert werden können~\cite{Martha2012}. Zudem wurde beobachtet, dass Kathodenmaterialien wie \ce{LiFePO4} einen geringeren Kapazitätsverlust mit Kohlenstofffasern als Trägermaterial aufweisen als bei Aluminium~\cite{Martha2011}.

\begin{table}[ht]
    \centering
    \caption{\label{tab:cathode_material}Übersicht verschiedener Aktivmaterialien für Kathoden.}
    \begin{tabular}[t]{lcccc}
    \toprule
    &
    &\makecell{Kapazität\\$\left[ \si{\mA \hour \per \g} \right]$} % \textsuperscript{*}
    &\makecell{Betriebsspannung\textsuperscript{*}\\$\left[ \si{\V} \right]$}
    %&\makecell{E-Modul\\ $\left[ \si{\GPa} \right]$}
    %&\makecell{Zugfestigkeit\\ $\left[ \si{\MPa} \right]$}
    &\makecell{Ref.}
    %&CR [\%] % Capacity Retention
    %&$\text{D}_{\text{Li}}$ %[$\text{cm^2/s}$]
    %&Ref.
    \\
    \midrule
    \ce{LiCoO2}&\makecell{\includegraphics[width=0.18\textwidth]{CathodeMaterials/LiCoO2.png}\vspace{-1.0em}}
    &140&4,1&\cite{Zhang2004,Lyu2020}\\
    \ce{Li2NiO2}&\makecell{\vspace{-0.5em}\includegraphics[width=0.18\textwidth]{CathodeMaterials/Li2NiO2.png}\vspace{-0.5em}}
    &235&4,0&\cite{Fan1998,Dahn1991,Arai1997}\\
    \ce{LiMnO2}&\makecell{\includegraphics[width=0.18\textwidth]{CathodeMaterials/LiMnO2.png}\vspace{-0.8em}}
    &150&4,7&\cite{Croguennec1996,Vitins1997}\\
    \ce{LiFePO4}&\makecell{\includegraphics[width=0.18\textwidth]{CathodeMaterials/LiFePO4.png}\vspace{-0.8em}}
    &138&3,6&\cite{Padhi1997} \\
    \ce{LiNMC}111&\makecell{\includegraphics[width=0.18\textwidth]{CathodeMaterials/LiNMC111.png}\vspace{-0.5em}}
    &145&4,5&\cite{Park2024,Schmiegel2019}\\
    \ce{LiNMC}811&\makecell{\includegraphics[width=0.18\textwidth]{CathodeMaterials/LiNMC811.png}\vspace{-0.7em}}
    &230&4.3&\cite{Bhowmik2025,Quilty2024}\\
    \bottomrule
    \end{tabular}\\
    \noindent{\footnotesize{\textsuperscript{*} Gemessen gegenüber \ce{Li}/\ce{Li+}.}}
\end{table}%

In der Batterieforschung wurde bereits eine Vielzahl an Aktivmaterialien für den Einsatz in Kathoden untersucht (siehe Tabelle~\ref{tab:cathode_material}). Lithium-Cobaltdioxid (\ce{LiCoO2} bzw. \ce{LCO}) stellt dabei das am frühesten kommerziell genutzte Kathodenmaterial dar~\cite{Amatucci1996}. Während \ce{LCO} eine hohe theoretische Kapazität von 274\,$\si{\mA\hour\per\g}$ aufweist~\cite{Ren2023}, begrenzen komplexe Phasenumwandlungen oberhalb einer Spannung von 4,1\,V die nutzbare Kapazität. In der Praxis können daher lediglich etwa 140\,$\si{\mA\hour\per\g}$ reversibel gespeichert werden~\cite{Vetter2005, Liu2018}. Zudem motivieren die hohen Kosten sowie die ökologischen Bedenken hinsichtlich Cobalt die Entwicklung kobaltarmer oder -freier Alternativen~\cite{Ren2023}.

Inspiriert durch den Erfolg von \ce{LCO} rückten weitere Übergangsmetalloxide in den Fokus. \ce{LiNiO2} weist im Vergleich zu \ce{LCO} ein geringfügig niedrigeres Spannungsplateau, jedoch eine höhere reversible Kapazität auf~\cite{Ren2023}. Ein signifikanter Nachteil ist jedoch die strukturelle Instabilität: Die durch Sauerstoff-Fehlstellen und Nickel-Migration bedingten Symmetrieänderungen machen das Gitter anfällig für den sogenannten Jahn-Teller-Effekt\footnote{Ein quantenmechanischer Mechanismus, der die spontane Verzerrung hochsymmetrischer Molekülstrukturen beschreibt, um energetisch entartete Elektronenzustände aufzuheben~\cite{Jahn1937}.}, was die chemische Stabilität reduziert~\cite{Kalyani2005}. Dies führt zu einem Kollaps von Interkalationsstellen, der sich mit fortschreitender Zyklenzahl auf benachbarte Gitterplätze ausbreitet und maßgeblich zum Kapazitätsverlust beiträgt~\cite{Vaelikangas2020}.

Ähnliche Herausforderungen zeigen sich bei \ce{LiMnO2}, das aufgrund ausgeprägter manganbasierter Gitterverzerrungen ebenfalls eine geringe strukturelle Stabilität besitzt. Infolgedessen sind von der theoretisch hohen spezifischen Kapazität (285\,$\si{\mA\hour\per\g}$) effektiv nur etwa 148\,$\si{\mA\hour\per\g}$ nutzbar~\cite{Chen2014}. Das wissenschaftliche Interesse an \ce{LiMnO2} begründet sich jedoch vor allem in der hohen Betriebsspannung von bis zu 4,7\,V (gegenüber \ce{Li/Li+})~\cite{Liang2020, Yu2021}.

Eine vielversprechende Maßnahme, um den Nachteilen der jeweiligen einzelnen Übergangsmetalloxide zu begegnen, ist das Kombinieren der jeweiligen Übergangsmetalle~\cite{Biasi2017}. Durch die Kombination können in einem gewissen Rahmen die Vorteile der einzelnen Systeme, wie etwa die höhere Stabilität von \ce{LiCoO2}, die höhere nutzbare Kapazität von \ce{LiNiO2} und die höhere Spannung von \ce{LiMnO2}, ohne die vorherigen Nachteile, wie Kosten, geringes Spannungsplateau und Instabilität, ausgenutzt werden~\cite{Ren2023}.

Die daraus resultierenden vorteilhaften Eigenschaften gegenüber reinem \ce{LCO} sind wichtige Kriterien, warum im Automobilbereich primär Nickel-Mangan-Cobaltoxid (\ce{NMC} bzw.\ \ce{NMC}111, \ce{LiNi_{1/3}Mn_{1/3}Co_{1/3}O2}) verwendet wird~\cite{Burow2016}. Das von \textsc{Ohzuku} et al.~\cite{Ohzuku2001, Yabuuchi2003} entdeckte Material ist im Vergleich zu \ce{LCO} kostengünstiger, weist eine höhere Kapazität auf und ist auch bei höheren Temperaturen noch chemisch stabil~\cite{Kim2016,Zheng2012}. 
Die höhere strukturelle Stabilität der Gitterstruktur erlaubt auch das reversible Zyklieren innerhalb eines größeren Stöchiometrie-Fensters\footnote{Anteil an eingelagerten Ionen.}~\cite{Dolotko2014,Choi2005}. 
Für kommerzielle Kathodenmaterialien wird meist aus Kostengründen und zum Erreichen einer höheren Kapazität eine Reduzierung des Kobaltanteils angestrebt. Systeme mit höherem Nickel- und Mangananteil wie etwa NMC442 und NMC811 erreichen dabei reversible Kapazitäten von jeweils 211~$\si{\milli \ampere \hour \per g}$~\cite{Ma2014} und 230~$\si{\milli \ampere \hour \per g}$~\cite{Xue2021}. Ein kompletter Verzicht auf Kobalt ist jedoch wegen der stabilisierenden Wirkung derzeit nicht möglich~\cite{Ren2023}.

Ein alternativer Ansatz ist hierbei phosphatbasiertes Kathodenmaterial (\ce{LiFePO4} und \ce{LiMn_{1-x}Fe_xPO4}). Lithiumeisenphosphat (LFP) ist sehr stabil, besitzt einen höheren Diffusionskoeffizienten und das Merkmal, dass auch bei Schädigung kein Sauerstoff frei wird~\cite{Ling2021}. Zudem sind Aktivpartikel aus \ce{LiFePO4} mit einem Durchmesser von 1,5~$\si{\um}$ klein genug, um als Aktivbeschichtung für eine Kohlenstofffaser zu dienen~\cite{Yuecel2023}.

Allerdings hat die geringe elektrische Leitfähigkeit von $10^{\text{-}9}$ bis $10^{\text{-}11}$~$\si{\milli \siemens \per \cm}$) zur Folge, dass bei Kontaktverlust mit den leitenden Passivmaterialien die elektrischen Verluste der Batterie signifikant ansteigen~\cite{Padhi1997}. Außerdem ist die Volumenänderung während der De- bzw. Lithiierungsphase fast dreimal so hoch wie bei NMC (Tabelle~\ref{tab:volume_change}).

\subsection{Elektrolyte}
In konventionellen Batterien dient das Elektrolyt hauptsächlich als Transportmedium für die ionischen Ladungsträger~\cite{Gerlach2020}. Im Kontext von elektrischen Strukturspeichern sind auch mechanische Eigenschaften wie Steifigkeit und Festigkeit von Bedeutung~\cite{Greenhalgh2023}. Des Weiteren hat das Elektrolytmaterial einen entscheidenden Einfluss auf die maximale elektrische Spannung~\cite{Xu2016}, die Betriebstemperatur~\cite{Chen2022a}, die Toxizität~\cite{Beard2019}, die Entflammbarkeit und das Brandverhalten~\cite{Roth2012}. Um einen Beitrag zur Steigerung der Multifunktionalität von Strukturspeichern zu leisten, wird von \textsc{Greenhalgh} et al. für Strukturelektrolyten ein Zugmodul von mehr als 1~$\si{GPa}$ und eine ionische Leitfähigkeit größer als 1~$\si{\milli \siemens \per \cm}$ als minimaler Grenzwerte angegeben~\cite{Greenhalgh2023} (siehe Bild~\ref{fig:electrolyte_data}).
\begin{figure}[ht]
	%\raggedleft
		%\def\svgwidth{\columnwidth}
        \center
		%\input{Abbildungen/02_SoA/electrolyte_data/}
		\import{Abbildungen/02_SoA/electrolyte_data/}{electrolyte_data.pdf_tex}
	%\includegraphics[width=\textwidth, angle=0]{bicontinous_electrolyte.pdf}
		\caption{\label{fig:electrolyte_data}Multifunktionale Performanz von verschiedenen Strukturelektrolyten~\cite{Greenhalgh2023}.}
\end{figure}

Diese Limitierungen als auch die Unfähigkeit Schubspannungen aufzunehmen schließen die in konventionellen Batterien etablierten flüssigen Elektrolytsysteme für Strukturbatterien kategorisch aus~\cite{Shirshova2013, Greenhalgh2023}. Gleiches gilt auch für Gelelektrolyte, deren hohe Ionenleitfähigkeit mit sehr geringen mechanischen Eigenschaften einhergeht~\cite{Gayet2009, Li2018, Zhao2020a}. Für den Einsatz in Strukturbatterien kommen daher nur zweiphasige oder feste Elektrolytsysteme in Frage~\cite{Greenhalgh2023}.

%\subsubsection{Zweiphasige Electrolytesysteme}
\begin{figure}[!ht]
	%\raggedleft
		%\def\svgwidth{\columnwidth}
        \center
		%\input{Abbildungen/02_SoA/electrolyte_data/}
		\includegraphics[width=\textwidth, angle=0]{SElectrolyte_PhaseSeparation.pdf}
	%\includegraphics[width=\textwidth, angle=0]{bicontinous_electrolyte.pdf}
		\caption{\label{fig:SE_PhaseSepearation}Herstellung von zweiphasigen Strukturelektrolyten durch thermische oder UV-induzierte Polymerisierung eines Mischsystems~\cite{Schneider2019}.}
\end{figure}

Zweiphasige Elektrolyte basieren auf einer funktionalen Phasenseparierung: Eine feste Phase gewährleistet die mechanische Stabilität, während eine flüssige oder gelförmige Phase den Ionentransport übernimmt~\cite{Ichino1995} (siehe Bild~\ref{fig:SE_PhaseSepearation}). Durch die gezielte Anpassung der Phasenanteile und der Porenarchitektur lassen sich die resultierenden Eigenschaften zwischen maximaler Leitfähigkeit und optimalen mechanischen Kennwerten variieren (Bild~\ref{fig:bicontinous_electrolyte}). Während simulative Studien bereits ideale Architekturen zur Maximierung der Multifunktionalität identifiziert haben~\cite{Lee2019,Tu2020}, erweist sich die präzise Fertigung solcher Nanostrukturen bisher als technologische Hürde~\cite{Zekoll2018}. Daher fokussieren sich aktuelle Forschungsarbeiten primär auf die Charakterisierung und Optimierung ungeordneter Strukturen~\cite{Greenhalgh2023}.
Bei der Auslegung des Phasenanteils spielt die Perkolationstheorie eine entscheidende Rolle, da sie die Bildung zusammenhängender Netzwerke in ungeordneten Systemen beschreibt. Untersuchungen an zufällig angeordneten Sphären belegen, dass ein kritischer Volumenanteil der leitfähigen Phase von etwa 29\,\% überschritten werden muss, um einen durchgehenden Ionentransport zwischen den Elektroden sicherzustellen~\cite{Li2020b}. Durch Abweichungen von der sphärischen Geometrie lässt sich dieser Schwellenwert auf bis zu 23\,\% reduzieren~\cite{Li2020b}. Die Perkolationstheorie liefert somit wesentliche Erklärungsansätze für das sprunghafte Ansteigen der Ionenleitfähigkeit um mehrere Größenordnungen bei Erreichen bestimmter Phasenkonzentrationen~\cite{Melodia2023}.

Für die feste Phase haben sich auf duroplastischen Kunststoffen basierende Systeme hauptsächlich wegen der einfacheren Löslichkeit ihrer Monomere und der damit einhergenden Bildung kleinerer Poren durchgesetzt~\cite{Snyder2009,Li2018,Choi2018,Lee2019}. Jedoch treten derzeit vermehrt thermoplastische Systeme in den Vordergrund~\cite{Melodia2023}. Diese sind leichter in den Fertigungsprozess zu integrieren und bieten zusätzliche Sicherheit bei auftretenden Kurzschlüssen, da bei der entstehenden Wärmeentwicklung der Thermoplast schmilzt und die Poren verschließt, was einen weiteren Ladungsaustausch unterbindet~\cite{Roth2012}.
\begin{figure}[ht]
	%\raggedleft
		%\def\svgwidth{\columnwidth}
        \center
	\includegraphics[width=\textwidth, angle=0]{bicontinous_electrolyte.pdf}
		\caption{\label{fig:bicontinous_electrolyte}Veränderung der Zugsteifigkeit und der Ionenleitfähigkeit mit zunehmenden festem Phasenanteil bei zweiphasigen Elektrolyten (a-d).}
\end{figure}
Als Ausgangsmaterialien für die flüssige Phase kommen ionische Flüssigkeiten~\cite{Huang2022,Shirshova2013,Wendong2021,Shirshova2014,Dzienia2020}, Lithiumsalzlösungen in organischen Lösemitteln~\cite{Gienger2015,Sakakibara2017}, deren Kombination~\cite{Shirshova2014,Yu2016} und andere Systeme~\cite{Feng2017} in Betracht.

Im Gegensatz dazu basieren Feststoffelektrolyte zumeist auf einer Polymermatrix mit flexiblen Kettensegmenten, welche die Migration gelöster Lithiumsalze ermöglichen. Der wesentliche Vorteil dieses Ansatzes liegt im Verzicht auf flüchtige oder brennbare Komponenten sowie in den vorteilhaften mechanischen Eigenschaften, die primär durch das Polymernetzwerk bestimmt werden. Dennoch bleibt die Ionenleitfähigkeit bei Raumtemperatur signifikant hinter der von zweiphasigen Systemen zurück. 
Die Herstellung von Feststoffelektrolyten kann auf verschiedenen Wegen erfolgen, wobei die In-situ-Polymerisation in Anwesenheit von Lithiumsalzen einen zentralen Ansatz darstellt. \textsc{Snyder} et al.~\cite{Snyder2007, Snyder2009} untersuchten hierfür Vinylester-basierte Systeme und verdeutlichten den ausgeprägten Zielkonflikt zwischen mechanischer Festigkeit und elektrochemischer Performance. Sie erreichten Ionenleitfähigkeiten in einem Bereich von $1,6 \times 10^{-6}$ bis $1,7 \times 10^{-3}$\,$\si{\milli \siemens \per\cm}$. Dabei korrelierten die höchsten Leitfähigkeitswerte mit einem starken Abfall der mechanischen Stabilität: Während die steifsten Proben einen Speichermodul von etwa 552\,$\si{\MPa}$ aufwiesen, sank dieser bei optimierter Ionenbeweglichkeit auf bis zu 15\,$\si{\MPa}$ oder darunter ab~\cite{Snyder2007}.
Der zweite, verbreitete Ansatz basiert auf dem Mischen von Polymeren mit Lithiumsalzen. Für diese festen, teils mehrphasigen Elektrolytsysteme werden häufig Epoxidharze~\cite{Matsumoto2011, Munoz2021, Wang2020b} oder Polyethylenoxid~\cite{Moreno2011, Ji2010, Guo2021} eingesetzt, um eine hinreichende mechanische Integrität bei akzeptabler Ionenbeweglichkeit zu gewährleisten.

\subsection{Separator}

Separatoren befinden sich zwischen den beiden Elektroden und dienen hauptsächlich dem Verhindern eines elektrischen Kurzschlusses. Daraus folgt die Anforderung, dass Separatormaterialien für Elektronen nicht durchlässig sein dürfen, aber Transportmechanismen für Ladungsträger bereitstellen müssen~\cite{Kurzweil2015}. Um Kurzschlüsse auch bei Rissen infolge überhörter mechanischer Belastungen zu verhindern, muss ein Bruch der Separatorschicht verhindert werden~\cite{Asp2015}. Weitere wichtige Eigenschaften sind eine hohe chemische Beständigkeit und thermische Stabilität, Korrosionsbeständigkeit, geringe Dichte, geringe Dicke, gute Verfügbarkeit und geringe Materialkosten~\cite{Beard2019}. Im Kontext von Strukturbatterien und der häufigen Verwendung von festen Elektrolyten ist der Einsatz von Separatoren theoretisch nicht notwendig, da im Gegensatz zu flüssigen Elektrolyten das Auseinanderhalten der Elektroden auch unter Druck gewährleistet wird. Jedoch ist im Rahmen von Sicherheitsbedenken und aufgrund einer signifikanten Erschwerung des Herstellungsprozesses ohne diese der Einsatz von Separatoren immer noch ein wichtiges Element in der Konzeption von Strukturbatterien~\cite{Asp2015, Hubert2022}. Das am meisten verwendete Separatormaterial ist Glasfasergewebe (GF-Gewebe)~\cite{Zhou2022}. Jedoch existieren auch andere vielversprechende Separatormaterialien, wie polymerbasierte Separatoren, Keramiken und Cellulose~\cite{Simon2008, Greenhalgh2023, Chaudhary2024a} (Tabelle~\ref{tab:separator_comp}).

\begin{table}[h!]
    \caption{Properties of different types of separators}
    \label{tab_separator_comp}
    %\begin{adjustwidth}{-\extralength}{0cm}
    \newcolumntype{C}[1]{>{\hsize=#1\hsize\centering\arraybackslash}X}%
    \begin{tabularx}{\textwidth}{
    %C{0.6}
    C{1} 
    C{1.8} 
    C{0.8} 
    C{0.8} 
    C{0.8} 
    C{0.6}
    }
        \toprule
        \textbf{Separatortyp}
        &\textbf{Separatormaterial} 
        &\textbf{Ionische Leit- fähigkeit\textsuperscript{*} (mS/cm)} 
        &\textbf{Zug- steifigkeit\textsuperscript{*} (GPa)}
        & \textbf{Festigkeit\textsuperscript{*} (MPa)}
        &\textbf{Ref.} \\
        \midrule
        %\legendsep{c0}&
        Glassfaser&Glassfaser&1.13&21
        &325
        &\cite{Deka2017}\\
        %\midrule
        \addlinespace
        %\legendsep{c10}&
        Polymer&RF/PLA&110&0.3271
        &15.2
        &\cite{Vargun2020}\\
        %\midrule
        \addlinespace
        %Gel polymer electrolyte&$\mathrm{PVA/KOH/K_3[Fe(CN)_6]}$&45.56&n.a.&n.a.&\cite{maHighPerformanceSolidstate2014}\\
        %%\midrule
        %\legendsep{c4}&
        Feststoff- elektrolyt&$\mathrm{PEGDGE/TETA/EMIBF_4}$&0.2&26
        &350
        &\cite{Hubert2022, Choi2022}\\
        %\midrule
        \addlinespace
        %\multirowcell{2}{\legendsep{c6}}&
        \multirowcell{2}{Keramik}
            &$\mathrm{PVDF/PPG/LiCl/CaTiO_3}$&n.a.&1.2
            &65
            &\cite{Alvarez‐Sanchez2019}\\
            &$\mathrm{PVB/Al_2O_3NW}$&13.5&n.a.
            &30
            &\cite{Liu2020a}\\
        %\midrule
        \addlinespace
        %Diode-like polymer electrolyte&PVP/PEI/SWCNT&n.a.&n.a.&n.a.&\cite{chowdhurySupercapacitorsElectricalGates2019}\\
        %%\midrule
        %Ceramic&NPs/PTFE/SiC&n.a.&n.a.&1.3&\cite{qinCeramicBasedSeparatorHighTemperature2018,zhaoInorganicCeramicFiber2017}\\
        %%\midrule
        %Tree-leave&Quercus rubra&n.a.&n.a.&n.a.&\cite{chenTrashTreasureFallen2022,wangMechanicalCharacteristicsTypical2010}\\
        %%\midrule
        %Eggshell membrane&Eggshell membrane&3.8&n.a.&6.59&\cite{yuUsingEggshellMembrane2012}\\
        %%\midrule
        %\legendsep{c8}&
        Cellulose&MCC/AMIM-Cl&298.6&5.43
        &71.71
        &\cite{Ahankari2022, Xu2020}\\
        %%\midrule
        %Graphene oxide&Graphene oxide paper&n.a.&n.a.&n.a.&\cite{shulgaSupercapacitorsGrapheneOxide2015,comptonTuningMechanicalProperties2012}\\
        %%\midrule
        %Metal-organic framework&Metal-organic framework&n.a.&n.a.&n.a.&\cite{mengMetalOrganicFrameworks2015,bundschuhMechanicalPropertiesMetalorganic2012}\\
        \bottomrule
    \end{tabularx}
    %\end{adjustwidth}
    \noindent{\footnotesize{\textsuperscript{*} The abbreviation not available (n.a.) is used.}}
\end{table}

Aufgrund ihrer ausgeprägten mechanischen Belastbarkeit, hohen thermischen Stabilität und elektrochemischen Inertheit finden glasfaserbasierte (GF) Separatoren breite Anwendung in der Batterieforschung~\cite{Luo2015, Asp2019, Asp2021, Liu2022}. Als textile Flächengebilde zeichnen sie sich zudem durch vergleichsweise geringe Materialkosten aus. Ein technologischer Nachteil ist jedoch ihre signifikante Dicke im Vergleich zu polymeren Membranen. Um einen hinreichenden Ionentransport zu gewährleisten, weisen diese Separatoren oft ein hohes Porenvolumen bzw. eine grobe Textur auf (siehe Bild~\ref{fig:separator_transportation})~\cite{Danzi2021}. Diese offene Struktur erhöht jedoch die Permeabilität für lokale Inhomogenitäten, was mit einem gesteigerten Kurzschlussrisiko einhergeht~\cite{Zhou2022}.

Alternativ wurden im Kontext faserbasierter Separatoren Aramidfasern (AF) eingehend untersucht~\cite{Jin2023}. Diese Hochleistungsfasern bieten aufgrund ihrer spezifischen Oberflächenbeschaffenheit und chemischen Struktur eine effektive Barriere gegen das Dendritenwachstum. Dies resultiert bei Lithium-Ionen-Batterien häufig in einer überlegenen Zyklenstabilität gegenüber klassischen Glasfaser-Gewebe- oder Vliesstrukturen~\cite{Tung2015, Wang2021a}.

\begin{figure}[ht]
	%\raggedleft
		%\def\svgwidth{\columnwidth}
        \center
	\includegraphics[width=\textwidth, angle=0]{separator_transportation.pdf}
		\caption{\label{fig:separator_transportation}Visualisierung verschiedener Separatormaterialien und der jeweiligen Ionentransportwege (gestrichelte Linie) für: (a) Gewebe, (b) Polymer-, (c) Keramikseparator und (d) Cellulose~\cite{Zschiebsch2024}.}
\end{figure}

Materialien für Polymerseparatoren sind oft identisch zu polymerbasierten Festelektrolyten. Bei zweiphasigen Elektrolyten ist besonders die gute Durchdringung des Elektrolyten von Vorteil, die eine um den Faktor von 100 höhere Ionenleitfähigkeit als Glasfaserseparatoren ermöglicht~\cite{Wang2021a}. Jedoch führt auch hier die höhere Porosität zu einer Reduktion der mechanischen Eigenschaften~\cite{Ahankari2022}. Dieser Zielkonflikt wird auch in einer Studie von \textsc{Karabelli} et al.~\cite{Karabelli2011} deutlich, bei der die entwickelte Polymermembran eine Zugsteifigkeit von 741~$\si{\MPa}$ erreichte. Im Vergleich reduzierte sich dieser Wert für das gleiche Ausgangsmaterial mit einer Porosität von 75~\% auf 53~$\si{\MPa}$. Im Vergleich zu Glasfaser sind die mechanischen Eigenschaften jedoch signifikant geringer, weshalb Polymerseparatoren außerhalb von flexiblen Anwendungen aktuell nur eine geringe Rolle spielen~\cite{Zschiebsch2024}.

Keramische Separatormaterialien zeichnen sich besonders durch ihre hohe thermische Stabilität aus und werden daher auch bereits in Batterien mit hohen Betriebstemperaturen eingesetzt~\cite{Qin2017,Cheong2012}. Außerdem bieten diese Materialien neben ausgezeichneten thermischen auch hervorragende physikalische sowie elektrochemische Eigenschaften. Jedoch besitzen derzeit verfügbare keramische Separatoren im Vergleich zu Glasfasern nur eine unzureichende Zugfestigkeit~\cite{Qin2017}. Ein diesbezüglich vielversprechender Entwicklungsansatz ist dabei die Integration von keramischen Fasern, die sowohl als Separatoren als auch zur Verstärkung der Elektrolyt-Matrix dienen. \textsc{Zhao} et al.~\cite{Zhao2017} entwickelten einen keramischen Faserseparator für Lithium-Ionen-Batterien, indem sie keramische Fasergewebe in die Matrix integrierten. Durch die Konsolidierung der Matrixmaterialien mit kontinuierlichen keramischen Fasern konnte die Festigkeit des Verbundmaterials verbessert werden. SiC-basierte Fasern weisen beispielsweise Zugfestigkeiten von bis zu 6~$\si{\GPa}$ und Elastizitätsmodulwerte von bis zu 420~$\si{\GPa}$ auf~\cite{Seydibeyoglu2017}. Zudem entwickelten \textsc{Yamamoto} et al.~\cite{Yamamoto2009} eine Methode, um ausgerichtete Kohlenstoffnanoröhren auf keramischen Fasern wachsen zu lassen, was die Bindung zwischen Faser und Matrix weiter verstärken und das Porenvolumen des keramischen Faserseparators erhöhen könnte, um die Ionenleitfähigkeit zu verbessern.

Parallel zum steigenden Interesse an nachhaltigen Energiespeichern und biobasierten Werkstoffsystemen rückte Cellulose als nachwachsender Rohstoff verstärkt in den Fokus der Separatorforschung~\cite{Liang2018, Teng2020}. Als natürliches Polymer zeichnet sich Cellulose durch eine exzellente Benetzbarkeit und damit einhergehende hohe Ionenleitfähigkeit aus. Während die mechanische Belastbarkeit und Zugsteifigkeit konventioneller Cellulose-Flächengebilde noch deutlich unter den Werten mineralischer Glasfasern liegen~\cite{Xu2020} (siehe Tabelle~\ref{tab:separator_comp}), bietet die Modifikation der morphologischen Struktur erhebliches Steigerungspotenzial.
Insbesondere durch den Einsatz von Nanocellulose lassen sich hochgeordnete Netzwerkarchitekturen realisieren. Aufgrund des hohen Kristallinitätsgrades der elementaren Fibrillen werden in diesen Strukturen auf molekularer Ebene Zugsteifigkeiten von bis zu 130\,$\si{\GPa}$ erreicht~\cite{Dufresne2013, Zhang2019}, was Cellulose zu einer vielversprechenden Basis für multifunktionale, biogene Separatoren macht.

\subsection{Pouchfolie}
Herkömmliche Pouchzellen sind mit einer kunststoffbeschichteten Aluminiumhülle vor Umwelteinflüssen geschützt. Insbesondere verhindert diese Hülle, dass Feuchtigkeit in die Batterie eindringt und giftige oder brennbare Stoffe aus der Batterie entweichen können~\cite{Beard2019}. Außerdem ermöglichen die hohen mechanischen Eigenschaften und hervorragende Wärmeleitfähigkeit der Aluminiumfolie eine geringe Gesamtmasse und eine effizientere Temperaturregulierung der Zellen~\cite{Boaretto2021}. Eine zunehmend wichtiger werdende Aufgabe, die allerdings noch nicht hinreichend erfüllt wird, ist das Aufbringen eines äußeren Zelldrucks~\cite{Sakamoto2019}.

In mehreren Studien konnte gezeigt werden, dass durch einen hohen externen Druck die Kontaktierung zwischen Elektrode und Elektrolyt verbessert wird, was einen besseren Ionen- und Elektronentransport bewirkt. Außerdem können unerwünschte Nebenreaktionen, wie etwa Gasbildung und Dendritwachstum,  unterdrückt werden, was den Lithiumverlust beim Laden und Entladen reduziert. Somit kann dem Kapazitätsverlust entgegenwirkt und das Batterieleben verlängert werden~\cite{Mussa2018,Mueller2019,Sakamoto2019}. Besonders Batterien mit Feststoffelektrolyten benötigen einen deutlich höheren Druck, um den Kontakt zwischen Elektrode und Elektrolyt zu gewährleisten \cite{Boaretto2021}. Hierbei wird bei der Herstellung mittels Verpressen der Elektroden ein gewisser Druck realisiert, allerdings können größere Drücke mit diesem Vorgehen nicht appliziert oder über längere Zeit aufrechterhalten werden \cite{Garayt2023}. Daher wird oft versucht, durch eine externe Einspannung auf Systemebene diesen Druck aufzubringen. Jedoch entsteht durch die innere Reibung der Batterien kein gleichmäßiger Druckverlauf, was zu stärkeren Belastungen der äußeren Zellen führt. Des Weiteren werden durch die mit steigenden Drücken verbundenen höheren mechanischen Belastungen  dickere Schutzfolien benötigt, was zu einer niedrigeren Gesamtenergiedichte führt~\cite{Ye2024,Asp2021}.

%Einzig die Knopfzellen, die durch eine integrierte Feder einen definierten Druck auf eine im Verhältnis zur Pouchzelle deutlich kleinere Fläche ausüben, sind die einzige bekannte Lösung zu diesem Problem. Hinzukommt, dass auch hier der Massenanteil von Gehäuse zu Zelle deutlich höher ist als bei Pouchzellen.

Für Strukturbatterien sind bisher keine Alternativen zur herkömmlichen Aluminiumpouchfolie untersucht worden \cite{Ye2024}. Jedoch gibt es viele Forschungsgruppen, die ihre Strukturbatterien mit Pouchfolie zusätzlich in einen kohlenstofffaserverstärkten Kunststoff einbetten \cite{Pattarakunnan2020,Asp2021}. Eine Realisierung der Druckvorspannung für diese Strukturbatterien wurde bisher noch nicht untersucht.

\section{Systemdesign und modellbasierte Auslegung von Strukturbatterien}

\subsection{Aktueller Stand der Materialkombinationen und Designkonzepte}
In der aktuellen Forschung hat sich ein dominantes Materialsystem etabliert, das als Referenz für die multifunktionale Charakterisierung dient. Dieses umfasst modifizierte Kohlenstofffasern als Anode, Lithium-Eisenphosphat (\ce{LiFePO4} bzw. LFP) als Kathoden-Aktivmaterial sowie zweiphasige Strukturelektrolyte~\cite{Asp2021, Asp2024, Chaudhary2024}. Die Wahl von LFP wird dabei primär durch dessen elektrochemische Stabilität und die bessere Vergleichbarkeit bei der Optimierung der Kohlenstofffasern begründet; ein Wechsel auf hochenergetische Materialien wie NMC würde zusätzliche Komplexität einführen, welche die isolierte Betrachtung der Fasereigenschaften erschweren würde~\cite{Asp2024}.

Hinsichtlich des Zell-Layouts hat eine Abkehr von der ursprünglichen Vision der \textit{Single-Fiber-Cell} (konzentrisches Design) stattgefunden~\cite{Ekstedt2010, Leijonmarck2013}. Aufgrund der hohen Fertigungskomplexität und des Ausfallrisikos konzentrieren sich aktuelle Arbeiten fast ausschließlich auf laminare Designs~\cite{Johannisson2018, Xu2022}. Dieser Aufbau orientiert sich am bewährten Pouchzellen-Design, ermöglicht jedoch durch die Integration gewebter Kohlenstofffaserstrukturen eine signifikante Steigerung der mechanischen Kennwerte in alle Raumrichtungen~\cite{Xu2022}. Die Gehäusekomponente (Pouchfolie) wird dabei in theoretischen Modellen oft idealisiert vernachlässigt oder massentechnisch herausgerechnet, um die intrinsischen Eigenschaften des strukturellen Verbunds hervorzuheben~\cite{Danzi2021, Ye2024}.

\subsection{\label{sec:soa_simulation}Multiphysikalische Modellierung und Simulationsframeworks}
Die Auslegung neuer Strukturbatterien stützt sich bisher stark auf empirische Daten, da eine universelle Vergleichbarkeit aufgrund mangelnder Standardisierung erschwert wird. Um diesen Prozess zu systematisieren, wurden in den letzten Jahren spezialisierte Simulationsframeworks entwickelt, maßgeblich geprägt durch die Arbeiten von \textsc{Carlstedt}~\cite{Carlstedt2022, Carlstedt2022a} und \textsc{Johansen}~\cite{Larsson2023, Johansen2024}. 

Das Verständnis dieser Modelle ist entscheidend, da sie die fundamentale Kopplung zwischen Elektrochemie und Mechanik auf der Meso- bzw. Partikelebene beschreiben. Sie kombinieren die klassische Batteriemodellierung nach \textsc{Newman} und \textsc{Doyle}~\cite{Doyle1993, Newman2021} mit kontinuumsmechanischen Ansätzen. Der Kern dieser Modelle liegt in der Abbildung lokaler Effekte:
\begin{itemize}
    \item \textbf{Mechanische Belastung durch Lithiierung:} Die Volumenexpansion der Kohlenstofffasern während der Ioneneinlagerung führt zu internen Spannungen im Verbund~\cite{Carlstedt2020b, Duan2021}.
    \item \textbf{Nicht-lineare Interaktionen:} Die Modelle berücksichtigen, dass die Ionenbeweglichkeit wiederum vom mechanischen Druckzustand des Elektrolyten abhängt.
\end{itemize}

Trotz des hohen Detailgrades bleibt der Rechenaufwand die größte Hürde für die praktische Anwendung. Während die Simulation einzelner Fasern bereits präzise physikalische Einblicke erlaubt, erfordert die Skalierung auf reale Systemgrößen (z.\,B. 25.000 Kurzfasern~\cite{Goudarzi2022}) immense Kapazitäten. Aktuelle Erweiterungen versuchen zudem, chemische Grenzflächeneffekte wie die Bildung der Solid Electrolyte Interphase (SEI) zu integrieren~\cite{Yuecel2024}. Da atomistische Simulationen (z.\,B. Dichtefunktionaltheorie) zur Vorhersage dieser Effekte jedoch rechnerisch kaum skalierbar sind~\cite{Franco2019, Rollin2023}, bleibt die Forschung derzeit auf einen hybriden Ansatz angewiesen: Physikalische Modelle definieren den Rahmen, während spezifische Materialparameter oft erst nachträglich durch Experimente und Regressionsverfahren bestimmt werden~\cite{Carlstedt2022, Carlstedt2023}.

\section{Herausforderungen in der Entwicklung und Skalierung}

Trotz signifikanter Fortschritte wird die Weiterentwicklung von Strukturbatterien durch fundamentale technologische und methodische Hürden gebremst. Während die Zugsteifigkeit seit 2009 von 0,7\,$\si{\GPa}$ auf beachtliche 26\,$\si{\GPa}$ gesteigert werden konnte, stagnierte die Energiedichte im gleichen Zeitraum auf einem vergleichsweise niedrigen Niveau von ca. 35 bis 41\,$\si{\watt \hour\per\kg}$~\cite{Liu2009, Siraj2023}. Dieser geringe Zuwachs resultiert unter anderem aus dem Zielkonflikt zwischen Kinetik und Mechanik: Die Steigerung der Diffusionsraten durch das Einbringen von Porosität in die Kohlenstofffaser führt zwangsläufig zu einem signifikanten Verlust der strukturellen Integrität. Zudem erschweren geometrische Restriktionen den Einsatz hochenergetischer Kathodenmaterialien wie NMC, da deren Partikelgröße in der Größenordnung des Faserdurchmessers liegt und somit eine effektive Integration in das Laminatdesign bisher verhindert~\cite{Asp2014, Asp2024}.

Ein zentrales Hemmnis für den wissenschaftlichen Durchbruch ist die methodische Diskrepanz zwischen Simulation und Experiment. Wie in Abschnitt~\ref{sec:soa_simulation} dargelegt, fokussieren sich aktuelle Simulations-Frameworks auf die Meso-Ebene, was zu einem hohen Bedarf an experimentell schwer zu bestimmenden Materialparametern führt~\cite{Carlstedt2022b}. Infolgedessen werden prädiktive, rein simulative Ansätze zur Vorhersage neuer Materialkombinationen bisher kaum genutzt. Stattdessen dienen Simulationen meist der retrospektiven Analyse, bei der fehlende Daten durch mathematische Regressionsverfahren approximiert werden~\cite{Carlstedt2022a, Carlstedt2023}. Die Forschung verbleibt somit in einem zyklischen, primär experimentell-empirischen Prozess, der durch den Mangel an standardisierten Prüfverfahren zusätzlich fragmentiert wird~\cite{Greenhalgh2024a}. Eine echte digitale Beschleunigung des Entwicklungsprozesses durch datengetriebene oder prädiktive Ansätze stellt daher eine bisher ungelöste Kernherausforderung dar.

%Diese methodischen Defizite wirken sich unmittelbar auf die Realisierbarkeit potenzieller Anwendungen aus. In der Luftfahrt verhindern strenge Zertifizierungsrichtlinien den Einsatz, solange keine deutlichen Leistungsvorteile und validierten Sicherheitsmodelle vorliegen~\cite{Scholz2018}. Im Automobilbau stehen der hohen funktionalen Integration zudem logistische Hürden gegenüber: Da die Batterie ein tragendes Teil der Fahrzeugstruktur ist, stellt die Reparatur oder der Austausch im Schadensfall ein bisher ungelöstes Problem dar~\cite{Kalnaus2021, Martins2021}. Die Überwindung dieser Barrieren erfordert daher nicht nur neue Materialien, sondern primär eine Evolution der Auslegungs- und Validierungsmethodik.
