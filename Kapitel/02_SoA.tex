\chapter{Faserbasierte Strukturbatterien für Leichtbauanwendungen}

Im folgenden Kapitel wird ein grundlegendes Verständnis für die Funktionsweise von Strukturbatterien vermittelt. Außerdem werden die Besonderheiten im Vergleich zu konventionellen Batterien und faserverstärkten Verbundwerkstoffen erläutert. Dazu werden die wichtigsten Eigenschaften und deren Ermittlungsverfahren dargestellt sowie die Rolle der einzelnen Komponenten im Zusammenhang mit der Materialauswahl näher erklärt. Im Folgenden werden aktuelle Entwicklungsansätze diskutiert. Abschließend werden die ungelösten Herausforderungen unter Nutzung der aktuellen Methoden eingehend analysiert.

\section{Funktionsweise und Bauweisen} 
%Strukturbatterien sind Batterien, die mechanisch belastbar sind und damit auch zur strukturellen Integrität beitragen können.

Strukturbatterien ermöglichen die Speicherung von elektrischer Energie durch reversible elektrochemische Prozesse~\cite{Newman2021}. Dabei werden unterschiedliche Bindungsenergien bei der Einlagerung von ionischen Ladungsträgern ausgenutzt~\cite{Ashcroft2021}. Auf diesem Prinzip basierende Systeme werden in der Literatur auch "`Shuttle-Clock"'-, "`Rocking-Chair"'- oder "`Swing"'-Zellen bezeichnet~\cite{Ohzuku1993,Tarascon1993,Bittihn1993}. Der Einlagerungsprozess selbst wird meist als "`Interkalation"' bezeichnet~\cite{Eichinger1976}. Die Materialien in der Interkaltionsprozess stattfindet, nehmen dadurch eine aktive Rolle in den Prozess ein, wodurch sich der Name "`Aktivmaterial"' ableitet ( Bild~\ref{fig:battery_function}). Häufig ist das Aktivmaterial für die Anode Graphit und für die Kathode Metalloxid (\ce{MO_2}).
\begin{figure}[ht]
	%\raggedleft
		%\def\svgwidth{\columnwidth}
        \center
	\includegraphics[width=\textwidth, angle=0]{battery_function.pdf}
		\caption{\label{fig:battery_function}Darstellung des skalenübergreifenden Ladungsaustausch einer faserbasierten Strukturbatterie mit Graphitstrukturen in den Kohlenstofffasern als Interkalationsort für die Anode und Phasenumwandlungen im Metalloxid als Bindungssystem auf Kathodenseite}
\end{figure}
Beim Entladen der Batteriezelle wandern die Ladungsträger von der Anode zur Kathode. An beiden Elektroden kommt es zum Ladungsaustausch, der mithilfe der nacfolgenden Redox-Gleichungen beschrieben werden kann~\cite{Goodenough2013}. 
\begin{align}
	\ce{C_6 + x Li^+ + x e^- &<=> Li_xC_6},\\ 
	\ce{LiMO_2 - x Li^+ - x e^- &<=> Li_{1-x}MO_2} 
\end{align} 
Der Einlagerungsprozess erlaubt im Vergleich zur alleinigen elektrostatischen Speicherung, wie etwa bei Kondensatoren, eine signifikant höhere Energiedichten, wodurch größere Energiemengen bei gleicher Masse oder Volumen gespeichert werden können~\cite{Newman2021}.

Entscheidend für die Funktionsfähigkeit ist hierbei, dass der Transport der Ionen durch den Elektrolyten und den Separator erfolgt. Die Elektronen können jedoch nur entlang des Stromkollektors geleitet werden und werden daher außerhalb der Zelle entlang geleitet~\cite{Plett2015} (Bild~\ref{fig:battery_function}). Die Bewegung der Elektronen ist dabei im Vergleich zum Ionentransport deutlich schneller, weshalb die Ent- und Beladungsgeschwindigkeit einzig von der Ionenmobilität begrenzt wird~\cite{Plett2024}.

Im Falle großer Ströme, wie etwa bei einem Kurzschluss, finden viele Ladungsumwandlungsreaktionen gleichzeitig statt~\cite{Berg2022}. Die dabei entstehende Wärme kann zum elektrochemischen Versagen oder Brennen der Zelle führen~\cite{Wang2019a}. Mögliche Ursachen für Kurzschlüsse sind Herstellungsfehler, Dendritwachstum oder mechanische Belastungen, bei denen die Elektroden durch das Versagen des Separators in direkten Kontakt kommen~\cite{Kaliaperumal2021,Berg2022}. Typische Beispiele hierfür sind Penetration durch Nägel oder Biegung aufgrund unsachgemäßer Lagerung~\cite{Beard2019}.

Darüber hinaus führt die mit den mechanischen Belastungen einhergehende Rissbildung zu einer Erhöhung des inneren elektrischen Widerstandes, was einen Verlust an Speichereffizienz zur Folge hat und daher vermieden werden sollte~\cite{Plett2024}. Die Maßnahmen, um eine Verbesserung der strukturellen Eigenschaften zu erreichen, lassen sich dabei in zwei Kategorien einteilen~\cite{Jin2023} (Bild~\ref{fig:sb_type_scales}):
\begin{enumerate}[label=(\roman*)]
	\item Einbettung bestehender Batterien in tragende Strukturen und
	\item Multifunktionale Materialien mit Fähigkeiten zur Energiespeicherung und Lastenaufnahme.
\end{enumerate}
\begin{figure}[!h]
	%\raggedleft
		%\def\svgwidth{\columnwidth}
        \center
		\includegraphics[width=\textwidth, angle=0]{sb_type_scales.pdf}
		\caption{\label{fig:sb_type_scales}Bauweisen von verschiedenen strukturtragenden Battereisystemen: a) Bessere strukturelle Eigenschaften durch Neugestaltung des Batteriepaketes~\cite{Tesla2020}. b) Verstärkung durch Einlaminierung bestehender Pouchbagsysteme~\cite{Pattarakunnan2020}. c) Strukturbatterie als Ergebnis eines geschichteten Aufbaus verschiedener strukturverstärkenden Materialien~\cite{Asp2015}. d) Faserbatteriedesign durch mehrere dünne Materialbeschichtungen~\cite{Thakur2020}}
\end{figure}

Kategorie (i) beinhaltet dabei alle Maßnahmen auf Zellebene, die meist durch Integration konventioneller Batterien zwischen verstärkenden Außenlagen aus Metall oder Kohlefasern~\cite{Galos2020,Attar2020} umgesetzt werden, um eine versagensrelevante Belastung der Batterien zu verhindern~\cite{Beard2019}. Dadurch liegt weiterhin eine Funktionstrennung vor, welche das Masseeinsparungspotenzial stark limitiert~\cite{Jin2023}. Allerdings ist dieser Ansatz sehr kompatibel mit bestehenden Produktionsprozessen und findet daher bereits Einsatz in der Massenproduktion von Batterien für elektrische Autos~\cite{Tesla2020}.

Bei Kategorie (ii) werden auf der Materialebene multifunktionale Materialien entwickelt, die sowohl Energie speichern als auch der mechanischen Beanspruchung aufnehmen~\cite{Asp2014}. Dadurch besteht ein größeres Potenzial für Masseeinsparungen, jedoch ist die Auslegung und Entwicklung solcher Materialien durch die zahlreichen Wechselwirkungen deutlich komplexer~\cite{Jin2023,Carlstedt2018}.
Zur Steigerung der Multifunktionalität neuer Strukturbatterien ist daher neben einem umfangreichen Verständnis der wichtigsten Eigenschaften auch ein Überblick über mögliche Materialien aus bestehenden Untersuchungen zu konventionellen Batterien sowie dem Leichtbaubereich notwendig~\cite{Asp2019}.

\section{\label{sec:Ermittlungsverfahren}Charakterisierungsverfahren}

Das Interesse an Strukturbatterien kommt aus zwei verschiedenen Fachbereichen: (1) der konventionellen Batterieforschung und (2) dem funktionsintegrativen Leichtbau. Beide Bereiche betrachten diese Technologie im Kontext unterschiedlicher Referenzsysteme. Daher erfolgt die Bewertung der Strukturbatterie sowohl mit Bezug auf aktuelle Hochleistungsbatterien, im Kontext auf elektrochemische Leistungsfähigkeit, als auch mechanischer Hochleistungsmaterialien, wie etwa Karbonfasern, im Kontext von Steifigkeit- und Festigkeitsbetrachtungen~\cite{O’Brien2011,Snyder2015}. Daraus leitet sich die Bedingung ab, dass jede Strukturbatterie so zu charakterisieren ist, dass ein Vergleich der Charakteristiken beider Fachdiziplinen (1) und (2) möglich ist. Darüber hinaus stellen perspektivische Anwendungen, wie der Automobilbereich~\cite{Martins2021,Pistoia2010}, die Luftfahrt~\cite{Ishfaq2022} und wachsende ökologische Bedenken~\cite{EU2023}, ein breites Spektrum an Anforderungen, dass den Charakterisierungsaufwand weiter erhöhen. 
Der große Anteil an nicht-standardisierten Ermittlungsverfahren stellt dabei zusätzlich eine ernst zu nehmende Herausforderung dar~\cite{Greenhalgh2023,Zschiebsch2024}. In diesem Kapitel soll daher ein Überblick über die wichtigsten Größen und in der Forschung verwendeten Bestimmungsmethoden gegeben werden.
Brennbarkeit~\cite{Shirshova2021,Shirshova2024}, sowie \ce{CO2}-Produktion und Recyclingfähigkeit~\cite{Heyadati2024} sind zwar von wachsender Bedeutung, ihre Bestimmungsmaßnahmen sind jedoch sehr wenig normiert und spielen in der aktuell frühen Strukturbatterieforschung eine vergleichsweise untergeordnete Rolle und werden im Rhamen dieser Arbeit nicht ausführlicher diskutiert.

\subsection{Elektrochemische Charakterisierungsverfahren}

%\subsection{Interkalation}
\subsubsection*{Ladungszustand}
Ein wesentliches Merkmal von Batterien ist die Interkalation von Ionen in das Aktivmaterial. Je nach Aktivmaterial steht dabei eine bestimmte Menge an Einlagerungsplätzen zur Verfügung. Der Ladungszustand (SOC\footnote{\textit{engl.} state of charge}) beschreibt den Anteil von aufgenommen zur maximal möglichen Menge an aufnehmbaren Ladungsträgern~\cite{Plett2015}.

Daraus folgt, dass bei einem SOC gleich 1 bzw. 100~\% keine weiteren Ladungsträger mehr aufgenommenwerden können, und wenn alle Ionen das Aktivmaterial verlassen, der SOC gegen 0~\% fällt. Alternativ wird der Beladungszustand auch durch die stöchiometrische Größe $x$ angegeben, die besonders häufig im Kontext der Beschreibung von Zwischenzuständen (z.B. \ce{Li_xMn2O4} oder \ce{Li_{1-x}C}) benutzt wird~\cite{Newman2021}. Der SOC hat maßgeblichen Einfluss auf das eletrochemische Potenzial (Bild~\ref{fig:battery_voltage}), die Volumenausdehnung (Tabelle~\ref{tab:volume_change}) und die Diffusionsgeschwindigkeiten~\cite{Plett2024}. Der SOC kann jedoch nicht direkt gemessen werden und wird stattdessen über die Elektrodenspannung ermittelt~\cite{Newman2021}.

\begin{table}[ht]
    \centering
    \caption{Volumenänderungen bo der Lithiierung für gängige Aktivmaterialien.}
    \begin{tabular}[t]{lccc}
        \toprule
        Material& Vöumenänderung $\Delta$V/V\textsubscript{0}&Verlauf&Quelle\\
        \midrule
        Graphit & +10\% - +13\% & nicht-linear & \cite{Qi2010,Woodford2012}\\
        NMC111 &-2,4\%&nicht-linear& \cite{Yabuuchi2005}\\
        NMC422 &+2,4\%&nicht-linear& \cite{Ma2007}\\
        LCO &-1,9\% & linear & \cite{Reimers1992}\\
        NCA &-1,6\% & nicht-linear& \cite{Itou2005}\\
        LFP &+6,5\% & linear & \cite{Padhi1997}\\
        LMO &+6,6\% & linear & \cite{Christensen2006}\\
        \bottomrule
    \end{tabular}
\end{table}

\subsubsection*{Elektrische Spannung}

Die elektrische Spannung ist neben dem Strom, der direkt mit der Bewegung der Ladungsträger verbunden ist, eine der wichtigsten Größen zur Beschreibung des Zustandes in einer Batteriezelle~\cite{Beard2019}. 
\begin{figure}[ht]
    \center
		\includegraphics[width=\textwidth, angle=0]{uocp.pdf}
		\caption{\label{fig:battery_voltage}Spannung über den stöchiometrischen Anteil der beladung einer Lithiumionenbatterie, sowie die Anteile der negativen (Graphit) und positiven (Manganoxid) Elektrode (angelehnt an~\cite{Newman2021}).}
\end{figure}
Im Kontext der Elektrochemie ist die elektrische Spannung ein Maß für das elektrochemische Potenzial einer Elektrode. Ein Potenzial kann jedoch nur durch den Vergleich zu einem Referenzpotenzial, das dann oft als Nullpotenzial bezeichnet wird, gemessen werden. In der Batterieforschung wird hierbei häufig die Referenzspannung gegen reines Lithium gemessen~\cite{Newman2021}. Entscheidend ist hierbei, dass diese Referenzspannung bei einer Elektrode nicht konstant ist, da sich durch das Ein- und Auslagern das chemische Potenzial verändert. Der resultierende Spannungsverlauf ist abhängig von der chemischen Struktur, wodurch Aufschlüsse auf das Einlagerungsverhalten und damit verbundene Phasenumwandlungen geben~\cite{Plett2015}, siehe Bild~\ref{fig:battery_voltage}.

Damit Batterien in der Anwendung sich nicht selbst entladen ist es wichtig, dass die Zellspannung\footnote{häufig auch als nominale Spannung bezeichnet} mit zunehmender Entladung stetig sinkt~\cite{Newman2021}~(Bild~\ref{fig:battery_voltage}). Dabei ist zu beachten, dass an der Elektrode das Potenzial immer dann sinkt, wenn sich diese auflädt\footnote{Anstieg des SOC}~\cite{Newman2021}. Da in einer Batterie durch den Ladungsaustausch immer eine Elektrode aufgeladen wird, während die andere entladen wird, kann der stetige Spannungsabfall während der Entladung sichergestellt werden, wenn die Referenzspannung einer Elektrode immer größer ist als die der anderen~\cite{Plett2024}. Aus dem stetigen Verlauf der Zellspannung leiten sich die Abschaltspannung\footnote{minimale erlaubte Spannung die den entladenen Zustand markiert}~\cite{Plett2015} und die Leerlaufspannung\footnote{Spannung wo kein Strom fließt und markiert damit den Gleichgewichtszustand}~\cite{Newman2021}.

\subsubsection*{C-Raten}
Die Auflade- oder Entladerate einer Batterie wird oft in sogenannten C-Raten angegeben. Dabei bedeutet 1~C, dass eine vollständig entladene/geladene Batterie in 1~h komplett aufgeladen/entladen wird. Bei einer doppelt so hohen C-Rate wird die Batterie folglich in der Hälfte der Zeit entladen bzw. aufgeladen. Bei einer halb so hohen Auflade- bzw. Entladerate (C/2) benötigt die Batterie 2~h zur vollständig Auf- oder Entladung. Die Wahl der C-Rate ist vor allem bei der Messung der Kapazität von Bedeutung. Je höher die C-Rate, desto geringer ist die gemessene Kapazität. Die Stärke des Kapazitätsabfalls wird durch eine Reihe von Faktoren, wie etwa Übergangsverhalten, Form und Art der chemischen Struktur der Elektrode bestimmt~\cite{Plett2015,Beard2019}.


\subsubsection*{Kapazität, Columbische Effizienz und Kapaziätserhalt}
Die Kapazität einer Elektrode beschreibt, wie viele Ladungsträger eingelagert oder entfernt werden können. Besonders in den ersten Zyklen besteht ein großer Unterschied zwischen Auflade- $\text{C}_{\text{aufl}}$ und Entladekapazität $\text{C}_{\text{entl}}$, weshalb die Kapazität für beide Prozesse getrennt bestimmt wird~\cite{Plett2015}.
Zur Ermittlung der Kapazität wird ein konstanter Auflade- $\text{I}_\text{C,aufl}$ oder Entladestrom $\text{I}_\text{C,entl}$ an eine Zelle aus Elektrode und Referenzelektrode, meist aus Lithiummetall, angelegt und die Zeit $\Delta \text{T}_\text{aufl}$ gemessen, die für die komplette Auf- bzw. Entladung benötigt wird. Da die Kapazität das zeitliche Integral des Stromes ist, kann die Bestimmung auf die folgenden Formeln vereinfacht werden~\cite{Newman2021}:
\begin{align}
	\text{C}_{\text{aufl}} &= \text{I}_\text{C,aufl} \cdot \Delta \text{T}_\text{aufl},\\
	\text{C}_{\text{entl}} &= \text{I}_\text{C,entl} \cdot \Delta \text{T}_\text{entl}.
\end{align}

%komplett entladen oder in der die Spannung von Abschalt- bzw. Leerspannung $\text{U}_{\text{leer}}$ zur vorher ermittelten maximal Spannung $\text{U}_{\text{voll}}$ benötigt. Die Aufladekapazit stellt dabei das Produkt aus Durch Umkehrung des Prozesses lässt sich die Entladekapazität bestimmen 

Bei der Entwicklung neuer Batteriematerialien wird die Kapazität meist auf die Masse des am Einlagerungsprozess teilnehmenden Materials (Aktivmaterial) normiert. %Diese spezifische Kapazität hat dann die Einheit [$\si{\A \hour \per \g}$]. Im Kontext der Batterieentwicklung wird allerdings die Kapazität je Elektrodenfläche [$\si{\A \hour \per \cm\squared}$] häufig angegeben.


%\subsection*{Columbische Effizienz}
Die coulombsche Effizienz (CE) dient zur Bewertung der internen Batteriereaktionen. Die CE des Zyklus \( n \) ist definiert als das Verhältnis der gemessenen Kapazität während des Entladevorgangs \( C_{Dch}(n) \) zur Kapazität des vorherigen Beladungsvorgangs \( C_{Ch}(n) \) \cite{Tornheim2020}.
Die Formel
\begin{equation}
CE = \frac{C_{Dch}(n)}{C_{Ch}(n)}
\end{equation}
gilt dabei für Aufbauten, die in einem Entladenzustand zusammengebaut werden und daher zuerst beladen werden müssen. Zellen, die in einem beladenen Zustand gefertigt werden, wie etwa Lithium-Schwefel-Batterien, beginnen allerdings zuerst mit einem Entladungszyklus. Die korrekte Formel lautet in einem solchen Fall
\begin{equation}
    CE = \frac{C_{Dch}(n+1)}{C_{Ch}(n)}.
\end{equation}

%\subsection{Kapaziätserhalt}
Die Kapazitätserhalt (CR\footnote{\textit{engl.} Capacity Retention}) dient zur Bemessung des Anteils an Nebenreaktionen, die zu einem Kapazitätsverlust in Batterien führen. Sie ist definiert als das Verhältnis von Entladungskapazität des $(n+1)$-ten Zyklus $C_{Dch}(n+1)$ und der des $n$-ten Zyklus $C_{Dch}(n)$:
\begin{equation}
    CR = \frac{C_{Dch}(n+1)}{C_{Dch}(n)}.
\end{equation}
In einigen Fällen wird CR auch im Verhältnis zur initialen Entlade-Kapazität bestimmt,
\begin{equation}
    CR = \frac{C_{Dch}(n)}{C_{Dch}(1)}.
\end{equation}
Dieser Ansatz ist besonders dann hilfreich, wenn die Langlebigkeit von Batterien zu quantifizieren ist \cite{Tornheim2020}.
Im Gegensatz zu der CE ist die CR meist relevanter für Hersteller und Endnutzer.

\subsubsection*{Energiedichte}
Die gravimetrische Energiedichte bzw. spezifische Energie bemisst, wie viel Energie pro eingesetzter Masse gespeichert werden kann. Alternativ wird mittels der volumetrischen Energiedichte die Menge an speicherbarer Energie pro Volumen angegeben. Beide Kenngrößen sind wichtige Größen, die bei der Entwicklung neuer Speichertechnologien nach Möglichkeit gesteigert werden sollen~\cite{Plett2015}.

Eine der größten Schwierigkeiten beim Umgang mit angegebenen Energiedichten aus der Literatur besteht in dem Umstand, dass oft bei der Masse oder dem Volumen auf verschiedene Komponenten Bezug genommen wird~\cite{Son2021}. Im Allgemeinen unterscheiden sich die Angaben darin, ob die Werte im Kontext von (1) Materialentwicklung, (2) Elektrodenentwicklung oder (3) Zellentwicklung
erhoben wurden (Tabelle~\ref{tab:energy_densities}).
So wird bei der Forschung an neuen Aktivmaterialien die spezifische Energie auf die eingesetzte Masse an Aktivmaterial bezogen~\cite{Son2021}. Im Bereich neuer Elektroden wird die Speicherkapazität entweder auf die Elektrodenmasse oder, im Falle einer Oberflächenbeschichtung, meist nur auf die Masse der Beschichtung normiert~\cite{Greenhalgh2023}. In Publikationen zu neuen Zellen gibt es noch mehr Varianten der massenbasierten Normierung. Je nach Autor finden sich hier Angaben in Referenz zur Masse der gebauten Knopfzelle oder der gebauten Pouchzelle~\cite{Akimoto1998,Liu2018}. Darüber hinaus gibt es bei Pouchzellen Varianten mit einer einzelnen Zelle oder einer mehrlagigen Ausführung~\cite{Schmuch2018}. Außerdem werden in manchen Publikationen die Mantelmaterialien herausgerechnet~\cite{Greenhalgh2023}.
\begin{table}[ht]
    \centering
    \caption{\label{tab:energy_densities}Spezifische Energie und Energiedichte für eine representative \ce{LiCoO2} (LCO) Kathode und eine Grafitanode in verscheidenen Referenzsystemen.\cite{Son2021}}
    \begin{tabular}[t]{lccccc}
    \toprule
    \multirow{2}{*}{}
    &\multirow{1}{*}{Materialebene} % \textsuperscript{*}
    &\multirow{1}{*}{Elektrodenebene}
    &\multicolumn{2}{c}{Zellebene}
    \\ \cmidrule{2-5}
    &\makecell{Aktivmaterial\\\includegraphics[width=0.125\textwidth]{CathodeMaterials/LiC6.png}\vspace{-1em}}
    &\makecell{Elektrode\\\includegraphics[width=0.1\textwidth]{EnergyDensitiesScales/Electrode.png}\vspace{-1em}}
    &\makecell{Knopfzelle\\\includegraphics[width=0.1\textwidth]{EnergyDensitiesScales/CoinCell.png}\vspace{-1em}}
    &\makecell{Pouchzelle\\(1 Ah)\\\includegraphics[width=0.1\textwidth]{EnergyDensitiesScales/PouchCell.png}\vspace{-1em}}
    \\
    \midrule
    \makecell{Spezifische Energie\\ $\left[ \si{\watt \hour \per \kg} \right]$} & 627 & 514 & 5 & 260\\
    \makecell{Energiedichte\\ $\left[ \si{\watt \hour \per \liter} \right]$} & 3166 & 1527 & 19 & 414\\
    \bottomrule
    \end{tabular}
    %\noindent{\footnotesize{\textsuperscript{*} Die Abkürzung nicht auffindbar (n.a.) wurde benutzt.}}
\end{table}%
Die beschriebene Uneinheitlichkeit in der Veröffentlichung der Energiedichte ist ein großer Diskussionspunkt in der Batterieforschung und macht es sehr aufwendig, vielversprechende Ansätze zu identifizieren, sowie dies zu vergleichen~\cite{Greenhalgh2023, Zschiebsch2024}.

%\subsection{Zyklenverhalten}
\subsection{Mechanische Charakterisierungen}
Steifigkeits- und Festigkeitsbetrachtungen im Kontext von Batteriesystemen spielen bislang kaum eine Rolle in der Forschung~\cite{Chen2024a} und rückten erst in den letzten Jahren mit einem zunehmenden Verständnis des Einflusses der mechanischen Eigenschaften auf das Batterieversagen in den Vordergrund~\cite{Zhu2023}. Anders ist dies bei Strukturbatterien, deren Fokus besonders auf mechanische Eigenschaften bisher eine Sonderrolle zukam~\cite{Asp2021}.

Zur Ermittlung wichtiger struktureller Kenngrößen, wie etwa Zugsteifigkeit und -festigkeit, werden üblicherweise Zugversuche nach ISO 527-4~\cite{ISO527-4-2023} und ISO 527-5~\cite{ISO527-5-2021} verwendet~\cite{Xu2022,Liu2022a}. Jedoch sorgen bei Strukturbatterien das vorhandene anisotrope Verhalten~\cite{Carlstedt2020b}, Sprödigkeit~\cite{Kalnin1982}, Temperatur-~\cite{Carlstedt2019a,Carlstedt2022b}, Feuchtigkeits-~\cite{Kosfeld2023} und Beladungsabhängigkeit~\cite{Qi2010,Duan2021} dafür, dass sich durch die Anzahl der Messungen für eine umfassende Klassifizierung ein enormer Zeit- und Ressourcenaufwand ergibt. Zusätzlich ist für komplexe Belastungen, wie etwa Biegung, auch das Deformationsverhalten unter Druck von Bedeutung, was allerdings mit entsprechenden Kompressionstests bisher nur in einigen Ausnahmen detaillierter untersucht wurde~\cite{Liu2022a,DiMauro2023}.

Eine oft benutzte Alternative stellt der nach ISO 178~\cite{ISO178-2019} beschriebene Drei-Punkt-Biegeversuch dar~\cite{AriefBudiman2022, Keshavarzi2022, Jin2021}. Der komplexe Belastungsfall generiert schnell Informationen über die Tragfähigkeit und ist für einige der diskutierten Anwendungsfälle, wie multifunktionelle Träger in Drohnen~\cite{Hollinger2019}, näher an dem Belastungsszenario als Zugversuche. Allerdings lassen sich nur schwer Materialkenngrößen für gängige Material- und Versagensmodelle, wie z.B. \textsc{Hook}~\cite{Atanackovic2000}, \textsc{Cuntze}~\cite{Cuntze2004} oder \textsc{Puck}~\cite{Puck2004}, aus diesem Versuch ableiten~\cite{Saba2019}.
 % Außerdem scheint der Mangel an aufwendigeren Zugversuchen dazu zu führen, dass in einigen Veröffentlichungen das Biegemodul als Ersatz für das Zugmodul benutzt wird, was nachweislich falsch ist.

%\subsection{Mechanische Spannung}
\subsection{Multifunktionale Effizienz}

Die einheitenlose multifunktionale Effizienz ist ein Maß, um zu bewerten, ob ein Vorteil durch den Einsatz von multifunktionalen Lösungen gegenüber einem kombinierten Einsatz von monofunktionalen Komponenten entsteht~\cite{Johannisson2020}.
Der von \textsc{Snyder} et al.~\cite{Snyder2015} im Jahr 2011 veröffentlichte Ansatz beschreibt die multifunktionale Effizienz als Summe der mechanischen und elektrochemischen Effizienz.
\begin{equation}
	\eta_{\text{mutli}} = \eta_{\text{mech}} + \eta_{\text{elchem}}= \frac{E}{E_{\text{ref}}} + \frac{\Gamma}{\Gamma_\text{ref}} 
\end{equation}
Dabei wird häufig die mechansiche Effizienz $\eta_{\text{mech}}$ durch das Verhältnis der E-Module ($E/E_{\text{ref}}$) von System und Referenz und $\eta_{\text{mech}}$ durch das Verhältnis der Energiedichten $\Gamma$ angenähert.
Diese Herangehensweise erlaubt eine vereinfachte Betrachtung der sonst komplexen multidisziplinären Optimierung und ist nach \textsc{Ashby}~\cite{Ashby2000} eine der möglichen Optimierungsstrategien für Materialdesign und -auswahl. In der Strukturbatterieforschung hat sich der Bewertungsansatz mittels multifunktionaler Effizienz weitgehend durchgesetzt~\cite{O’Brien2011,Freund2018}.

Im Kontext von Strukturspeichern wird für als Referenzsystem für die mechanische Effizienz ein UD-Kohlefasergelege und für die elektrochemischen Anteil eine kommerzielle Li-Ionenbatterie, die beide allerdings an neue Entwicklungen stehts anzupassen ist~\cite{Sha2021}. Ein multifunktionaler Effizienzwert größer oder gleich eins bedeutet, dass durch den Einsatz eine Reduktion der Gesamtmasse gegenüber der monofunktionalen Lösung erreicht wird~\cite{Snyder2015}.

\section{Materailauswahl für Komponenten der Strukturbatterie}

Jedes Material in einer Strukturbatterie erfüllt gleichzeitig mehrere Aufgaben. Dies bedeutet, dass 
die am häufigsten verwendete Untergliederung die Materialien nach ihrer elektrochemischen Funktion innerhalb des Ladungsspeichers eingeteilt sind. Im nachfolgenden werden daher Anforderungen und typische Materialien für Andode, Kathode, Elektrolyt, Seperator und Pouchfolie näher erläutert.


\subsection{Anode}
Die Anode sollte ein niedriges elektrochemisches Potenzial und eine schnelle Interkalation\footnote{Ioneneinlagerung in das Material} für eine möglichst hohe Energiedichte und Leistungsdichte aufweisen. Insbesondere bei Strukturbatterien sind außerdem materialien mit hohen Festigkeits- und Steifigkeitswerten interessant.

Die Verwendung von Kohlenstoff in Lithium-Ionen-Batterien wurde erstmals von \textsc{Yoshino} \cite{Yoshino1986} im Jahr 1986 veröffentlicht.
Heute ist Kohlenstoff eines der meistbenutzten Materialien in wiederaufladbaren nicht-wässrigen Batterien \cite{Ahmad2021}. Am weitesten verbreitet ist dabei die Kombination von Graphit als Anode und einer Kathode aus Phosphat, welche eine maximale Energiedichte von 200 bis 250~$\si{\watt \hour \per \kg}$ erreicht. 
Für die elektrochemischen Eigenschaften des Kohlenstoff ist eine Unterteilung in geordneten und ungeordneten Kohlenstoff von besonderer Bedeutung~\cite{Ghosh2024}, siehe Bild~\ref{fig:carbon_types}.

\begin{figure}[!h]
	%\raggedleft
		%\def\svgwidth{\columnwidth}
        \center
		%\input{Abbildungen/02_SoA/electrolyte_data/}
		\includegraphics[width=\textwidth, angle=0]{carbon_types.pdf}
	%\includegraphics[width=\textwidth, angle=0]{bicontinous_electrolyte.pdf}
		\caption{\label{fig:carbon_types}Unterteilung des Kohlenstoffarten nach \textsc{Ghosh} und \textsc{Wang}~\cite{Kundu2020,Wang2021,Liu2022b,Ghosh2024}.}
\end{figure}

Geordneter Kohlenstoff ist ein Sammelbegriff für sp2-hybridisierte Kohlenstoffverbindungen\footnote{Dieser Begriff leitet sich aus dem Orbitalmodell ab. Diese beschreibt die Elektronenhülle eines Atoms. Für Kohlenstoffverbindungen sind vorallem die Hybridisierungsformen sp2, resultiert in ebene Strukturen, und sp3, ergibt räumliche bzw. tetraedrische Verbindungen, von Bedeutung.} mit einer weitreichenden Ordnung und damit vebunden hohen Kristallinität. Die Ordnung kann dabei entlang (1) einer Achse (CNTs), (2)einer Ebene (Graphen) oder (3) aller drei Raumkoordinaten (Graphit) existieren~\cite{Wang2021}.

CNTs sind geordnete 1D-Kohlenstoffstrukturen, die 1991 von \textsc{Iijima} \cite{Iijima1991} erstmals entdeckt wurden. Diese zylindrischen Formen des Kohlenstoffs haben einen Durchmesser von 1 bis 20~$\si{\nano\metre}$ und meist ein hohes Verhältnis von Länge zu Durchmesser, mit der höchsten bisher dokumentierten Länge von 55~$\si{\centi\metre}$. CNTs werden meist durch ihre Schichtanzahl in SWCNT\footnote{\textit{engl.} Single Wall Carbon Nano Tube} und MWCNT\footnote{\textit{engl.} Multi Wall Carbon Nano Tube} unterschieden. Darüber hinaus können SWCNTs, je nach Winkel des graphenähnlichen Gitters im Mantel gegenüber der Zylinderachse, metallische oder halbleiterähnliche Eigenschaften aufweisen. 
SWCNTs und MWCNTs besitzen hohe spezifische Oberflächen von etwa 1300~$\si{\m^2\per g}$, eine sehr hohe elektrische Leitfähigkeit von bis zu 5000~$\si{\siemens \per \cm}$ und eine hohe Ionenleitfähigkeit von über 100000~$\si{\cm \squared \per \V \per \s}$ \cite{Xu2011,Uetani2014,Charlier2007}.

Seit seiner Entdeckung im Jahr 2004 \cite{Novoselov2004} ist Graphen zunehmend in den Fokus der Batterieforschung geraten. Mit einer theoretischen Kapazität von über 1000~$\si{\mA \hour \per \g}$, einer hohen mechanischen Zugfestigkeit von annähernd 130~$\si{\GPa}$ und einer Zugsteifigkeit von ungefähr 1~$\si{\tera \Pa}$ stellt es ein ideales Material für den Einsatz in Strukturbatterien dar \cite{Novoselov2012}. Jedoch konnte das Material bisher nur im Labormaßstab und in unzureichenden Mengen synthetisiert werden~\cite{Shams2015}. Auch ist bisher umstritten, wie die Einlagerung von Lithium bei Graphen genau abläuft, was einen starken Einfluss auf die theoretische Kapazität hat~\cite{Safie2023, Singh2024}. Bisherige Experimente mit zweilagigem Graphen kommen zu unterschiedlichen Ergebnissen. \textsc{Ji} et al. beobachteten einen Mechanismus, der auf einen ähnlichs Verhalten wie bei Graphit schließen lässt~\cite{Ji2019}, während \textsc{Kühne} et al. sogenannte superdichte Lithiumeinlagerungen zwischen den beiden Graphenschichten beschreiben~\cite{Kuehne2017}. Derzeitig geht die Produktion von Graphen nicht über den Labormaßstab hinaus und bleibt daher für den Einsatz in Strukturbatterien bis auf Weiteres ungeeignet~\cite{Zhu2014, Singh2024}.


Graphit hat eine hochkristalline Struktur und besitzt eine große Fernordnung. Die $\text{sp}^\text{2}$-hybridisierten Graphenschichten sind entlang der c-Achse gestapelt\footnote{Die Abfolge der Graphenschichten unterscheidet ist für die hexagonalen AB-Sequenz und der rhomboedrischen ABC-Folge unterschiedlich.}~\cite{Inagaki2014}, siehe Bild~\ref{fig:carbon_types}. Durch die Delokalisierung der $\pi$-Elektroenen liegt  eine gute Leitfähigkeit von $10^3$-$10^4$~$\si{\siemens \per \cm}$ in der Ebenenrichtung vor~\cite{Dutta1953}. Die Graphenschichten haben einen Abstand von 0,335~$\si{\nano\meter}$ entlang der c-Achse und werden nur durch relativ schwache van der Waals-Kräfte mit einem Wert von 16 bis 17~$\si{\kJ \per \mol}$ zusammengehalten~\cite{Xu2012}. Der relativ hohe Abstand und die schwachen Bindungskräfte erleichtert die Einlagerung kleinerer Atome, wie z.B. Lithium oder Kalium~\cite{Wang2021}.

\begin{figure}[!h]
	%\raggedleft
		%\def\svgwidth{\columnwidth}
        \center
		%\input{Abbildungen/02_SoA/electrolyte_data/}
		\includegraphics[width=\textwidth, angle=0]{graphite_intercalation.pdf}
	%\includegraphics[width=\textwidth, angle=0]{bicontinous_electrolyte.pdf}
		\caption{\label{fig:graphite_intercalation}Der Interkalationsprozess von Lithiumionen in Graphit unter Andeutung der vorwiegenden Gitterverzerrungen~\cite{Nishidate2005,Markevich2005}.}
\end{figure}

Der Interkalationsprozess läuft dabei in mehreren Stufen ab, was sich im Potenzialverlauf erkennen lässt. Das Lithium-Ion ($\text{Li}^{+}$) wird dabei zwischen zwei benachbarten Graphenschichten eingelagert, wobei jedes $\text{Li}^{+}$ den niedrigsten Energiezustand einnimmt, der im Zentrum eines hexagonalen Kohlenstoffrings existiert \cite{Sole2014,Weng2023} (Bild~\ref{fig:graphite_intercalation}). Allerdings können $\text{Li}^{+}$ sich nicht durch die Graphenschichten hindurch bewegen, weshalb diese Transportbewegung nur durch Gitterdefekte möglich ist \cite{Nishidate2005}. Die Einlagerungsgeschwindigkeit zwischen den Schichten ist dabei nicht konstant und kann während jeder Stufe um teilweise das Tausendfache schwanken \cite{Levi1997}. Dieses Verhalten kommt nach \textsc{Aurbach} et al. durch die Bildung von Lithium-Clustern zwischen den beiden Graphenschichten zustande, welche die Diffusion weiterer $\text{Li}^{+}$ am Anfang einer neuen Phase verhindern \cite{Markevich2005}. Die maximale Einlagerungsmenge ist mit der $\text{LiC}_\text{6}$-Konfiguration erreicht, bei der zwischen jeder Graphitschicht alle möglichen Plätze belegt sind. Die Menge an eingelagerten $\text{Li}^{+}$ entspricht dabei einer theoretischen spezifischen Kapazität von 372~$\si{\mA \hour \per \g}$ \cite{Winter1998}. 
Eine weitere wichtige Eigenschaft ist die relativ hohe Dichte von über 2~$\si{\g \per \cm \cubed}$, wodurch möglichst viel Aktivmaterial in einem geringen Volumen untergebracht und damit kleine Batterien mit einer hohen Energiedichte erzeugt werden können.

Im Gegensatz zum geordenten Kohlenstoff werden unter ungeordnetem Kohlenstoff alle Strukturen zusammengefasst, die durch amorphe sp3-hybridisierte Bereiche die weitreichende periodische Struktur in zufällig ausgerichteten sp2-graphitischen Mikrobereichen zergliedern~\cite{Inagaki2014}. Dabei entscheidet der Graphitisierungsgrad\footnote{Anteil an sp2-hybridisierten Bereichen. Dieser lässt sich u.a. durch Ramanspektroskopie bestimmen~\cite{Yu2014}.}, ob eine Graphitisierung bei Temperaturen bis zu 3000~$\si{\degreeCelsius}$ in einer inerten Atmosphäre möglich ist\footnote{Dieser Prozess wird auch als Pyrolyse bezeichnet~\cite{Kim2017a}.}~\cite{Inagaki2014}. Dies führt zu einer Unterscheidung in sogenannten weichen bzw. graphitisierender und harte bzw. nicht-graphitisierenden Kohlenstoff, siehe Bild~\ref{fig:carbon_types}.

Bei weichem oder graphitisierendem Kohlenstoff reduzieren die amorphen sp3 hybridisierten Bereiche die Ionenspeicherkapazität bei langsamen Beladungsgeschwinigkeiten von C/10, wobei graphitischer Kohlenstoff etwa 250~$\si{\mA \hour \per \g}$ und Graphit 372~$\si{\mA \hour \per \g}$ erreicht~\cite{Schroeder2014}. Jedoch erfolgt die Einlagerung deutlich schneller, was bei höheren Beladungs- und Entladungstests ab 10C zu einer mehr als dreimal höheren Kapazität von 90~$\si{\mA \hour \per \g}$ im Vergleich zu Graphit 25~$\si{\mA \hour \per \g}$ führt \cite{Schroeder2014}. Auch zeigt graphitisierender Kohlenstoff im Gegensatz zu Graphit keine Einbrüche im Diffusionsverhalten, was dafür spricht, dass die Einlagerung stufenlos erfolgt~\cite{Huajun2007}. Allerdings bleiben auch in den weniger geordneten Strukturen mehr $\text{Li}^{+}$ gefangen, weshalb die CE während des ersten Zyklus für weichen Kohlenstoff nur bei etwa 72~\% liegt, während Graphit 82~\% erreicht~\cite{Schroeder2014}. Die Verluste zeigen sich allerdings nur in den ersten Zyklen. Nach einem ausreichenden Prelithiierungsprozess\footnote{Das, machmal auch mehrfache, Beladen mit Lithium.} die CE auch für graphitisierenden Kohlenstoff über 99~\% \cite{Schroeder2014}. Für die Herstellung von weichem Kohlenstoff wird häufig die thermische Zersetzung von verschiedenen organischen Precursoren in einer inerten Atmosphäre bei hohen Temperaturen von 1000 bis 1700~$\si{\degreeCelsius}$\footnote{In der Literatur dies auch als Karbonisierung bezeichnet.} benutzt~\cite{Ghosh2024,Kim2017a}. Besonders geeignet sind hierbei pyrolytische aromatische Verbindungen wie etwa Pech, Benzol, Petrolkoks, Polyvinylacetat und Polyvinylchlorid \cite{Wang2021}.

Harter oder nicht-graphitisierender Kohlenstoff wird meist aus der Karbonisierung von Precursoren mit wenigen aromatischen Strukturen, wie etwa Zucker, Holzkohle, Cellulose und Kokosnussschalen, gewonnen \cite{Wang2021}. Die komplexeren organischen Strukturen der Ausgangsmaterialien sorgen dafür, dass nach der Karbonisierung eine signifikante Anzahl an kleineren Poren und Rissen in der Mikrostruktur verbleiben~\cite{Liu2019a}. Diese Defekte beweirken eine hohe aktive Oberfläche und erlauben einen schnellen Zugang zu den graphitischen Bereichen, wo die Interkalation stattfindet, und ~\cite{Fujimoto2010}. Des Weiteren besitzen nicht-graphitisierende Koohlenstoffe eine hohe zyklische Stabilität, mit einer CR von 85~\% nach hunderttausend Zyklen \cite{Cao2014}. Allerdings sorgt der geringere Graphitisierungsgrad für signifikanten eine Reduktion der Kapazität zu ungefähr 53~$\si{\mA \hour \per \g}$ bei 1C und 48~$\si{\mA \hour \per \g}$ bei 20C~\cite{Sun2017}.

% Eine CAG ist ein hartcarbon?

%Eines der am frühsten und immer noch am weitverbreitesten Aktivematerialien anodenseitig ist Graphit. Zwischen den Graphitschichten können Lithiumionen eingelagert werden. In herkömmlichen monofunktionalen Batterien werden oft dünne Kupferfolien mit einer Graphitpartikelbeschichtung verwendet. Die zusätzliche Additive in der Pulvermischung halten die Partikel zusammen und sorgen für einen geringen Widerstand beim Transport der Elektronen zur Kupferelektrode. Die Bindungen zwischen den Partikeln sind jedoch sehr schwach und tragen nicht zur Steigerung der mechanischen Eigenschaften bei \cite{Chen2024}. Außerdem ~mAh/gsorgt die Ausdehnung infolge von Lithiierung mit der Zeit für Risse durch die mit der Zeit der Leitungswiderstand steigt, was einer von vielen beobachten Alterungsmechanismen von Batterien ist \cite{Xiong2020}.

%Die begrenzte Kapazität, langsame Diffusionskinetik, geringe mechanische Eigenschaften, sind einige der Faktoren die Untersuchungen Kohlenstoff-Nanostrukturen und andere Morphologien bewegen.
\begin{figure}[!h]
	%\raggedleft
		%\def\svgwidth{\columnwidth}
        \center
		%\input{Abbildungen/02_SoA/electrolyte_data/}
		\includegraphics[width=0.90\textwidth, angle=0]{fiber_comparison.pdf}
	%\includegraphics[width=\textwidth, angle=0]{bicontinous_electrolyte.pdf}
		\caption{\label{fig:fiber_comp}(1) Vergleich der Entladungskapazität nach dem ersten Zyklus für PAN-basierte und Pech-basierte Fasern im Bezug auf a) Zugmodul und b) Zugfestigkeit der Fasern. (2) Vergleich der a) Entladekapazität und b) columbischen Effizienz für beschichtete und unbeschichte PAN-basierte Fasern~\cite{Kjell2011,Snyder2009a,Hagberg2016,Ye2024}.}
\end{figure}
Kohlenstofffasern sind einer der vielversprechendsten Kandidaten für lasttragende Anoden~\cite{Greenhalgh2024a}. Ungefähr 96~\% aller Fasern weltweit werden aus Polyacrylnitril (PAN) hergestellt~\cite{Das2016}. Die restlichen werden aus Precursorn wie Pech, Rayon oder Lignin gewonnen. Kohlenstofffasern besitzen im Allgemeinen hohe Festigkeits- und Steifigkeitswerte sowie eine elektrisch gut leitende Oberfläche, die mit 0,2~$\si{\metre\squared\per\g}$ für konventionelle Kohlenstofffaser zu kleien für Batterieanwendungen ist~\cite{Qian2013,Senokos2023}. Durch verschiedene Oberflächenmodifikationen kann diese allerdings auf über 200~$\si{\metre\squared\per\g}$ gesteigert werden~\cite{Zenkert2024}. Jedoch haben die Wahl des sogenannten Precursormaterials sowie die Verfahrensparameter während des Spinnens, Stabilisierens und Karbonisierens einen entscheidenden Einfluss auf die Struktur der Faser~\cite{Newcomb2015}. Diese hat wiederum signifikanten Einfluss auf die mechanischen, elektrischen und elektrochemischen Eigenschaften.

Verallgemeinert lässt sich feststellen, dass ein höherer Anteil an kristallinen Graphitstrukturen in der Faser zu einer höheren Steifigkeit, Festigkeit sowie thermischen und elektrischen Leitfähigkeit führt~\cite{Zenkert2024}. Jedoch ist die Kapazität von 150~$\si{\mA\per\g}$ (C/10) bei diesen hochmoduligen Fasern, wie etwa M60J, deutlich geringer als bei Fasern mit niedrigem Kristallinitätsanteil, wie etwa T800 mit 265~$\si{\mA\per\g}$ und IMS65 mit 358~$\si{\mA\per\g}$ \cite{Fredi2018} (Bild~\ref{fig:fiber_comp}). Die geringere Kapazität kommt durch die relativ großen, sich wie ein Mantel um die Faser ausbildenden Kristallstrukturen und turbostatischen Graphitstrukturen zustande, die einen radialen Ionentransport stark behindern~\cite{Zenkert2024,He2021}. Bei Fasern mit weniger ausgeprägter Graphitkristallausbildung bieten die zahlreichen Gitterdefekte, ähnlich wie bei nicht-graphitisierenden Kohlenstoff, genug Zugang für die $\text{Li}^{+}$, um sich bei kleineren Beladungsraten vollständig einlagern zu können \cite{Fredi2018}. Dies deckt sich mit Beobachtungen, dass sich Lithium zunächst in den ungeordneten, amorphen, Bereichen einlagert und erst bei höherer Beladung auch die graphitischen Strukturen besetzt werden \cite{Fang2022}. Wie auch bei graphitischen Kohlenstoffen verlieren Kohlenstofffasern einen großen Teil ihrer Ladungsträger während des ersten Zyklus \cite{Jacques2013} (Bild~\ref{fig:fiber_comp}). Jedoch bleibt die CE auch nach zehn Zyklen bei einem Wert über 99,9~\% \cite{Hagberg2016}, was bedeutet, dass der weitere Beladungs- und Entladungsprozess nahezu verlustfrei ist. Allerdings hat die Einlagerung von Ionen auch zur Folge, dass sich die mechanischen Eigenschaften der Fasern ändern. Dabei verdoppelte sich der E-Modul quer zur Faserrichtung im lithiierten Zustand und geht während der Delithiierung nahezu vollständig auf die Werte im Ursprungszustand zurück~\cite{Duan2021}. Für das Modul in Faserrichtung ergeben sich keine Veränderungen. Weiterführende Zugversuche im lithiierten und delithiierten Zustand zeigten außerdem, dass die Zugfestigkeit während der Lithiierung um 25-30~\% zurückgeht und selbst nach der Entladung um 5-10~\% geringer ist als im ursprünglichen Zustand \cite{Jacques2012}. Versuche mit verschiedenen Lithiierungsgraden konnten dabei eine direkte Abhängigkeit zur Zugfestigkeit feststellen \cite{Jacques2014}, was darauf schließen lässt, dass die durch die Einlagerung verursachten Dehnungen im Material maßgeblich den Festigkeitsverlust beeinflussen \cite{Zenkert2024}. Der Festigkeitsverlust im Zusammenhang mit einer multifunktionalen Nutzung muss damit zwar unbedingt berücksichtigt werden, spielt aber besonders bei Anwendungen mit hoher geforderter Steifigkeit eine untergeordnete Rolle, da eine weitere Degradierung der Fasern nicht beobachtet wurde \cite{Zenkert2024}.

\begin{table}[ht]
    \centering
    \caption{Übersicht bisher entwickelter Strukturbatterien.}
    \begin{tabular}[t]{lcccc}
    \toprule
    &Entlade Kapazität\textsuperscript{*} [mAh/g]
    &CR [\%] % Capacity Retention
    &$\text{D}_{\text{Li}}$ %[$\text{cm^2/s}$]
    &Ref.\\
    \midrule
    Graphit
        &356-372
        &98
        &$10^{-7}-10^{-6}$ ($10^{-11}$\textsuperscript{,K})
        &\cite{Persson2010,Wang2021,Olutogun2024}\\
    Graphen
        &770/1115
        %&100
        &90
        &$7 \times 10^{-5}$
        &\cite{Zhu2014,Wang2017,Kuehne2017}\\
    Kohlenstofff Nanoröhren
        &1115
        &90
        &$10^{-14}-10^{-11}$
        &\cite{Maurin1999,Zhao2000,Meunier2002,Shin2002,Nishidate2005,Schauerman2012}\\
    Harter Kohlenstoff
        &200-600 % 0.2C
        %802-1063 lade capacitität
        % 27.9-47.3 lade/entlade effizienz / Columbic Efficiency
        &72-90 % nach 50 Zyklen
        &$10^{-9}$-$10^{-8}$
        &\cite{Fujimoto2010,Bridges2012,Yang2012}\\
    Karbon Aerogel
        &349-570,2
        &31,9-97%(836.9-570.2)/836.9
        &n.a.
        &\cite{Yang2015,Pham2024,Li2022a}\\
    T300
        &91
        &46,5 % (170-91)/170
        &$10^-12-10^-11$
        &\cite{Uchida1996,Kjell2011,Johansen2022}\\
    T300 unbeschichtet
        &130
        &62,9 %(350-130)/350
        &$10^-12-10^-11$
        &\cite{Uchida1996,Kjell2011,Johansen2022}\\
    T800
        &98
        &42,4 % (170-98)/170
        &n.a.
        &\cite{Kjell2011,Johansen2022,Johansen2024}\\
    T800 unbeschichtet
        &112
        &42,3 %(194-112)/194
        &n.a.
        &\cite{Kjell2011,Johansen2022,Johansen2024}\\
    IMS65
        &108
        &34,9 %(166-108)/166
        &$10^{-8}-10^{-6}$
        &\cite{Kjell2011}\\
    IMS65 unbeschichtet
        &177
        &52,3 %(360-177)/350
        & $10^{-8}-10^{-6}$
        &\cite{Kjell2011,Kjell2013}\\
    \bottomrule
    \end{tabular}
    \noindent{\footnotesize{\textsuperscript{1} gemessen nach min. 10 Zyklen.}}
    \noindent{\footnotesize{\textsuperscript{K} Korngrenze.}}
    \noindent{\footnotesize{\textsuperscript{*} Die Abkürzung nicht auffindbar (n.a.) wurde benutzt.}}
\end{table}%

Abschließend sei erwähnt, dass Umwandlungsmetalle wie etwa Silicium zwar Energiedichten größer als 250~$\si{\watt \hour \per \kg}$ erreichen, jedoch ist der Interkalationsprozess mit großen Volumenänderungen verbunden, wodurch sich die Zyklenzahl drastisch reduzieren~\cite{Gayet2009, Pereira2019}. Die großen Dehnungsunterschiede und die geringe mechanische Belastbarkeit machen diese Art von Anodenmaterial daher uninteressant für den Einsatz in Strukturbatterien~\cite{Javaid2018}.

\subsection{Kathode}

Kohlefasern spielen nicht nur bei faserbasierten Anoden für Strukturbatterien eine entscheidende Rolle~\cite{Ye2024}. Im Kontext von Faserkathoden dienen sie jedoch nicht als Aktivmaterial, sondern übernehmen hauptsächlich die Aufgabe des Stromkollektors, wodurch Materialien wie Aluminium effektiv substituiert werden können~\cite{Martha2012}. Zudem wurde beobachtet, dass Kathodenmaterialien wie \ce{LiFePO4} einen geringeren Kapazitätsverlust mit Kohlefasern als Trägermaterial aufweisen als bei Aluminium~\cite{Martha2011}.

Aus der Batterieforschung sind bereits eine Vielzahl an verschiedenen Aktivmaterialien für den Einsatz in Kathoden untersucht worden (Tabelle~\ref{tab:cathode_material}). Lithium-Cobaltdioxid (\ce{LiCoO2} bzw. \ce{LCO}) ist dabei das am frühesten kommerziell eingesetzte Kathodenmaterial\cite{Amatucci1996}. \ce{LCO} zeichnet sich durch eine hohe theoretische Kapazität von 274~$\si{\milli \ampere \hour \per g}$ aus~\cite{Ren2023}. Allerdings sorgen komplexe Phasenumwandlungen zwischen geordneten und ungeordneten Phasen oberhalb von 4,1~V dafür, dass nur ein Teil, etwa 140~$\si{\milli \ampere \hour \per g}$, reversibel gespeichert werden kann~\cite{Vetter2005, Liu2018}. Des Weiteren ist Kobalt relativ teuer, was eine Reduktion des Kobaltanteils in Batteriesystemen motiviert~\cite{Ren2023}.
Durch den ersten Erfolg mit \ce{LCO} inspiriert, wurden weitere Übergangsmetalloxide untersucht. \ce{LiNiO2} hat dabei im Vergleich zu \ce{LCO} ein leicht geringeres Spannungsplateau, jedoch auch eine leicht höhere reversible Kapazität~\cite{Ren2023}. Allerdings sind die durch Sauerstoff bedingten Struktursymmetrien anfällig für den sogenannten Jahn-Teller-Effekt\footnote{Ein wichtiger Mechanismus das spontane Verzerrungen der Molekularstruktur beschreibt um einen degenerirten Elektronenzustand aufzulösen~\cite{Jahn1937}.}, weshalb es weniger chemisch stabil ist~\cite{Kalyani2005}. Dies führt zum Kollaps von Interkalationsstellen, der sich mit steigender Zyklenzahl auch auf benachbarte Stellen ausbreitet, was einer der Hauptgründe für den stärkeren Verlust der spezifischen Kapazität ist~\cite{Vaelikangas2020}. Auch \ce{LiMnO2} hat aufgrund der starken manganbasierten Gitterverzerrung eine geringe Stabilität. Dies führt dazu, dass auch hier nur etwa 148~$\si{\milli \ampere \hour \per g}$ von der sehr hohen theoretischen spezifischen Kapazität von 285~$\si{\milli \ampere \hour \per g}$ effektiv genutzt werden können~\cite{Chen2014}. Das Interesse der Batterieforschung an \ce{LiMnO2} basiert vor allem auf der hohen Betriebsspannung von 4,7~V\footnote{Gemessen gegenüber dem Li/Li+ Potenzial.}~\cite{Liang2020,Yu2021,Liang2020a}.
\begin{table}[ht]
    \centering
    \caption{\label{tab:cathode_material}Übersicht verschiedener Aktivmaterialien für Kathoden.}
    \begin{tabular}[t]{lcccc}
    \toprule
    &
    &\makecell{Kapazität\\$\left[ \si{\mA \hour \per \g} \right]$} % \textsuperscript{*}
    &\makecell{Betriebsspannung\textsuperscript{*}\\$\left[ \si{\V} \right]$}
    %&\makecell{E-Modul\\ $\left[ \si{\GPa} \right]$}
    %&\makecell{Zugfestigkeit\\ $\left[ \si{\MPa} \right]$}
    &\makecell{Ref.}
    %&CR [\%] % Capacity Retention
    %&$\text{D}_{\text{Li}}$ %[$\text{cm^2/s}$]
    %&Ref.
    \\
    \midrule
    \ce{LiCoO2}&\makecell{\includegraphics[width=0.18\textwidth]{CathodeMaterials/LiCoO2.png}\vspace{-1.0em}}
    &140&4,1&\cite{Zhang2004,Lyu2020}\\
    \ce{Li2NiO2}&\makecell{\vspace{-0.5em}\includegraphics[width=0.18\textwidth]{CathodeMaterials/Li2NiO2.png}\vspace{-0.5em}}
    &235&4,0&\cite{Fan1998,Dahn1991,Arai1997}\\
    \ce{LiMnO2}&\makecell{\includegraphics[width=0.18\textwidth]{CathodeMaterials/LiMnO2.png}\vspace{-0.8em}}
    &150&4,7&\cite{Croguennec1996,Vitins1997}\\
    \ce{LiFePO4}&\makecell{\includegraphics[width=0.18\textwidth]{CathodeMaterials/LiFePO4.png}\vspace{-0.8em}}
    &138&3,6&\cite{Padhi1997} \\
    \ce{LiNMC}111&\makecell{\includegraphics[width=0.18\textwidth]{CathodeMaterials/LiNMC111.png}\vspace{-0.5em}}
    &145&4,5&\cite{Park2024,Schmiegel2019}\\
    \ce{LiNMC}811&\makecell{\includegraphics[width=0.18\textwidth]{CathodeMaterials/LiNMC811.png}\vspace{-0.7em}}
    &230&4.3&\cite{Bhowmik2025,Quilty2024}\\
    \bottomrule
    \end{tabular}\\
    \noindent{\footnotesize{\textsuperscript{*} Gemessen gegenüber \ce{Li}/\ce{Li+}.}}
\end{table}%
Eine vielversprechende Maßnahme, um den Nachteilen der jeweiligen einzelnen Übergangsmetalloxide zu begegnen, ist das Kombinieren der jeweiligen Übergangsmetalle~\cite{Biasi2017}. Durch die Kombination können in einem gewissen Rahmen die Vorteile der einzelnen Systeme, wie etwa die höhere Stabilität von \ce{LiCoO2}, die höhere nutzbare Kapazität von \ce{LiNiO2} und die höhere Spannung von \ce{LiMnO2}, ohne die vorherigen Nachteile, wie Kosten, geringes Spannungsplateau und Instabilität, ausgenutzt werden~\cite{Ren2023}.
Die daraus resultierenden vorteilhaften Eigenschaften gegenüber reinem \ce{LCO} sind wichtige Kriterien, warum im Automobilbereich primär Nickel-Mangan-Cobaltoxid (\ce{NMC} bzw.\ \ce{NMC}111, \ce{LiNi_{1/3}Mn_{1/3}Co_{1/3}O2}) verwendet wird~\cite{Burow2016}. Das von \textsc{Ohzuku} et al.~\cite{Ohzuku2001, Yabuuchi2003} entdeckte Material ist im Vergleich zu \ce{LCO} kostengünstiger, weist eine höhere Kapazität auf und ist auch bei höheren Temperaturen noch chemisch stabil~\cite{Kim2016,Zheng2012}. 
Die höhere strukturelle Stabilität der Gitterstruktur erlaubt auch das reversible Zyklieren innerhalb eines größeren Stöchiometrie-Fensters\footnote{Anteil an eingelagerten Ionen.}~\cite{Dolotko2014,Choi2005}. 
Für kommerzielle Kathodenmaterialien wird meist aus Kostengründen und zum Erreichen einer höheren Kapazität eine Reduzierung des Kobaltanteils angestrebt. Systeme mit höherem Nickel- und Mangananteil wie etwa NMC442 und NMC811 erreichen dabei reversible Kapazitäten von jeweils 211~$\si{\milli \ampere \hour \per g}$~\cite{Ma2014} und 230~$\si{\milli \ampere \hour \per g}$~\cite{Xue2021}. Ein kompletter Verzicht auf Kobalt ist jedoch wegen der stabilisierenden Wirkung nicht möglich~\cite{Ren2023}.

Ein alternativer Ansatz ist hierbei phosphatbasiertes Kathodenmaterial (\ce{LiFePO4} und \ce{LiMn_{1-x}Fe_xPO4}). Lithiumeisenphosphat (LFP) ist sehr stabil, besitzt einen höheren Diffusionskoeffizienten und das Merkmal, dass auch bei Schädigung kein Sauerstoff frei wird~\cite{Ling2021}. Zudem sind Aktivpartikel aus \ce{LiFePO4} mit einem Durchmesser von 1,5~$\si{\um}$ klein genug, um als Aktivbeschichtung für eine Kohlefaser zu dienen~\cite{Yuecel2023}.
Allerdings hat die geringe elektrische Leitfähigkeit von $10^{\text{-}9}$ bis $10^{\text{-}11}$~$\si{\milli \siemens \per \cm}$) zur Folge, dass bei Kontaktverlust mit den leitenden Passivmaterialien die elektrischen Verluste der Batterie signifikant ansteigen~\cite{Padhi1997}. Außerdem ist die Volumenänderung während der De- bzw. Lithiierungsphase fast dreimal so hoch wie bei NMC (Tabelle~\ref{tab:volume_change}).

\subsection{Elektrolyte}
In konventionellen Batterien dient das Elektrolyt hauptsächlich als Transportmedium für die ionischen Ladungsträger~\cite{Gerlach2020}. Im Kontext von elektrischen Strukturspeichern sind auch mechanische Eigenschaften wie Steifigkeit und Festigkeit von Bedeutung~\cite{Greenhalgh2023}. Des Weiteren hat das Elektrolytmaterial einen entscheidenden Einfluss auf die maximale elektrische Spannung~\cite{Xu2016}, die Betriebstemperatur~\cite{Chen2022a}, Toxizität~\cite{Beard2019}, Entflammbarkeit und das Brandverhalten~\cite{Roth2012}. Um einen Beitrag zur Steigerung der Multifunktionalität von Strukturspeichern zu leisten, wird von \textsc{Greenhalgh} et al. für Strukturelektrolyten ein Zugmodul von mehr als 1~$\si{GPa}$ und eine ionische Leitfähigkeit größer als 1~$\si{\milli \siemens \per \cm}$ als zu überschreitende Grenzwerte angegeben~\cite{Greenhalgh2023} (siehe Bild~\ref{fig:electrolyte_data}).
\begin{figure}[ht]
	%\raggedleft
		%\def\svgwidth{\columnwidth}
        \center
		%\input{Abbildungen/02_SoA/electrolyte_data/}
		\import{Abbildungen/02_SoA/electrolyte_data/}{electrolyte_data.pdf_tex}
	%\includegraphics[width=\textwidth, angle=0]{bicontinous_electrolyte.pdf}
		\caption{\label{fig:electrolyte_data}Multifunktionale Performanz von verschiedenen Strukturelektrolyten~\cite{Greenhalgh2023}.}
\end{figure}

Diese Limitierungen als auch die Unfähigkeit Schubspannungen aufzunehmen schließen die in konventionellen Batterien etablierten flüssigen Elektrolytsysteme für Strukturbatterien kategorisch aus~\cite{Shirshova2013, Greenhalgh2023}. Gleiches gilt auch für Gelelektrolyte, deren hohe Ionenleitfähigkeit mit sehr geringen mechanischen Eigenschaften einhergeht~\cite{Gayet2009, Li2018, Zhao2020a}. Für den Einsatz in Strukturbatterien kommen daher nur zweiphasige oder feste Elektrolytsysteme in Frage~\cite{Greenhalgh2023}.

%\subsubsection{Zweiphasige Electrolytesysteme}
\begin{figure}[!h]
	%\raggedleft
		%\def\svgwidth{\columnwidth}
        \center
		%\input{Abbildungen/02_SoA/electrolyte_data/}
		\includegraphics[width=\textwidth, angle=0]{SElectrolyte_PhaseSeparation.pdf}
	%\includegraphics[width=\textwidth, angle=0]{bicontinous_electrolyte.pdf}
		\caption{\label{fig:SE_PhaseSepearation}Herstellungs von zweiphasigen Strukturelektrolyten durch thermische oder UV-induzierte Polymerisierung eines Mischsystems~\cite{Schneider2019}.}
\end{figure}

Zweiphasige Elektrolyte bestehen aus einer festen Phase, die für die mechanischen Eigenschaften verantwortlich ist, und einer flüssigen oder gelförmigen Phase, die für die Leitung der Ionen zuständig ist~\cite{Ichino1995} (Bild~\ref{fig:SE_PhaseSepearation}). Durch die Einstellung des Phasenanteils und die Kontrolle der Porenarchitektur können die resultierenden Eigenschaften zwischen maximaler Leitfähigkeit und minimalen mechanischen Eigenschaften und umgekehrt eingestellt werden (Bild~\ref{fig:bicontinous_electrolyte}). In simulativen Studien wurden mögliche ideale Architekturen und Phasenanteile zur maximalen Steigerung der Multifunktionalität bereits bestimmt~\cite{Lee2019,Tu2020}. Jedoch gibt es bisher nur eine bekannte Studie, die diese Nanostrukturen mithilfe von 3D-Druck fertigte~\cite{Zekoll2018}. In dieser Studie kennte jedoch der Grenzwert von $2,7 \times 10^4$ $\si{\milli \siemens \per \cm}$ nicht überschritten werden. Daher konzentrieren sich die aktuellsten Arbeiten hauptsächlich auf die Fertigung relativ ungeordneter Strukturen~\cite{Greenhalgh2023}. Bei der Wahl des Phasenanteils sind dabei auch Untersuchungen aus der Perkolationstheorie, die sich mit der Bildung von weitreichenden Verbindungen in zufälligen Systemen beschäftigt, zu beachten. Untersuchungen mit zufällig angeordneten Kugeln zeigen, dass erst bei einem Anteil von 29~\% an leitender Phase ein Grenzwert überschritten wird, bei dem eine durchgehende Verbindung und damit die Möglichkeit des Ionentransports von einer Elektrode zur anderen sichergestellt werden kann~\cite{Li2020b}. Die gleiche Studie zeigt auch, dass durch Abweichung von der sphärischen Struktur dieser Anteil auf 23~\% reduziert wird. Die Perkolationstheorie ist somit ein wichtiges Mittel zur Untersuchung des Verhaltens von zweiphasigen Elektrolyten und bietet zum Beispiel Erklärungsansätze, warum bei bestimmten Anteilen von flüssiger Phase scheinbar plötzlich eine Steigerung der Ionenleitfähigkeit um mehrere Größenordnungen beobachtet wird~\cite{Melodia2023}.

Für die feste Phase haben sich auf duroplastischen Kunstoffen basierende Systeme hauptsächlich wegen der einfacheren Löslichkeit ihrer Monomere und damit kleineren Porenbildung durchgesetzt~\cite{Snyder2009,Li2018,Choi2018,Lee2019}. Jedoch treten vermehrt thermoplastische Systeme in den Vordergrund~\cite{Melodia2023}. Diese sind leichter in den Fertigungsprozess zu integrieren und bieten zusätzliche Sicherheit bei auftretenden Kurzschlüssen, da bei der entstehenden Wärmeentwicklung der Thermoplast schmilzt und die Poren verschließt, was einen weiteren Ladungsaustausch unterbindet~\cite{Roth2012}.
\begin{figure}[ht]
	%\raggedleft
		%\def\svgwidth{\columnwidth}
        \center
	\includegraphics[width=\textwidth, angle=0]{bicontinous_electrolyte.pdf}
		\caption{\label{fig:bicontinous_electrolyte}Veränderung der Zugsteifigkeit und der Ionenleitfähigkeit mit zunehmenden festem Phasenanteil bei zweiphasigen Elektrolyten (a-d).}
\end{figure}
Als Ausgangsmaterialien für die flüssige Phase kommen ionische Flüssigkeiten~\cite{Huang2022,Shirshova2013,Wendong2021,Shirshova2014,Dzienia2020}, Lithiumsalzlösungen in organischen Lösemitteln~\cite{Gienger2015,Sakakibara2017}, deren Kombination~\cite{Shirshova2014,Yu2016} und andere Systeme~\cite{Feng2017} in Betracht.

Dem gegenüber bestehen Feststoffelektrolyten meist aus einer Polymermatrix mit flexiblen Ketten, um die Bewegung eines gelösten Salzes zu ermöglichen. Der Hauptvorteil dieses Ansatzes liegt im Verzicht auf flüchtige oder brennbare Bestandteile und in den durch das Polymer bestimmten vergleichsweise guten mechanischen Eigenschaften. Allerdings ist die ionische Leitfähigkeit bei Raumtemperatur deutlich geringer als bei zweiphasigen Elektrolyt. Die Herstellung von Feststoffelektrolyten kann auf zwei Wegen erfolgen. Eine Möglichkeit stellt dabei die Polymerisation in Anwesenheit von Lithiumsalz dar. \textsc{Snyder} et al.~\cite{Snyder2007, Snyder2009} erreichten mit diesem Ansatz Ionenleitfähigkeiten von $1,6 \times 10^{-5}$ bis $1,7 \times 10^{-3}$ $\si{\milli \siemens \per \cm}$ und damit verbundene Zugmodule von 552 bis 15~$\si{\MPa}$. Der zweite, häufig verwendete Ansatz nutzt eine Mischung von Polymeren mit Lithiumsalz. Für diese komplett festen zweiphasigen Elektrolytsysteme kommen häufig Epoxidharze~\cite{Matsumoto2011,Munoz2021,Wang2020b} oder Polyethylenoxid-~\cite{Moreno2011,Ji2010,Guo2021} zum Einsatz.

\subsection{Separator}

Separatoren befinden sich zwischen den beiden Elektroden und dienen hauptsächlich dem Verhindern eines elektrischen Kurzschlusses. Daraus folgt die Anforderung, dass Separatormaterialien für Elektronen nicht durchlässig sind, aber Transportmechanismen für Ladungsträger bereitstellen~\cite{Kurzweil2015}. Um Kurzschlüsse auch bei Rissen infolge mechanischer Belastungen zu verhindern, muss ein Bruch der Separatorschicht verhindert werden~\cite{Asp2015}. Weitere wichtige Eigenschaften sind eine hohe chemische Beständigkeit und thermische Stabilität, Korrosionsbeständigkeit, geringe Dichte, geringe Dicke, gute Verfügbarkeit und geringe Materialkosten~\cite{Beard2019}. Im Kontext von Strukturbatterien und der häufigen Verwendung von festen Elektrolyten ist der Einsatz von Separatoren theoretisch nicht notwendig, da im Gegensatz zu flüssigen Elektrolyten das Auseinanderhalten der Elektroden auch unter Druck gewährleistet wird. Jedoch ist im Rahmen von Sicherheitsbedenken und aufgrund einer signifikanten Erschwerung des Herstellungsprozesses ohne diese der Einsatz von Separatoren immer noch ein wichtiges Element in der Konzeptionierung von Strukturbatterien~\cite{Asp2015, Hubert2022}. Das am meisten verwendete Separatormaterial ist Glasfasergewebe (GF)~\cite{Zhou2022}. Jedoch existieren auch andere vielversprechende Separatormaterialien, wie polymerbasierte Separatoren, Keramiken und Cellulose~\cite{Simon2008, Greenhalgh2023, Chaudhary2024a} (Tabelle~\ref{tab:separator_comp}).

\begin{table}[h!]
    \caption{Properties of different types of separators}
    \label{tab_separator_comp}
    %\begin{adjustwidth}{-\extralength}{0cm}
    \newcolumntype{C}[1]{>{\hsize=#1\hsize\centering\arraybackslash}X}%
    \begin{tabularx}{\textwidth}{
    %C{0.6}
    C{1} 
    C{1.8} 
    C{0.8} 
    C{0.8} 
    C{0.8} 
    C{0.6}
    }
        \toprule
        \textbf{Separatortyp}
        &\textbf{Separatormaterial} 
        &\textbf{Ionische Leit- fähigkeit\textsuperscript{*} (mS/cm)} 
        &\textbf{Zug- steifigkeit\textsuperscript{*} (GPa)}
        & \textbf{Festigkeit\textsuperscript{*} (MPa)}
        &\textbf{Ref.} \\
        \midrule
        %\legendsep{c0}&
        Glassfaser&Glassfaser&1.13&21
        &325
        &\cite{Deka2017}\\
        %\midrule
        \addlinespace
        %\legendsep{c10}&
        Polymer&RF/PLA&110&0.3271
        &15.2
        &\cite{Vargun2020}\\
        %\midrule
        \addlinespace
        %Gel polymer electrolyte&$\mathrm{PVA/KOH/K_3[Fe(CN)_6]}$&45.56&n.a.&n.a.&\cite{maHighPerformanceSolidstate2014}\\
        %%\midrule
        %\legendsep{c4}&
        Feststoff- elektrolyt&$\mathrm{PEGDGE/TETA/EMIBF_4}$&0.2&26
        &350
        &\cite{Hubert2022, Choi2022}\\
        %\midrule
        \addlinespace
        %\multirowcell{2}{\legendsep{c6}}&
        \multirowcell{2}{Keramik}
            &$\mathrm{PVDF/PPG/LiCl/CaTiO_3}$&n.a.&1.2
            &65
            &\cite{Alvarez‐Sanchez2019}\\
            &$\mathrm{PVB/Al_2O_3NW}$&13.5&n.a.
            &30
            &\cite{Liu2020a}\\
        %\midrule
        \addlinespace
        %Diode-like polymer electrolyte&PVP/PEI/SWCNT&n.a.&n.a.&n.a.&\cite{chowdhurySupercapacitorsElectricalGates2019}\\
        %%\midrule
        %Ceramic&NPs/PTFE/SiC&n.a.&n.a.&1.3&\cite{qinCeramicBasedSeparatorHighTemperature2018,zhaoInorganicCeramicFiber2017}\\
        %%\midrule
        %Tree-leave&Quercus rubra&n.a.&n.a.&n.a.&\cite{chenTrashTreasureFallen2022,wangMechanicalCharacteristicsTypical2010}\\
        %%\midrule
        %Eggshell membrane&Eggshell membrane&3.8&n.a.&6.59&\cite{yuUsingEggshellMembrane2012}\\
        %%\midrule
        %\legendsep{c8}&
        Cellulose&MCC/AMIM-Cl&298.6&5.43
        &71.71
        &\cite{Ahankari2022, Xu2020}\\
        %%\midrule
        %Graphene oxide&Graphene oxide paper&n.a.&n.a.&n.a.&\cite{shulgaSupercapacitorsGrapheneOxide2015,comptonTuningMechanicalProperties2012}\\
        %%\midrule
        %Metal-organic framework&Metal-organic framework&n.a.&n.a.&n.a.&\cite{mengMetalOrganicFrameworks2015,bundschuhMechanicalPropertiesMetalorganic2012}\\
        \bottomrule
    \end{tabularx}
    %\end{adjustwidth}
    \noindent{\footnotesize{\textsuperscript{*} The abbreviation not available (n.a.) is used.}}
\end{table}

GFbasierte Separatoren finden großen Einsatz wegen ihrer hohen mechanischen Belastbarkeit, ihrer sehr hohen thermischen Stabilität, elektrochemischen Beständigkeit und relativ geringen Materialkosten~\cite{Luo2015,Asp2019,Asp2021,Liu2022}. Allerdings haben sie im Vergleich zu anderen Materialien eine signifikant größere Dicke und benötigen eine hohe Maschenweite, um einen hohen Ionenaustausch zu ermöglichen, (Bild~\ref{fig:separator_transportation})~\cite{Danzi2021}. Dies geht allerdings mit einem höheren Kurzschlussrisiko einher~\cite{Zhou2022}.

Im Kontext von faserbasierten Separatoren wurden außer auch Aramidfasern (AF) eingehend betrachtet~\cite{Jin2023}. Hervorzuheben ist hierbei ihre Fähigkeit, Dendritwachstum zu verhindern, was bei Lithiumionenbatterien oft zu einem besseren Zyklenstabilität als bei GF führt~\cite{Tung2015,Wang2021a}.

\begin{figure}[ht]
	%\raggedleft
		%\def\svgwidth{\columnwidth}
        \center
	\includegraphics[width=\textwidth, angle=0]{separator_transportation.pdf}
		\caption{\label{fig:separator_transportation}Visualisierung verschiedener Separatormaterialien und der jeweligen Ionentransportwege (gestrichelte Linie) für: (a) Gewebe, (b) Polymer-, (c) Keramikseparator und (d) Cellulose~\cite{Zschiebsch2024}.}
\end{figure}


Materialien für Polymerseparatoren sind oft identisch zu polymerbasierten Festelektrolyten. Bei zweiphasigen Elektrolyten ist besonders die gute Durchdringung des Elektrolyten von Vorteil, die eine um den Faktor von 100 höhere Ionenleitfähigkeit als Glasfaserseparatoren ermöglicht~\cite{Wang2021a}. Jedoch führt auch hier die höhere Porosität zu einer Reduktion der mechanischen Eigenschaften~\cite{Ahankari2022}. Dieser Zielkonflikt wird auch in einer Studie von \textsc{Karabelli} et al.~\cite{Karabelli2011} deutlich, bei der die entwickelte Polymermembran eine Zugsteifigkeit von 741~$\si{\MPa}$ erreichte. Im Vergleich reduzierte sich dieser Wert für das gleiche Ausgangsmaterial mit einer Porosität von 75~\% auf 53~$\si{\MPa}$. Im Vergleich zu Glasfaser sind die mechanischen Eigenschaften jedoch signifikant geringer, weshalb Polymerseparatoren außerhalb von flexiblen Anwendungen aktuell eine geringe Rolle spielen~\cite{Zschiebsch2024}.

Keramische Separatormaterialien zeichnen sich besonders durch ihre hohe thermische Stabilität aus und werden daher auch bereits in Batterien mit hohen Betriebstemperaturen eingesetzt~\cite{Qin2017,Cheong2012}. Außerdem bieten diese Materialien neben ausgezeichneten thermischen auch hervorragende physikalische sowie elektrochemische Eigenschaften. Jedoch besitzen derzeit verfügbare keramische Separatoren im Vergleich zu Glasfaser nur eine unzureichende Zugfestigkeit~\cite{Qin2017}. Ein diesbezüglich vielversprechender Entwicklungsansatz ist dabei die Integration von keramischen Fasern, die sowohl als Separatoren als auch zur Verstärkung der Elektrolyt-Matrix dienen. \textsc{Zhao} et al.~\cite{Zhao2017} entwickelten einen keramischen Faserseparator für Lithium-Ionen-Batterien, indem sie unregelmäßige keramische Kurzfasern in die Matrix integrierten. Durch die Konsolidierung der Matrixmaterialien mit kontinuierlichen keramischen Fasern konnte die Festigkeit des Verbundmaterials verbessert werden. SiC-basierte Fasern weisen beispielsweise Zugfestigkeiten von bis zu 6~$\si{\GPa}$ und Elastizitätsmodulwerte von bis zu 420~$\si{\GPa}$ auf~\cite{Seydibeyoglu2017}. Zudem entwickelten \textsc{Yamamoto} et al.~\cite{Yamamoto2009} eine Methode, um ausgerichtete Kohlenstoffnanoröhren auf keramischen Fasern wachsen zu lassen, was die Bindung zwischen Faser und Matrix weiter verstärken und das Porenvolumen des keramischen Faserseparators erhöhen könnte, um die Ionenleitfähigkeit zu verbessern.

Das Interesse an Cellulose ist in den letzten Jahren zusammen mit einem wachsenden Forschungsfelder nachwachsende Rohstoffe und recyclebare elektrische Speicher immer mehr gewachsen~\cite{Liang2018,Teng2020}. Cellulose zeigt dabei eine sehr hohe Ionenleitfähigkeit und eine gute, wenn auch im Vergleich zu Glasfasern deutlich geringere, mechanische Festigkeit und Zugsteifigkeit~\cite{Xu2020} (Tabelle~\ref{tab:separator_comp}). Eine Möglichkeit zur Steigerung der Steifigkeit stellt dabei die Verwendung von Nanocellulose dar, die Zugsteifigkeiten von bis zu 130~$\si{\GPa}$ erreicht~\cite{Dufresne2013,Zhang2019}.

\subsection{Pouchfolie}
Herkömmliche Pouchzellen sind mit einer kunststoffbeschichteten Aluminiumhülle vor Umwelteinflüssen geschützt. Insbesondere verhindert diese Hülle, dass Feuchtigkeit in die Batterie eindringt und giftige oder brennbare Stoffe aus der Batterie entweichen können~\cite{Beard2019}. Außerdem ermöglichen die hohen mechanischen Eigenschaften und hervorragende Wärmeleitfähigkeit der Aluminiumfolie eine geringe Gesamtmasse und eine effizientere Temperaturregulierung der Zellen~\cite{Boaretto2021}. Eine zunehmend wichtiger werdende Aufgabe, die allerdings noch nicht hinreichend erfüllt wird, ist das Aufbringen eines äußeren Zelldrucks~\cite{Sakamoto2019}.

In mehreren Studien konnte gezeigt werden, dass durch einen hohen externen Druck die Kontaktierung zwischen Elektrode und Elektrolyt verbessert wird, was einen besseren Ionen- und Elektronentransport bewirkt. Außerdem können unerwünschte Nebenreaktionen, wie etwa Gasbildung und Dendritwachstum,  unterdrückt werden, was den Lithiumverlust beim Laden und Entladen reduziert. Somit kann dem Kapazitätsverlust entgegenwirkt und das Batterieleben verlängert werden~\cite{Mussa2018,Mueller2019,Sakamoto2019}. Besonders Batterien mit Feststoffelektrolyten benötigen einen deutlich höheren Druck, um den Kontakt zwischen Elektrode und Elektrolyt zu gewährleisten \cite{Boaretto2021}. Hierbei wird bei der Herrstellung mittles Verpressen der Elektroden ein gewisser Druck realisiert, allerdings können größere Drücke mit diesem Vorgehen nicht appliziert oder über längere Zeit aufrechterhalten werden \cite{Garayt2023}. Daher wird oft versucht, durch eine externe Einspannung auf Systemebene diesen Druck aufzubringen. Jedoch entsteht durch die innere Reibung der Batterien kein gleichmäßiger Druckverlauf, was zu stärkeren Belastungen der äußeren Zellen führt. Des Weiteren werden durch die mit steigenden Drücken verbundenen höheren mechansichen Belastungen  dickere Schutzfolien benötigt, was zu einer niedrigeren Gesamtenergiedichte führt~\cite{Ye2024,Asp2021}.

%Einzig die Knopfzellen, die durch eine integrierte Feder einen definierten Druck auf eine im Verhältnis zur Pouchzelle deutlich kleinere Fläche ausüben, sind die einzige bekannte Lösung zu diesem Problem. Hinzukommt, dass auch hier der Massenanteil von Gehäuse zu Zelle deutlich höher ist als bei Pouchzellen.

Für Strukturbatterien sind bisher keine Alternativen zur herkömmlichen Aluminiumpouchfolie untersucht worden \cite{Ye2024}. Jedoch gibt es viele Gruppen, die ihre Strukturbatterien mit Pouchfolie zusätzlich in einen kohlefaserverstärkten Kunststoff einbetten \cite{Pattarakunnan2020,Asp2021}. Eine Realisierung der Druckvorspannun für diese Strukturbatterien wurde bisher noch nicht untersucht.

\section{Ansätze zur Entwicklung und Auslegung von Strukturbatterien}

In aktuellen Forschungsarbeiten konzentriert sich die Forschungsgemeinschaft hauptsächlich auf verschiedene Kohlefasern als Anode, \ce{FePO4} als Aktivmaterial für die Kathode, eine Vielzahl an verschiedenen zweiphasigen Strukturelektrolyten, GF als Separatorschicht und konventionellen Pouchfolien als Komponenten für verschiedene Strukturbatterien~\cite{Asp2021,Jin2023, Asp2024,Chaudhary2024}. Bei den Kohlefasern zielt die aktuelle Forschung darauf ab, die Eigenschaften über die Herstellung zielgerichtet auf die Anforderungen der Strukturbatterie anzupassen~\cite{Asp2024}. Unter Anderem wird in diesem Kontext versucht die Lithiumdiffusion in die Faser zu beschleunigen, ohne dabei die Faserstabilität zu reduzieren~\cite{Duan2021,Larsson2023,Johansen2024,Asp2024}. Forschungsgruppen wie \textsc{Asp} et al. setzen auf \ce{FePO4} als Kathodenmaterial wegen seiner höheren elektrochemischen Stabilität und leichteren Verarbeitbarkeit~\cite{Asp2021, Siraj2023, Ye2024, Chaudhary2024}. Auch spielt bei dieser Entscheidung die Vergleichbarkeit eine wichtige Rolle, die für die Verbesserung der Kohlefasern notwendig ist und mit einem Wechsel auf z.B. \ce{NMC} nicht mehr gegeben wäre~\cite{Asp2024}. Die aktuellen Strukturelektrolyte sind aus mechanischer Sicht das schwächste Element~\cite{Lee2019,Jin2023}. Die Kombination aus fester Polymermatrix und flüssigem Transportmedium für die Ionen wird allgemein als der vielversprechendste Ansatz betrachtet~\cite{Lee2019,Asp2021, Greenhalgh2023}. Dabei spielt besonders die Kontrolle über die Porenstruktur eine wichtige Rolle~\cite{Lee2019}. Der Einsatz von Glasfasern als Separator und konventionellen Pouchfolien hängt mit der Abkehr von faserbasierten Aufbauten hin zu einem laminatbasierten Design zusammen~\cite{Zhao2020,Xu2022}.

Die ursprüngliche Idee, bei der jede Faser eine zylinderförmige Zelle ist, wurde besonders zu Beginn als vielversprechendes Ziel angesehen~\cite{Ekstedt2010, Leijonmarck2013, Asp2014}. Die gewonnenen Erkenntnisse zeigen jedoch die Probleme, wie etwa komplexe Herstellung, ein hohes Ausfallrisiko und mangelnder struktureller Eigenschaften auf~\cite{Asp2015,Johannisson2018,Asp2021, Ye2024}. Der laminare Aufbau aus mehreren aufeinander liegenden Schichten orientiert sich an dem bereits viel untersuchten Pouchzelldesign und erlaubt somit eine einfachere Herstellung~\cite{Johannisson2018, Xu2022, Siraj2023}. Darüber hinaus ermöglicht die vergleichsweise einfache Integration von gewebten Kohlefaserstrukturen mit signifikant höheren mechanischen Eigenschaften in alle Richtungen~\cite{Xu2022}. Des Weiteren kann durch die Einbringung von Glasfaserseparatoren das Kurzschlussrisiko deutlich reduziert werden~\cite{Siraj2023}. Pouchfolien spielen in den aktuellen Forschungsarbeiten zu Strukturspeichern nur eine untergeordnete Rolle~\cite{Jin2023}. \textsc{Asp}, \textsc{Greenhalgh} und \textsc{Zenkert} argumentieren sorgar, dass diese bei der Betrachtung von Strukturspeichern keine Rolle spielen sollten~\cite{Asp2024}. Des Weiteren werden der Massenanteil der Pouchfolie am Gesamtsystem in aktuellen Studien zu Strukturbatterien herausgerechnet~\cite{Danzi2021,Ye2024}. Leichter Zugang zu konventionellen Pouchbag-Systemen, leichte Verarbeitung und geringe Versagensraten führen dazu, dass fast ausschließlich konventionelle Pouchbagsysteme zum Einsatz kommen~\cite{Jin2023,Ye2024}.

Die Entwicklung neuer Strukturbatterien baut hauptsächlich auf eigenen Erfahrungswerten auf. Der angegebene Hauptgrund dafür ist der Mangel an Vergleichbarkeit zwischen den einzelnen Arbeiten, welcher auf einen Mangel an standardisierten Charakterisierungsmethoden zurückzuführen ist. Zukünftig sollen hierfür die über die Jahre ausgebauten Simulationsframeworks von \textsc{Carlstedt}~\cite{Carlstedt2022,Carlstedt2022a,Carlstedt2022b} und \textsc{Johansen}~\cite{Larsson2023,Siraj2023,Johansen2024} Abhilfe schaffen. Diese Modelle beschreiben die Vorgänge auf der Meso- bzw. Partikelebene~\cite{Carlstedt2020b,Carlstedt2022a} und kombinieren bestehende Batteriemodelle von \textsc{Newman}~\cite{Bernardi1985,Pals1995,Pals1995a,Christensen2006,Newman2021} und \textsc{Doyle}~\cite{Doyle1993,Doyle1995} mit klassischer Mechanik. Durch den Fokus auf kleinere Skalen können z.B. Effekte durch Faserausdehnung infolge von Lithiierung berücksichtigt werden~\cite{Carlstedt2019,Carlstedt2020b,Duan2021,Johansen2024}. Jedoch werden dazu eine Vielzahl an teils aufwendig zu bestimmenden Materialgrößen benötigt~\cite{Plett2015}. Die Simulation größerer Systeme erfordert aufgrund des hohen Detailgrades eine signifikante Menge an Rechenkapazität~\cite{Carlstedt2019,Carlstedt2020b,Carlstedt2022a}. Die Forschungsgruppe um \textsc{Simone} konnte mit ihrer Methode die Anzahl der zu simulierenden Fasern von 20 auf 25.000 Kurzfasern steigern~\cite{Goudarzi2022}. Jedoch erfordert die Simulation immer noch einen sehr hohen Rechenbedarf. Hinzu kommt, dass der bisherige Fokus darauf liegt, die bestehenden Modelle in ihrer gesamten Komplexität unter Berücksichtigung aller Interaktionen hinsichtlich ihrer oft nicht-linearen thermisch, mechanisch, elektrochemisch Zusammenhänge zu implementieren~\cite{Carlstedt2019,Carlstedt2020b,Carlstedt2022a,Johansen2024}. Aktuelle Arbeiten zur Bildung von SEI könnten diesbezüglich zur ersten Erweiterung der bestehenden Modelle über das bereits in der Batterieforschung bekannte hinausführen~\cite{Yuecel2024}. Bisherige Forschungen zu diesem Themenkomplex zeigen jedoch große Varianzen, die vermutlich durch verschiedene chemische Interaktionen hervorgerufen werden~\cite{Rollin2023,Yuecel2024}. Eine Erweiterung der multiskaligen Methode um atomistische Betrachtungen, um diese Effekte vorherzusagen, ist jedoch mit der Gefahr verbunden, den Rechenaufwand durch rechenintensive Dichtefunktionaltheorie oder molekular dynamische Simulationen weiter zu erhöhen~\cite{Franco2019,Li2020a,Rollin2023}. Dies führt dazu, dass die aktuelle Forschung primär auf Experimente vertraut und die aus den physikalischen Simulationsmodellen folgenden Materialparameter erst im Nachhinein, z.B. durch Regressionsverfahren, bestimmt werden~\cite{Franco2013, Carlstedt2022, Carlstedt2023}.

\section{Ungelöste Herausforderungen in der Entwicklung von Strukturbatterien}
% Aktuell besteht diese überschaubare Forschungsgemeinschaft hauptsächlich aus Fachleuten der Kohlefasertechnologie, namentlich \textsc{Greenhalgh}~\cite{Greenhalgh2023}, \textsc{Asp}~\cite{Asp2019,Asp2021}, \textsc{Zenkert}~\cite{Zenkert2024} etc., Polymerespezialisten, wie etwa \textsc{Bismarck}~\cite{Bismarck2012}, \textsc{Shirshova}~\cite{Shirshova2013} und Nanomaterialexperten, wie z.B. \textsc{Shaffer}~\cite{Senokos2023}. Fachgebietsübergreifende Expertisen gibt es aktuell kaum, obwohl sich dies in Zukunft mit der Etablierung spezieller Masterprogramme ändern könnte. Auch gibt es aktuell nur sehr wenige Experten mit einem Hintergrund in Batteriechemie und -herstellung~\cite{Asp2013US}. Hinzu kommt, dass es außer den Forschungsgruppen um \textsc{Asp} und \textsc{Zenkert} nur wenige gibt, die ganze Strukturbatterien entwickeln und fertigen können. Dies liegt vor allem an den Schwierigkeiten, eine reproduzierbare und skalierbare Fertigung aufzubauen~\cite{Siraj2023}.

Aufgrund der sehr limitierten Fertigungsmöglichkeiten wird der Bau neuer Strukturbatterien maßgeblich durch die Froschungsteams rund um \textsc{Asp} und \textsc{Zenkert} geprägt. Bei beiden Forschern liegt der Fokus hauptsächlich auf den mechanischen~\cite{Carlstedt2019a,Asp2021,Duan2021} und aktuatorischen Eigenschaften~\cite{Carlstedt2023}. Seit 2009 wurde die Energiedichte von 35~$\si{\watt \hour \per \kg}$~\cite{Liu2009} auf 41~$\si{\watt \hour \per \kg}$~\cite{Siraj2023} gesteigert, während die Zugsteifigkeit von 0,7~$\si{\GPa}$ auf 26~$\si{\GPa}$ anstieg. Eine weitere Ursache könnte allerdings auch in dem zunehmenden Umschwenken von Lithium zu Natrium liegen, welches hauptsächlich durch wachsendes Umweltbewusstsein motiviert ist~\cite{Peters2022}. Jedoch sind die theoretisch erreichbaren Energiedichten von Natrium signifikant geringer~\cite{Kundu2015}. Hinzu kommen Herausforderungen durch die vergleichsweise langsame Diffusion in die Kohlefasern, deren Steigerung durch das Einbringen von Ionentransportwegen wie etwa Poren direkt mit einem signifikanten Verlust der mechanischen Eigenschaften einhergeht. Des Weiteren wurden aus elektrochemischer Sicht vielversprechendere Kathodenmaterialien, wie NMC, bisher zugunsten von \ce{FePO4} kaum untersucht~\cite{Asp2024}. Ausschlaggebend ist hierbei die bisher ungelöste Herausforderung, die NMC-Partikel, die in etwa die gleiche Größe wie der Faserdurchmesser haben, effektiv in das Strukturbatteriedesign zu integrieren~\cite{Asp2014}.

Die Einbeziehung von weniger vertrauten Materialien in die eigene Forschung wird dabei auch durch den in Kapitel~\ref{sec:Ermittlungsverfahren} beschriebenen Mangel an standardisierten Tests bestärkt~\cite{Greenhalgh2024a}. Dieser führt dazu, dass viele Forschungsgruppen eher den eigenen Arbeiten vertrauen und darauf aufbauen, während neue externe Erkenntnisse spärlicher mit einbezogen werden~\cite{Greenhalgh2023, Greenhalgh2024a}. Hinzu kommt, dass simulative Tools zwar beständig weiterentwickelt werden, jedoch oft ein oder zwei Ebenen unterhalb der erstmals evaluierbaren Zellebene beginnen und daher Skalierungsproblemen auftreten~\cite{Franco2019,Giessen2020}. Zusätzlich führt ihr höherer Detailgrad zu einer enormen Menge an zu bestimmenden Materialdaten, die oft signifikant von mehreren Parametern, wie etwa Temperatur und Ladegeschwindigkeit, abhängen~\cite{Duan2021,Carlstedt2022b}, oft nur sehr aufwendig bestimmt werden können und teils lange Simulationszeiten benötigen~\cite{Franco2019,Carlstedt2022}. Dies führt dazu, dass die meisten simulativen Methoden erst im Nachhinein eingesetzt werden, bei denen fehlende Daten durch Regression bestimmt wurden~\cite{Carlstedt2022a, Carlstedt2023}. Damit dienen die aktuellen Simualtionen der Verbesserung des Verständnisses des jeweiligen Einflusses, aber nicht um die aufwendigen Experimente zu ersetzen~\cite{Franco2013}. Die Herausforderung Entwicklungsprozess von Strukturbatterien durch digitale und datengetriebene Ansätze zu beschleunigen ist daher bisher ungelöst.

Strukturbatterien sind ein vergleichsweise noch wenig untersuchter Ansatz~\cite{Asp2024}. Daher gibt es noch zahlreiche Herausforderungen zu lösen. Das oft erwähnte Ziel des elektrischen Fliegens ist aufgrund der vielen Regularien eine Anwendung, die erst nach der deutlichen Leistungsvorteilen gegenüber den konventionellen Lösungen überhaupt in Betracht zu ziehen~\cite{Scholz2018,Schaefer2018,Ishfaq2022}. Auch für Anwendungen in Automobilbau sind die aktuell zu geringen Energiedichten und Sicherheitsbedenken, verbunden mit dem Auswechseln dieser hochintegrativen Lösung im Schadensfall, ein bisher ungelöste Herausforderung~\cite{Kalnaus2021,Martins2021, Carlstedt2020a}. Eine fundierte Betrachtung und Erörterung potenzieller Anwendungen ist daher von großer Relevanz.

%(siehe Abschnitt \ref{ch:anhang})